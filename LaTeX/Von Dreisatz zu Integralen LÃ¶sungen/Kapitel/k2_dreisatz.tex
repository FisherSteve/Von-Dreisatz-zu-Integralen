% --- ANNAHME: Alle Pakete und Box-Definitionen aus dem Hauptdokument sind hier gültig ---
% --- Dieser Code-Block ist als Ersatz für das bestehende Kapitel 2 gedacht ---

\section{Der Dreisatz – Dein Werkzeug für Verhältnisse}
\label{sec:dreisatz_ueberarbeitet}


\begin{aufgabenumgebung}[aufg:X.Y]{Bücherkauf} % Passen Sie das Label und die Aufgabennummer an
Wenn 4 gleiche Notizbücher zusammen 10 Euro kosten, wie viel kosten dann 7 dieser Notizbücher?
Zeichne auch ein kleines Schema wie im Beispiel (mit den Pfeilen $:4$ und $\cdot7$).
\end{aufgabenumgebung}

\begin{loesungsumgebung}[loes:2.1]{} % Passen Sie das Label und die Aufgabennummer an, z.B. loes:4.3 für Aufgabe 4.3
Diese Aufgabe lässt sich hervorragend mit dem \textbf{Dreisatz} lösen. Der Dreisatz hilft uns, Verhältnisse zu berechnen, indem wir zunächst den Wert für eine Einheit bestimmen.

1.  \textbf{Schritt 1: Preis für ein Notizbuch berechnen.}
    Wir wissen, dass 4 Notizbücher 10 Euro kosten. Um den Preis für ein Notizbuch zu erfahren, teilen wir den Gesamtpreis durch die Anzahl der Notizbücher:
    $10 \text{ Euro} : 4 \text{ Notizbücher} = 2,50 \text{ Euro pro Notizbuch}$

2.  \textbf{Schritt 2: Preis für 7 Notizbücher berechnen.}
    Da wir nun den Preis für ein Notizbuch kennen (2,50 Euro), können wir diesen mit der gewünschten Anzahl (7 Notizbücher) multiplizieren:
    $2,50 \text{ Euro/Notizbuch} \cdot 7 \text{ Notizbücher} = 17,50 \text{ Euro}$

\textbf{Schema zur Veranschaulichung:}

\begin{center}
\begin{tabular}{c c c}
    4 Notizbücher & $\xrightarrow{\text{ : } 4}$ & 1 Notizbuch \\
    10 Euro & $\xrightarrow{\text{ : } 4}$ & 2,50 Euro \\
    \quad \\ % Kleiner vertikaler Abstand
    1 Notizbuch & $\xrightarrow{\text{ } \cdot 7}$ & 7 Notizbücher \\
    2,50 Euro & $\xrightarrow{\text{ } \cdot 7}$ & 17,50 Euro \\
\end{tabular}
\end{center}

\vspace{0.5em} % Kleiner Abstand nach der Tabelle

\textbf{Antwort:} 7 Notizbücher kosten 17,50 Euro.

\begin{tippumgebung}{Dreisatz-Prinzip verstehen}
Der Dreisatz basiert auf der \textbf{direkten Proportionalität}. Wenn zwei Größen direkt proportional sind, bedeutet das, dass sich die eine Größe im gleichen Verhältnis ändert wie die andere. Im Beispiel: Wenn die Anzahl der Notizbücher steigt, steigt auch der Preis proportional. Dies ist die Grundlage für viele Alltagsberechnungen und ein wichtiges Fundament in der Mathematik.
\end{tippumgebung}

\end{loesungsumgebung}

% Beispiel für die Überschrift, passen Sie diese an Ihre Kapitelstruktur an
\begin{aufgabenumgebung}[aufg:2.2]{Autoverbrauch} % Passen Sie das Label und die Nummer an
Ein Auto verbraucht auf einer Strecke von 150 km genau 12 Liter Benzin.
\begin{enumerate}
    \item Wie viel Benzin verbraucht es auf einer Strecke von 250 km?
    \item Wie weit kommt das Auto mit einem vollen Tank von 40 Litern? (Tipp: Hier ist die Literzahl gegeben und die Kilometer sind gesucht! Du rechnest also erst aus, wie weit es mit 1 Liter kommt.)
\end{enumerate}
\end{aufgabenumgebung}

\begin{loesungsumgebung}[loes:2.2]{} % Passen Sie das Label und die Nummer an, z.B. loes:2.2
Diese Aufgabe besteht aus zwei Teilen, die beide mit dem Dreisatz gelöst werden können.

\textbf{Gegebene Informationen:}
\begin{itemize}
    \item Verbrauch: 12 Liter für 150 km
\end{itemize}

\subsubsection*{Teilaufgabe a) Benzinverbrauch auf 250 km}

Hier wollen wir herausfinden, wie viele Liter für eine gegebene Kilometerzahl benötigt werden.

1.  \textbf{Schritt 1: Verbrauch pro Kilometer berechnen.}
    Um zu wissen, wie viel Benzin pro 1 km verbraucht wird, teilen wir die Liter durch die Kilometer:
    $\frac{12 \text{ Liter}}{150 \text{ km}} = 0,08 \text{ Liter/km}$

2.  \textbf{Schritt 2: Benzin für 250 km berechnen.}
    Jetzt multiplizieren wir den Verbrauch pro Kilometer mit der neuen Strecke:
    $0,08 \text{ Liter/km} \cdot 250 \text{ km} = 20 \text{ Liter}$

\textbf{Schema zur Veranschaulichung (Teil a):}

\begin{center}
\begin{tabular}{c c c}
    150 km & $\xrightarrow{\text{ : } 150}$ & 1 km \\
    12 Liter & $\xrightarrow{\text{ : } 150}$ & 0,08 Liter \\
    \quad \\
    1 km & $\xrightarrow{\text{ } \cdot 250}$ & 250 km \\
    0,08 Liter & $\xrightarrow{\text{ } \cdot 250}$ & 20 Liter \\
\end{tabular}
\end{center}

\textbf{Antwort zu a):} Das Auto verbraucht auf einer Strecke von 250 km genau 20 Liter Benzin.

\subsubsection*{Teilaufgabe b) Gefahrene Strecke mit 40 Litern}

Hier ist die Situation umgekehrt: Wir wissen, wie viele Liter Benzin wir haben, und wollen wissen, wie weit wir damit fahren können.

1.  \textbf{Schritt 1: Reichweite pro Liter berechnen.}
    Um zu wissen, wie viele Kilometer man mit 1 Liter fahren kann, teilen wir die Kilometer durch die Liter:
    $\frac{150 \text{ km}}{12 \text{ Liter}} = 12,5 \text{ km/Liter}$

2.  \textbf{Schritt 2: Reichweite mit 40 Litern berechnen.}
    Nun multiplizieren wir die Reichweite pro Liter mit dem gegebenen Tankvolumen:
    $12,5 \text{ km/Liter} \cdot 40 \text{ Liter} = 500 \text{ km}$

\textbf{Schema zur Veranschaulichung (Teil b):}

\begin{center}
\begin{tabular}{c c c}
    12 Liter & $\xrightarrow{\text{ : } 12}$ & 1 Liter \\
    150 km & $\xrightarrow{\text{ : } 12}$ & 12,5 km \\
    \quad \\
    1 Liter & $\xrightarrow{\text{ } \cdot 40}$ & 40 Liter \\
    12,5 km & $\xrightarrow{\text{ } \cdot 40}$ & 500 km \\
\end{tabular}
\end{center}

\textbf{Antwort zu b):} Das Auto kommt mit einem vollen Tank von 40 Litern genau 500 km weit.

\begin{warumwichtigumgebung}{Anwendung des Dreisatzes in verschiedenen Situationen}
Der Dreisatz ist ein grundlegendes Werkzeug für viele Berechnungen im Alltag und in der Wissenschaft. Es ist wichtig zu erkennen, ob Sie eine direkte oder indirekte Proportionalität vorliegen haben. In diesem Beispiel handelt es sich um \textbf{direkte Proportionalität}: Je mehr Benzin, desto mehr Kilometer (und umgekehrt). Achten Sie immer darauf, die Einheit zu bestimmen, die Sie für den nächsten Rechenschritt benötigen.
\end{warumwichtigumgebung}

\end{loesungsumgebung}

\subsection*{Aufgabe 2.3} % Oder die entsprechende Nummer in Ihrem Kapitel
\begin{aufgabenumgebung}[aufg:2.3]{Dreisatz – Weitere Übungen}
Überlege bei jeder Aufgabe zuerst, ob sie proportional oder antiproportional ist! Schreibe deine Überlegung kurz auf.
\begin{itemize}
    \item 6 gleiche Laborflaschen kosten 15 Euro. Wie viel kosten 10 solcher Flaschen?
    \item 4 Pumpen füllen ein Schwimmbecken in 12 Stunden. Wie lange würden 6 gleiche Pumpen brauchen, wenn sie alle die gleiche Leistung haben?
    \item 2 kg Äpfel kosten 3,80 Euro. Bestimme den Preis für 750 g Äpfel. (Tipp: Wandle Gramm in Kilogramm um: $750\,\text{g} = 0,75\,\text{kg}$, oder rechne alles in Gramm.)
    \item Ein Futtervorrat reicht für 12 Pferde 20 Tage lang. Wie lange reicht der gleiche Vorrat für 15 Pferde (wenn alle Pferde gleich viel fressen)?
\end{itemize}
\end{aufgabenumgebung}

\begin{loesungsumgebung}[loes:2.3]{} % Oder die entsprechende Nummer in Ihrem Kapitel
Diese Aufgabe besteht aus mehreren Dreisatz-Problemen, die sowohl proportionale als auch antiproportionale Zuordnungen umfassen. Es ist wichtig, den Unterschied zu verstehen, um den Dreisatz korrekt anzuwenden.

\subsubsection*{Lösung zu 6 Laborflaschen kosten 15 Euro. Wie viel kosten 10 solcher Flaschen?}

\textbf{Überlegung:} Wenn die Anzahl der Flaschen steigt, steigen auch die Kosten. Dies ist eine \textbf{proportionale Zuordnung}.

1.  \textbf{Schritt 1: Preis pro Flasche berechnen.}
    Wir wissen, dass 6 Flaschen 15 Euro kosten. Um den Preis pro 1 Flasche zu erhalten, teilen wir die Gesamtkosten durch die Anzahl der Flaschen:
    $\frac{15 \text{ Euro}}{6 \text{ Flaschen}} = 2,50 \text{ Euro/Flasche}$

2.  \textbf{Schritt 2: Preis für 10 Flaschen berechnen.}
    Nun multiplizieren wir den Preis pro Flasche mit der gewünschten Anzahl von Flaschen:
    $2,50 \text{ Euro/Flasche} \cdot 10 \text{ Flaschen} = 25 \text{ Euro}$

\textbf{Antwort:} 10 Laborflaschen kosten 25 Euro.

\begin{center}
\begin{tabular}{c c c}
    6 Flaschen & $\xrightarrow{\text{ : } 6}$ & 1 Flasche \\
    15 Euro & $\xrightarrow{\text{ : } 6}$ & 2,50 Euro \\
    \quad \\
    1 Flasche & $\xrightarrow{\text{ } \cdot 10}$ & 10 Flaschen \\
    2,50 Euro & $\xrightarrow{\text{ } \cdot 10}$ & 25 Euro \\
\end{tabular}
\end{center}

\subsubsection*{Lösung zu 4 Pumpen füllen ein Schwimmbecken in 12 Stunden. Wie lange würden 6 gleiche Pumpen brauchen?}

\textbf{Überlegung:} Wenn die Anzahl der Pumpen steigt, sinkt die Zeit, die zum Füllen des Beckens benötigt wird. Dies ist eine \textbf{antiproportionale (umgekehrt proportionale) Zuordnung}. Bei antiproportionalen Zuordnungen ist das Produkt der zusammengehörenden Größen konstant.

1.  \textbf{Schritt 1: Gesamtarbeitseinheiten berechnen (oder Zeit für 1 Pumpe).}
    Die Gesamt'arbeit' entspricht der Anzahl der Pumpen multipliziert mit der Zeit. Diese Menge bleibt konstant.
    4 Pumpen $\cdot$ 12 Stunden = 48 'Pumpenstunden' (Gesamtarbeit)

    Alternativ: Wenn eine Pumpe allein arbeiten würde, würde sie 4 Mal so lange brauchen:
    12 Stunden $\cdot$ 4 = 48 Stunden (Zeit für 1 Pumpe)

2.  \textbf{Schritt 2: Zeit für 6 Pumpen berechnen.}
    Teile die Gesamtarbeit durch die neue Anzahl der Pumpen:
    $\frac{48 \text{ Pumpenstunden}}{6 \text{ Pumpen}} = 8 \text{ Stunden}$

\textbf{Antwort:} 6 gleiche Pumpen würden 8 Stunden brauchen, um das Schwimmbecken zu füllen.

\begin{center}
\begin{tabular}{c c c}
    4 Pumpen & $\xrightarrow{\text{ } \cdot 4}$ & 1 Pumpe \\
    12 Stunden & $\xrightarrow{\text{ } \cdot 4}$ & 48 Stunden \\
    \quad \\
    1 Pumpe & $\xrightarrow{\text{ : } 6}$ & 6 Pumpen \\
    48 Stunden & $\xrightarrow{\text{ : } 6}$ & 8 Stunden \\
\end{tabular}
\end{center}
\begin{merksatzumgebung}{Antiproportionale Zuordnung}
Bei einer \textbf{antiproportionalen} Zuordnung gilt: Wenn die eine Größe auf das Doppelte, Dreifache, etc. steigt, sinkt die andere Größe auf die Hälfte, ein Drittel, etc. Das Produkt der beiden Größen bleibt konstant. Oft wird hier erst der Wert für die 'Einheit 1' durch Multiplikation statt Division bestimmt, und dann durch Division der neuen Menge.
\end{merksatzumgebung}

\subsubsection*{Lösung zu 2 kg Äpfel kosten 3,80 Euro. Bestimme den Preis für 750 g Äpfel.}

\textbf{Überlegung:} Wenn die Menge der Äpfel sinkt, sinkt auch der Preis. Dies ist eine \textbf{proportionale Zuordnung}.
\textbf{Wichtig:} Einheiten angleichen! Wir rechnen alles in Kilogramm. $750\,\text{g} = 0,75\,\text{kg}$.

1.  \textbf{Schritt 1: Preis pro Kilogramm Äpfel berechnen.}
    Wir teilen den Gesamtpreis durch die Kilogramm:
    $\frac{3,80 \text{ Euro}}{2 \text{ kg}} = 1,90 \text{ Euro/kg}$

2.  \textbf{Schritt 2: Preis für 0,75 kg Äpfel berechnen.}
    Nun multiplizieren wir den Preis pro Kilogramm mit der gewünschten Menge:
    $1,90 \text{ Euro/kg} \cdot 0,75 \text{ kg} = 1,425 \text{ Euro}$

    Da es sich um Geld handelt, runden wir auf zwei Dezimalstellen: 1,43 Euro.

\textbf{Antwort:} 750 g Äpfel kosten 1,43 Euro.

\begin{center}
\begin{tabular}{c c c}
    2 kg & $\xrightarrow{\text{ : } 2}$ & 1 kg \\
    3,80 Euro & $\xrightarrow{\text{ : } 2}$ & 1,90 Euro \\
    \quad \\
    1 kg & $\xrightarrow{\text{ } \cdot 0,75}$ & 0,75 kg \\
    1,90 Euro & $\xrightarrow{\text{ } \cdot 0,75}$ & 1,425 Euro $\approx$ 1,43 Euro \\
\end{tabular}
\end{center}
\begin{tippumgebung}{Einheitenumrechnung}
Achten Sie bei Aufgaben, die unterschiedliche Einheiten für die gleiche Größe verwenden (z.B. Gramm und Kilogramm), immer darauf, diese vor der Berechnung anzugleichen. Wählen Sie entweder die kleinere oder die größere Einheit für die gesamte Rechnung, um Fehler zu vermeiden.
\end{tippumgebung}

\subsubsection*{Lösung zu Ein Futtervorrat reicht für 12 Pferde 20 Tage lang. Wie lange reicht der gleiche Vorrat für 15 Pferde?}

\textbf{Überlegung:} Wenn die Anzahl der Pferde steigt, sinkt die Dauer, für die der Futtervorrat reicht. Dies ist eine \textbf{antiproportionale (umgekehrt proportionale) Zuordnung}.

1.  \textbf{Schritt 1: Gesamtfuttereinheiten berechnen (oder Dauer für 1 Pferd).}
    Die Gesamt'futtertage' entsprechen der Anzahl der Pferde multipliziert mit der Dauer. Diese Menge bleibt konstant.
    12 Pferde $\cdot$ 20 Tage = 240 'Pferdetage' (Gesamtfuttermenge)

    Alternativ: Wenn nur 1 Pferd das Futter fressen würde, würde es 12 Mal so lange reichen:
    20 Tage $\cdot$ 12 = 240 Tage (Reichweite für 1 Pferd)

2.  \textbf{Schritt 2: Reichweite für 15 Pferde berechnen.}
    Teile die Gesamtfuttermenge durch die neue Anzahl der Pferde:
    $\frac{240 \text{ Pferdetage}}{15 \text{ Pferde}} = 16 \text{ Tage}$

\textbf{Antwort:} Der gleiche Futtervorrat reicht für 15 Pferde 16 Tage lang.

\begin{center}
\begin{tabular}{c c c}
    12 Pferde & $\xrightarrow{\text{ } \cdot 12}$ & 1 Pferd \\
    20 Tage & $\xrightarrow{\text{ } \cdot 12}$ & 240 Tage \\
    \quad \\
    1 Pferd & $\xrightarrow{\text{ : } 15}$ & 15 Pferde \\
    240 Tage & $\xrightarrow{\text{ : } 15}$ & 16 Tage \\
\end{tabular}
\end{center}
\end{loesungsumgebung}