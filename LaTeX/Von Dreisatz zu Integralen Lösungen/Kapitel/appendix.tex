\appendix
\section*{Anhang} % Verwendung von \chapter* für unnummerierten Eintrag im Inhaltsverzeichnis
\addcontentsline{toc}{section}{Anhang} % Manuelles Hinzufügen zum Inhaltsverzeichnis

\section*{A: Algebraische Formelsammlung}
\label{anhang:formelsammlung}

Diese Sammlung fasst einige wichtige algebraische Regeln und Formeln zusammen, die im gesamten Skript nützlich sind.

\subsection*{A.1 Bruchrechnung}
Für $a, b, c, d \in \mathbb{R}$ und Nenner $\neq 0$:
\begin{itemize}
    \item \textbf{Erweitern:} $\frac{a}{b} = \frac{a \cdot c}{b \cdot c}$ \quad (für $c \neq 0$)
    \item \textbf{Kürzen:} $\frac{a \cdot c}{b \cdot c} = \frac{a}{b}$ \quad (für $c \neq 0$)
    \item \textbf{Addition/Subtraktion (gleichnamig):} $\frac{a}{c} \pm \frac{b}{c} = \frac{a \pm b}{c}$
    \item \textbf{Addition/Subtraktion (ungleichnamig):} $\frac{a}{b} \pm \frac{c}{d} = \frac{a \cdot d \pm b \cdot c}{b \cdot d}$ (Hauptnenner bilden)
    \item \textbf{Multiplikation:} $\frac{a}{b} \cdot \frac{c}{d} = \frac{a \cdot c}{b \cdot d}$
    \item \textbf{Division (Kehrwert):} $\frac{a}{b} : \frac{c}{d} = \frac{a}{b} \cdot \frac{d}{c} = \frac{a \cdot d}{b \cdot c}$
    \item \textbf{Doppelbruch:} $\frac{\frac{a}{b}}{\frac{c}{d}} = \frac{a}{b} \cdot \frac{d}{c} = \frac{ad}{bc}$
    \item \textbf{Umwandlung Dezimalzahl in Bruch:} z.B. $0.75 = \frac{75}{100} = \frac{3}{4}$
    \item \textbf{Umwandlung gemischte Zahl in Bruch:} z.B. $2\frac{3}{5} = \frac{2 \cdot 5 + 3}{5} = \frac{13}{5}$
\end{itemize}

\begin{multicols}{2} % Start der zweispaltigen Umgebung

\subsection*{A.2 Potenzgesetze}
Für $a, b \in \mathbb{R}$, $a,b \neq 0$ und $m, n \in \mathbb{Q}$ (bzw. $\mathbb{Z}$ für einige Regeln, wenn die Basis negativ sein kann):
\begin{itemize}
    \item $a^m \cdot a^n = a^{m+n}$
    \item $\frac{a^m}{a^n} = a^{m-n}$
    \item $(a^m)^n = a^{m \cdot n}$
    \item $(a \cdot b)^n = a^n \cdot b^n$
    \item $\left(\frac{a}{b}\right)^n = \frac{a^n}{b^n}$
    \item $a^0 = 1$
    \item $a^1 = a$
    \item $a^{-n} = \frac{1}{a^n}$
    \item $a^{\frac{1}{n}} = \sqrt[n]{a}$ \quad (für $a \ge 0$, wenn $n$ gerade)
    \item $a^{\frac{m}{n}} = \sqrt[n]{a^m} = (\sqrt[n]{a})^m$ \quad (für $a \ge 0$, wenn $n$ gerade)
\end{itemize}

\subsection*{A.3 Wurzelgesetze}
Für $a, b \ge 0$ und $n, m \in \mathbb{N}, n,m \ge 2$:
\begin{itemize}
    \item $\sqrt[n]{a} \cdot \sqrt[n]{b} = \sqrt[n]{a \cdot b}$
    \item $\frac{\sqrt[n]{a}}{\sqrt[n]{b}} = \sqrt[n]{\frac{a}{b}}$ \quad (für $b \neq 0$)
    \item $\sqrt[m]{\sqrt[n]{a}} = \sqrt[m \cdot n]{a}$
    \item $(\sqrt[n]{a})^m = \sqrt[n]{a^m}$
    \item $\sqrt[n]{a^n} = a$ (für $a \ge 0$)
    \item $a \sqrt[n]{b} = \sqrt[n]{a^n \cdot b}$ (Teilweises Radizieren / unter die Wurzel bringen)
\end{itemize}

\subsection*{A.4 Binomische Formeln}
Für $a, b \in \mathbb{R}$:
\begin{itemize}
    \item $(a+b)^2 = a^2 + 2ab + b^2$
    \item $(a-b)^2 = a^2 - 2ab + b^2$
    \item $(a+b)(a-b) = a^2 - b^2$
\end{itemize}
Erweiterungen:
\begin{itemize}
    \item $(a+b)^3 = a^3 + 3a^2b + 3ab^2 + b^3$
    \item $(a-b)^3 = a^3 - 3a^2b + 3ab^2 - b^3$
\end{itemize}

\subsection*{A.5 Logarithmengesetze}
Für $u, v > 0$, Basis $b>0, b\neq 1$ und $r \in \mathbb{R}$:
\begin{itemize}
    \item $\log_b(u \cdot v) = \log_b(u) + \log_b(v)$
    \item $\log_b\left(\frac{u}{v}\right) = \log_b(u) - \log_b(v)$
    \item $\log_b(u^r) = r \cdot \log_b(u)$
\end{itemize}
Speziell für den natürlichen Logarithmus ($\ln$, Basis $e$):
\begin{itemize}
    \item $\ln(e) = 1$
    \item $\ln(1) = 0$
    \item $\ln(e^x) = x$
    \item $e^{\ln(x)} = x$ (für $x>0$)
\end{itemize}

\end{multicols}

\section*{B: Beispiele zur Termumformung und zum Lösen von Gleichungen}
\label{anhang:termumformungen}

\subsection*{B.1 Lineare Gleichung lösen}
Löse die folgende Gleichung nach $x$:
\[ 5(x-2) + 3x = 2(x+7) - 4 \]
\textbf{Lösungsweg:}
\begin{center}
\begin{tabular}{r @{\,} c @{\,} l @{\quad\quad} l}
$5(x-2) + 3x$ & $=$ & $2(x+7) - 4$ & \small{Klammern ausmultiplizieren} \\
$5x - 10 + 3x$ & $=$ & $2x + 14 - 4$ & \small{Terme zusammenfassen} \\
$8x - 10$ & $=$ & $2x + 10$ & $| -2x$ \\
$6x - 10$ & $=$ & $10$ & $| +10$ \\
$6x$ & $=$ & $20$ & $| :6$ \\
$x$ & $=$ & $\frac{20}{6}$ & \small{Kürzen} \\
$x$ & $=$ & $\frac{10}{3}$ & \\
\end{tabular}
\end{center}
Die Lösung der Gleichung ist $x = \frac{10}{3}$.

\subsection*{B.2 Komplexen Bruchterm vereinfachen}
Vereinfache den folgenden Bruchterm so weit wie möglich (alle Variablen seien so gewählt, dass die Ausdrücke definiert sind und keine Nenner Null werden):
\[ \frac{(a^2b^{-3})^2 \cdot (ab)^4}{a^5 b^{-1}} : \frac{(a^2b)^3}{b^5 a^{-1}} \]
\textbf{Lösungsweg:}
Wir vereinfachen zuerst den ersten Bruch (Dividend) und den zweiten Bruch (Divisor) getrennt und wenden dann die Divisionsregel für Brüche an.

\textbf{1. Dividend vereinfachen:} $D_1 = \frac{(a^2b^{-3})^2 \cdot (ab)^4}{a^5 b^{-1}}$
\begin{align*} D_1 &= \frac{(a^{2\cdot2}b^{-3\cdot2}) \cdot (a^1b^1)^4}{a^5 b^{-1}} && \text{(Potenzgesetz: $(x^m)^n = x^{mn}$)} \\ &= \frac{a^4 b^{-6} \cdot a^4 b^4}{a^5 b^{-1}} && \text{(Potenzgesetz: $(xy)^n = x^n y^n$)} \\ &= \frac{a^{4+4} b^{-6+4}}{a^5 b^{-1}} && \text{(Potenzgesetz: $x^m x^n = x^{m+n}$)} \\ &= \frac{a^8 b^{-2}}{a^5 b^{-1}} \\ &= a^{8-5} \cdot b^{-2-(-1)} && \text{(Potenzgesetz: $x^m/x^n = x^{m-n}$)} \\ &= a^3 \cdot b^{-2+1} \\ &= a^3 b^{-1} = \frac{a^3}{b} \end{align*}

\textbf{2. Divisor vereinfachen:} $D_2 = \frac{(a^2b)^3}{b^5 a^{-1}}$
\begin{align*} D_2 &= \frac{(a^{2\cdot3}b^{1\cdot3})}{b^5 a^{-1}} && \text{(Potenzgesetz: $(x^m)^n = x^{mn}$ und $(xy)^n = x^n y^n$)} \\ &= \frac{a^6 b^3}{b^5 a^{-1}} \\ &= a^{6-(-1)} \cdot b^{3-5} && \text{(Potenzgesetz: $x^m/x^n = x^{m-n}$)} \\ &= a^{6+1} \cdot b^{-2} \\ &= a^7 b^{-2} = \frac{a^7}{b^2} \end{align*}

\textbf{3. Division durchführen:} $\text{Dividend} : \text{Divisor} = D_1 : D_2 = D_1 \cdot \frac{1}{D_2}$ (bzw. mit Kehrwert multiplizieren)
\begin{align*} \frac{a^3}{b} : \frac{a^7}{b^2} &= \frac{a^3}{b} \cdot \frac{b^2}{a^7} \\ &= \frac{a^3 b^2}{b a^7} \\ &= a^{3-7} \cdot b^{2-1} && \text{(Kürzen bzw. Potenzgesetze)} \\ &= a^{-4} b^1 \\ &= \frac{b}{a^4} \end{align*}
Der vereinfachte Term ist $\frac{b}{a^4}$.