\documentclass[twoside, a4paper,12pt]{article}
\usepackage[ngerman]{babel}
\usepackage[utf8]{inputenc}
\usepackage[T1]{fontenc}
\usepackage{amsmath, amssymb, amsfonts}
\usepackage{graphicx}
\usepackage{geometry}
\usepackage{hyperref}
\usepackage{booktabs}
\geometry{margin=2cm, left=1cm, right=1cm, top=1cm}
\usepackage{polynom}
\usepackage{longdivision}
\usepackage{tcolorbox}
\tcbuselibrary{skins,breakable,hooks}
\usepackage{wasysym}
\usepackage{caption}
\usepackage{array}
\usepackage{enumitem}
\usepackage{multicol}

% Farben definieren für Boxen
\definecolor{merkboxcolorbg}{RGB}{220, 230, 255}
\definecolor{beispielboxcolorbg}{RGB}{220, 255, 220}
\definecolor{aufgabeboxcolorbg}{RGB}{255, 240, 220}
\definecolor{infoboxcolorbg}{RGB}{230, 230, 230}
\definecolor{kurzknappcolorbg}{RGB}{255, 255, 200}
\definecolor{fehlerboxcolorbg}{RGB}{255, 220, 220}
\definecolor{warumwichtigcolorbg}{RGB}{220, 240, 255}
\definecolor{tippboxcolorbg}{RGB}{230,250,230}

\definecolor{merkboxframe}{RGB}{0, 0, 150}
\definecolor{beispielboxframe}{RGB}{0, 100, 0}
\definecolor{aufgabeboxframe}{RGB}{150, 75, 0}
\definecolor{infoboxframe}{RGB}{100, 100, 100}
\definecolor{kurzknappframe}{RGB}{180, 180, 0}
\definecolor{fehlerboxframe}{RGB}{180, 0, 0}
\definecolor{warumwichtigframe}{RGB}{0, 100, 150}
\definecolor{tippboxframe}{RGB}{50,150,50}

% Zähler und Nummerierung pro Section
\newcounter{aufgabe}
\newcounter{beispiel}
\newcounter{merksatz}
\newcounter{kurzknapp}
\newcounter{fehlerbox}
\newcounter{warumwichtigbox}
\newcounter{tippbox}

\counterwithin{aufgabe}{section}
\counterwithin{beispiel}{section}
\counterwithin{merksatz}{section}
\counterwithin{kurzknapp}{section}
\counterwithin{fehlerbox}{section}
\counterwithin{warumwichtigbox}{section}
\counterwithin{tippbox}{section}
\counterwithin{figure}{section}
\counterwithin{table}{section}

\renewcommand{\theaufgabe}{\thesection.\arabic{aufgabe}}
\renewcommand{\thebeispiel}{\thesection.\arabic{beispiel}}
\renewcommand{\themerksatz}{\thesection.\arabic{merksatz}}
\renewcommand{\thekurzknapp}{\thesection.\arabic{kurzknapp}}
\renewcommand{\thefehlerbox}{\thesection.\arabic{fehlerbox}}
\renewcommand{\thewarumwichtigbox}{\thesection.\arabic{warumwichtigbox}}
\renewcommand{\thetippbox}{\thesection.\arabic{tippbox}}
\renewcommand{\figurename}{Abb.}
\renewcommand{\tablename}{Tab.}

% Allgemeine Box-Optionen
\tcbset{
  breakable,
  fonttitle=\bfseries,
  colbacktitle=white,
  coltitle=black,
  enhanced,
  attach boxed title to top left={yshift=-2mm, xshift=3mm},
  boxed title style={size=small, colback=white, boxrule=0.5pt},
  underlay boxed title=\tcbset{interior style={fill=white}}
}

% Umgebung für Aufgaben:
\newtcolorbox{aufgabenumgebung}[2][]{%
  colback=aufgabeboxcolorbg!30!white,
  colframe=aufgabeboxframe,
  borderline west={2pt}{0pt}{aufgabeboxframe!50!black, dotted},
  title={Aufgabe \theaufgabe{} #2}, % Explizite Gruppierung des Titels
  phantom=\refstepcounter{aufgabe}\label{#1} % #1 ist das Label
}

% Umgebung für Beispiele:
\newtcolorbox{beispielumgebung}[2][]{%
  colback=beispielboxcolorbg!30!white,
  colframe=beispielboxframe,
  borderline west={2pt}{0pt}{beispielboxframe!50!black, dashed},
  title={Beispiel \thebeispiel{} #2}, % Explizite Gruppierung des Titels
  phantom=\refstepcounter{beispiel}\label{#1} % #1 ist das Label
}

% Umgebung für Merksätze (Boxen):
\newtcolorbox{merksatzumgebung}[2][Merksatz]{%
  colback=merkboxcolorbg!30!white,
  colframe=merkboxframe,
  borderline west={2pt}{0pt}{merkboxframe!70!black, solid},
  title={%
    {\Large\normalfont\lightning}% <-- Hier ist der Unterschied: \normalfont und gruppiert
    \hspace{0.1cm} % Abstand zwischen Symbol und Text
    {\bfseries\itshape\themerksatz{} #2}% <-- Der Textteil
  },
  phantom=\refstepcounter{merksatz}\label{#1}
}

% Umgebung für Infoboxen:
\newtcolorbox{infoboxumgebung}[1]{%
  colback=infoboxcolorbg!30!white,
  colframe=infoboxframe,
  borderline west={2pt}{0pt}{infoboxframe!50!black, double},
  title={#1} % Explizite Gruppierung
  % Keine Nummerierung oder Label für reine Infoboxen vorgesehen
}

% Umgebung für "Kurz & Knapp" Boxen:
\newtcolorbox{kurzknappumgebung}[1]{%
  colback=kurzknappcolorbg!30!white,
  colframe=kurzknappframe,
  fonttitle=\bfseries\itshape,
  title={Kurz \& Knapp \thekurzknapp: #1}, % Explizite Gruppierung
  borderline west={2pt}{0pt}{kurzknappframe!60!black, solid},
  phantom=\refstepcounter{kurzknapp}
}

% Umgebung für "Typische Fehler" Boxen:
\newtcolorbox{fehlerboxumgebung}[1]{%
  colback=fehlerboxcolorbg!30!white,
  colframe=fehlerboxframe,
  fonttitle=\bfseries,
  title={Achtung Stolperstein! \thefehlerbox: #1}, % Explizite Gruppierung
  borderline west={2pt}{0pt}{fehlerboxframe!60!black, solid},
  phantom=\refstepcounter{fehlerbox}
}

% Umgebung für "Warum ist das wichtig?" Boxen:
\newtcolorbox{warumwichtigumgebung}[1]{%
  colback=warumwichtigcolorbg!30!white,
  colframe=warumwichtigframe,
  fonttitle=\bfseries\itshape,
  title={Warum ist das wichtig? \thewarumwichtigbox: #1}, % Explizite Gruppierung
  borderline west={2pt}{0pt}{warumwichtigframe!60!black, solid},
  phantom=\refstepcounter{warumwichtigbox}
}

% Umgebung für "Tipp" Boxen:
\newtcolorbox{tippumgebung}[1]{%
  colback=tippboxcolorbg!30!white,
  colframe=tippboxframe,
  fonttitle=\bfseries\itshape,
  title={Tipp \thetippbox: #1}, % Explizite Gruppierung
  borderline west={2pt}{0pt}{tippboxframe!60!black, solid},
  phantom=\refstepcounter{tippbox}
}


% Zähler für Lösungen
\newcounter{loesung}
\counterwithin{loesung}{section}
\renewcommand{\theloesung}{\thesection.\arabic{loesung}}

% Farben für Lösungsbox
\definecolor{loesungboxcolorbg}{RGB}{230, 245, 255} % Ein helles Blau
\definecolor{loesungboxframe}{RGB}{0, 100, 150} % Ein passendes Dunkelblau

% Umgebung für Lösungen:
\newtcolorbox{loesungsumgebung}[2][]{%
  colback=loesungboxcolorbg!30!white,
  colframe=loesungboxframe,
  borderline west={2pt}{0pt}{loesungboxframe!50!black, solid},
  title={Lösung zu Aufgabe \theloesung{} #2}, % Titel: "Lösung zu Aufgabe X.Y"
  phantom=\refstepcounter{loesung}\label{#1} % #1 ist das Label, falls benötigt
}

% Makro für Äquivalenzumformungen
\newcommand{\umformung}[4]{%
\ensuremath{#1} & \ensuremath{=} & \ensuremath{#2} & \vline & \ensuremath{#3 #4} \\%
}
\newcommand{\umformungend}[2]{%
\ensuremath{#1} & \ensuremath{=} & \ensuremath{#2} & & \\%
}

\begin{document}
\begin{center}
    {\Huge \textbf{Lösungsbuch: Von Dreisatz zu Integralen}}\\
    \vspace{0.5cm}
    {\Large Ausführliche Lösungen für Selbstlerner}\\
    \vspace{1cm}
    % \includegraphics[width=5cm]{placeholder_logo.png} % Platzhalter für Logo, falls gewünscht
    \vspace{1cm}
\end{center}
\setcounter{section}{0}
\tableofcontents
\newpage

% Hier beginnen die Lösungen für die einzelnen Kapitel
% Ersetzen Sie die folgenden Zeilen durch Ihre tatsächlichen Lösungen
% Jede Lösung sollte ausführlich und schrittweise erklärt werden.
% Diesen Block kannst du am Anfang deines Lösungsbuch-Dokumentes einfügen,
% z.B. nach dem Inhaltsverzeichnis oder als erste Seite des Vorworts.

\section{Ein paar Worte vorab: Dein Wegweiser durch die Lösungen}
\label{sec:vorwort_loesungsbuch}

Liebe Schülerin, lieber Schüler,

herzlich willkommen zum Lösungsbuch für dein Lernmaterial 'Von Dreisatz zu Integralen'! Dieses Begleitheft soll dir eine wertvolle Unterstützung auf deiner mathematischen Entdeckungsreise sein. Es ist dafür gedacht, dir Sicherheit bei der Überprüfung deiner eigenen Ergebnisse zu geben, dir alternative Lösungswege aufzuzeigen und dir zu helfen, eventuelle Denkfehler zu erkennen und zu verstehen.

Mathematik lernt man am besten, indem man sie selbst macht – durch eigenes Knobeln, Ausprobieren und auch durch das Überwinden von Schwierigkeiten. Der größte Lernerfolg stellt sich ein, wenn du dich zunächst selbst intensiv und ohne fremde Hilfe mit den Aufgaben auseinandersetzt.  Dieses Lösungsbuch ist daher nicht als schnelle Abkürzung gedacht, sondern als ein Werkzeug, das dich auf deinem Lernweg begleitet und bestärkt.

\subsection*{Wie du dieses Lösungsbuch am besten für dich nutzt}

Damit dieses Lösungsbuch dir den größten Nutzen bringt und deinen Lernfortschritt optimal unterstützt, möchten wir dir ein paar Ratschläge mit auf den Weg geben, ähnlich wie im Vorwort deines Hauptskripts:        

\begin{itemize}
    \item \textbf{Der erste Versuch gehört dir:} Versuche jede Aufgabe immer zuerst vollständig eigenständig zu lösen, bevor du einen Blick in dieses Lösungsbuch wirfst.  Nutze dein Wissen aus dem Hauptskript und deine Notizen. Der eigene Denkprozess ist der wichtigste Schritt beim Lernen!
    \item \textbf{Lösungen gezielt einsetzen:} Wenn du feststeckst oder deine eigene Lösung überprüfen möchtest, nutze die hier angebotenen Lösungswege. Vergleiche sie mit deinem Ansatz. Wo gab es Unterschiede? Was kannst du daraus lernen?
    \item \textbf{Verstehen statt nur Vergleichen:} Lies die Lösungen aktiv durch.  Versuche, jeden Schritt nachzuvollziehen und zu verstehen, warum er so gemacht wurde.         Decke die Lösung vielleicht sogar ab und versuche, sie Schritt für Schritt selbst zu entwickeln, nachdem du dir nur den ersten Denkanstoß geholt hast.        
    \item \textbf{Aus Fehlern lernen:} Wenn deine Lösung von der hier präsentierten abweicht oder fehlerhaft war, sieh das als Chance! Analysiere genau, wo der Fehler lag. Das Verstehen eigener Fehler ist ein unglaublich effektiver Lernprozess.        
    \item \textbf{Lösungen aktiv nacharbeiten:} Wenn du eine Lösung nachschlagen musstest, lege sie nach dem Lesen beiseite und versuche, die Aufgabe später noch einmal komplett selbstständig zu lösen.         Erkläre dir den Lösungsweg mit eigenen Worten, als würdest du ihn einem Freund oder einer Freundin erklären – das ist der berühmte 'Feynman-Effekt' und zeigt, ob du es wirklich verstanden hast!        
    \item \textbf{Abschreiben als letzter Ausweg – aber mit Köpfchen!:} Manchmal ist man vielleicht versucht, eine Lösung einfach abzuschreiben. Auch wenn das den geringsten Lerneffekt hat, ist selbst das bewusste und nachdenkliche Abschreiben einer Lösung, bei dem du versuchst, die Struktur und die einzelnen Schritte zu erfassen, immer noch besser, als sie nur flüchtig zu überfliegen. Konzentriere dich dabei darauf, was genau passiert.
\end{itemize}

Dieses Lösungsbuch möchte dir helfen, die Mathematik hinter den Aufgaben zu durchdringen und deine Fähigkeiten kontinuierlich zu verbessern.       Es ist ein Werkzeug auf deinem Weg – nutze es klug und sei stolz auf jeden eigenen Schritt, den du machst!

Wir wünschen dir viel Erfolg und auch Freude beim Vergleichen, Verstehen und Weiterlernen!
% --- ANNAHME: Alle Pakete und Box-Definitionen aus dem Hauptdokument sind hier gültig ---
% --- Dieser Code-Block ist als Ersatz für das bestehende Kapitel 2 gedacht ---

\section{Der Dreisatz – Dein Werkzeug für Verhältnisse}
\label{sec:dreisatz_ueberarbeitet}


\begin{aufgabenumgebung}[aufg:X.Y]{Bücherkauf} % Passen Sie das Label und die Aufgabennummer an
Wenn 4 gleiche Notizbücher zusammen 10 Euro kosten, wie viel kosten dann 7 dieser Notizbücher?
Zeichne auch ein kleines Schema wie im Beispiel (mit den Pfeilen $:4$ und $\cdot7$).
\end{aufgabenumgebung}

\begin{loesungsumgebung}[loes:2.1]{} % Passen Sie das Label und die Aufgabennummer an, z.B. loes:4.3 für Aufgabe 4.3
Diese Aufgabe lässt sich hervorragend mit dem \textbf{Dreisatz} lösen. Der Dreisatz hilft uns, Verhältnisse zu berechnen, indem wir zunächst den Wert für eine Einheit bestimmen.

1.  \textbf{Schritt 1: Preis für ein Notizbuch berechnen.}
    Wir wissen, dass 4 Notizbücher 10 Euro kosten. Um den Preis für ein Notizbuch zu erfahren, teilen wir den Gesamtpreis durch die Anzahl der Notizbücher:
    $10 \text{ Euro} : 4 \text{ Notizbücher} = 2,50 \text{ Euro pro Notizbuch}$

2.  \textbf{Schritt 2: Preis für 7 Notizbücher berechnen.}
    Da wir nun den Preis für ein Notizbuch kennen (2,50 Euro), können wir diesen mit der gewünschten Anzahl (7 Notizbücher) multiplizieren:
    $2,50 \text{ Euro/Notizbuch} \cdot 7 \text{ Notizbücher} = 17,50 \text{ Euro}$

\textbf{Schema zur Veranschaulichung:}

\begin{center}
\begin{tabular}{c c c}
    4 Notizbücher & $\xrightarrow{\text{ : } 4}$ & 1 Notizbuch \\
    10 Euro & $\xrightarrow{\text{ : } 4}$ & 2,50 Euro \\
    \quad \\ % Kleiner vertikaler Abstand
    1 Notizbuch & $\xrightarrow{\text{ } \cdot 7}$ & 7 Notizbücher \\
    2,50 Euro & $\xrightarrow{\text{ } \cdot 7}$ & 17,50 Euro \\
\end{tabular}
\end{center}

\vspace{0.5em} % Kleiner Abstand nach der Tabelle

\textbf{Antwort:} 7 Notizbücher kosten 17,50 Euro.

\begin{tippumgebung}{Dreisatz-Prinzip verstehen}
Der Dreisatz basiert auf der \textbf{direkten Proportionalität}. Wenn zwei Größen direkt proportional sind, bedeutet das, dass sich die eine Größe im gleichen Verhältnis ändert wie die andere. Im Beispiel: Wenn die Anzahl der Notizbücher steigt, steigt auch der Preis proportional. Dies ist die Grundlage für viele Alltagsberechnungen und ein wichtiges Fundament in der Mathematik.
\end{tippumgebung}

\end{loesungsumgebung}

% Beispiel für die Überschrift, passen Sie diese an Ihre Kapitelstruktur an
\begin{aufgabenumgebung}[aufg:2.2]{Autoverbrauch} % Passen Sie das Label und die Nummer an
Ein Auto verbraucht auf einer Strecke von 150 km genau 12 Liter Benzin.
\begin{enumerate}
    \item Wie viel Benzin verbraucht es auf einer Strecke von 250 km?
    \item Wie weit kommt das Auto mit einem vollen Tank von 40 Litern? (Tipp: Hier ist die Literzahl gegeben und die Kilometer sind gesucht! Du rechnest also erst aus, wie weit es mit 1 Liter kommt.)
\end{enumerate}
\end{aufgabenumgebung}

\begin{loesungsumgebung}[loes:2.2]{} % Passen Sie das Label und die Nummer an, z.B. loes:2.2
Diese Aufgabe besteht aus zwei Teilen, die beide mit dem Dreisatz gelöst werden können.

\textbf{Gegebene Informationen:}
\begin{itemize}
    \item Verbrauch: 12 Liter für 150 km
\end{itemize}

\subsubsection*{Teilaufgabe a) Benzinverbrauch auf 250 km}

Hier wollen wir herausfinden, wie viele Liter für eine gegebene Kilometerzahl benötigt werden.

1.  \textbf{Schritt 1: Verbrauch pro Kilometer berechnen.}
    Um zu wissen, wie viel Benzin pro 1 km verbraucht wird, teilen wir die Liter durch die Kilometer:
    $\frac{12 \text{ Liter}}{150 \text{ km}} = 0,08 \text{ Liter/km}$

2.  \textbf{Schritt 2: Benzin für 250 km berechnen.}
    Jetzt multiplizieren wir den Verbrauch pro Kilometer mit der neuen Strecke:
    $0,08 \text{ Liter/km} \cdot 250 \text{ km} = 20 \text{ Liter}$

\textbf{Schema zur Veranschaulichung (Teil a):}

\begin{center}
\begin{tabular}{c c c}
    150 km & $\xrightarrow{\text{ : } 150}$ & 1 km \\
    12 Liter & $\xrightarrow{\text{ : } 150}$ & 0,08 Liter \\
    \quad \\
    1 km & $\xrightarrow{\text{ } \cdot 250}$ & 250 km \\
    0,08 Liter & $\xrightarrow{\text{ } \cdot 250}$ & 20 Liter \\
\end{tabular}
\end{center}

\textbf{Antwort zu a):} Das Auto verbraucht auf einer Strecke von 250 km genau 20 Liter Benzin.

\subsubsection*{Teilaufgabe b) Gefahrene Strecke mit 40 Litern}

Hier ist die Situation umgekehrt: Wir wissen, wie viele Liter Benzin wir haben, und wollen wissen, wie weit wir damit fahren können.

1.  \textbf{Schritt 1: Reichweite pro Liter berechnen.}
    Um zu wissen, wie viele Kilometer man mit 1 Liter fahren kann, teilen wir die Kilometer durch die Liter:
    $\frac{150 \text{ km}}{12 \text{ Liter}} = 12,5 \text{ km/Liter}$

2.  \textbf{Schritt 2: Reichweite mit 40 Litern berechnen.}
    Nun multiplizieren wir die Reichweite pro Liter mit dem gegebenen Tankvolumen:
    $12,5 \text{ km/Liter} \cdot 40 \text{ Liter} = 500 \text{ km}$

\textbf{Schema zur Veranschaulichung (Teil b):}

\begin{center}
\begin{tabular}{c c c}
    12 Liter & $\xrightarrow{\text{ : } 12}$ & 1 Liter \\
    150 km & $\xrightarrow{\text{ : } 12}$ & 12,5 km \\
    \quad \\
    1 Liter & $\xrightarrow{\text{ } \cdot 40}$ & 40 Liter \\
    12,5 km & $\xrightarrow{\text{ } \cdot 40}$ & 500 km \\
\end{tabular}
\end{center}

\textbf{Antwort zu b):} Das Auto kommt mit einem vollen Tank von 40 Litern genau 500 km weit.

\begin{warumwichtigumgebung}{Anwendung des Dreisatzes in verschiedenen Situationen}
Der Dreisatz ist ein grundlegendes Werkzeug für viele Berechnungen im Alltag und in der Wissenschaft. Es ist wichtig zu erkennen, ob Sie eine direkte oder indirekte Proportionalität vorliegen haben. In diesem Beispiel handelt es sich um \textbf{direkte Proportionalität}: Je mehr Benzin, desto mehr Kilometer (und umgekehrt). Achten Sie immer darauf, die Einheit zu bestimmen, die Sie für den nächsten Rechenschritt benötigen.
\end{warumwichtigumgebung}

\end{loesungsumgebung}

\subsection*{Aufgabe 2.3} % Oder die entsprechende Nummer in Ihrem Kapitel
\begin{aufgabenumgebung}[aufg:2.3]{Dreisatz – Weitere Übungen}
Überlege bei jeder Aufgabe zuerst, ob sie proportional oder antiproportional ist! Schreibe deine Überlegung kurz auf.
\begin{itemize}
    \item 6 gleiche Laborflaschen kosten 15 Euro. Wie viel kosten 10 solcher Flaschen?
    \item 4 Pumpen füllen ein Schwimmbecken in 12 Stunden. Wie lange würden 6 gleiche Pumpen brauchen, wenn sie alle die gleiche Leistung haben?
    \item 2 kg Äpfel kosten 3,80 Euro. Bestimme den Preis für 750 g Äpfel. (Tipp: Wandle Gramm in Kilogramm um: $750\,\text{g} = 0,75\,\text{kg}$, oder rechne alles in Gramm.)
    \item Ein Futtervorrat reicht für 12 Pferde 20 Tage lang. Wie lange reicht der gleiche Vorrat für 15 Pferde (wenn alle Pferde gleich viel fressen)?
\end{itemize}
\end{aufgabenumgebung}

\begin{loesungsumgebung}[loes:2.3]{} % Oder die entsprechende Nummer in Ihrem Kapitel
Diese Aufgabe besteht aus mehreren Dreisatz-Problemen, die sowohl proportionale als auch antiproportionale Zuordnungen umfassen. Es ist wichtig, den Unterschied zu verstehen, um den Dreisatz korrekt anzuwenden.

\subsubsection*{Lösung zu 6 Laborflaschen kosten 15 Euro. Wie viel kosten 10 solcher Flaschen?}

\textbf{Überlegung:} Wenn die Anzahl der Flaschen steigt, steigen auch die Kosten. Dies ist eine \textbf{proportionale Zuordnung}.

1.  \textbf{Schritt 1: Preis pro Flasche berechnen.}
    Wir wissen, dass 6 Flaschen 15 Euro kosten. Um den Preis pro 1 Flasche zu erhalten, teilen wir die Gesamtkosten durch die Anzahl der Flaschen:
    $\frac{15 \text{ Euro}}{6 \text{ Flaschen}} = 2,50 \text{ Euro/Flasche}$

2.  \textbf{Schritt 2: Preis für 10 Flaschen berechnen.}
    Nun multiplizieren wir den Preis pro Flasche mit der gewünschten Anzahl von Flaschen:
    $2,50 \text{ Euro/Flasche} \cdot 10 \text{ Flaschen} = 25 \text{ Euro}$

\textbf{Antwort:} 10 Laborflaschen kosten 25 Euro.

\begin{center}
\begin{tabular}{c c c}
    6 Flaschen & $\xrightarrow{\text{ : } 6}$ & 1 Flasche \\
    15 Euro & $\xrightarrow{\text{ : } 6}$ & 2,50 Euro \\
    \quad \\
    1 Flasche & $\xrightarrow{\text{ } \cdot 10}$ & 10 Flaschen \\
    2,50 Euro & $\xrightarrow{\text{ } \cdot 10}$ & 25 Euro \\
\end{tabular}
\end{center}

\subsubsection*{Lösung zu 4 Pumpen füllen ein Schwimmbecken in 12 Stunden. Wie lange würden 6 gleiche Pumpen brauchen?}

\textbf{Überlegung:} Wenn die Anzahl der Pumpen steigt, sinkt die Zeit, die zum Füllen des Beckens benötigt wird. Dies ist eine \textbf{antiproportionale (umgekehrt proportionale) Zuordnung}. Bei antiproportionalen Zuordnungen ist das Produkt der zusammengehörenden Größen konstant.

1.  \textbf{Schritt 1: Gesamtarbeitseinheiten berechnen (oder Zeit für 1 Pumpe).}
    Die Gesamt'arbeit' entspricht der Anzahl der Pumpen multipliziert mit der Zeit. Diese Menge bleibt konstant.
    4 Pumpen $\cdot$ 12 Stunden = 48 'Pumpenstunden' (Gesamtarbeit)

    Alternativ: Wenn eine Pumpe allein arbeiten würde, würde sie 4 Mal so lange brauchen:
    12 Stunden $\cdot$ 4 = 48 Stunden (Zeit für 1 Pumpe)

2.  \textbf{Schritt 2: Zeit für 6 Pumpen berechnen.}
    Teile die Gesamtarbeit durch die neue Anzahl der Pumpen:
    $\frac{48 \text{ Pumpenstunden}}{6 \text{ Pumpen}} = 8 \text{ Stunden}$

\textbf{Antwort:} 6 gleiche Pumpen würden 8 Stunden brauchen, um das Schwimmbecken zu füllen.

\begin{center}
\begin{tabular}{c c c}
    4 Pumpen & $\xrightarrow{\text{ } \cdot 4}$ & 1 Pumpe \\
    12 Stunden & $\xrightarrow{\text{ } \cdot 4}$ & 48 Stunden \\
    \quad \\
    1 Pumpe & $\xrightarrow{\text{ : } 6}$ & 6 Pumpen \\
    48 Stunden & $\xrightarrow{\text{ : } 6}$ & 8 Stunden \\
\end{tabular}
\end{center}
\begin{merksatzumgebung}{Antiproportionale Zuordnung}
Bei einer \textbf{antiproportionalen} Zuordnung gilt: Wenn die eine Größe auf das Doppelte, Dreifache, etc. steigt, sinkt die andere Größe auf die Hälfte, ein Drittel, etc. Das Produkt der beiden Größen bleibt konstant. Oft wird hier erst der Wert für die 'Einheit 1' durch Multiplikation statt Division bestimmt, und dann durch Division der neuen Menge.
\end{merksatzumgebung}

\subsubsection*{Lösung zu 2 kg Äpfel kosten 3,80 Euro. Bestimme den Preis für 750 g Äpfel.}

\textbf{Überlegung:} Wenn die Menge der Äpfel sinkt, sinkt auch der Preis. Dies ist eine \textbf{proportionale Zuordnung}.
\textbf{Wichtig:} Einheiten angleichen! Wir rechnen alles in Kilogramm. $750\,\text{g} = 0,75\,\text{kg}$.

1.  \textbf{Schritt 1: Preis pro Kilogramm Äpfel berechnen.}
    Wir teilen den Gesamtpreis durch die Kilogramm:
    $\frac{3,80 \text{ Euro}}{2 \text{ kg}} = 1,90 \text{ Euro/kg}$

2.  \textbf{Schritt 2: Preis für 0,75 kg Äpfel berechnen.}
    Nun multiplizieren wir den Preis pro Kilogramm mit der gewünschten Menge:
    $1,90 \text{ Euro/kg} \cdot 0,75 \text{ kg} = 1,425 \text{ Euro}$

    Da es sich um Geld handelt, runden wir auf zwei Dezimalstellen: 1,43 Euro.

\textbf{Antwort:} 750 g Äpfel kosten 1,43 Euro.

\begin{center}
\begin{tabular}{c c c}
    2 kg & $\xrightarrow{\text{ : } 2}$ & 1 kg \\
    3,80 Euro & $\xrightarrow{\text{ : } 2}$ & 1,90 Euro \\
    \quad \\
    1 kg & $\xrightarrow{\text{ } \cdot 0,75}$ & 0,75 kg \\
    1,90 Euro & $\xrightarrow{\text{ } \cdot 0,75}$ & 1,425 Euro $\approx$ 1,43 Euro \\
\end{tabular}
\end{center}
\begin{tippumgebung}{Einheitenumrechnung}
Achten Sie bei Aufgaben, die unterschiedliche Einheiten für die gleiche Größe verwenden (z.B. Gramm und Kilogramm), immer darauf, diese vor der Berechnung anzugleichen. Wählen Sie entweder die kleinere oder die größere Einheit für die gesamte Rechnung, um Fehler zu vermeiden.
\end{tippumgebung}

\subsubsection*{Lösung zu Ein Futtervorrat reicht für 12 Pferde 20 Tage lang. Wie lange reicht der gleiche Vorrat für 15 Pferde?}

\textbf{Überlegung:} Wenn die Anzahl der Pferde steigt, sinkt die Dauer, für die der Futtervorrat reicht. Dies ist eine \textbf{antiproportionale (umgekehrt proportionale) Zuordnung}.

1.  \textbf{Schritt 1: Gesamtfuttereinheiten berechnen (oder Dauer für 1 Pferd).}
    Die Gesamt'futtertage' entsprechen der Anzahl der Pferde multipliziert mit der Dauer. Diese Menge bleibt konstant.
    12 Pferde $\cdot$ 20 Tage = 240 'Pferdetage' (Gesamtfuttermenge)

    Alternativ: Wenn nur 1 Pferd das Futter fressen würde, würde es 12 Mal so lange reichen:
    20 Tage $\cdot$ 12 = 240 Tage (Reichweite für 1 Pferd)

2.  \textbf{Schritt 2: Reichweite für 15 Pferde berechnen.}
    Teile die Gesamtfuttermenge durch die neue Anzahl der Pferde:
    $\frac{240 \text{ Pferdetage}}{15 \text{ Pferde}} = 16 \text{ Tage}$

\textbf{Antwort:} Der gleiche Futtervorrat reicht für 15 Pferde 16 Tage lang.

\begin{center}
\begin{tabular}{c c c}
    12 Pferde & $\xrightarrow{\text{ } \cdot 12}$ & 1 Pferd \\
    20 Tage & $\xrightarrow{\text{ } \cdot 12}$ & 240 Tage \\
    \quad \\
    1 Pferd & $\xrightarrow{\text{ : } 15}$ & 15 Pferde \\
    240 Tage & $\xrightarrow{\text{ : } 15}$ & 16 Tage \\
\end{tabular}
\end{center}
\end{loesungsumgebung}
\section{Lineare Funktionen – Geraden verstehen}

Nachdem wir gesehen haben, dass der Dreisatz eng mit linearen Zusammenhängen verknüpft ist, wollen wir uns nun systematisch mit \textbf{linearen Funktionen} beschäftigen. Sie sind die einfachste Art von Funktionen, aber unglaublich wichtig als Grundlage für komplexere Modelle.


\begin{tcolorbox}[colback=blue!5!white, colframe=blue!75!black, title=Was du in diesem Kapitel lernen wirst:]
Nachdem du dieses Kapitel durchgearbeitet hast, wirst du in der Lage sein:
\begin{itemize}[noitemsep, topsep=0pt]
    \item zu erklären, was eine lineare Funktion ist und wie ihre allgemeine Form lautet.
    \item die Bedeutung der Parameter $a$ (Steigung) und $b$ (y-Achsenabschnitt) zu verstehen und zu interpretieren.
    \item den Graphen einer linearen Funktion zu zeichnen und aus einem Graphen Informationen abzulesen.
    \item die Steigung einer Geraden aus zwei Punkten zu berechnen.
    \item die Funktionsgleichung einer linearen Funktion aus verschiedenen Angaben (z.B. zwei Punkte, Steigung und ein Punkt) aufzustellen.
    \item Nullstellen linearer Funktionen zu berechnen und zu interpretieren.
    \item Wertetabellen zu erstellen und zu nutzen.
    \item Schnittpunkte von zwei Geraden zu berechnen.
    \item Anwendungsaufgaben mit linearen Funktionen zu modellieren und zu lösen.
    \item das Konzept der durchschnittlichen Änderungsrate zu verstehen.
\end{itemize}
\end{tcolorbox}
\bigskip


\begin{infoboxumgebung}{Was bedeutet 'linear'?}
'Linear' kommt vom lateinischen Wort 'linea', was 'Linie' bedeutet. Eine lineare Funktion stellt also immer eine \textbf{gerade Linie} dar, wenn du sie zeichnest. Denk an ein Lineal – das ist ein gutes Bild für etwas Lineares!
Im Alltag begegnen uns lineare Zusammenhänge oft:
\begin{itemize}
    \item Dein Taschengeld pro Woche (wenn es immer gleich viel ist und du bei Null startest oder einen festen Betrag schon hast).
    \item Die Kosten für Benzin, wenn der Preis pro Liter fest ist und du eine bestimmte Menge tankst (ohne Grundgebühr).
    \item Die Strecke, die du mit konstanter Geschwindigkeit zurücklegst, wenn du die Zeit als Variable nimmst.
    \item Viele Gebührenmodelle: eine feste Grundgebühr plus ein Betrag pro Einheit (z.B. pro Minute telefonieren, pro gefahrenem Kilometer).
\end{itemize}
Lineare Funktionen beschreiben also Situationen, in denen die Änderung einer Größe immer gleichmäßig erfolgt.
\end{infoboxumgebung}

Jetzt wird es etwas formaler, aber keine Sorge, wir erklären alles Schritt für Schritt.

\begin{merksatzumgebung}[Lineare Funktion]{Die allgemeine Form}
Eine lineare Funktion hat die allgemeine Form:
\[ f(x) = a \cdot x + b \]
oder auch oft geschrieben als (besonders in der Geometrie):
\[ y = a \cdot x + b \]
Dabei bedeuten die Buchstaben:
\begin{itemize}
    \item $f(x)$ oder $y$: Der \textbf{Funktionswert} (oder y-Wert). Das ist das Ergebnis, das die Funktion liefert, wenn du einen bestimmten Wert für $x$ einsetzt. Im Koordinatensystem ist das die Höhe des Punktes auf der Geraden.
    \item $x$: Die \textbf{Variable} (oder x-Wert, Argument). Für $x$ kannst du verschiedene Zahlen einsetzen. Im Koordinatensystem ist das die Position auf der horizontalen Achse.
    \item $a$: Die \textbf{Steigung} der Geraden. Sie sagt dir, wie steil die Gerade ansteigt oder abfällt, wenn du auf der x-Achse um eine Einheit nach rechts gehst. Eine positive Steigung bedeutet 'bergauf', eine negative Steigung 'bergab'.
    \item $b$: Der \textbf{y-Achsenabschnitt}. Er sagt dir, an welcher Stelle (bei welchem y-Wert) die Gerade die y-Achse schneidet. Das ist immer der Funktionswert an der Stelle $x=0$, denn $f(0) = a \cdot 0 + b = b$. Der Punkt ist also $(0|b)$.
\end{itemize}
Die Zahlen $a$ und $b$ nennt man auch \textbf{Parameter} der Funktion. Sie bestimmen, wie genau die Gerade aussieht und wo sie liegt.
\end{merksatzumgebung}



\begin{warumwichtigumgebung}{Parameter $a$ und $b$}
\begin{itemize}
    \item \textbf{Die Steigung $a$ ist entscheidend!} Sie beschreibt die \textit{Rate der Veränderung}. In Anwendungsaufgaben ist $a$ oft ein Preis pro Stück, eine Geschwindigkeit, ein Verbrauch pro Kilometer etc. Ein tiefes Verständnis der Steigung ist der Schlüssel zu vielen Problemen.
    \item \textbf{Der y-Achsenabschnitt $b$ ist der Startpunkt!} In vielen Anwendungen ist $b$ ein fester Grundbetrag, ein Anfangsbestand oder ein Wert zu Beginn einer Messung ($x=0$).
\end{itemize}
Wenn du $a$ und $b$ in einer Textaufgabe identifizieren kannst, hast du oft schon die halbe Miete!
\end{warumwichtigumgebung}

Eine lineare Funktion beschreibt also einen eindeutigen Zusammenhang: Jedem $x$-Wert wird genau ein $y$-Wert zugeordnet, und diese Wertepaare $(x|y)$ liegen alle auf einer Geraden.

% Dein bisheriger Code bis zum Ende der merksatzumgebung


\bigskip % Ein kleiner vertikaler Abstand für bessere Lesbarkeit

Bevor wir uns nun genauer anschauen, wie sich die Parameter $a$ (Steigung) und $b$ (y-Achsenabschnitt) auf den Graphen einer linearen Funktion auswirken und wie wir mit ihnen rechnen, wollen wir noch einmal einen Schritt zurücktreten. Was genau verstehen wir eigentlich unter einer \textbf{Funktion}? Du hast diesen Begriff sicher schon oft gehört, aber eine klare und vielseitige Vorstellung davon ist Gold wert, nicht nur für lineare Funktionen, sondern für alles, was in der Mathematik noch kommt!

\begin{infoboxumgebung}{Was ist eigentlich eine Funktion? – Eine vielseitige Zuordnung}
Ganz grundlegend ist eine Funktion eine Art \textbf{Zuordnungsvorschrift}. Sie nimmt sich etwas aus einer Menge (der sogenannten \textbf{Definitionsmenge}, z.B. Zahlen, die du einsetzen darfst) und ordnet diesem \textbf{eindeutig} etwas aus einer anderen Menge zu (der sogenannten \textbf{Wertemenge}, z.B. die Ergebnis-Zahlen). Stell es dir so vor: Für jede erlaubte Eingabe gibt es genau eine festgelegte Ausgabe.

Hier sind verschiedene Bilder und Ideen, die dir helfen, das Konzept „Funktion“ besser zu greifen:

\paragraph{0. Die Rezept-Analogie: Portionen und Mehl}
Stell dir vor, du möchtest Pfannkuchen backen. Dein Rezept sagt: Für eine Portion Pfannkuchen benötigst du 100 Gramm Mehl.
\begin{itemize}
    \item Für \textbf{1 Portion} ($x=1$) brauchst du $1 \cdot 100\,\text{g} = \mathbf{100\,\textbf{g}}$ Mehl.
    \item Für \textbf{3 Portionen} ($x=3$) brauchst du $3 \cdot 100\,\text{g} = \mathbf{300\,\textbf{g}}$ Mehl.
    \item Für eine \textbf{halbe Portion} ($x=0,5$) brauchst du $0,5 \cdot 100\,\text{g} = \mathbf{50\,\textbf{g}}$ Mehl.
\end{itemize}
Die Anzahl der Portionen (deine Eingabe $x$) bestimmt eindeutig, wie viel Mehl (deine Ausgabe $f(x)$) du benötigst. Als Funktionsgleichung könnten wir schreiben: $f(x) = 100 \cdot x$. Die Funktion $f$ ordnet also der Anzahl der Portionen $x$ die benötigte Mehlmenge $100x$ zu.

\paragraph{1. Die Funktionsmaschine}
Eine sehr beliebte und hilfreiche Vorstellung ist die einer \textbf{Maschine}:
\begin{itemize}
    \item \textbf{Eingabe (Input):} Du wirfst eine Zahl (unser $x$) oben in die Maschine hinein.
    \item \textbf{Verarbeitung:} Im Inneren der Maschine wird mit dieser Zahl nach einer festen Regel etwas Bestimmtes gemacht – sie wird zum Beispiel verdoppelt ($f(x)=2x$), es wird eine Zahl addiert ($f(x)=x+5$), oder es werden mehrere Rechenschritte kombiniert (z.B. $f(x) = 2 \cdot (x+2) + 3$). Diese Regel ist die \textbf{Funktionsvorschrift}.
    \item \textbf{Ausgabe (Output):} Unten kommt genau eine neue Zahl (unser Funktionswert $f(x)$ oder $y$) heraus.
\end{itemize}
Für jede erlaubte Eingabe $x$ liefert die Maschine also eine eindeutig bestimmte Ausgabe $f(x)$.
% Hier könntest du später eine kleine Skizze einfügen, z.B.:
% \begin{center} \includegraphics[width=0.4\textwidth]{pfad/zu/deiner/skizze.png} \end{center}
% (Eine einfache Box mit 'x rein', 'f(x) raus' und 'Rechenregel' in der Mitte)

\paragraph{2. Der (ideale) Getränkeautomat}
Ein bisschen wie ein Getränkeautomat (der immer funktioniert und nie leer ist und nichts kostet \smiley{}):
\begin{itemize}
    \item \textbf{Eingabe:} Du drückst eine Taste (z.B. Taste Nr. 1 für 'Apfelschorle'). Die Tastenwahl ist dein $x$.
    \item \textbf{Ausgabe:} Du erhältst ein bestimmtes Getränk (z.B. eine Apfelschorle). Das Getränk ist dein $f(x)$.
\end{itemize}
Wichtig ist hier: \textbf{Jede Taste ist genau einem Getränk zugeordnet.} Es kann nicht sein, dass du Taste 1 drückst und mal eine Cola, mal eine Limo bekommst. Das ist das Prinzip der \textbf{Eindeutigkeit} einer Funktion! (Es kann aber sein, dass verschiedene Tasten zum selben Getränk führen – z.B. Taste 1 und Taste 5 geben beide Apfelschorle. Das ist bei Funktionen auch erlaubt: Unterschiedliche $x$-Werte können denselben $f(x)$-Wert haben. Aber ein $x$-Wert darf nicht mehrere verschiedene $f(x)$-Werte haben!)

\paragraph{3. Die mathematische Sicht: Eine Abbildung}
In der Mathematik nennt man eine Funktion oft auch eine \textbf{Abbildung}. Sie bildet Elemente einer Menge auf Elemente einer anderen Menge ab.
Wenn wir zum Beispiel sagen, dass wir für $x$ beliebige reelle Zahlen (die Zahlen, die du vom Zahlenstrahl kennst, inklusive Brüchen, Wurzeln etc., geschrieben als $\mathbb{R}$) einsetzen dürfen und als Ergebnis auch wieder reelle Zahlen erhalten, schreiben Mathematiker das manchmal so:
\[ f: \mathbb{R} \to \mathbb{R} \]
Das liest sich als: 'Die Funktion $f$ bildet von den reellen Zahlen in die reellen Zahlen ab.'
Die genaue Vorschrift, was mit einem $x$ passiert, schreibt man dann oft mit einem Pfeil mit einem senkrechten Strich am Anfang:
\[ x \mapsto f(x) \]
Für eine lineare Funktion wie $f(x) = 3x + 1$ würde das bedeuten:
\[ x \mapsto 3x + 1 \]
Das ist einfach eine kompakte Art zu sagen: 'Nimm ein beliebiges $x$, multipliziere es mit 3 und addiere dann 1.'

Das Wichtigste ist: Eine Funktion ist eine \textbf{eindeutige Zuordnung}. Zu jeder Eingabe (aus dem erlaubten Bereich) gibt es genau eine Ausgabe.
\end{infoboxumgebung}

Jetzt, wo wir eine bessere Vorstellung davon haben, was eine Funktion allgemein ist, können wir uns ein paar typische Aufgabenstellungen rund um diese Funktionsmaschinen ansehen.

\begin{aufgabenumgebung}{Funktionsmaschinen-Logik}{}
Stell dir die folgenden Funktionsmaschinen vor. Gib jeweils die Funktionsgleichung $f(x) = \dots$ an, die beschreibt, was die Maschine mit der Eingabe $x$ macht.
\begin{enumerate}
    \item Maschine M1: Verdoppelt die Eingabe $x$ und addiert anschließend 5.
    \item Maschine M2: Subtrahiert von der Eingabe $x$ die Zahl 7 und multipliziert das Ergebnis mit 3.
    \item Maschine M3: Multipliziert die Eingabe $x$ mit sich selbst (quadriert sie) und zieht dann 4 ab.
    \item Maschine M4: Addiert zur Eingabe $x$ die Zahl 2, dividiert das Ergebnis durch 4 und addiert dann $x$.
\end{enumerate}
\end{aufgabenumgebung}

\begin{aufgabenumgebung}{Werte aus der Maschine}{}
Gegeben sind die folgenden Funktionsmaschinen durch ihre Funktionsgleichungen. Welche Ausgabe $f(x)$ (oder $y$) erzeugt die Maschine, wenn die angegebene Zahl $x$ eingegeben wird?
\begin{enumerate}
    \item $f(x) = 4x - 7$
    \begin{itemize}
        \item Was kommt raus bei $x=3$?
        \item Was kommt raus bei $x=0$?
        \item Was kommt raus bei $x=-2$?
    \end{itemize}
    \item $g(x) = -2(x+3)$
    \begin{itemize}
        \item Was kommt raus bei $x=1$?
        \item Was kommt raus bei $x=-3$?
        \item Was kommt raus bei $x=-5$?
    \end{itemize}
\end{enumerate}
\end{aufgabenumgebung}

\begin{erinnerungsboxumgebung}{Grundrechenarten, Vorzeichen und Brüche}
Für lineare Funktionen und das Umstellen von Gleichungen brauchen wir ein paar grundlegende Rechenregeln immer wieder. Das gilt besonders für Steigungen, wo oft Brüche auftauchen. Lass uns kurz checken, ob alles sitzt!

\paragraph{1. Multiplikation und Division mit Vorzeichen}
Beim Multiplizieren und Dividieren von Zahlen mit unterschiedlichen Vorzeichen gilt:
\begin{itemize}
    \item \textbf{Gleiche Vorzeichen:} Das Ergebnis ist \textbf{positiv (+)}.
    \begin{itemize}
        \item $(+) \cdot (+) = (+)$, z.B. $3 \cdot 4 = 12$
        \item $(-) \cdot (-) = (+)$, z.B. $(-3) \cdot (-4) = 12$
        \item $(+) : (+) = (+)$, z.B. $10 : 2 = 5$
        \item $(-) : (-) = (+)$, z.B. $(-10) : (-2) = 5$
    \end{itemize}
    \item \textbf{Unterschiedliche Vorzeichen:} Das Ergebnis ist \textbf{negativ (-)}.
    \begin{itemize}
        \item $(+) \cdot (-) = (-)$, z.B. $3 \cdot (-4) = -12$
        \item $(-) \cdot (+) = (-)$, z.B. $(-3) \cdot 4 = -12$
        \item $(+) : (-) = (-)$, z.B. $10 : (-2) = -5$
        \item $(-) : (+) = (-)$, z.B. $(-10) : 2 = -5$
    \end{itemize}
\end{itemize}

\paragraph{2. Minus vor der Klammer / Subtraktion einer negativen Zahl}
Ein Minuszeichen vor einer Zahl oder Klammer kehrt das Vorzeichen um. Besonders wichtig:
\begin{itemize}
    \item $a - (-b) = a + b$
    \item Beispiel: $7 - (-5) = 7 + 5 = 12$
    \item Beispiel: $-2 - (-8) = -2 + 8 = 6$
\end{itemize}

\paragraph{3. Punkt- vor Strichrechnung}
Denk immer dran: Multiplikation ($\cdot$) und Division (:) werden \textbf{vor} Addition (+) und Subtraktion (-) ausgeführt.
\begin{itemize}
    \item Regel: Zuerst Klammern, dann Punktrechnung, dann Strichrechnung.
    \item Beispiel 1: $5 + 2 \cdot 3 = 5 + 6 = 11$
    \item Beispiel 2: $(5+2) \cdot 3 = 7 \cdot 3 = 21$
    \item Beispiel 3: $10 - 8 : 2 = 10 - 4 = 6$
\end{itemize}

\paragraph{4. Rechnen mit Brüchen}
Brüche begegnen uns oft, z.B. bei Steigungsdreiecken.
\subparagraph{4.1 Kürzen und Erweitern}
\begin{itemize}
    \item \textbf{Kürzen:} Zähler und Nenner durch dieselbe Zahl (außer 0) teilen. Der Wert des Bruchs ändert sich nicht.
    \textit{Formel:} $\frac{a \cdot k}{b \cdot k} = \frac{a}{b}$
    Beispiel: $\frac{6}{9} = \frac{6:3}{9:3} = \frac{2}{3}$
    \item \textbf{Erweitern:} Zähler und Nenner mit derselben Zahl (außer 0) multiplizieren. Der Wert des Bruchs ändert sich nicht.
    \textit{Formel:} $\frac{a}{b} = \frac{a \cdot k}{b \cdot k}$
    Beispiel: $\frac{2}{3} = \frac{2 \cdot 4}{3 \cdot 4} = \frac{8}{12}$
\end{itemize}

\subparagraph{4.2 Addition und Subtraktion von Brüchen}
\begin{itemize}
    \item \textbf{Gleicher Nenner:} Zähler addieren/subtrahieren, Nenner beibehalten.
    \textit{Formeln:} $\frac{a}{c} + \frac{b}{c} = \frac{a+b}{c}$ \quad und \quad $\frac{a}{c} - \frac{b}{c} = \frac{a-b}{c}$
    Beispiel: $\frac{1}{7} + \frac{3}{7} = \frac{1+3}{7} = \frac{4}{7}$
    \item \textbf{Ungleiche Nenner:} Erst auf einen \textbf{gemeinsamen Nenner} (Hauptnenner) erweitern, dann addieren/subtrahieren.
    Beispiel: $\frac{1}{2} - \frac{1}{3} = \frac{1 \cdot 3}{2 \cdot 3} - \frac{1 \cdot 2}{3 \cdot 2} = \frac{3}{6} - \frac{2}{6} = \frac{1}{6}$
\end{itemize}

\subparagraph{4.3 Multiplikation von Brüchen}
\begin{itemize}
    \item Regel: Zähler mal Zähler und Nenner mal Nenner.
    \textit{Formel:} $\frac{a}{b} \cdot \frac{c}{d} = \frac{a \cdot c}{b \cdot d}$
    Beispiel: $\frac{2}{5} \cdot \frac{3}{4} = \frac{2 \cdot 3}{5 \cdot 4} = \frac{6}{20} = \frac{3}{10}$ (Kürzen nicht vergessen!)
\end{itemize}

\subparagraph{4.4 Division von Brüchen}
\begin{itemize}
    \item Regel: Mit dem \textbf{Kehrwert (Reziprokwert)} des zweiten Bruchs multiplizieren.
    \textit{Formel:} $\frac{a}{b} : \frac{c}{d} = \frac{a}{b} \cdot \frac{d}{c} = \frac{a \cdot d}{b \cdot c}$
    Beispiel: $\frac{2}{3} : \frac{5}{7} = \frac{2}{3} \cdot \frac{7}{5} = \frac{2 \cdot 7}{3 \cdot 5} = \frac{14}{15}$
    \item Einen \textbf{Doppelbruch} löst du ebenfalls so auf: $\frac{\frac{a}{b}}{\frac{c}{d}} = \frac{a}{b} : \frac{c}{d} = \frac{a}{b} \cdot \frac{d}{c}$
    Beispiel: $\frac{\frac{1}{2}}{\frac{3}{4}} = \frac{1}{2} \cdot \frac{4}{3} = \frac{4}{6} = \frac{2}{3}$
\end{itemize}


\textbf{Kurze Übungen dazu:}
Berechne die folgenden Terme.
\begin{multicols}{3}
\begin{enumerate}[label=(\alph*)]
    \item $(-5) \cdot 6 = ?$
    \item $7 \cdot (-3) = ?$
    \item $(-8) \cdot (-4) = ?$
    \item $12 : (-3) = ?$
    \item $(-21) : 7 = ?$
    \item $(-30) : (-5) = ?$
    \item $9 - (-4) = ?$
    \item $-6 - (-10) = ?$
    \item $3 - (-3) + (-3) = ?$
    \item $4 + 6 \cdot (-2) = ?$
    \item $15 : 3 - 7 = ?$
    \item $(-2) \cdot (5 - 1) = ?$
    \item $20 - (-5) \cdot 2 = ?$
    \item $18 : (-3) - (-4) = ?$
    \item $\frac{3}{8} + \frac{2}{8} = ?$
    \item $\frac{5}{6} - \frac{1}{3} = ?$
    \item $\frac{2}{5} \cdot \frac{3}{4} = ?$
    \item $\frac{3}{7} : \frac{2}{5} = ?$
    \item $3 \cdot \frac{2}{11} = ?$
    \item $(-\frac{1}{3}) \cdot \frac{2}{5} = ?$
    \item $\frac{4}{5} + (-\frac{1}{2}) = ?$
    \item $(-\frac{2}{3}) : (-\frac{4}{9}) = ?$
    \item $\frac{\frac{2}{5}}{\frac{3}{10}} = ?$
    \item $1 - \frac{3}{5} \cdot \frac{5}{9} = ?$
\end{enumerate}
\end{multicols}

Wenn du bei diesen Aufgaben sicher bist, fällt dir der Umgang mit Termen und Gleichungen bei linearen Funktionen gleich viel leichter!
\end{erinnerungsboxumgebung}


\subsection{Grafische Darstellung – Wie sieht eine lineare Funktion aus?}

Wie schon gesagt: Der Graph (die zeichnerische Darstellung) einer linearen Funktion ist immer eine Gerade. Die Parameter $a$ und $b$ bestimmen ihr Aussehen:

\begin{itemize}
    \item \textbf{Die Steigung $a$ bestimmt die Richtung und Neigung:}
        \begin{itemize}
            \item Ist $a > 0$ (positiv), \textbf{steigt} die Gerade an (wenn man von links nach rechts schaut). Je größer $a$, desto steiler steigt sie.
            \item Ist $a < 0$ (negativ), \textbf{fällt} die Gerade ab. Je kleiner $a$ (d.h. je größer der Betrag von $a$), desto steiler fällt sie.
            \item Ist $a = 0$, ist die Gerade \textbf{waagerecht} (parallel zur x-Achse). Die Funktion ist dann $f(x)=b$, also eine konstante Funktion. Jeder x-Wert hat denselben y-Wert $b$.
        \end{itemize}
    \item \textbf{Der y-Achsenabschnitt $b$ bestimmt die Lage:}
        Er gibt an, wo die Gerade die y-Achse schneidet. Eine Änderung von $b$ verschiebt die Gerade parallel nach oben oder unten.
\end{itemize}

Hier sind einige Beispiele grafisch dargestellt:
\begin{center} % Zentriert den Inhalt
    \includegraphics[scale=0.5]{grafiken/Lineare_Funktionen_Beispiele.png}
    \captionof{figure}{Beispiele für lineare Funktionen mit unterschiedlichen Steigungen und y-Achsenabschnitten}
    \label{fig:lineare_funktionen_beispiele}
\end{center}
% Der Text geht hier direkt weiter


Es ist sehr hilfreich, wenn du dir vorstellen kannst, wie sich die Gerade verändert, wenn du $a$ oder $b$ variierst.

\begin{infoboxumgebung}{Trick zum Zeichnen einer Geraden (mit $a$ und $b$):}
Um eine Gerade $f(x)=ax+b$ schnell zu zeichnen, brauchst du nur zwei Punkte. So findest du sie leicht:
\begin{enumerate}
    \item \textbf{Der y-Achsenabschnitt $b$:} Markiere den Punkt $P_1(0|b)$ auf der y-Achse. Das ist dein erster Punkt, denn für $x=0$ ist $f(0)=b$.
    \item \textbf{Das Steigungsdreieck für $a$:} Die Steigung $a$ kannst du oft als Bruch schreiben, z.B. $a = \frac{\Delta y}{\Delta x}$ (Veränderung in y-Richtung geteilt durch Veränderung in x-Richtung).
        \begin{itemize}
            \item Gehe vom Punkt $P_1(0|b)$ aus $\Delta x$ Einheiten nach \textbf{rechts} (entlang der x-Achse).
            \item Von dort gehe $\Delta y$ Einheiten nach \textbf{oben} (wenn $a$ positiv ist, also $\Delta y > 0$) oder $\Delta y$ Einheiten nach \textbf{unten} (wenn $a$ negativ ist, also $\Delta y < 0$).
            \item Der Punkt, den du so erreichst, ist dein zweiter Punkt $P_2$.
        \end{itemize}
        Beispiele für das Steigungsdreieck:
        \begin{itemize}
            \item $a = 2 = \frac{2}{1}$: Gehe 1 nach rechts, 2 nach oben.
            \item $a = -0.5 = -\frac{1}{2} = \frac{-1}{2}$: Gehe 2 nach rechts, 1 nach unten.
            \item $a = \frac{3}{4}$: Gehe 4 nach rechts, 3 nach oben.
        \end{itemize}
        Wenn $a$ eine ganze Zahl ist (z.B. $a=3$), kannst du sie als Bruch $a = \frac{a}{1}$ schreiben (also 1 nach rechts, $a$ nach oben/unten).
    \item \textbf{Verbinden:} Zeichne eine Gerade durch die beiden Punkte $P_1$ und $P_2$. Fertig!
\end{enumerate}
Dieser 'Zeichentrick' ist sehr nützlich!
\end{infoboxumgebung}

Die Steigung ist ein zentrales Konzept, nicht nur bei linearen Funktionen. Schauen wir sie uns genauer an.

\subsection{Die Steigung $a$ – Wie steil ist es?}

Die Steigung $a$ gibt an, um wie viele Einheiten sich der y-Wert ändert, wenn der x-Wert um eine Einheit zunimmt.

\begin{merksatzumgebung}[Steigung berechnen]{Die Steigung $a$ aus zwei Punkten}
Wenn eine Gerade durch zwei gegebene Punkte $P_1(x_1|y_1)$ und $P_2(x_2|y_2)$ verläuft, kannst du ihre Steigung $a$ wie folgt berechnen:
\[ a = \frac{\text{Unterschied der y-Werte}}{\text{Unterschied der x-Werte}} = \frac{y_2 - y_1}{x_2 - x_1} = \frac{\Delta y}{\Delta x} \]
Das Symbol $\Delta$ (Delta, ein griechischer Buchstabe, oft als Dreieck geschrieben) steht in der Mathematik häufig für eine 'Differenz' oder einen 'Unterschied'.
Die Formel $a = \frac{\Delta y}{\Delta x}$ ist fundamental. Man kann sie sich auch als 'Höhenunterschied geteilt durch Längenunterschied' vorstellen, wenn man ein \textbf{Steigungsdreieck} zwischen den beiden Punkten zeichnet.
\end{merksatzumgebung}

Ein Steigungsdreieck ist ein rechtwinkliges Dreieck, das du dir unter (oder über) der Geraden zwischen zwei Punkten vorstellen kannst. Die horizontale Kathete hat die Länge $\Delta x = |x_2-x_1|$ und die vertikale Kathete die Länge $\Delta y = |y_2-y_1|$. Das Vorzeichen von $a$ ergibt sich dann aus der Richtung.

\begin{beispielumgebung}[Steigungsdreieck]{Steigung aus zwei Punkten berechnen und visualisieren}
Gegeben sind die Punkte $P_1(1|2)$ und $P_2(3|6)$. Wie groß ist die Steigung der Geraden durch diese Punkte?

Wir identifizieren die Koordinaten:
$x_1 = 1, y_1 = 2$
$x_2 = 3, y_2 = 6$

Nun berechnen wir die Differenzen:
Unterschied der y-Werte: $\Delta y = y_2 - y_1 = 6 - 2 = 4$.
Unterschied der x-Werte: $\Delta x = x_2 - x_1 = 3 - 1 = 2$.

Jetzt setzen wir in die Formel für die Steigung ein:
\[ a = \frac{\Delta y}{\Delta x} = \frac{4}{2} = 2 \]
Die Steigung der Geraden ist $a=2$. Das bedeutet: Wenn du auf der Geraden 1 Einheit nach rechts gehst, gehst du 2 Einheiten nach oben.

\begin{center}
    \includegraphics[width=0.8\textwidth]{grafiken/Lineare_Funktion_Steigungsdreieck.png}
    \captionof{figure}{Steigungsdreieck zur Berechnung von $a$}
    \label{fig:steigungsdreieck_beispiel}
\end{center}
% Der Text geht hier direkt weiter
Es ist egal, welchen Punkt du als $P_1$ und welchen als $P_2$ wählst, das Ergebnis für $a$ ist dasselbe:
$a = \frac{y_1 - y_2}{x_1 - x_2} = \frac{2 - 6}{1 - 3} = \frac{-4}{-2} = 2$.
\end{beispielumgebung}

\textit{Selbst-Check:} Was wäre die Steigung, wenn $P_2(3|2)$ wäre? Wie würde die Gerade dann aussehen? (Antwort: $a = (2-2)/(3-1) = 0/2 = 0$. Die Gerade wäre waagerecht.)

Üben wir das gleich.

\begin{aufgabenumgebung}{Steigung zwischen zwei Punkten}
Berechne die Steigung der Geraden, die durch die folgenden Punktepaare verläuft. Versuche auch, dir vorzustellen oder zu skizzieren, wie die Gerade ungefähr aussieht (steigend/fallend, steil/flach).
\begin{enumerate}
    \item $A(-1|1)$ und $B(2|7)$
    \item $C(0|4)$ und $D(3|1)$
    \item $E(-2|-3)$ und $F(4|-3)$ (Was ist hier besonders?)
    \item $G(2|1)$ und $H(2|5)$ (Was ist hier besonders? Ist das noch eine Funktion $y=f(x)$? Begründe!)
\end{enumerate}
\end{aufgabenumgebung}


\begin{infoboxumgebung}{Sonderfall: Senkrechte Geraden}
Du hast vielleicht bemerkt, dass bei der Berechnung der Steigung $a = \frac{y_2-y_1}{x_2-x_1}$ der Nenner $x_2-x_1$ nicht Null sein darf. Was passiert, wenn $x_1=x_2$ ist, wie z.B. bei den Punkten $G(2|1)$ und $H(2|5)$?
Dann liegen die Punkte übereinander und bilden eine \textbf{senkrechte Gerade} (parallel zur y-Achse). Die Gleichung einer solchen Geraden ist z.B. $x=2$.
Für senkrechte Geraden ist die Steigung \textbf{nicht definiert} (man würde durch Null teilen). Wichtig ist auch: Eine senkrechte Gerade ist \textbf{keine Funktion} im Sinne von $y=f(x)$, da einem $x$-Wert (hier $x=2$) unendlich viele $y$-Werte zugeordnet werden. Das widerspricht der Eindeutigkeit einer Funktion.
Merke dir also: Lineare Funktionen $f(x)=ax+b$ können steigend, fallend oder waagerecht sein, aber niemals senkrecht.
\end{infoboxumgebung}

Mit der Steigung $a$ und dem y-Achsenabschnitt $b$ ist eine lineare Funktion vollständig bestimmt. Wenn wir diese beiden Werte kennen, können wir die Funktionsgleichung aufschreiben.

\subsection{Funktionsgleichung aufstellen}

Es gibt verschiedene Wege, die Gleichung einer linearen Funktion $f(x)=ax+b$ zu bestimmen, je nachdem, welche Informationen gegeben sind.

\subsubsection{Aus zwei Punkten eine Gerade machen}

Oft hat man nicht direkt $a$ und $b$ gegeben, sondern zum Beispiel zwei Punkte, durch die die Gerade verlaufen soll. Daraus lässt sich die Funktionsgleichung $f(x) = ax + b$ eindeutig bestimmen.

\begin{merksatzumgebung}{Geradengleichung aus 2 Punkten bestimmen}
So gehst du vor, um die Gleichung einer Geraden durch die Punkte $P_1(x_1|y_1)$ und $P_2(x_2|y_2)$ zu finden:
\begin{enumerate}
    \item \textbf{Steigung $a$ berechnen:}
    Benutze die dir bekannte Formel:
    \[ a = \frac{y_2 - y_1}{x_2 - x_1} \]
    (Voraussetzung: $x_1 \neq x_2$, sonst wäre es eine senkrechte Linie und keine Funktion der Form $f(x)=ax+b$).

    \item \textbf{y-Achsenabschnitt $b$ berechnen:}
    Setze die berechnete Steigung $a$ und die Koordinaten \textbf{eines der beiden Punkte} (egal welchen, nimm den, der dir sympathischer ist oder einfacher aussieht) in die allgemeine Geradengleichung $y = ax + b$ ein. Du erhältst dann eine Gleichung, in der nur noch $b$ die Unbekannte ist. Löse diese Gleichung nach $b$ auf.
    Also, wenn du $P_1(x_1|y_1)$ nimmst: $y_1 = a \cdot x_1 + b \implies b = y_1 - a \cdot x_1$.

    \item \textbf{Funktionsgleichung hinschreiben:}
    Setze die berechneten Werte für $a$ und $b$ in die allgemeine Form $f(x) = ax + b$ ein. Fertig!
\end{enumerate}
\end{merksatzumgebung}

Das klingt vielleicht kompliziert, ist aber mit etwas Übung ein Standardverfahren.

\begin{infoboxumgebung}{Gleichungen umformen – So machen wir das (Äquivalenzumformungen)}
Wenn wir Gleichungen lösen, wenden wir \textbf{Äquivalenzumformungen} an. Das bedeutet, wir verändern beide Seiten der Gleichung auf die gleiche Weise, sodass die Lösung der Gleichung erhalten bleibt. Wir schreiben das so auf, dass jeder Schritt klar nachvollziehbar ist. Hinter einen senkrechten Strich schreiben wir, welche Operation wir auf beiden Seiten durchführen.

Beispiel: Löse $3x - 5 = 7$ nach $x$ auf.
\begin{center}
\begin{tabular}{r @{\,} c @{\,} l @{\quad\quad} l} % rcll für Ausrichtung
$3x - 5$ & $=$ & $7$ & $| +5$ \\
$3x$ & $=$ & $12$ & $| :3$ \\
$x$ & $=$ & $4$ & \\
\end{tabular}
\end{center}
Achte darauf, dass die Gleichheitszeichen schön untereinander stehen. Das hilft, den Überblick zu behalten.
\end{infoboxumgebung}

\begin{beispielumgebung}[Funktionsgleichung durch zwei Punkte]{Gerade durch $P(1|3)$ und $Q(4|9)$}
Gegeben sind die Punkte $P(1|3)$ und $Q(4|9)$.
Also: $x_P = 1, y_P = 3$ und $x_Q = 4, y_Q = 9$.

\textbf{Schritt 1: Steigung $a$ berechnen}
\[ a = \frac{y_Q - y_P}{x_Q - x_P} = \frac{9 - 3}{4 - 1} = \frac{6}{3} = 2 \]
Die Steigung ist also $a=2$.

\textbf{Schritt 2: y-Achsenabschnitt $b$ berechnen}
Wir haben $a=2$. Nun setzen wir die Koordinaten von Punkt $P(1|3)$ in die Geradengleichung $y = ax + b$ ein:
$y_P = a \cdot x_P + b$
\begin{center}
\begin{tabular}{r @{\,} c @{\,} l @{\quad\quad} l}
$3$ & $=$ & $2 \cdot 1 + b$ & \\
$3$ & $=$ & $2 + b$ & $| -2$ \\
$3-2$ & $=$ & $b$ & \\
$1$ & $=$ & $b$ & \\
\end{tabular}
\end{center}
Der y-Achsenabschnitt ist $b=1$.

(Zur Probe könntest du auch Punkt $Q(4|9)$ und $a=2$ einsetzen:
$y_Q = a \cdot x_Q + b$
\begin{center}
\begin{tabular}{r @{\,} c @{\,} l @{\quad\quad} l}
$9$ & $=$ & $2 \cdot 4 + b$ & \\
$9$ & $=$ & $8 + b$ & $| -8$ \\
$1$ & $=$ & $b$ & \\
\end{tabular}
\end{center}
Es kommt dasselbe Ergebnis für $b$ heraus, was gut ist!)

\textbf{Schritt 3: Funktionsgleichung angeben}
Mit der berechneten Steigung $a=2$ und dem y-Achsenabschnitt $b=1$ lautet die Funktionsgleichung der Geraden:
\[ f(x) = 2x + 1 \]

\begin{center}
    \includegraphics[width=0.8\textwidth]{grafiken/Lineare_Funktion_aus_2_Punkten.png}
    \captionof{figure}{Die Gerade $f(x)=2x+1$ durch die Punkte P und Q}
    \label{fig:gerade_aus_2_punkten}
\end{center}
% Der Text geht hier direkt weiter

\end{beispielumgebung}

Jetzt eine Aufgabe für dich, um das Verfahren zu üben.

\begin{aufgabenumgebung}{Lineare Funktionen umfassend bestimmen und analysieren}

\textbf{Teil 1: Gerade mit positiver Steigung} \\
Gegeben sind die Punkte $C(-2|0)$ und $D(2|8)$, durch die eine lineare Funktion $f(x) = ax+b$ verläuft.

\begin{enumerate}[label=(\alph*)]
    \item \textbf{Steigung berechnen:} Berechne die Steigung $a$ der Geraden durch die Punkte C und D.
    \item \textbf{Y-Achsenabschnitt berechnen:} Bestimme den y-Achsenabschnitt $b$ der Geraden.
    \item \textbf{Funktionsgleichung aufstellen:} Gib die vollständige Funktionsgleichung $f(x)$ an.
    \item \textbf{Nullstelle bestimmen:} Berechne die Nullstelle $x_N$ der Funktion $f(x)$. Welcher der gegebenen Punkte entspricht der Nullstelle?
    \item \textbf{Funktionswerte berechnen:}
        \begin{itemize}
            \item Welchen Wert hat $f(0)$? Was sagt dieser Wert über den Graphen aus?
            \item Berechne $f(1)$.
        \end{itemize}
    \item \textbf{Argument für einen Funktionswert finden:} Für welchen Wert von $x$ gilt $f(x) = 6$?
    \item \textbf{Punktprobe:} Liegt der Punkt $P(3|10)$ auf der Geraden? Begründe deine Antwort rechnerisch.
    \item \textbf{Skizze und Überprüfung:} Zeichne den Graphen der Funktion $f(x)$ in ein Koordinatensystem. Markiere die Punkte C und D sowie den y-Achsenabschnitt und die Nullstelle. Überprüfe anhand deiner Zeichnung, ob deine berechneten Werte für $a$ (Steigungsdreieck) und $b$ plausibel sind.
    \item \textbf{Vorzeichen der Funktionswerte:} Was kannst du über das Vorzeichen der Funktionswerte $f(x)$ sagen für $x$-Werte, die kleiner als die Nullstelle sind ($x < x_N$), und für $x$-Werte, die größer als die Nullstelle sind ($x > x_N$)? Begründe dies anhand der Steigung und der Nullstelle.
\end{enumerate}

\bigskip % Fügt einen größeren vertikalen Abstand ein

\textbf{Teil 2: Gerade mit negativer Steigung} \\
Gegeben sind die Punkte $E(-1|3)$ und $F(1|-1)$, durch die eine lineare Funktion $g(x) = ax+b$ verläuft.

\begin{enumerate}[label=(\alph*)]
    \item \textbf{Steigung berechnen:} Berechne die Steigung $a$ der Geraden durch die Punkte E und F.
    \item \textbf{Y-Achsenabschnitt berechnen:} Bestimme den y-Achsenabschnitt $b$ der Geraden.
    \item \textbf{Funktionsgleichung aufstellen:} Gib die vollständige Funktionsgleichung $g(x)$ an.
    \item \textbf{Nullstelle bestimmen:} Berechne die Nullstelle $x_N$ der Funktion $g(x)$.
    \item \textbf{Funktionswerte berechnen:}
        \begin{itemize}
            \item Welchen Wert hat $g(0)$? Was sagt dieser Wert über den Graphen aus?
            \item Berechne $g(3)$.
        \end{itemize}
    \item \textbf{Argument für einen Funktionswert finden:} Für welchen Wert von $x$ gilt $g(x) = 7$?
    \item \textbf{Punktprobe:} Liegt der Punkt $Q(0.5|0)$ auf der Geraden? Begründe deine Antwort rechnerisch. (Tipp: Vergleiche mit deiner Berechnung aus Teil d)).
    \item \textbf{Skizze und Überprüfung:} Zeichne den Graphen der Funktion $g(x)$ in ein Koordinatensystem. Markiere die Punkte E und F sowie den y-Achsenabschnitt und die Nullstelle. Überprüfe anhand deiner Zeichnung, ob deine berechneten Werte für $a$ (Ist die Gerade fallend? Wie ist das Steigungsdreieck?) und $b$ plausibel sind.
    \item \textbf{Vorzeichen der Funktionswerte:} Was kannst du über das Vorzeichen der Funktionswerte $g(x)$ sagen für $x$-Werte, die kleiner als die Nullstelle sind ($x < x_N$), und für $x$-Werte, die größer als die Nullstelle sind ($x > x_N$)? Begründe dies anhand der (negativen) Steigung und der Nullstelle.
\end{enumerate}
\end{aufgabenumgebung}

\subsubsection{Nullstellen linearer Funktionen – Wo die Gerade die x-Achse trifft}
Ein wichtiger Punkt einer Funktion ist ihre \textbf{Nullstelle}. Das ist der x-Wert, an dem der Funktionswert $f(x)$ gleich Null ist. Grafisch ist das der \textbf{Schnittpunkt der Geraden mit der x-Achse}.

\begin{merksatzumgebung}{Nullstelle einer linearen Funktion berechnen}
Um die Nullstelle $x_N$ einer linearen Funktion $f(x) = ax+b$ zu finden, setzt du den Funktionsterm gleich Null und löst nach $x$ auf:
\[ f(x_N) = 0 \]
\[ ax_N + b = 0 \]
Wenn $a \neq 0$ ist, kannst du die Gleichung umformen:
\begin{center}
\begin{tabular}{r @{\,} c @{\,} l @{\quad\quad} l}
$ax_N + b$ & $=$ & $0$ & $| -b$ \\
$ax_N$ & $=$ & $-b$ & $| :a \quad (\text{falls } a \neq 0)$ \\
$x_N$ & $=$ & $-\frac{b}{a}$ & \\
\end{tabular}
\end{center}
Die Nullstelle ist also $x_N = -b/a$. Der Punkt auf der x-Achse ist $N(-b/a | 0)$.

\textbf{Sonderfälle:}
\begin{itemize}
    \item \textbf{Fall 1: $a \neq 0$ und $b=0$ (proportionale Funktion $f(x)=ax$)}
    Dann ist $x_N = -0/a = 0$. Die Nullstelle ist im Ursprung $(0|0)$. Die Gerade geht durch den Ursprung.
    \item \textbf{Fall 2: $a = 0$ (konstante Funktion $f(x)=b$)}
    \begin{itemize}
        \item Wenn $b \neq 0$ (z.B. $f(x)=3$): Die Gleichung $b=0$ ist ein Widerspruch. Die Gerade ist parallel zur x-Achse und schneidet sie nie. Es gibt \textbf{keine Nullstelle}.
        \item Wenn $b = 0$ (also $f(x)=0$): Die Gleichung $0=0$ ist immer wahr. Die Gerade ist die x-Achse selbst. Es gibt \textbf{unendlich viele Nullstellen} (jeder x-Wert ist eine Nullstelle).
    \end{itemize}
\end{itemize}
\end{merksatzumgebung}

\begin{beispielumgebung}{Nullstelle berechnen}
Gegeben ist die Funktion $f(x) = 2x - 4$.
Wir suchen die Nullstelle, also setzen wir $f(x)=0$:
\begin{center}
\begin{tabular}{r @{\,} c @{\,} l @{\quad\quad} l}
$2x - 4$ & $=$ & $0$ & $| +4$ \\
$2x$ & $=$ & $4$ & $| :2$ \\
$x$ & $=$ & $2$ & \\
\end{tabular}
\end{center}
Die Nullstelle ist $x_N=2$. Der Schnittpunkt mit der x-Achse ist $N(2|0)$.
Mit der Formel: $x_N = -b/a = -(-4)/2 = 4/2 = 2$. Passt!
\end{beispielumgebung}

\begin{aufgabenumgebung}{Nullstellen finden}
Berechne die Nullstellen der folgenden linearen Funktionen. Gib auch den Schnittpunkt mit der x-Achse an.
\begin{enumerate}
    \item $f(x) = 3x + 6$
    \item $g(x) = -0.5x + 2$
    \item $h(x) = 4x$
    \item $k(x) = 5$ (Was passiert hier?)
\end{enumerate}
\end{aufgabenumgebung}

\subsubsection{Wertetabellen erstellen und nutzen}
Eine \textbf{Wertetabelle} ist eine Tabelle, die zu ausgewählten x-Werten die zugehörigen y-Werte (Funktionswerte) einer Funktion auflistet. Sie ist sehr nützlich, um:
\begin{itemize}
    \item Einen ersten Überblick über den Verlauf der Funktion zu bekommen.
    \item Punkte zu sammeln, um den Graphen der Funktion (hier eine Gerade) zu zeichnen.
    \item Spezifische Funktionswerte schnell nachzuschlagen.
\end{itemize}

\begin{merksatzumgebung}{Wertetabelle erstellen}
So erstellst du eine Wertetabelle für eine Funktion $f(x)$:
\begin{enumerate}
    \item \textbf{Wähle x-Werte aus:} Entscheide dich für einige x-Werte, für die du die Funktionswerte berechnen möchtest. Oft wählt man ganze Zahlen rund um den Ursprung (z.B. -2, -1, 0, 1, 2) oder Werte, die für eine Anwendungsaufgabe relevant sind.
    \item \textbf{Berechne die y-Werte:} Setze jeden gewählten x-Wert in die Funktionsgleichung $f(x)$ ein und berechne den zugehörigen y-Wert.
    \item \textbf{Trage die Wertepaare in eine Tabelle ein:}
\end{enumerate}
\end{merksatzumgebung}

\begin{beispielumgebung}{Wertetabelle für $f(x) = 0.5x + 1$}
Wir erstellen eine Wertetabelle für die Funktion $f(x) = 0.5x + 1$ für die x-Werte -2, -1, 0, 1, 2.

\begin{itemize}
    \item $x = -2 \implies f(-2) = 0.5 \cdot (-2) + 1 = -1 + 1 = 0$
    \item $x = -1 \implies f(-1) = 0.5 \cdot (-1) + 1 = -0.5 + 1 = 0.5$
    \item $x =  0 \implies f(0)  = 0.5 \cdot 0 + 1 = 0 + 1 = 1$ (Das ist der y-Achsenabschnitt!)
    \item $x =  1 \implies f(1)  = 0.5 \cdot 1 + 1 = 0.5 + 1 = 1.5$
    \item $x =  2 \implies f(2)  = 0.5 \cdot 2 + 1 = 1 + 1 = 2$
\end{itemize}

Die Wertetabelle sieht dann so aus:
\begin{center}
\begin{tabular}{c|c}
$x$ & $f(x) = 0.5x + 1$ \\
\hline
-2  & 0 \\
-1  & 0.5 \\
0   & 1 \\
1   & 1.5 \\
2   & 2 \\
\end{tabular}
\captionof{table}{Wertetabelle für $f(x)=0.5x+1$}
\end{center}
Mit diesen Punkten $(-2|0)$, $(-1|0.5)$, $(0|1)$, $(1|1.5)$, $(2|2)$ könntest du nun die Gerade zeichnen. Für eine Gerade reichen eigentlich zwei Punkte, aber eine Wertetabelle mit mehr Punkten gibt mehr Sicherheit beim Zeichnen und hilft, Rechenfehler zu entdecken.
\end{beispielumgebung}

\begin{aufgabenumgebung}{Deine Wertetabelle}
Erstelle eine Wertetabelle für die Funktion $g(x) = -2x + 3$ für die x-Werte von -3 bis 3 in Einerschritten. Zeichne anschließend den Graphen der Funktion mithilfe der Punkte aus deiner Wertetabelle.
\end{aufgabenumgebung}

\subsubsection{Einen x-Wert oder y-Wert bestimmen}
Manchmal ist ein x-Wert gegeben und der zugehörige y-Wert (Funktionswert) ist gesucht. Manchmal ist es umgekehrt: ein y-Wert ist bekannt und man möchte wissen, welcher x-Wert dazu gehört.

\begin{merksatzumgebung}{x- oder y-Wert bestimmen}
Gegeben sei eine lineare Funktion $f(x) = ax+b$.
\begin{itemize}
    \item \textbf{y-Wert (Funktionswert) zu einem gegebenen x-Wert bestimmen:}
    Setze den gegebenen x-Wert einfach in die Funktionsgleichung ein und berechne $f(x)$.
    Beispiel: $f(x)=2x+1$. Was ist $f(3)$? $f(3) = 2 \cdot 3 + 1 = 6+1=7$. Der y-Wert ist 7. Der Punkt ist $(3|7)$.

    \item \textbf{x-Wert zu einem gegebenen y-Wert (Funktionswert) bestimmen:}
    Setze den gegebenen y-Wert für $f(x)$ in die Funktionsgleichung ein: $y_{gegeben} = ax+b$.
    Löse diese Gleichung dann nach $x$ auf:
    \begin{center}
    \begin{tabular}{r @{\,} c @{\,} l @{\quad\quad} l}
    $y_{gegeben}$ & $=$ & $ax+b$ & $|-b$ \\
    $y_{gegeben} - b$ & $=$ & $ax$ & $|:a \quad (\text{falls } a \neq 0)$ \\
    $\frac{y_{gegeben} - b}{a}$ & $=$ & $x$ & \\
    \end{tabular}
    \end{center}
    Beispiel: $f(x)=2x+1$. Für welchen x-Wert ist $f(x)=9$?
    \begin{center}
    \begin{tabular}{r @{\,} c @{\,} l @{\quad\quad} l}
    $9$ & $=$ & $2x+1$ & $|-1$ \\
    $8$ & $=$ & $2x$ & $|:2$ \\
    $4$ & $=$ & $x$ & \\
    \end{tabular}
    \end{center}
    Der gesuchte x-Wert ist 4. Der Punkt ist $(4|9)$.
\end{itemize}
\end{merksatzumgebung}

Diese beiden Operationen sind grundlegend im Umgang mit Funktionen.

\begin{aufgabenumgebung}{Umgang mit linearen Funktionen: Werte und Schnittpunkte}

\textbf{Teil 1: Funktion $f(x) = -3x + 7$} \\
Gegeben ist die Funktion $f(x) = -3x + 7$.
\begin{enumerate}[label=(\alph*)]
    \item Berechne $f(-2)$, $f(0)$ und $f(4)$.
    \item Für welchen Wert von $x$ gilt $f(x) = 10$?
    \item Für welchen Wert von $x$ gilt $f(x) = -5$?
    \item An welcher Stelle schneidet der Graph die y-Achse? An welcher die x-Achse (Nullstelle)?
\end{enumerate}

\bigskip % Fügt einen größeren vertikalen Abstand ein

\textbf{Teil 2: Funktion $g(x) = \frac{1}{2}x - 1$} \\
Gegeben ist die Funktion $g(x) = \frac{1}{2}x - 1$.
\begin{enumerate}[label=(\alph*)]
    \item Berechne $g(-4)$, $g(0)$ und $g(6)$.
    \item Für welchen Wert von $x$ gilt $g(x) = 3$?
    \item Für welchen Wert von $x$ gilt $g(x) = -2.5$?
    \item An welcher Stelle schneidet der Graph die y-Achse? An welcher die x-Achse (Nullstelle)?
\end{enumerate}

\bigskip % Fügt einen größeren vertikalen Abstand ein

\textbf{Teil 3: Funktion $h(x) = -4x$} \\
Gegeben ist die Funktion $h(x) = -4x$.
\begin{enumerate}[label=(\alph*)]
    \item Berechne $h(-1.5)$, $h(0)$ und $h(2.5)$.
    \item Für welchen Wert von $x$ gilt $h(x) = 12$?
    \item Für welchen Wert von $x$ gilt $h(x) = -1$?
    \item An welcher Stelle schneidet der Graph die y-Achse? An welcher die x-Achse (Nullstelle)? Was ist hier besonders?
\end{enumerate}
\end{aufgabenumgebung}

% Dieser Block sollte in Kapitel 2: Lineare Funktionen eingefügt werden,
% z.B. nach dem Unterabschnitt 'Funktionsgleichung aufstellen' und 
% vor 'Anwendungsaufgaben – Lineare Funktionen im echten Leben'.

\subsubsection{Schnittpunkte zweier Geraden – Wo treffen sie sich?}
\label{subsubsec:schnittpunkte_geraden}

Oft haben wir nicht nur eine Gerade, sondern zwei (oder mehr) und möchten wissen, ob und wo sie sich schneiden. Der Schnittpunkt zweier Geraden $f(x)$ und $g(x)$ ist der Punkt $(x_S|y_S)$, der auf beiden Geraden liegt. Das bedeutet, an der Stelle $x_S$ müssen beide Funktionen denselben Funktionswert $y_S$ haben.

\begin{merksatzumgebung}{Schnittpunkt zweier Funktionen $f(x)$ und $g(x)$ berechnen}
Um den Schnittpunkt (oder die Schnittpunkte) zweier Funktionen $f(x)$ und $g(x)$ zu finden, gehst du wie folgt vor:
\begin{enumerate}
    \item \textbf{Funktionsterme gleichsetzen:}
    Setze die beiden Funktionsterme gleich:
    \[ f(x) = g(x) \]
    \item \textbf{Gleichung nach $x$ auflösen:}
    Löse die entstandene Gleichung nach der Variablen $x$. Der Wert (oder die Werte), den du für $x$ erhältst, ist die x-Koordinate des Schnittpunkts (der Schnittpunkte), $x_S$.
    \item \textbf{y-Koordinate berechnen:}
    Setze den gefundenen x-Wert $x_S$ in \textbf{eine der beiden ursprünglichen Funktionsgleichungen} ($f(x)$ oder $g(x)$ – es ist egal welche, da der Punkt ja auf beiden liegen soll) ein, um die zugehörige y-Koordinate $y_S$ zu berechnen.
    \[ y_S = f(x_S) \quad \text{oder} \quad y_S = g(x_S) \]
    \item \textbf{Schnittpunkt angeben:}
    Der Schnittpunkt ist $S(x_S|y_S)$.
\end{enumerate}
Zwei nicht-parallele Geraden haben immer genau einen Schnittpunkt. Parallele Geraden haben keinen Schnittpunkt (wenn sie verschieden sind) oder unendlich viele (wenn sie identisch sind).
\end{merksatzumgebung}

\begin{beispielumgebung}{Schnittpunkt zweier Geraden berechnen}
Gegeben sind die beiden linearen Funktionen:
$f(x) = 2x - 1$
$g(x) = -0.5x + 4$

Wir suchen den Schnittpunkt $S(x_S|y_S)$.

\textbf{Schritt 1: Funktionsterme gleichsetzen.}
$f(x) = g(x)$
$2x - 1 = -0.5x + 4$

\textbf{Schritt 2: Gleichung nach $x$ auflösen.}
Wir verwenden unsere Äquivalenzumformungen:
\begin{center}
\begin{tabular}{r @{\,} c @{\,} l @{\quad\quad} l}
$2x - 1$ & $=$ & $-0.5x + 4$ & $| +0.5x$ \\
$2.5x - 1$ & $=$ & $4$ & $| +1$ \\
$2.5x$ & $=$ & $5$ & $| :2.5$ \\
$x$ & $=$ & $2$ & \\
\end{tabular}
\end{center}
Die x-Koordinate des Schnittpunkts ist $x_S = 2$.

\textbf{Schritt 3: y-Koordinate berechnen.}
Wir setzen $x_S=2$ in eine der beiden ursprünglichen Gleichungen ein, z.B. in $f(x)$:
$y_S = f(2) = 2 \cdot 2 - 1 = 4 - 1 = 3$.
Zur Probe können wir auch in $g(x)$ einsetzen:
$y_S = g(2) = -0.5 \cdot 2 + 4 = -1 + 4 = 3$.
Das Ergebnis ist dasselbe, also $y_S=3$.

\textbf{Schritt 4: Schnittpunkt angeben.}
Der Schnittpunkt der beiden Geraden ist $S(2|3)$.

\begin{center}
    \includegraphics[scale = 0.5]{grafiken/Schnittpunkt_Zweier_Geraden.png}
    \captionof{figure}{Schnittpunkt der Geraden $f(x)=2x-1$ und $g(x)=-0.5x+4$}
    \label{fig:schnittpunkt_geraden_bsp}
\end{center}
% Der Text geht hier direkt weiter

\end{beispielumgebung}

\begin{aufgabenumgebung}{Schnittpunkte berechnen und zeichnen}
\begin{enumerate}
    \item Berechne den Schnittpunkt der folgenden Geradenpaare:
        \begin{itemize}
            \item $f(x) = x + 3$ und $g(x) = -2x + 9$
            \item $h(x) = 0.25x - 2$ und $k(x) = 0.25x + 1$ (Was stellst du hier fest? Wie liegen die Geraden zueinander?)
            \item $m(x) = \frac{1}{3}x + 1$ und $n(x) = -\frac{2}{3}x + 4$
        \end{itemize}
    \item Gegeben sind die Funktionen $f(x) = -x+5$ und $g(x) = 2x-1$.
        \begin{itemize}
            \item Berechne ihren Schnittpunkt $S$.
            \item Zeichne beide Geraden und ihren Schnittpunkt in ein Koordinatensystem.
        \end{itemize}
\end{enumerate}
\end{aufgabenumgebung}

\begin{warumwichtigumgebung}{Schnittpunkte}
Die Fähigkeit, Schnittpunkte von Funktionen zu berechnen, ist fundamental. Sie wird nicht nur bei Geraden benötigt, sondern auch bei Parabeln, Exponentialfunktionen und allen anderen Funktionstypen. Immer wenn gefragt wird, wann zwei Größen gleich sind oder wann sich zwei Prozesse treffen, führt dies mathematisch auf die Berechnung von Schnittpunkten. Die Nullstellenbestimmung ist ein Spezialfall davon: der Schnittpunkt mit der x-Achse (der Funktion $y=0$).
\end{warumwichtigumgebung}

% Hier würde dann der nächste Abschnitt des Kapitels 'Lineare Funktionen' folgen,
% z.B. 'Anwendungsaufgaben – Lineare Funktionen im echten Leben'.


Lineare Funktionen sind nicht nur abstrakte Gebilde, sondern haben viele praktische Anwendungen.

\subsection{Anwendungsaufgaben – Lineare Funktionen im echten Leben}

Viele reale Situationen lassen sich durch lineare Funktionen modellieren, zumindest näherungsweise. Oft gibt es eine Art 'Startwert' (entspricht $b$) und eine konstante Änderungsrate (entspricht $a$).

\begin{beispielumgebung}[Handyvertrag]{Kosten für einen Handyvertrag}
Ein Handyvertrag kostet monatlich eine Grundgebühr von 10 Euro. Jede SMS kostet zusätzlich 0,05 Euro.
Stelle eine Funktion auf, die die Gesamtkosten $K(x)$ in Abhängigkeit von der Anzahl $x$ der SMS beschreibt.

\textbf{Überlegung:}
\begin{itemize}
    \item Die Grundgebühr von 10 Euro zahlst du immer, auch wenn du keine SMS schreibst. Das ist dein fester Anteil, der nicht von $x$ abhängt. Das ist also der y-Achsenabschnitt $b=10$.
    \item Jede SMS kostet 0,05 Euro. Dieser Betrag wird mit der Anzahl $x$ der SMS multipliziert. Das ist also die Steigung $a=0,05$ (Kosten pro SMS).
\end{itemize}
\textbf{Funktionsgleichung:}
Die Kostenfunktion $K(x)$ lautet:
\[ K(x) = 0,05x + 10 \]
Hier ist $x$ die Anzahl der SMS und $K(x)$ die Gesamtkosten in Euro für einen Monat.

\textbf{Frage:} Wie viel kostet es, wenn man 100 SMS schreibt?
Wir setzen $x=100$ in die Funktion ein:
$K(100) = 0,05 \cdot 100 + 10 = 5 + 10 = 15$.
Antwort: Es kostet 15 Euro, wenn man 100 SMS schreibt.

\textbf{Frage:} Was bedeutet $K(0)$?
$K(0) = 0,05 \cdot 0 + 10 = 10$. Das sind die Kosten, wenn man keine SMS schreibt – also die reine Grundgebühr.
\end{beispielumgebung}

Solche Kostenfunktionen sind typische Anwendungsbeispiele.

\begin{aufgabenumgebung}{Taxifahrt}
Ein Taxifahrer verlangt eine Grundgebühr von 3,50 Euro für jede Fahrt. Zusätzlich kostet jeder gefahrene Kilometer 2 Euro.
\begin{enumerate}
    \item Stelle die Kostenfunktion $K(x)$ auf, wobei $x$ die Anzahl der gefahrenen Kilometer ist. Identifiziere $a$ und $b$ und erkläre ihre Bedeutung in diesem Kontext.
    \item Wie viel kostet eine Fahrt von 15 km?
    \item Du hast 25 Euro dabei. Wie viele Kilometer kannst du maximal fahren, wenn du den vollen Betrag ausgeben möchtest? (Tipp: Setze $K(x)=25$ und löse die Gleichung nach $x$ auf.)
    \item Zeichne den Graphen der Funktion $K(x)$ für $x$-Werte von 0 km bis 20 km. Wähle passende Einheiten für die Achsen.
\end{enumerate}
\end{aufgabenumgebung}

\begin{fehlerboxumgebung}{Typische Fehler bei linearen Funktionen}
\begin{itemize}
    \item \textbf{Vorzeichenfehler bei der Steigung:} Achte genau auf die Vorzeichen von $y_2, y_1, x_2, x_1$ bei der Formel $a = \frac{y_2-y_1}{x_2-x_1}$. Ein falsches Vorzeichen kehrt die Richtung der Geraden um!
    \item \textbf{Verwechslung von $a$ und $b$:} Die Steigung $a$ gibt an, wie steil es ist, der y-Achsenabschnitt $b$ ist der Startwert auf der y-Achse.
    \item \textbf{Fehler beim Umstellen von Gleichungen:} Übe das sichere Umstellen von Gleichungen (Äquivalenzumformungen), besonders wenn du $x$ bei gegebenem $y$ suchst oder $b$ aus einem Punkt und der Steigung.
    \item \textbf{Steigungsdreieck falsch ablesen/zeichnen:} Denke daran: $a = \frac{\Delta y}{\Delta x}$ (erst 'nach oben/unten', dann 'nach rechts').
\end{itemize}
\end{fehlerboxumgebung}

Hier sind noch ein paar weitere Übungen, um dein Verständnis zu festigen.

\begin{aufgabenumgebung}[labelA:LinUeb]{Lineare Funktionen – Übungen querbeet}
\begin{enumerate}
    \item \textbf{Paketversand:} Ein Paketversand kostet 5 Euro Grundgebühr. Jedes Kilogramm Gewicht kostet zusätzlich 1,50 Euro.
        \begin{itemize}
            \item Wie lautet die Kostenfunktion $K(g)$, wenn $g$ das Gewicht in Kilogramm ist?
            \item Was kostet ein Paket, das 3,2 kg wiegt?
            \item Ein Paket kostet 15,50 Euro. Wie schwer war es?
        \end{itemize}
    \item \textbf{Kerzenlänge:} Eine Kerze ist anfangs 20 cm lang. Pro Stunde brennt sie gleichmäßig 2,5 cm ab.
        \begin{itemize}
            \item Stelle eine Funktion $L(t)$ auf, die die Länge der Kerze nach $t$ Stunden beschreibt. (Achtung: Die Länge nimmt ab. Was bedeutet das für das Vorzeichen der Steigung $a$?)
            \item Wie lang ist die Kerze nach 3 Stunden?
            \item Nach wie vielen Stunden ist die Kerze komplett abgebrannt? (Das bedeutet, ihre Länge ist 0 cm.)
            \item Für welchen Zeitraum $t$ ist diese Funktion realistisch? (Definitionsbereich der Anwendung)
        \end{itemize}
    \item \textbf{Funktion aus Punkten:} Eine Gerade geht durch die Punkte $P_1(2|5)$ und $P_2(5|14)$. Bestimme ihre Funktionsgleichung $f(x)=ax+b$.
    \item \textbf{Funktion aus Steigung und Punkt:} Die Steigung einer Geraden ist $a = -2$. Die Gerade geht außerdem durch den Punkt $P(1|3)$. Bestimme den y-Achsenabschnitt $b$ und gib die vollständige Funktionsgleichung an. (Tipp: Setze $a$ und die Koordinaten von $P$ in $y=ax+b$ ein und löse nach $b$.)
\end{enumerate}
\end{aufgabenumgebung}

Das Konzept der Steigung als Änderungsrate ist fundamental und führt uns zu einem wichtigen Begriff in der Analysis.

\subsection{Die durchschnittliche Änderungsrate – Ein Vorgeschmack auf mehr}

Bei linearen Funktionen ist die Steigung $a$ konstant. Das bedeutet, die Änderungsrate ist immer gleich, egal welches Intervall man betrachtet. Bei kurvigen Funktionen ist das anders. Dort spricht man von der \textbf{durchschnittlichen Änderungsrate} über ein bestimmtes Intervall.

\begin{merksatzumgebung}[Durchschnittliche Änderungsrate]{Was ist das und wie berechnet man sie?}
Die durchschnittliche Änderungsrate einer Funktion $f$ im Intervall $[x_1, x_2]$ gibt an, wie stark sich der Funktionswert $f(x)$ \textbf{im Durchschnitt} ändert, wenn sich der $x$-Wert von $x_1$ nach $x_2$ ändert.
Sie ist nichts anderes als die \textbf{Steigung der Sekante} (eine Gerade, die durch die beiden Punkte $(x_1|f(x_1))$ und $(x_2|f(x_2))$ auf dem Funktionsgraphen geht).

Die Formel lautet:
\[ \text{Durchschnittliche Änderungsrate im Intervall } [x_1, x_2] = m_{Sekante} = \frac{f(x_2) - f(x_1)}{x_2 - x_1} = \frac{\Delta y}{\Delta x} \]
Du siehst, das ist genau die gleiche Formel wie für die Steigung $a$ einer linearen Funktion!
Bei \textbf{linearen Funktionen} ist die durchschnittliche Änderungsrate in jedem Intervall konstant und entspricht genau der Steigung $a$ der Geraden. Bei anderen (nicht-linearen) Funktionen ändert sich die durchschnittliche Änderungsrate jedoch, je nachdem, welches Intervall $[x_1, x_2]$ man wählt.
\end{merksatzumgebung}

Ein Beispiel aus dem Alltag macht das klarer.

\begin{beispielumgebung}[Kilometerstand eines Autos]{Durchschnittliche Fahrleistung}
Am 1. Mai (Tag 0 eines Beobachtungszeitraums) zeigt der Kilometerzähler eines Autos 10.000 km. Am 1. Oktober (nach 5 Monaten, also am Tag $5 \times 30 = 150$ ungefähr, oder einfacher: Monat 0 bis Monat 5) zeigt er 25.000 km.
Wie viele Kilometer ist das Auto \textbf{durchschnittlich} pro Monat gefahren?

Hier können wir die Monate als 'x-Werte' und die Kilometerstände als 'Funktionswerte' betrachten.
$x_1 = 0$ Monate (Startzeitpunkt)
$f(x_1) = 10.000$ km (Kilometerstand zu Beginn)

$x_2 = 5$ Monate (Endzeitpunkt)
$f(x_2) = 25.000$ km (Kilometerstand am Ende)

Die Veränderung der Monate ist $\Delta x = x_2 - x_1 = 5 - 0 = 5$ Monate.
Die Veränderung des Kilometerstandes ist $\Delta y = f(x_2) - f(x_1) = 25.000 \text{ km} - 10.000 \text{ km} = 15.000 \text{ km}$.

Die durchschnittliche Fahrleistung (Änderungsrate) ist:
\[ \frac{\Delta y}{\Delta x} = \frac{15.000 \text{ km}}{5 \text{ Monate}} = 3.000 \text{ km/Monat} \]
Das Auto ist also im Durchschnitt 3000 km pro Monat gefahren. Das heißt nicht, dass es jeden Monat genau 3000 km gefahren ist – mal mehr, mal weniger – aber im Schnitt über die 5 Monate waren es 3000 km/Monat.
\end{beispielumgebung}

Dieser Begriff ist sehr nützlich, um Veränderungen über Zeiträume zu analysieren.

\begin{aufgabenumgebung}{Deine durchschnittliche Rate}
Denke dir ein eigenes Beispiel aus dem Alltag aus, bei dem eine Größe sich über die Zeit oder eine andere Einheit verändert (z.B. Wasserverbrauch einer Familie über mehrere Tage, Wachstum einer Pflanze über Wochen, Temperaturänderung im Laufe eines Tages, Anzahl der gelesenen Seiten eines Buches über mehrere Abende).
\begin{enumerate}
    \item Beschreibe die Situation klar. Welche Größe ändert sich in Abhängigkeit von welcher anderen Größe?
    \item Lege zwei Messpunkte (Anfangswert mit zugehörigem 'x1' und Endwert mit zugehörigem 'x2') fest.
    \item Berechne die durchschnittliche Änderungsrate. Was sagt diese Rate in deinem Beispiel konkret aus? Welche Einheit hat sie?
\end{enumerate}
\end{aufgabenumgebung}

\begin{kurzknappumgebung}{Lineare Funktionen}
\begin{itemize}
    \item \textbf{Form:} $f(x) = ax+b$ (Graph ist eine Gerade).
    \item \textbf{$a$ = Steigung:} Gibt an, wie stark die Gerade steigt/fällt. $a = \frac{\Delta y}{\Delta x}$.
    \item \textbf{$b$ = y-Achsenabschnitt:} Schnittpunkt mit der y-Achse bei $(0|b)$.
    \item \textbf{Nullstelle $x_N$:} Schnittpunkt mit der x-Achse. Lösung von $ax+b=0 \implies x_N = -b/a$ (für $a \neq 0$).
    \item \textbf{Wertetabelle:} Hilft beim Zeichnen und Verstehen.
    \item \textbf{Durchschnittliche Änderungsrate:} Ist bei linearen Funktionen immer gleich der Steigung $a$.
\end{itemize}
\end{kurzknappumgebung}

\begin{infoboxumgebung}{Ausblick: Von der durchschnittlichen zur momentanen Änderungsrate}
Die durchschnittliche Änderungsrate betrachtet, wie der Name schon sagt, einen Durchschnitt über ein Intervall. In der \textbf{Differentialrechnung}, einem Kerngebiet der Analysis, wollen wir aber oft wissen, wie schnell sich etwas in einem \textbf{einzigen Moment} ändert. Das nennt man die \textbf{momentane Änderungsrate} oder auch die \textbf{Ableitung} einer Funktion an einer bestimmten Stelle.
Stell dir vor, du machst das Intervall $[x_1, x_2]$ immer kleiner und kleiner, sodass $x_2$ ganz nah an $x_1$ heranrückt. Was passiert dann mit der Steigung der Sekante? Sie nähert sich der Steigung der \textbf{Tangente} an den Graphen im Punkt $(x_1|f(x_1))$. Diese Tangentensteigung ist dann die momentane Änderungsrate. Das ist eine der faszinierendsten Ideen der Analysis, die wir später genauer untersuchen werden!
\begin{center}
    \includegraphics[width=0.8\textwidth]{grafiken/Sekante_zur_Tangente.png}
    \captionof{figure}{Von der Sekantensteigung zur Tangentensteigung}
    \label{fig:sekante_tangente}
\end{center}
% Der Text geht hier direkt weiter

\end{infoboxumgebung}


\begin{aufgabenumgebung}{Checkliste: Eigenschaften einer linearen Funktion analysieren}
Betrachte die lineare Funktion $f(x) = \frac{1}{2}x - 1$.
Bearbeite die folgenden Punkte, um dein Verständnis dieser Funktion zu überprüfen:

\begin{enumerate}[label=(\alph*)]
    \item \textbf{Nullstelle:} Berechne die Nullstelle $x_0$ der Funktion $f(x)$ (d.h. die Stelle, an der $f(x_0)=0$ ist).
    \item \textbf{Y-Achsenabschnitt:} Bestimme den Schnittpunkt mit der y-Achse (den y-Achsenabschnitt $b$). Welchen Wert hat die Funktion an der Stelle $x=0$?
    \item \textbf{Skizze:} Zeichne den Graphen der Funktion $f(x)$ sorgfältig in ein Koordinatensystem. Markiere deutlich die Nullstelle auf der x-Achse und den y-Achsenabschnitt auf der y-Achse.
    \item \textbf{Funktionswerte positiv/negativ:}
    \begin{itemize}
        \item Für welche $x$-Werte sind die Funktionswerte $f(x)$ \textbf{positiv} (d.h. der Graph verläuft oberhalb der x-Achse)? Gib den Bereich an.
        \item Für welche $x$-Werte sind die Funktionswerte $f(x)$ \textbf{negativ} (d.h. der Graph verläuft unterhalb der x-Achse)? Gib den Bereich an.
    \end{itemize}
    \item \textbf{Rolle der Steigung:} Wie hilft dir das Vorzeichen der Steigung $a = \frac{1}{2}$ dabei, die Bereiche aus Teil (d) schnell zu bestimmen, sobald du die Nullstelle kennst? Erkläre in eigenen Worten.
    \item \textbf{Rolle des y-Achsenabschnitts:} Betrachte den y-Achsenabschnitt $b=-1$. Hilft dir dieser Wert auch dabei, das Vorzeichen der Funktion in einem bestimmten Bereich (z.B. für $x$-Werte nahe Null) zu bestimmen? Begründe.
    \item \textbf{Wertebereich:} Der Wertebereich einer nicht-konstanten linearen Funktion ist die Menge aller reellen Zahlen $\mathbb{R}$. Wie kannst du das an deinem Graphen erkennen oder dir vorstellen, dass jeder y-Wert irgendwann einmal getroffen wird?
\end{enumerate}
\end{aufgabenumgebung}

\begin{aufgabenumgebung}{Checkliste: Verhalten linearer Funktionen verstehen und interpretieren}
Gegeben sei die lineare Funktion $g(x) = -2x + 4$.
Führe eine Analyse dieser Funktion anhand der folgenden Punkte durch:

\begin{enumerate}[label=(\alph*)]
    \item \textbf{Nullstelle und y-Achsenabschnitt:} Berechne die Nullstelle von $g(x)$ und gib den y-Achsenabschnitt an.
    \item \textbf{Skizze:} Zeichne den Graphen der Funktion $g(x)$. Achte darauf, die Achsenschnittpunkte klar zu kennzeichnen.
    \item \textbf{Vorzeichen der Funktionswerte:}
    \begin{itemize}
        \item In welchem Intervall ist $g(x) > 0$?
        \item In welchem Intervall ist $g(x) < 0$?
    \end{itemize}
    \item \textbf{Argumentation mit Steigung und Nullstelle:} Erkläre, wie du allein aus dem Vorzeichen der Steigung $a=-2$ und der Lage der Nullstelle $x_0$ schlussfolgern kannst, ob die Funktionswerte für $x$-Werte, die größer als die Nullstelle sind ($x > x_0$), positiv oder negativ sein müssen.
    \item \textbf{Vergleich mit dem y-Achsenabschnitt:} Bestätigt der y-Achsenabschnitt deine Überlegungen zum Vorzeichen der Funktion für $x=0$? Liegt $x=0$ in dem von dir bestimmten positiven oder negativen Bereich?
    \item \textbf{Gedankenexperiment:} Stell dir eine lineare Funktion $h(x)$ vor, von der du nur weißt: Ihre Steigung ist positiv ($a > 0$) und ihre Nullstelle ist ebenfalls positiv ($x_0 > 0$).
    \begin{itemize}
        \item Mache eine grobe Skizze, wie solch eine Funktion aussehen könnte.
        \item Welches Vorzeichen muss der y-Achsenabschnitt dieser Funktion $h(x)$ haben? Begründe deine Antwort mathematisch oder anhand deiner Skizze.
    \end{itemize}
\end{enumerate}
\end{aufgabenumgebung}
% --- ANNAHME: Alle Pakete und Box-Definitionen aus dem Hauptdokument sind hier gültig ---
% --- Dieser Code-Block ist als Ersatz für das bestehende Kapitel 4 gedacht ---

\section{Quadratische Funktionen – Die Welt der Parabeln}
\label{sec:quadratische-funktionen_ueberarbeitet}

\begin{aufgabenumgebung}{Parameter-Check}
Betrachte die folgenden quadratischen Funktionen. Gib für jede Funktion die Werte der Koeffizienten $a, b$ und $c$ an. Was kannst du sofort über die Öffnungsrichtung, die Form (schmaler/breiter als Normalparabel) und den y-Achsenabschnitt sagen?
\begin{enumerate}
    \item $f(x) = 2x^2 - 4x + 5$
    \item $g(x) = -x^2 + 3x - 1$
    \item $h(x) = 0.25x^2 + 2$ (Achtung, welcher Koeffizient ist hier Null?)
    \item $k(x) = -3x^2 - 6x$ (Und hier?)
\end{enumerate}
\end{aufgabenumgebung}


\begin{loesungsumgebung}[loes:parameter-check-quadratisch]{Parameter-Check}
Die allgemeine Form einer quadratischen Funktion ist $f(x) = ax^2 + bx + c$. Die Koeffizienten $a, b$ und $c$ geben Aufschluss über verschiedene Eigenschaften der Parabel.

\begin{enumerate}[label=(\alph*)]
    \item \textbf{Funktion $f(x) = 2x^2 - 4x + 5$}
    \begin{itemize}
        \item Koeffizienten: $a = 2$, $b = -4$, $c = 5$.
        \item \textbf{Öffnungsrichtung:} Da $a = 2 > 0$ ist, ist die Parabel \textbf{nach oben geöffnet}.
        \item \textbf{Form:} Da $|a| = |2| = 2 > 1$ ist, ist die Parabel \textbf{schmaler} als die Normalparabel (sie ist gestreckt).
        \item \textbf{y-Achsenabschnitt:} Der Koeffizient $c = 5$ gibt den y-Achsenabschnitt an. Die Parabel schneidet die y-Achse im Punkt $(0|5)$.
    \end{itemize}

    \item \textbf{Funktion $g(x) = -x^2 + 3x - 1$}
    \begin{itemize}
        \item Koeffizienten: $a = -1$, $b = 3$, $c = -1$.
        \item \textbf{Öffnungsrichtung:} Da $a = -1 < 0$ ist, ist die Parabel \textbf{nach unten geöffnet}.
        \item \textbf{Form:} Da $|a| = |-1| = 1$ ist, hat die Parabel die \textbf{gleiche Form wie die Normalparabel} (ist weder schmaler noch breiter, nur nach unten geöffnet).
        \item \textbf{y-Achsenabschnitt:} Der Koeffizient $c = -1$ gibt den y-Achsenabschnitt an. Die Parabel schneidet die y-Achse im Punkt $(0|-1)$.
    \end{itemize}

    \item \textbf{Funktion $h(x) = 0.25x^2 + 2$}
    Um die Koeffizienten klar zu sehen, kann man die Funktion als $h(x) = 0.25x^2 + 0x + 2$ schreiben.
    \begin{itemize}
        \item Koeffizienten: $a = 0.25$, $b = 0$, $c = 2$.
        \item \textbf{Öffnungsrichtung:} Da $a = 0.25 > 0$ ist, ist die Parabel \textbf{nach oben geöffnet}.
        \item \textbf{Form:} Da $|a| = |0.25| = 0.25 < 1$ ist, ist die Parabel \textbf{breiter} als die Normalparabel (sie ist gestaucht).
        \item \textbf{y-Achsenabschnitt:} Der Koeffizient $c = 2$ gibt den y-Achsenabschnitt an. Die Parabel schneidet die y-Achse im Punkt $(0|2)$.
        \item \textit{Anmerkung (Achtung, welcher Koeffizient ist hier Null?):} Der Koeffizient $b$ ist hier Null ($b=0$). Dies bedeutet, dass die Parabel achsensymmetrisch zur y-Achse ist und ihr Scheitelpunkt auf der y-Achse bei $(0|c)$ liegt, also bei $(0|2)$.
    \end{itemize}

    \item \textbf{Funktion $k(x) = -3x^2 - 6x$}
    Um die Koeffizienten klar zu sehen, kann man die Funktion als $k(x) = -3x^2 - 6x + 0$ schreiben.
    \begin{itemize}
        \item Koeffizienten: $a = -3$, $b = -6$, $c = 0$.
        \item \textbf{Öffnungsrichtung:} Da $a = -3 < 0$ ist, ist die Parabel \textbf{nach unten geöffnet}.
        \item \textbf{Form:} Da $|a| = |-3| = 3 > 1$ ist, ist die Parabel \textbf{schmaler} als die Normalparabel (sie ist gestreckt).
        \item \textbf{y-Achsenabschnitt:} Der Koeffizient $c = 0$ gibt den y-Achsenabschnitt an. Die Parabel schneidet die y-Achse im Punkt $(0|0)$, d.h., sie verläuft durch den Ursprung des Koordinatensystems.
        \item \textit{Anmerkung (Und hier?):} Der Koeffizient $c$ ist hier Null ($c=0$). Dies bedeutet, dass die Parabel durch den Ursprung geht.
    \end{itemize}
\end{enumerate}

\end{loesungsumgebung}


\begin{aufgabenumgebung}{Parabeln skizzieren mit Wertetabelle}
\begin{enumerate}
    \item Erstelle eine Wertetabelle für die Normalparabel $f(x)=x^2$ für $x$-Werte von $-3$ bis $3$ (in Einerschritten). Zeichne den Graphen.
    \item Erstelle eine Wertetabelle für $g(x)=2x^2$ für dieselben x-Werte. Zeichne den Graphen in dasselbe Koordinatensystem wie $f(x)$. Was beobachtest du im Vergleich zur Normalparabel?
    \item Erstelle eine Wertetabelle für $h(x)=-0.5x^2+1$ für dieselben x-Werte. Zeichne den Graphen ebenfalls in dasselbe Koordinatensystem. Was beobachtest du?
\end{enumerate}
\end{aufgabenumgebung}


\begin{loesungsumgebung}[loes:parabeln-skizzieren-wertetabelle]{Parabeln skizzieren mit Wertetabelle}

\begin{enumerate}[label=(\alph*)]
    \item \textbf{Normalparabel $f(x)=x^2$} \\
    Wir erstellen eine Wertetabelle für $f(x)=x^2$ für die $x$-Werte von $-3$ bis $3$:
    \begin{itemize}
        \item $f(-3) = (-3)^2 = 9$
        \item $f(-2) = (-2)^2 = 4$
        \item $f(-1) = (-1)^2 = 1$
        \item $f(0) = (0)^2 = 0$
        \item $f(1) = (1)^2 = 1$
        \item $f(2) = (2)^2 = 4$
        \item $f(3) = (3)^2 = 9$
    \end{itemize}
    \textbf{Wertetabelle für $f(x)=x^2$:}
    \begin{center}
    \begin{tabular}{r r}
    \toprule
    \multicolumn{1}{c}{$x$} & \multicolumn{1}{c}{$f(x)=x^2$} \\
    \midrule
    $-3$ & $9$ \\
    $-2$ & $4$ \\
    $-1$ & $1$ \\
    $0$ & $0$ \\
    $1$ & $1$ \\
    $2$ & $4$ \\
    $3$ & $9$ \\
    \bottomrule
    \end{tabular}
    \end{center}
    Der Graph dieser Funktion ist die Normalparabel mit dem Scheitelpunkt im Ursprung $(0|0)$ und nach oben geöffnet.

    \item \textbf{Funktion $g(x)=2x^2$} \\
    Wir erstellen eine Wertetabelle für $g(x)=2x^2$ für dieselben $x$-Werte:
    \begin{itemize}
        \item $g(-3) = 2(-3)^2 = 2 \cdot 9 = 18$
        \item $g(-2) = 2(-2)^2 = 2 \cdot 4 = 8$
        \item $g(-1) = 2(-1)^2 = 2 \cdot 1 = 2$
        \item $g(0) = 2(0)^2 = 0$
        \item $g(1) = 2(1)^2 = 2$
        \item $g(2) = 2(2)^2 = 8$
        \item $g(3) = 2(3)^2 = 18$
    \end{itemize}
    \textbf{Wertetabelle für $g(x)=2x^2$:}
    \begin{center}
    \begin{tabular}{r r}
    \toprule
    \multicolumn{1}{c}{$x$} & \multicolumn{1}{c}{$g(x)=2x^2$} \\
    \midrule
    $-3$ & $18$ \\
    $-2$ & $8$ \\
    $-1$ & $2$ \\
    $0$ & $0$ \\
    $1$ & $2$ \\
    $2$ & $8$ \\
    $3$ & $18$ \\
    \bottomrule
    \end{tabular}
    \end{center}
    \textbf{Beobachtung im Vergleich zur Normalparabel:}
    Die Parabel $g(x)=2x^2$ ist ebenfalls nach oben geöffnet und hat ihren Scheitelpunkt im Ursprung $(0|0)$. Allerdings sind ihre $y$-Werte (außer bei $x=0$) doppelt so groß wie die der Normalparabel $f(x)=x^2$ für denselben $x$-Wert. Dadurch erscheint der Graph von $g(x)$ \textbf{schmaler} bzw. steiler als die Normalparabel. Man sagt auch, sie ist in y-Richtung mit dem Faktor 2 gestreckt.

    \item \textbf{Funktion $h(x)=-0.5x^2+1$} \\
    Wir erstellen eine Wertetabelle für $h(x)=-0.5x^2+1$ für dieselben $x$-Werte:
    \begin{itemize}
        \item $h(-3) = -0.5(-3)^2 + 1 = -0.5 \cdot 9 + 1 = -4.5 + 1 = -3.5$
        \item $h(-2) = -0.5(-2)^2 + 1 = -0.5 \cdot 4 + 1 = -2 + 1 = -1$
        \item $h(-1) = -0.5(-1)^2 + 1 = -0.5 \cdot 1 + 1 = -0.5 + 1 = 0.5$
        \item $h(0) = -0.5(0)^2 + 1 = 0 + 1 = 1$
        \item $h(1) = -0.5(1)^2 + 1 = -0.5 + 1 = 0.5$
        \item $h(2) = -0.5(2)^2 + 1 = -0.5 \cdot 4 + 1 = -2 + 1 = -1$
        \item $h(3) = -0.5(3)^2 + 1 = -0.5 \cdot 9 + 1 = -4.5 + 1 = -3.5$
    \end{itemize}
    \textbf{Wertetabelle für $h(x)=-0.5x^2+1$:}
    \begin{center}
    \begin{tabular}{r r}
    \toprule
    \multicolumn{1}{c}{$x$} & \multicolumn{1}{c}{$h(x)=-0.5x^2+1$} \\
    \midrule
    $-3$ & $-3.5$ \\
    $-2$ & $-1$ \\
    $-1$ & $0.5$ \\
    $0$ & $1$ \\
    $1$ & $0.5$ \\
    $2$ & $-1$ \\
    $3$ & $-3.5$ \\
    \bottomrule
    \end{tabular}
    \end{center}
    \textbf{Beobachtung:}
    Der Graph von $h(x)=-0.5x^2+1$ unterscheidet sich in mehreren Punkten von der Normalparabel:
    \begin{itemize}
        \item Er ist \textbf{nach unten geöffnet} (aufgrund des negativen Koeffizienten $-0.5$).
        \item Er ist \textbf{breiter} als die Normalparabel (da $|-0.5|=0.5 < 1$, d.h. in y-Richtung gestaucht).
        \item Sein Scheitelpunkt ist um 1 Einheit \textbf{nach oben verschoben} auf $(0|1)$ (aufgrund des Terms $+1$).
    \end{itemize}
\end{enumerate}

\subsection*{Gemeinsame Skizze der Graphen}
Die drei Funktionen $f(x)$, $g(x)$ und $h(x)$ werden nun in ein gemeinsames Koordinatensystem gezeichnet, um die Unterschiede und Gemeinsamkeiten direkt vergleichen zu können.

\begin{center}
\includegraphics[width=0.8\textwidth]{grafiken/parabeln_vergleich_graph.png}
% --- Beschreibung der Grafik ---
% Die Grafik zeigt ein kartesisches Koordinatensystem.
% Die x-Achse ist beschriftet und skaliert von etwa -4 bis 4.
% Die y-Achse ist beschriftet und skaliert, um alle Funktionswerte abzudecken, z.B. von -5 bis 20.
% Drei Parabeln sind eingezeichnet und idealerweise durch unterschiedliche Farben oder Linienstile gekennzeichnet:
% 1. f(x) = x^2 (Normalparabel): Scheitelpunkt (0|0), nach oben geöffnet, geht durch (1|1), (2|4), (3|9).
% 2. g(x) = 2x^2: Scheitelpunkt (0|0), nach oben geöffnet, schmaler als f(x), geht durch (1|2), (2|8), (3|18).
% 3. h(x) = -0.5x^2 + 1: Scheitelpunkt (0|1), nach unten geöffnet, breiter als f(x), geht durch (1|0.5), (2|-1), (3|-3.5).
% Die Beschriftung der Achsen und der Funktionen (oder ihrer Graphen) ist klar.
\captionof{figure}{Graphen der Funktionen $f(x)=x^2$, $g(x)=2x^2$ und $h(x)=-0.5x^2+1$ im Vergleich.}
\label{fig:parabeln_vergleich}
\end{center}

\end{loesungsumgebung}


\begin{aufgabenumgebung}{Übung zur quadratischen Ergänzung und Scheitelpunktbestimmung}
Bestimme für die folgenden Funktionen den Scheitelpunkt, indem du die Normalform durch quadratische Ergänzung in die Scheitelpunktform überführst. Gib auch an, ob es sich um ein Maximum oder Minimum handelt.
\begin{enumerate}
    \item $f(x) = x^2 - 6x + 5$
    \item $g(x) = x^2 + 8x + 10$
    \item $h(x) = 2x^2 + 4x - 1$ (Tipp: Erst den Faktor 2 ausklammern!)
    \item $k(x) = -x^2 - 2x + 3$ (Tipp: Erst den Faktor -1 ausklammern!)
\end{enumerate}
Vergleiche deine Ergebnisse für $x_S$ mit der Formel $x_S = -b/(2a)$.
\end{aufgabenumgebung}

\begin{loesungsumgebung}[loes:quadratische-ergaenzung-scheitelpunkt]{Übung zur quadratischen Ergänzung und Scheitelpunktbestimmung}
Die Scheitelpunktform einer quadratischen Funktion lautet $f(x) = a(x-x_S)^2 + y_S$, wobei $S(x_S|y_S)$ der Scheitelpunkt der Parabel ist.

\begin{enumerate}[label=(\alph*)]
    \item \textbf{Funktion $f(x) = x^2 - 6x + 5$} \\
    Hier sind $a=1$, $b=-6$, $c=5$.
    \textbf{Quadratische Ergänzung:}
    \begin{align*}
    f(x) &= (x^2 - 6x) + 5 \\
         &= \left(x^2 - 6x + \left(\frac{-6}{2}\right)^2 - \left(\frac{-6}{2}\right)^2\right) + 5 \\
         &= (x^2 - 6x + (-3)^2 - (-3)^2) + 5 \\
         &= (x^2 - 6x + 9) - 9 + 5 \\
         &= (x - 3)^2 - 4
    \end{align*}
    \textbf{Scheitelpunktform:} $f(x) = (x - 3)^2 - 4$. \\
    \textbf{Scheitelpunkt:} $S(3|-4)$. \\
    \textbf{Maximum/Minimum:} Da $a=1 > 0$ (Parabel nach oben geöffnet), handelt es sich um ein \textbf{Minimum}. \\
    \textbf{Vergleich mit Formel $x_S = -b/(2a)$:}
    $x_S = \frac{-(-6)}{2 \cdot 1} = \frac{6}{2} = 3$. Das Ergebnis stimmt mit der quadratischen Ergänzung überein.

    \item \textbf{Funktion $g(x) = x^2 + 8x + 10$} \\
    Hier sind $a=1$, $b=8$, $c=10$.
    \textbf{Quadratische Ergänzung:}
    \begin{align*}
    g(x) &= (x^2 + 8x) + 10 \\
         &= \left(x^2 + 8x + \left(\frac{8}{2}\right)^2 - \left(\frac{8}{2}\right)^2\right) + 10 \\
         &= (x^2 + 8x + 4^2 - 4^2) + 10 \\
         &= (x^2 + 8x + 16) - 16 + 10 \\
         &= (x + 4)^2 - 6
    \end{align*}
    \textbf{Scheitelpunktform:} $g(x) = (x + 4)^2 - 6$. \\
    \textbf{Scheitelpunkt:} $S(-4|-6)$. \\
    \textbf{Maximum/Minimum:} Da $a=1 > 0$ (Parabel nach oben geöffnet), handelt es sich um ein \textbf{Minimum}. \\
    \textbf{Vergleich mit Formel $x_S = -b/(2a)$:}
    $x_S = \frac{-(8)}{2 \cdot 1} = \frac{-8}{2} = -4$. Das Ergebnis stimmt mit der quadratischen Ergänzung überein.

    \item \textbf{Funktion $h(x) = 2x^2 + 4x - 1$} \\
    Hier sind $a=2$, $b=4$, $c=-1$.
    \textbf{Quadratische Ergänzung (Faktor 2 ausklammern):}
    \begin{align*}
    h(x) &= 2(x^2 + 2x) - 1 \\
         &= 2\left(x^2 + 2x + \left(\frac{2}{2}\right)^2 - \left(\frac{2}{2}\right)^2\right) - 1 \\
         &= 2(x^2 + 2x + 1^2 - 1^2) - 1 \\
         &= 2((x^2 + 2x + 1) - 1) - 1 \\
         &= 2(x + 1)^2 - 2 \cdot 1 - 1 \\
         &= 2(x + 1)^2 - 2 - 1 \\
         &= 2(x + 1)^2 - 3
    \end{align*}
    \textbf{Scheitelpunktform:} $h(x) = 2(x + 1)^2 - 3$. \\
    \textbf{Scheitelpunkt:} $S(-1|-3)$. \\
    \textbf{Maximum/Minimum:} Da $a=2 > 0$ (Parabel nach oben geöffnet), handelt es sich um ein \textbf{Minimum}. \\
    \textbf{Vergleich mit Formel $x_S = -b/(2a)$:}
    $x_S = \frac{-(4)}{2 \cdot 2} = \frac{-4}{4} = -1$. Das Ergebnis stimmt mit der quadratischen Ergänzung überein.

    \item \textbf{Funktion $k(x) = -x^2 - 2x + 3$} \\
    Hier sind $a=-1$, $b=-2$, $c=3$.
    \textbf{Quadratische Ergänzung (Faktor -1 ausklammern):}
    \begin{align*}
    k(x) &= -(x^2 + 2x) + 3 \\ % Achten Sie auf das Vorzeichen beim Ausklammern von -1
         &= -\left(x^2 + 2x + \left(\frac{2}{2}\right)^2 - \left(\frac{2}{2}\right)^2\right) + 3 \\
         &= -(x^2 + 2x + 1^2 - 1^2) + 3 \\
         &= -((x^2 + 2x + 1) - 1) + 3 \\
         &= -(x + 1)^2 - (-1) \cdot 1 + 3 \\
         &= -(x + 1)^2 + 1 + 3 \\
         &= -(x + 1)^2 + 4
    \end{align*}
    \textbf{Scheitelpunktform:} $k(x) = -(x + 1)^2 + 4$. \\
    \textbf{Scheitelpunkt:} $S(-1|4)$. \\
    \textbf{Maximum/Minimum:} Da $a=-1 < 0$ (Parabel nach unten geöffnet), handelt es sich um ein \textbf{Maximum}. \\
    \textbf{Vergleich mit Formel $x_S = -b/(2a)$:}
    $x_S = \frac{-(-2)}{2 \cdot (-1)} = \frac{2}{-2} = -1$. Das Ergebnis stimmt mit der quadratischen Ergänzung überein.
\end{enumerate}

\end{loesungsumgebung}


\begin{aufgabenumgebung}{Symmetrie prüfen und bestimmen}
\begin{enumerate}
    \item Untersuche die folgenden Funktionen rechnerisch auf Achsensymmetrie zur y-Achse, indem du $f(-x)$ berechnest und mit $f(x)$ vergleichst.
        \begin{itemize}
            \item $f_1(x) = -3x^2 + 5$
            \item $f_2(x) = x^2 - 2x + 1$
            \item $f_3(x) = 4x^2$
        \end{itemize}
    \item Bestimme für die Funktion $f(x) = 0.5x^2 - 3x + 1$ die Gleichung ihrer Symmetrieachse. Überprüfe dann für zwei verschiedene x-Werte, die symmetrisch zu dieser Achse liegen, ob ihre Funktionswerte gleich sind.
\end{enumerate}
\end{aufgabenumgebung}


\begin{loesungsumgebung}[loes:symmetrie-pruefen-bestimmen]{Symmetrie prüfen und bestimmen}

\begin{enumerate}[label=(\alph*)]
    \item \textbf{Untersuchung auf Achsensymmetrie zur y-Achse} \\
    Eine Funktion $f(x)$ ist achsensymmetrisch zur y-Achse, wenn für alle $x$ aus dem Definitionsbereich gilt: $f(-x) = f(x)$. Dies ist bei quadratischen Funktionen $f(x)=ax^2+bx+c$ genau dann der Fall, wenn der Koeffizient $b=0$ ist (also der lineare Term $bx$ fehlt).

    \begin{itemize}
        \item \textbf{Funktion $f_1(x) = -3x^2 + 5$}
        $$ f_1(-x) = -3(-x)^2 + 5 = -3x^2 + 5 $$
        Da $f_1(-x) = -3x^2 + 5 = f_1(x)$ ist, ist die Funktion $f_1(x)$ \textbf{achsensymmetrisch zur y-Achse}. (Hier ist $b=0$).

        \item \textbf{Funktion $f_2(x) = x^2 - 2x + 1$}
        $$ f_2(-x) = (-x)^2 - 2(-x) + 1 = x^2 + 2x + 1 $$
        Da $f_2(-x) = x^2 + 2x + 1 \neq x^2 - 2x + 1 = f_2(x)$ (für $x \neq 0$), ist die Funktion $f_2(x)$ \textbf{nicht achsensymmetrisch zur y-Achse}. (Hier ist $b=-2 \neq 0$).

        \item \textbf{Funktion $f_3(x) = 4x^2$}
        $$ f_3(-x) = 4(-x)^2 = 4x^2 $$
        Da $f_3(-x) = 4x^2 = f_3(x)$ ist, ist die Funktion $f_3(x)$ \textbf{achsensymmetrisch zur y-Achse}. (Hier ist $b=0$ und $c=0$).
    \end{itemize}

    \item \textbf{Bestimmung der Symmetrieachse für $f(x) = 0.5x^2 - 3x + 1$} \\
    Die Symmetrieachse einer Parabel mit der Gleichung $f(x) = ax^2+bx+c$ ist eine vertikale Gerade mit der Gleichung $x = x_S = \frac{-b}{2a}$.
    Für die gegebene Funktion $f(x) = 0.5x^2 - 3x + 1$ sind die Koeffizienten:
    $a = 0.5$, $b = -3$, $c = 1$.

    Berechnung von $x_S$:
    $$ x_S = \frac{-(-3)}{2 \cdot 0.5} = \frac{3}{1} = 3 $$
    Die Gleichung der \textbf{Symmetrieachse ist $x = 3$}.

    \textbf{Überprüfung der Symmetrie:}
    Wir wählen zwei $x$-Werte, die symmetrisch zur Achse $x=3$ liegen. Das bedeutet, sie haben den gleichen Abstand $d$ von der Achse $x=3$.
    Sei z.B. der Abstand $d=1$:
    \begin{itemize}
        \item $x_1 = x_S - d = 3 - 1 = 2$
        \item $x_2 = x_S + d = 3 + 1 = 4$
    \end{itemize}
    Nun berechnen wir die Funktionswerte an diesen Stellen:
    \begin{itemize}
        \item $f(x_1) = f(2) = 0.5(2)^2 - 3(2) + 1 = 0.5 \cdot 4 - 6 + 1 = 2 - 6 + 1 = -3$.
        \item $f(x_2) = f(4) = 0.5(4)^2 - 3(4) + 1 = 0.5 \cdot 16 - 12 + 1 = 8 - 12 + 1 = -3$.
    \end{itemize}
    Da $f(2) = -3$ und $f(4) = -3$ ist, sind die Funktionswerte gleich ($f(x_S-d) = f(x_S+d)$).

    Wählen wir einen anderen Abstand, z.B. $d=2$:
    \begin{itemize}
        \item $x_3 = x_S - d = 3 - 2 = 1$
        \item $x_4 = x_S + d = 3 + 2 = 5$
    \end{itemize}
    Funktionswerte:
    \begin{itemize}
        \item $f(x_3) = f(1) = 0.5(1)^2 - 3(1) + 1 = 0.5 \cdot 1 - 3 + 1 = 0.5 - 3 + 1 = -1.5$.
        \item $f(x_4) = f(5) = 0.5(5)^2 - 3(5) + 1 = 0.5 \cdot 25 - 15 + 1 = 12.5 - 15 + 1 = -1.5$.
    \end{itemize}
    Auch hier sind die Funktionswerte $f(1) = -1.5$ und $f(5) = -1.5$ gleich.
    Die Überprüfung bestätigt, dass $x=3$ die Symmetrieachse der Funktion $f(x)$ ist.
\end{enumerate}

\end{loesungsumgebung}



\begin{aufgabenumgebung}{Nullstellen finden – Übung und Vertiefung}
Berechne die Nullstellen der folgenden Funktionen. Entscheide selbst, welche Methode (Ausklammern, p-q-Formel, Mitternachtsformel, Substitution) am besten geeignet ist. Überprüfe bei quadratischen Gleichungen immer zuerst die Diskriminante, um die Anzahl der erwarteten reellen Nullstellen zu bestimmen.
\begin{enumerate}
    \item $f(x) = x^2 - x - 6$


    \item $g(x) = -2x^2 + 12x - 18$ 


    \item $h(x) = x^2 + 2x + 5$


    \item $k(x) = 3x^2 - 12$ 


    \item $m(x) = -0.5x^2 + 2x$


    \item \textbf{Polynom 3. Grades durch Ausklammern:}
        $p(x) = x^3 - 5x^2 + 6x$


    \item \textbf{Biquadratische Funktion:}
        $q(x) = x^4 - 10x^2 + 9$


    \item \textbf{Produkt aus Linearfaktoren (versteckt):}
        $r(x) = (x^2-4)(x^2+x-2)$


    \item \textbf{Funktion mit bekannter Nullstelle (für Knobler):}
        Gegeben ist die Funktion $s(x) = x^3 - 2x^2 - 5x + 6$. Es ist bekannt, dass $x_1=1$ eine Nullstelle ist. Finde die anderen Nullstellen.



    \item \textbf{Nullstellen und Parameter:}
        Für welche Werte des Parameters $k$ hat die Funktion $f_k(x) = x^2 - 2kx + (k+2)$ genau eine, zwei oder keine reelle(n) Nullstelle(n)?


\end{enumerate}
\end{aufgabenumgebung}


\begin{loesungsumgebung}[loes:nullstellen-uebung-vertiefung]{Nullstellen finden – Übung und Vertiefung}
Um die Nullstellen einer Funktion $f(x)$ zu finden, setzen wir $f(x)=0$ und lösen die entstehende Gleichung nach $x$.

\begin{enumerate}[label=(\alph*)]
    \item \textbf{Funktion $f(x) = x^2 - x - 6$} \\
    Dies ist eine quadratische Gleichung in der Normalform $ax^2+bx+c=0$ mit $a=1, b=-1, c=-6$. Wir verwenden die Mitternachtsformel (oder p-q-Formel, da $a=1$).
    \textbf{Diskriminante:} $D = b^2 - 4ac = (-1)^2 - 4(1)(-6) = 1 + 24 = 25$.
    Da $D > 0$, gibt es zwei verschiedene reelle Nullstellen.
    \textbf{Nullstellenberechnung (Mitternachtsformel):}
    $$ x_{1,2} = \frac{-b \pm \sqrt{D}}{2a} = \frac{-(-1) \pm \sqrt{25}}{2(1)} = \frac{1 \pm 5}{2} $$
    $$ x_1 = \frac{1+5}{2} = \frac{6}{2} = 3 $$
    $$ x_2 = \frac{1-5}{2} = \frac{-4}{2} = -2 $$
    Die Nullstellen sind $x_1=3$ und $x_2=-2$.

    \item \textbf{Funktion $g(x) = -2x^2 + 12x - 18$} \\
    Quadratische Gleichung mit $a=-2, b=12, c=-18$.
    \textbf{Diskriminante:} $D = b^2 - 4ac = (12)^2 - 4(-2)(-18) = 144 - 144 = 0$.
    \textbf{Anmerkung zur Diskriminante (Tipp):} Da $D=0$, gibt es genau eine (doppelte) reelle Nullstelle. Der Scheitelpunkt der Parabel liegt auf der x-Achse, d.h., der Graph berührt die x-Achse an dieser Stelle.
    \textbf{Nullstellenberechnung (Mitternachtsformel):}
    $$ x_0 = \frac{-b \pm \sqrt{D}}{2a} = \frac{-12 \pm \sqrt{0}}{2(-2)} = \frac{-12}{-4} = 3 $$
    Die Nullstelle ist $x_0=3$ (doppelte Nullstelle).

    \item \textbf{Funktion $h(x) = x^2 + 2x + 5$} \\
    Quadratische Gleichung mit $a=1, b=2, c=5$.
    \textbf{Diskriminante:} $D = b^2 - 4ac = (2)^2 - 4(1)(5) = 4 - 20 = -16$.
    \textbf{Anmerkung zur Diskriminante (Tipp):} Da $D < 0$, gibt es keine reellen Nullstellen. Der Graph der Parabel schneidet die x-Achse nicht. Da $a=1>0$ (Parabel nach oben geöffnet), verläuft der Graph vollständig oberhalb der x-Achse.
    Die Funktion $h(x)$ hat keine reellen Nullstellen.

    \item \textbf{Funktion $k(x) = 3x^2 - 12$} \\
    Setze $k(x)=0$: $3x^2 - 12 = 0$. Diese reinquadratische Gleichung lässt sich direkt auflösen.
    \textbf{Umformungsschritte:}
    $$
    \begin{array}{r c l c l}
    \umformung{3x^2 - 12}{0}{+}{12}
    \umformung{3x^2}{12}{\div}{3}
    \umformungend{x^2}{4}
    \end{array}
    $$
    Aus $x^2=4$ folgt durch Wurzelziehen: $x = \pm\sqrt{4}$.
    Die Nullstellen sind $x_1=2$ und $x_2=-2$.

    \item \textbf{Funktion $m(x) = -0.5x^2 + 2x$} \\
    Setze $m(x)=0$: $-0.5x^2 + 2x = 0$. Hier fehlt der konstante Term ($c=0$), daher ist Ausklammern geeignet.
    \textbf{Ausklammern:}
    $$ x(-0.5x + 2) = 0 $$
    Nach dem Satz vom Nullprodukt ist ein Produkt genau dann Null, wenn mindestens einer der Faktoren Null ist:
    \begin{enumerate}
        \item $x_1 = 0$
        \item $-0.5x + 2 = 0$
            $$
            \begin{array}{r c l c l}
            \umformung{-0.5x + 2}{0}{-}{2}
            \umformung{-0.5x}{-2}{\div}{(-0.5)}
            \umformungend{x_2}{4}
            \end{array}
            $$
    \end{enumerate}
    Die Nullstellen sind $x_1=0$ und $x_2=4$.

    \item \textbf{Polynom 3. Grades durch Ausklammern: $p(x) = x^3 - 5x^2 + 6x$} \\
    Setze $p(x)=0$: $x^3 - 5x^2 + 6x = 0$.
    \textbf{Strategie (Tipp):} Klammere den gemeinsamen Faktor $x$ aus.
    $$ x(x^2 - 5x + 6) = 0 $$
    Nach dem Satz vom Nullprodukt:
    \begin{enumerate}
        \item $x_1 = 0$
        \item $x^2 - 5x + 6 = 0$ (Quadratische Gleichung) \\
        Hier $a=1, b=-5, c=6$.
        Diskriminante: $D = (-5)^2 - 4(1)(6) = 25 - 24 = 1$.
        $x_{2,3} = \frac{-(-5) \pm \sqrt{1}}{2(1)} = \frac{5 \pm 1}{2}$.
        $x_2 = \frac{5+1}{2} = 3$.
        $x_3 = \frac{5-1}{2} = 2$.
    \end{enumerate}
    Die Nullstellen sind $x_1=0$, $x_2=3$ und $x_3=2$.

    \item \textbf{Biquadratische Funktion: $q(x) = x^4 - 10x^2 + 9$} \\
    Setze $q(x)=0$: $x^4 - 10x^2 + 9 = 0$.
    \textbf{Strategie (Tipp):} Substitution $z = x^2$. Dann ist $z^2 = x^4$.
    Die Gleichung wird zu einer quadratischen Gleichung in $z$:
    $$ z^2 - 10z + 9 = 0 $$
    Lösen für $z$ (mit $a=1, b=-10, c=9$):
    Diskriminante: $D_z = (-10)^2 - 4(1)(9) = 100 - 36 = 64$.
    $z_{1,2} = \frac{-(-10) \pm \sqrt{64}}{2(1)} = \frac{10 \pm 8}{2}$.
    $z_1 = \frac{10+8}{2} = 9$.
    $z_2 = \frac{10-8}{2} = 1$.
    \textbf{Rücksubstitution:}
    \begin{itemize}
        \item Für $z_1 = 9$: $x^2 = 9 \implies x = \pm\sqrt{9} \implies x_{1,2} = \pm 3$.
        \item Für $z_2 = 1$: $x^2 = 1 \implies x = \pm\sqrt{1} \implies x_{3,4} = \pm 1$.
    \end{itemize}
    (Beide $z$-Werte sind positiv, daher führen sie zu reellen $x$-Lösungen).
    Die Nullstellen sind $x_1=3$, $x_2=-3$, $x_3=1$ und $x_4=-1$.

    \item \textbf{Produkt aus Linearfaktoren (versteckt): $r(x) = (x^2-4)(x^2+x-2)$} \\
    Setze $r(x)=0$: $(x^2-4)(x^2+x-2) = 0$.
    \textbf{Strategie (Tipp):} Nach dem Satz vom Nullprodukt setzen wir jeden Faktor gleich Null.
    \begin{enumerate}
        \item Faktor 1: $x^2-4 = 0$
            $$ x^2 = 4 \implies x = \pm\sqrt{4} \implies x_{1,2} = \pm 2 $$
            (Also $x_1=2, x_2=-2$)
        \item Faktor 2: $x^2+x-2 = 0$ \\
            Hier $a=1, b=1, c=-2$.
            Diskriminante: $D = (1)^2 - 4(1)(-2) = 1 + 8 = 9$.
            $x_{3,4} = \frac{-1 \pm \sqrt{9}}{2(1)} = \frac{-1 \pm 3}{2}$.
            $x_3 = \frac{-1+3}{2} = 1$.
            $x_4 = \frac{-1-3}{2} = -2$.
    \end{enumerate}
    Die Menge der Nullstellen ist $\{2, -2, 1\}$. Die Nullstelle $x=-2$ tritt in beiden Faktoren auf (ist eine mehrfache Nullstelle des Gesamtpolynoms). Die verschiedenen Nullstellen sind $x_1=2$, $x_2=-2$ und $x_3=1$.

    \item \textbf{Funktion mit bekannter Nullstelle (für Knobler): $s(x) = x^3 - 2x^2 - 5x + 6$} \\
    Gegeben ist $x_1=1$ als Nullstelle.
    \textbf{Strategie (Tipp):} Koeffizientenvergleich. Wenn $x_1=1$ eine Nullstelle ist, dann ist $(x-1)$ ein Linearfaktor. Wir suchen $ax^2+bx+c$, sodass $s(x)=(x-1)(ax^2+bx+c)$.
    Aus dem Tipp ergibt sich $a=1, c=-6, b=-1$. Der quadratische Faktor ist also $(x^2-x-6)$.
    Wir müssen nun die Nullstellen dieses quadratischen Faktors finden:
    $$ x^2-x-6=0 $$
    Diese Gleichung wurde bereits in Teil (a) gelöst. Die Nullstellen sind $x=3$ und $x=-2$.
    Die Nullstellen der Funktion $s(x)$ sind also $x_1=1$, $x_2=3$ und $x_3=-2$.

    \item \textbf{Nullstellen und Parameter: $f_k(x) = x^2 - 2kx + (k+2)$} \\
    Wir untersuchen die Diskriminante $D = B^2-4AC$ der quadratischen Gleichung $f_k(x)=0$. Hier sind $A=1$, $B=-2k$, $C=(k+2)$.
    $$ D = (-2k)^2 - 4(1)(k+2) = 4k^2 - 4(k+2) = 4k^2 - 4k - 8 $$
    \begin{itemize}
        \item \textbf{Genau eine reelle Nullstelle:} $D=0$.
        $$ 4k^2 - 4k - 8 = 0 $$
        Teilen durch 4:
        $$ k^2 - k - 2 = 0 $$
        Lösen für $k$ (mit $a_k=1, b_k=-1, c_k=-2$):
        $D_k = (-1)^2 - 4(1)(-2) = 1+8=9$.
        $k_{1,2} = \frac{-(-1) \pm \sqrt{9}}{2(1)} = \frac{1 \pm 3}{2}$.
        $k_1 = \frac{1+3}{2} = 2$.
        $k_2 = \frac{1-3}{2} = -1$.
        Für $k=2$ oder $k=-1$ hat die Funktion $f_k(x)$ genau eine reelle Nullstelle.
        \item \textbf{Zwei verschiedene reelle Nullstellen:} $D>0$.
        $$ 4k^2 - 4k - 8 > 0 \implies k^2 - k - 2 > 0 $$
        Die Parabel $P(k) = k^2 - k - 2$ ist nach oben geöffnet und hat Nullstellen bei $k=-1$ und $k=2$. Sie ist positiv für Werte außerhalb ihrer Nullstellen.
        Also für $k < -1$ oder $k > 2$. In Intervallschreibweise: $k \in (-\infty, -1) \cup (2, \infty)$.
        \item \textbf{Keine reelle(n) Nullstelle(n):} $D<0$.
        $$ 4k^2 - 4k - 8 < 0 \implies k^2 - k - 2 < 0 $$
        Die Parabel $P(k) = k^2 - k - 2$ ist negativ für Werte zwischen ihren Nullstellen.
        Also für $-1 < k < 2$. In Intervallschreibweise: $k \in (-1, 2)$.
    \end{itemize}
\end{enumerate}

\end{loesungsumgebung}



\begin{aufgabenumgebung}
        \textbf{Anwendung des Satzes von Vieta (Kopfrechnen für Profis):}
        Versuche, die Nullstellen der folgenden quadratischen Funktionen (in Normalform $x^2+px+q=0$) durch 'scharfes Hinsehen' mit dem Satz von Vieta zu finden. Suche also zwei Zahlen $x_1, x_2$, für die gilt: $x_1+x_2 = -p$ und $x_1 \cdot x_2 = q$.
        \begin{enumerate}[label=(\alph*)]
            \item $f(x) = x^2 - 5x + 6$
            \item $g(x) = x^2 - 3x - 4$
            \item $h(x) = x^2 + 7x + 10$
        \end{enumerate}
\end{aufgabenumgebung}


\begin{loesungsumgebung}[loes:satz-von-vieta]{Anwendung des Satzes von Vieta (Kopfrechnen für Profis)}
Der Satz von Vieta bietet eine elegante Methode, um die Nullstellen einer quadratischen Funktion der Form $f(x) = x^2+px+q$ zu finden, sofern diese ganzzahlig (oder leicht zu erraten) sind. Man sucht zwei Zahlen $x_1$ und $x_2$, für die gilt:
\begin{itemize}
    \item $x_1 + x_2 = -p$
    \item $x_1 \cdot x_2 = q$
\end{itemize}
Diese Zahlen $x_1$ und $x_2$ sind dann die Nullstellen der Funktion.

\begin{enumerate}[label=(\alph*)]
    \item \textbf{Funktion $f(x) = x^2 - 5x + 6$} \\
    Hier ist $p=-5$ und $q=6$.
    Wir suchen zwei Zahlen $x_1, x_2$, für die gilt:
    \begin{itemize}
        \item $x_1 + x_2 = -(-5) = 5$
        \item $x_1 \cdot x_2 = 6$
    \end{itemize}
    Wir betrachten die ganzzahligen Teiler von $q=6$: $\pm 1, \pm 2, \pm 3, \pm 6$.
    Mögliche Paare für das Produkt $x_1 \cdot x_2 = 6$:
    \begin{itemize}
        \item $1 \cdot 6 = 6 \implies 1+6 = 7 \neq 5$
        \item $2 \cdot 3 = 6 \implies 2+3 = 5$ \textbf{(Treffer!)}
        \item $(-1) \cdot (-6) = 6 \implies (-1)+(-6) = -7 \neq 5$
        \item $(-2) \cdot (-3) = 6 \implies (-2)+(-3) = -5 \neq 5$
    \end{itemize}
    Die gesuchten Zahlen sind $2$ und $3$.
    Die Nullstellen sind also $x_1 = 2$ und $x_2 = 3$.

    \item \textbf{Funktion $g(x) = x^2 - 3x - 4$} \\
    Hier ist $p=-3$ und $q=-4$.
    Wir suchen zwei Zahlen $x_1, x_2$, für die gilt:
    \begin{itemize}
        \item $x_1 + x_2 = -(-3) = 3$
        \item $x_1 \cdot x_2 = -4$
    \end{itemize}
    Da das Produkt $q=-4$ negativ ist, müssen die beiden Zahlen unterschiedliche Vorzeichen haben.
    Mögliche Paare für das Produkt $x_1 \cdot x_2 = -4$:
    \begin{itemize}
        \item $1 \cdot (-4) = -4 \implies 1+(-4) = -3 \neq 3$
        \item $(-1) \cdot 4 = -4 \implies (-1)+4 = 3$ \textbf{(Treffer!)}
        \item $2 \cdot (-2) = -4 \implies 2+(-2) = 0 \neq 3$
    \end{itemize}
    Die gesuchten Zahlen sind $-1$ und $4$.
    Die Nullstellen sind also $x_1 = -1$ und $x_2 = 4$.

    \item \textbf{Funktion $h(x) = x^2 + 7x + 10$} \\
    Hier ist $p=7$ und $q=10$.
    Wir suchen zwei Zahlen $x_1, x_2$, für die gilt:
    \begin{itemize}
        \item $x_1 + x_2 = -(7) = -7$
        \item $x_1 \cdot x_2 = 10$
    \end{itemize}
    Da das Produkt $q=10$ positiv ist und die Summe $x_1+x_2=-7$ negativ ist, müssen beide Zahlen negativ sein.
    Mögliche Paare (negative Teiler von 10):
    \begin{itemize}
        \item $(-1) \cdot (-10) = 10 \implies (-1)+(-10) = -11 \neq -7$
        \item $(-2) \cdot (-5) = 10 \implies (-2)+(-5) = -7$ \textbf{(Treffer!)}
    \end{itemize}
    Die gesuchten Zahlen sind $-2$ und $-5$.
    Die Nullstellen sind also $x_1 = -2$ und $x_2 = -5$.
\end{enumerate}

\end{loesungsumgebung}


\begin{aufgabenumgebung}{Deine Kurvendiskussion – Quadratische Funktionen}
Führe eine vollständige Kurvendiskussion (gemäß der obigen Checkliste) für die folgenden quadratischen Funktionen durch und skizziere jeweils den Graphen.
\begin{enumerate}
    \item $f(x) = x^2 - 4x + 3$
    \item $g(x) = -2x^2 + 8x - 6$
    \item $h(x) = x^2 + 2x + 2$ (Was ist hier bei den Nullstellen und dem Scheitelpunkt besonders?)
    \item $k(x) = x^2 - 2x + 3$ (Untersuche, ob diese Funktion Nullstellen besitzt. Wie wirkt sich das auf den Graphen und den Wertebereich aus?)
    \item $m(x) = -x^2 - 3x + 4$ (Achte auf die Vorzeichen bei der Berechnung des Scheitelpunkts und der Nullstellen.)
\end{enumerate}
\end{aufgabenumgebung}



\begin{loesungsumgebung}[loes:kurvendiskussion-quadratisch]{Deine Kurvendiskussion – Quadratische Funktionen}
Wir führen für jede der gegebenen quadratischen Funktionen eine vollständige Kurvendiskussion gemäß der Checkliste durch.

\subsection*{1. Kurvendiskussion von $f(x) = x^2 - 4x + 3$}
\begin{enumerate}[label=(\alph*)]
    \item \textbf{Grundlegende Eigenschaften und Definitionsbereich $D_f$:}
    \begin{itemize}
        \item Koeffizienten: $a=1, b=-4, c=3$.
        \item Öffnungsrichtung: Da $a=1 > 0$, ist die Parabel nach \textbf{oben geöffnet}.
        \item Form: Da $|a|=|1|=1$, hat die Parabel die \textbf{Normalform} (Standardöffnung).
        \item Definitionsbereich: $D_f = \mathbb{R}$.
    \end{itemize}
    \item \textbf{Symmetrie:}
    Die Parabel ist achsensymmetrisch zur senkrechten Geraden $x = x_S$. Die genaue Lage von $x_S$ wird beim Scheitelpunkt berechnet. Da $b=-4 \neq 0$, ist die Parabel nicht achsensymmetrisch zur y-Achse.
    \item \textbf{Verhalten für $x \to \pm \infty$ (Globalverhalten):}
    Da $a=1 > 0$:
    \begin{itemize}
        \item $f(x) \to \infty$ für $x \to \infty$.
        \item $f(x) \to \infty$ für $x \to -\infty$.
    \end{itemize}
    \item \textbf{Schnittpunkt mit der y-Achse ($P_y$):}
    $f(0) = (0)^2 - 4(0) + 3 = 3$. Der Punkt ist $P_y(0|3)$.
    \item \textbf{Nullstellen (Schnittpunkte mit der x-Achse, $N_1, N_2$):}
    Setze $f(x)=0 \Rightarrow x^2 - 4x + 3 = 0$.
    Diskriminante: $D = b^2 - 4ac = (-4)^2 - 4(1)(3) = 16 - 12 = 4$.
    Da $D=4 > 0$, gibt es zwei verschiedene reelle Nullstellen.
    $x_{1,2} = \frac{-b \pm \sqrt{D}}{2a} = \frac{-(-4) \pm \sqrt{4}}{2(1)} = \frac{4 \pm 2}{2}$.
    $x_1 = \frac{4+2}{2} = 3$.
    $x_2 = \frac{4-2}{2} = 1$.
    Die Nullstellenpunkte sind $N_1(3|0)$ und $N_2(1|0)$.
    \item \textbf{Scheitelpunkt $S(x_S|y_S)$ (Extrempunkt):}
    $x_S = -\frac{b}{2a} = -\frac{-4}{2(1)} = \frac{4}{2} = 2$.
    $y_S = f(x_S) = f(2) = (2)^2 - 4(2) + 3 = 4 - 8 + 3 = -1$.
    Scheitelpunkt $S(2|-1)$. Da $a=1 > 0$, ist dies ein \textbf{Tiefpunkt (Minimum)}.
    Die Symmetrieachse ist die Gerade $x=2$.
    \item \textbf{Wertebereich $W_f$:}
    Da $a>0$ und der Tiefpunkt bei $y_S=-1$ liegt: $W_f = \{ y \in \mathbb{R} \,|\, y \ge -1 \} = [-1, \infty)$.
\end{enumerate}

\subsection*{2. Kurvendiskussion von $g(x) = -2x^2 + 8x - 6$}
\begin{enumerate}[label=(\alph*)]
    \item \textbf{Grundlegende Eigenschaften und Definitionsbereich $D_g$:}
    \begin{itemize}
        \item Koeffizienten: $a=-2, b=8, c=-6$.
        \item Öffnungsrichtung: Da $a=-2 < 0$, ist die Parabel nach \textbf{unten geöffnet}.
        \item Form: Da $|a|=|-2|=2 > 1$, ist die Parabel \textbf{schmaler} als die Normalparabel (gestreckt).
        \item Definitionsbereich: $D_g = \mathbb{R}$.
    \end{itemize}
    \item \textbf{Symmetrie:}
    Symmetrieachse $x = x_S$. Da $b=8 \neq 0$, nicht symmetrisch zur y-Achse.
    \item \textbf{Verhalten für $x \to \pm \infty$:}
    Da $a=-2 < 0$:
    \begin{itemize}
        \item $g(x) \to -\infty$ für $x \to \infty$.
        \item $g(x) \to -\infty$ für $x \to -\infty$.
    \end{itemize}
    \item \textbf{Schnittpunkt mit der y-Achse ($P_y$):}
    $g(0) = -2(0)^2 + 8(0) - 6 = -6$. Der Punkt ist $P_y(0|-6)$.
    \item \textbf{Nullstellen ($N_1, N_2$):}
    Setze $g(x)=0 \Rightarrow -2x^2 + 8x - 6 = 0$.
    Diskriminante: $D = b^2 - 4ac = (8)^2 - 4(-2)(-6) = 64 - 48 = 16$.
    Da $D=16 > 0$, gibt es zwei verschiedene reelle Nullstellen.
    $x_{1,2} = \frac{-b \pm \sqrt{D}}{2a} = \frac{-8 \pm \sqrt{16}}{2(-2)} = \frac{-8 \pm 4}{-4}$.
    $x_1 = \frac{-8+4}{-4} = \frac{-4}{-4} = 1$.
    $x_2 = \frac{-8-4}{-4} = \frac{-12}{-4} = 3$.
    Die Nullstellenpunkte sind $N_1(1|0)$ und $N_2(3|0)$.
    \item \textbf{Scheitelpunkt $S(x_S|y_S)$ (Extrempunkt):}
    $x_S = -\frac{b}{2a} = -\frac{8}{2(-2)} = -\frac{8}{-4} = 2$.
    $y_S = g(x_S) = g(2) = -2(2)^2 + 8(2) - 6 = -2(4) + 16 - 6 = -8 + 16 - 6 = 2$.
    Scheitelpunkt $S(2|2)$. Da $a=-2 < 0$, ist dies ein \textbf{Hochpunkt (Maximum)}.
    Die Symmetrieachse ist die Gerade $x=2$.
    \item \textbf{Wertebereich $W_g$:}
    Da $a<0$ und der Hochpunkt bei $y_S=2$ liegt: $W_g = \{ y \in \mathbb{R} \,|\, y \le 2 \} = (-\infty, 2]$.
\end{enumerate}

\subsection*{3. Kurvendiskussion von $h(x) = x^2 + 2x + 2$}
\begin{enumerate}[label=(\alph*)]
    \item \textbf{Grundlegende Eigenschaften und Definitionsbereich $D_h$:}
    \begin{itemize}
        \item Koeffizienten: $a=1, b=2, c=2$.
        \item Öffnungsrichtung: Da $a=1 > 0$, ist die Parabel nach \textbf{oben geöffnet}.
        \item Form: Da $|a|=|1|=1$, hat die Parabel die \textbf{Normalform}.
        \item Definitionsbereich: $D_h = \mathbb{R}$.
    \end{itemize}
    \item \textbf{Symmetrie:}
    Symmetrieachse $x = x_S$. Da $b=2 \neq 0$, nicht symmetrisch zur y-Achse.
    \item \textbf{Verhalten für $x \to \pm \infty$:}
    Da $a=1 > 0$: $h(x) \to \infty$ für $x \to \pm \infty$.
    \item \textbf{Schnittpunkt mit der y-Achse ($P_y$):}
    $h(0) = (0)^2 + 2(0) + 2 = 2$. Der Punkt ist $P_y(0|2)$.
    \item \textbf{Nullstellen:}
    Setze $h(x)=0 \Rightarrow x^2 + 2x + 2 = 0$.
    Diskriminante: $D = b^2 - 4ac = (2)^2 - 4(1)(2) = 4 - 8 = -4$.
    Da $D < 0$, gibt es \textbf{keine reellen Nullstellen}.
    \item \textbf{Scheitelpunkt $S(x_S|y_S)$ (Extrempunkt):}
    $x_S = -\frac{b}{2a} = -\frac{2}{2(1)} = -1$.
    $y_S = h(x_S) = h(-1) = (-1)^2 + 2(-1) + 2 = 1 - 2 + 2 = 1$.
    Scheitelpunkt $S(-1|1)$. Da $a=1 > 0$, ist dies ein \textbf{Tiefpunkt (Minimum)}.
    Die Symmetrieachse ist die Gerade $x=-1$.
    \textit{Anmerkung (Was ist hier bei den Nullstellen und dem Scheitelpunkt besonders?):} Die Funktion hat keine reellen Nullstellen. Der Scheitelpunkt $S(-1|1)$ liegt oberhalb der x-Achse ($y_S=1 > 0$), und da die Parabel nach oben geöffnet ist, schneidet sie die x-Achse nicht. Der y-Wert des Scheitelpunkts ist der kleinste Funktionswert.
    \item \textbf{Wertebereich $W_h$:}
    Da $a>0$ und der Tiefpunkt bei $y_S=1$ liegt: $W_h = [1, \infty)$.
\end{enumerate}

\subsection*{4. Kurvendiskussion von $k(x) = x^2 - 2x + 3$}
\begin{enumerate}[label=(\alph*)]
    \item \textbf{Grundlegende Eigenschaften und Definitionsbereich $D_k$:}
    \begin{itemize}
        \item Koeffizienten: $a=1, b=-2, c=3$.
        \item Öffnungsrichtung: Da $a=1 > 0$, ist die Parabel nach \textbf{oben geöffnet}.
        \item Form: Da $|a|=|1|=1$, hat die Parabel die \textbf{Normalform}.
        \item Definitionsbereich: $D_k = \mathbb{R}$.
    \end{itemize}
    \item \textbf{Symmetrie:}
    Symmetrieachse $x = x_S$. Da $b=-2 \neq 0$, nicht symmetrisch zur y-Achse.
    \item \textbf{Verhalten für $x \to \pm \infty$:}
    Da $a=1 > 0$: $k(x) \to \infty$ für $x \to \pm \infty$.
    \item \textbf{Schnittpunkt mit der y-Achse ($P_y$):}
    $k(0) = (0)^2 - 2(0) + 3 = 3$. Der Punkt ist $P_y(0|3)$.
    \item \textbf{Nullstellen:}
    Setze $k(x)=0 \Rightarrow x^2 - 2x + 3 = 0$.
    Diskriminante: $D = b^2 - 4ac = (-2)^2 - 4(1)(3) = 4 - 12 = -8$.
    Da $D < 0$, besitzt die Funktion \textbf{keine reellen Nullstellen}.
    \item \textbf{Scheitelpunkt $S(x_S|y_S)$ (Extrempunkt):}
    $x_S = -\frac{b}{2a} = -\frac{-2}{2(1)} = \frac{2}{2} = 1$.
    $y_S = k(x_S) = k(1) = (1)^2 - 2(1) + 3 = 1 - 2 + 3 = 2$.
    Scheitelpunkt $S(1|2)$. Da $a=1 > 0$, ist dies ein \textbf{Tiefpunkt (Minimum)}.
    Die Symmetrieachse ist die Gerade $x=1$.
    \item \textbf{Wertebereich $W_k$:}
    Da $a>0$ und der Tiefpunkt bei $y_S=2$ liegt: $W_k = [2, \infty)$.
    \textit{Anmerkung (Wie wirkt sich das [keine Nullstellen] auf den Graphen und den Wertebereich aus?):} Da die Parabel nach oben geöffnet ist und keine Nullstellen hat, muss ihr Scheitelpunkt (und somit die gesamte Parabel) oberhalb der x-Achse liegen. Der y-Wert des Scheitelpunkts ($y_S=2$) ist der kleinste Wert, den die Funktion annehmen kann. Der Wertebereich beginnt daher bei $y_S=2$ und geht bis unendlich.
\end{enumerate}

\subsection*{5. Kurvendiskussion von $m(x) = -x^2 - 3x + 4$}
\begin{enumerate}[label=(\alph*)]
    \item \textbf{Grundlegende Eigenschaften und Definitionsbereich $D_m$:}
    \begin{itemize}
        \item Koeffizienten: $a=-1, b=-3, c=4$.
        \item Öffnungsrichtung: Da $a=-1 < 0$, ist die Parabel nach \textbf{unten geöffnet}.
        \item Form: Da $|a|=|-1|=1$, hat die Parabel die \textbf{Normalform}.
        \item Definitionsbereich: $D_m = \mathbb{R}$.
    \end{itemize}
    \item \textbf{Symmetrie:}
    Symmetrieachse $x = x_S$. Da $b=-3 \neq 0$, nicht symmetrisch zur y-Achse.
    \item \textbf{Verhalten für $x \to \pm \infty$:}
    Da $a=-1 < 0$: $m(x) \to -\infty$ für $x \to \pm \infty$.
    \item \textbf{Schnittpunkt mit der y-Achse ($P_y$):}
    $m(0) = -(0)^2 - 3(0) + 4 = 4$. Der Punkt ist $P_y(0|4)$.
    \item \textbf{Nullstellen ($N_1, N_2$):}
    Setze $m(x)=0 \Rightarrow -x^2 - 3x + 4 = 0$.
    Diskriminante: $D = b^2 - 4ac = (-3)^2 - 4(-1)(4) = 9 + 16 = 25$.
    Da $D=25 > 0$, gibt es zwei verschiedene reelle Nullstellen.
    $x_{1,2} = \frac{-b \pm \sqrt{D}}{2a} = \frac{-(-3) \pm \sqrt{25}}{2(-1)} = \frac{3 \pm 5}{-2}$.
    $x_1 = \frac{3+5}{-2} = \frac{8}{-2} = -4$.
    $x_2 = \frac{3-5}{-2} = \frac{-2}{-2} = 1$.
    Die Nullstellenpunkte sind $N_1(-4|0)$ und $N_2(1|0)$.
    \item \textbf{Scheitelpunkt $S(x_S|y_S)$ (Extrempunkt):}
    $x_S = -\frac{b}{2a} = -\frac{-3}{2(-1)} = \frac{3}{-2} = -1.5$.
    $y_S = m(x_S) = m(-1.5) = -(-1.5)^2 - 3(-1.5) + 4 = -(2.25) - (-4.5) + 4 = -2.25 + 4.5 + 4 = 2.25 + 4 = 6.25$.
    Scheitelpunkt $S(-1.5|6.25)$. Da $a=-1 < 0$, ist dies ein \textbf{Hochpunkt (Maximum)}.
    Die Symmetrieachse ist die Gerade $x=-1.5$.
    \item \textbf{Wertebereich $W_m$:}
    Da $a<0$ und der Hochpunkt bei $y_S=6.25$ liegt: $W_m = (-\infty, 6.25]$.
\end{enumerate}

\subsection*{Gemeinsame Skizze der Graphen}
Die fünf Funktionen werden nun in ein gemeinsames Koordinatensystem gezeichnet.

\begin{center}
\includegraphics[width=0.8\textwidth]{grafiken/kurvendiskussion_kombi_graph.png}
% --- Beschreibung der Grafik ---
% Die Grafik zeigt ein kartesisches Koordinatensystem.
% Die x-Achse ist beschriftet und skaliert, um die relevanten x-Werte aller Parabeln darzustellen (z.B. von -5 bis 5).
% Die y-Achse ist beschriftet und skaliert, um die relevanten y-Werte aller Parabeln darzustellen (z.B. von -7 bis 8).
% Fünf Parabeln sind eingezeichnet und idealerweise durch unterschiedliche Farben oder Linienstile sowie Beschriftungen (f(x), g(x), h(x), k(x), m(x)) gekennzeichnet:
% 1. f(x) = x^2 - 4x + 3: S(2|-1), N(1|0), N(3|0), Py(0|3), nach oben offen.
% 2. g(x) = -2x^2 + 8x - 6: S(2|2), N(1|0), N(3|0), Py(0|-6), nach unten offen, schmaler.
% 3. h(x) = x^2 + 2x + 2: S(-1|1), keine Nullstellen, Py(0|2), nach oben offen.
% 4. k(x) = x^2 - 2x + 3: S(1|2), keine Nullstellen, Py(0|3), nach oben offen.
% 5. m(x) = -x^2 - 3x + 4: S(-1.5|6.25), N(-4|0), N(1|0), Py(0|4), nach unten offen.
% Wichtige Punkte (Scheitelpunkte, Achsenschnittpunkte) sollten für jede Parabel durch die Zeichnung plausibel erscheinen.
\captionof{figure}{Graphen der Funktionen $f(x)$, $g(x)$, $h(x)$, $k(x)$ und $m(x)$ im Vergleich.}
\label{fig:kurvendiskussion_kombi}
\end{center}

\end{loesungsumgebung}



\begin{aufgabenumgebung}[Brückenbogen]{Der Brückenbogen – eine klassische Anwendung}
Ein parabelförmiger Brückenbogen wird durch eine quadratische Funktion $f(x)$ beschrieben. Der Ursprung des Koordinatensystems $(0|0)$ liegt direkt unter der Mitte des Bogens auf Wasserniveau (x-Achse).
Der Bogen beginnt und endet auf Wasserniveau bei $x=-20\,$m und $x=20\,$m. In der Mitte (also bei $x=0$) ist der Bogen $8\,$m hoch.
\begin{enumerate}
    \item \textbf{Informationen sammeln:} Welche drei Punkte des Graphen der Funktion $f(x)$ sind dir damit bekannt? Notiere ihre Koordinaten.
    \item \textbf{Funktionsgleichung bestimmen:}
        \begin{itemize}
            \item Da der Scheitelpunkt offensichtlich auf der y-Achse liegt (genau in der Mitte bei $x=0$), welche Form hat die quadratische Funktion dann? (Tipp: Welcher Koeffizient ist Null?)
            \item Nutze den Punkt $(0|8)$, um den Koeffizienten $c$ zu bestimmen.
            \item Nutze einen der anderen Punkte (z.B. $(20|0)$), um den Koeffizienten $a$ zu bestimmen.
            \item Schreibe die vollständige Funktionsgleichung $f(x)$ des Brückenbogens auf.
        \end{itemize}
    \item \textbf{Anwendung der Funktion:} Wie hoch ist der Brückenbogen an der Stelle $x=10\,$m vom Mittelpunkt aus gemessen?
    \item \textbf{Skizze:} Erstelle eine Skizze des Brückenbogens im Koordinatensystem.
\end{enumerate}
\end{aufgabenumgebung}


\begin{loesungsumgebung}[loes:Brueckenbogen]{Der Brückenbogen – eine klassische Anwendung}

\begin{enumerate}[label=(\alph*)]
    \item \textbf{Informationen sammeln:}
    Aus der Problembeschreibung können wir die folgenden drei Punkte des Graphen der Funktion $f(x)$ entnehmen:
    \begin{itemize}
        \item Der Bogen beginnt auf Wasserniveau bei $x=-20\,$m: $N_1(-20|0)$.
        \item Der Bogen endet auf Wasserniveau bei $x=20\,$m: $N_2(20|0)$.
        \item In der Mitte (bei $x=0$) ist der Bogen $8\,$m hoch: Dies ist der Scheitelpunkt $S(0|8)$.
    \end{itemize}

    \item \textbf{Funktionsgleichung bestimmen:}
    \begin{itemize}
        \item Da der Scheitelpunkt bei $x=0$ auf der y-Achse liegt, ist die Parabel achsensymmetrisch zur y-Achse. Das bedeutet, dass der Koeffizient $b$ in der allgemeinen quadratischen Funktionsgleichung $f(x) = ax^2+bx+c$ Null ist ($b=0$). Die Funktion hat also die Form $f(x) = ax^2+c$.
        \item Der Scheitelpunkt einer solchen Parabel ist $S(0|c)$. Da wir wissen, dass der Scheitelpunkt $S(0|8)$ ist, können wir direkt ablesen, dass $c=8$.
        Die Funktion lautet somit bisher: $f(x) = ax^2+8$.
        \item Um den Koeffizienten $a$ zu bestimmen, setzen wir die Koordinaten eines der anderen bekannten Punkte (z.B. $N_2(20|0)$) in die Funktionsgleichung $f(x) = ax^2+8$ ein:
        $f(20) = 0 \Rightarrow a(20)^2+8 = 0$.
        $$a(400)+8 = 0$$
        Umformungsschritte:
        $$
        \begin{array}{r c l c l}
        \umformung{400a + 8}{0}{-}{8}
        \umformung{400a}{-8}{\div}{400}
        \umformungend{a}{-\frac{8}{400}}
        \end{array}
        $$
        $a = -\frac{8}{400} = -\frac{1}{50} = -0.02$.
        \item Die vollständige Funktionsgleichung des Brückenbogens lautet:
        $$f(x) = -0.02x^2 + 8$$
    \end{itemize}

    \item \textbf{Anwendung der Funktion:}
    Um die Höhe des Brückenbogens an der Stelle $x=10\,$m vom Mittelpunkt aus zu bestimmen, setzen wir $x=10$ in die Funktionsgleichung ein:
    $$f(10) = -0.02(10)^2 + 8$$
    $$f(10) = -0.02(100) + 8$$
    $$f(10) = -2 + 8$$
    $$f(10) = 6$$
    An der Stelle $x=10\,$m ist der Brückenbogen $6\,$m hoch.

    \item \textbf{Skizze:}
    Eine Skizze des Brückenbogens $f(x) = -0.02x^2 + 8$ würde Folgendes zeigen:
    \begin{itemize}
        \item Eine nach unten geöffnete Parabel (da $a=-0.02 < 0$).
        \item Der Scheitelpunkt (höchster Punkt) liegt bei $S(0|8)$.
        \item Die Nullstellen (Schnittpunkte mit der x-Achse, dem Wasserniveau) liegen bei $N_1(-20|0)$ und $N_2(20|0)$.
        \item Der Punkt $P(10|6)$ (und symmetrisch $P'(-10|6)$) liegt auf dem Bogen.
        \item Die x-Achse repräsentiert das Wasserniveau, die y-Achse die Höhe. Der Bogen verläuft nur für $x \in [-20, 20]$ im relevanten Bereich oberhalb oder auf dem Wasserniveau.
    \end{itemize}
    Die im Aufgabenstellungstext bereits eingefügte Abbildung gibt eine allgemeine Vorstellung eines solchen Brückenbogens. 

\end{enumerate}
\end{loesungsumgebung}

\begin{aufgabenumgebung}{Flugbahn eines Balls}
Ein Ball wird schräg nach oben geworfen. Seine Flughöhe $h(x)$ in Metern über dem Boden kann in Abhängigkeit von der horizontalen Entfernung $x$ (in Metern vom Abwurfpunkt) durch die Funktion
\[ h(x) = -0.1x^2 + 1.2x + 1.6 \]
beschrieben werden (für $x \ge 0$ und solange $h(x) \ge 0$).
\begin{enumerate}
    \item \textbf{Abwurfhöhe:} Aus welcher Höhe wurde der Ball abgeworfen? (Tipp: Das ist die Höhe an der Stelle $x=0$.)
    \item \textbf{Maximale Flughöhe:} Berechne die maximale Flughöhe des Balls. (Tipp: Das ist die y-Koordinate des Scheitelpunkts.)
    \item \textbf{Horizontale Entfernung bei max. Höhe:} Bei welcher horizontalen Entfernung vom Abwurfpunkt erreicht der Ball seine maximale Höhe? (Tipp: Das ist die x-Koordinate des Scheitelpunkts.)
    \item \textbf{Wurfweite:} Bei welcher horizontalen Entfernung trifft der Ball wieder auf dem Boden auf? (Tipp: Gesucht ist die positive Nullstelle der Funktion, da $h(x)=0$ bedeutet, dass der Ball auf dem Boden ist.)
    \item \textbf{Skizze:} Skizziere die Flugbahn des Balls (den Graphen von $h(x)$ im relevanten Bereich). Markiere Abwurfpunkt, höchsten Punkt und Auftreffpunkt.
\end{enumerate}
\end{aufgabenumgebung}

\begin{loesungsumgebung}[loes:flugbahn-ball]{Flugbahn eines Balls}
Die Flughöhe des Balls wird durch die quadratische Funktion $h(x) = -0.1x^2 + 1.2x + 1.6$ beschrieben, wobei $a=-0.1$, $b=1.2$ und $c=1.6$.

\begin{enumerate}[label=(\alph*)]
    \item \textbf{Abwurfhöhe:}
    Die Abwurfhöhe ist die Höhe an der Stelle $x=0$ (horizontale Entfernung vom Abwurfpunkt ist Null).
    $$ h(0) = -0.1(0)^2 + 1.2(0) + 1.6 = 0 + 0 + 1.6 = 1.6 $$
    Der Ball wurde aus einer Höhe von \textbf{1,6 Metern} abgeworfen.

    \item \textbf{Maximale Flughöhe:}
    Die maximale Flughöhe ist die y-Koordinate des Scheitelpunkts ($y_S$) der Parabel. Zuerst berechnen wir die x-Koordinate des Scheitelpunkts $x_S = -\frac{b}{2a}$.
    $$ x_S = -\frac{1.2}{2 \cdot (-0.1)} = -\frac{1.2}{-0.2} = \frac{1.2}{0.2} = 6 $$
    Nun setzen wir $x_S=6$ in die Funktion $h(x)$ ein, um $y_S$ zu erhalten:
    $$ y_S = h(6) = -0.1(6)^2 + 1.2(6) + 1.6 $$
    $$ y_S = -0.1(36) + 7.2 + 1.6 $$
    $$ y_S = -3.6 + 7.2 + 1.6 $$
    $$ y_S = 3.6 + 1.6 = 5.2 $$
    Die maximale Flughöhe des Balls beträgt \textbf{5,2 Meter}.

    \item \textbf{Horizontale Entfernung bei max. Höhe:}
    Dies ist die x-Koordinate des Scheitelpunkts, $x_S$, die wir in Teil (b) berechnet haben.
    Der Ball erreicht seine maximale Höhe bei einer horizontalen Entfernung von \textbf{6 Metern} vom Abwurfpunkt.

    \item \textbf{Wurfweite:}
    Die Wurfweite ist die horizontale Entfernung, bei der der Ball wieder auf dem Boden auftrifft, d.h. $h(x)=0$. Wir suchen die positive Nullstelle der Funktion.
    $$ -0.1x^2 + 1.2x + 1.6 = 0 $$
    Wir verwenden die Mitternachtsformel (ABC-Formel) $x_{1,2} = \frac{-b \pm \sqrt{b^2-4ac}}{2a}$.
    Die Diskriminante $D$ ist:
    $$ D = b^2 - 4ac = (1.2)^2 - 4(-0.1)(1.6) = 1.44 - (-0.64) = 1.44 + 0.64 = 2.08 $$
    Die Nullstellen sind:
    $$ x_{1,2} = \frac{-1.2 \pm \sqrt{2.08}}{2(-0.1)} = \frac{-1.2 \pm \sqrt{2.08}}{-0.2} $$
    Wir können $\sqrt{2.08}$ auch als $\sqrt{\frac{208}{100}} = \frac{\sqrt{16 \cdot 13}}{10} = \frac{4\sqrt{13}}{10} = \frac{2\sqrt{13}}{5}$ schreiben.
    $$ x_{1,2} = \frac{-1.2 \pm \frac{2\sqrt{13}}{5}}{-0.2} = \frac{-\frac{6}{5} \pm \frac{2\sqrt{13}}{5}}{-\frac{1}{5}} $$
    Multiplikation von Zähler und Nenner mit $-5$:
    $$ x_{1,2} = 6 \mp 2\sqrt{13} $$
    Die beiden Nullstellen sind:
    $$ x_1 = 6 - 2\sqrt{13} \approx 6 - 2 \cdot 3.60555 \approx 6 - 7.2111 \approx -1.2111 $$
    $$ x_2 = 6 + 2\sqrt{13} \approx 6 + 2 \cdot 3.60555 \approx 6 + 7.2111 \approx 13.2111 $$
    Da die horizontale Entfernung $x$ nicht negativ sein kann ($x \ge 0$), ist die Wurfweite die positive Nullstelle $x_2$.
    Der Ball trifft bei einer horizontalen Entfernung von $6 + 2\sqrt{13}$ Metern (ca. \textbf{13,21 Metern}) wieder auf dem Boden auf.
\end{enumerate}

\end{loesungsumgebung}


\begin{aufgabenumgebung}[labelA:QuadratischeAnw]{Weitere Anwendungs- und Verständnisaufgaben}
\begin{enumerate}
    \item \textbf{Rechteck mit maximaler Fläche:} Ein Bauer hat 40 Meter Zaun und möchte damit ein rechteckiges Feld an einer langen, geraden Mauer abgrenzen. Die Seite an der Mauer braucht keinen Zaun. Welche Abmessungen (Länge und Breite) sollte das Feld haben, damit seine Fläche maximal wird?
        \begin{itemize}
            \item Sei $x$ die Länge der Seite, die senkrecht zur Mauer steht. Drücke die andere Zaunseite (parallel zur Mauer) ebenfalls durch $x$ aus (bedenke die 40m Gesamtzaunlänge).
            \item Stelle eine Funktion $A(x)$ für die Fläche des Rechtecks in Abhängigkeit von $x$ auf.
            \item Bestimme den Scheitelpunkt dieser quadratischen Funktion $A(x)$. Was bedeuten die Koordinaten des Scheitelpunkts für das Problem?
            \item Gib die optimalen Abmessungen und die maximale Fläche an.
        \end{itemize}
    \item \textbf{Funktion aus Scheitelpunkt und Punkt finden:} Eine Parabel hat ihren Scheitelpunkt bei $S(2|-1)$ und geht durch den Punkt $P(4|7)$. Bestimme ihre Funktionsgleichung. (Tipp: Setze den Scheitelpunkt in die Scheitelpunktform $f(x)=a(x-x_S)^2+y_S$ ein. Setze dann die Koordinaten von $P$ ein, um $a$ zu berechnen.)
    \item \textbf{Funktion aus Nullstellen und Punkt finden:} Eine Parabel schneidet die x-Achse bei $x_1=-1$ und $x_2=3$. Außerdem geht sie durch den Punkt $P(1|-4)$. Bestimme ihre Funktionsgleichung. (Tipp: Nutze die faktorisierte Form einer quadratischen Funktion: $f(x)=a(x-x_1)(x-x_2)$, die wir im Merksatz \ref{merksatz:faktorisierte_form} kennengelernt haben. Setze die Nullstellen ein und dann den Punkt $P$, um $a$ zu berechnen.)
    \item \textbf{Parameter untersuchen:} Gegeben ist die Funktionenschar $f_k(x) = x^2 - 2kx + 4$ mit dem Parameter $k \in \mathbb{R}$.
        \begin{itemize}
            \item Bestimme die Koordinaten des Scheitelpunkts in Abhängigkeit von $k$.
            \item Für welche Werte von $k$ hat die Funktion genau eine Nullstelle? Keine Nullstellen? Zwei Nullstellen? (Tipp: Untersuche die Diskriminante $D$ in Abhängigkeit von $k$.)
            \item Auf welcher Kurve liegen alle Scheitelpunkte der Schar $f_k(x)$? (Tipp: Drücke $y_S$ durch $x_S$ aus. Diese Kurve nennt man auch \textit{Ortskurve} der Scheitelpunkte.)
        \end{itemize}
\end{enumerate}
\end{aufgabenumgebung}


\begin{loesungsumgebung}[loes:QuadratischeAnw]{Weitere Anwendungs- und Verständnisaufgaben}

\begin{enumerate}
    \item \textbf{Rechteck mit maximaler Fläche:}
    Ein Bauer hat 40 Meter Zaun für ein rechteckiges Feld an einer Mauer.
    \begin{itemize}
        \item \textbf{Seitenlängen ausdrücken:} \\
        Sei $x$ die Länge der Seite, die senkrecht zur Mauer steht (in Metern). Es gibt zwei solcher Seiten. Der Zaunverbrauch für diese beiden Seiten ist $2x$.
        Die Länge der Seite parallel zur Mauer ist dann der verbleibende Zaun: $L_{parallel} = 40 - 2x$ (in Metern).

        \item \textbf{Flächenfunktion $A(x)$ aufstellen:} \\
        Die Fläche $A$ des Rechtecks ist $A = \text{Breite} \cdot \text{Länge}$. Hier:
        $A(x) = x \cdot (40 - 2x) = 40x - 2x^2$.
        In Standardform: $A(x) = -2x^2 + 40x$.

        \item \textbf{Scheitelpunkt dieser quadratischen Funktion $A(x)$ bestimmen:} \\
        Die Funktion $A(x) = -2x^2 + 40x$ hat die Koeffizienten $a=-2$, $b=40$, $c=0$.
        Die x-Koordinate des Scheitelpunkts ist $x_S = -\frac{b}{2a}$:
        $$ x_S = -\frac{40}{2 \cdot (-2)} = -\frac{40}{-4} = 10 $$
        Die y-Koordinate des Scheitelpunkts (maximale Fläche) ist $y_S = A(x_S) = A(10)$:
        $$ A(10) = -2(10)^2 + 40(10) = -2(100) + 400 = -200 + 400 = 200 $$
        Der Scheitelpunkt ist $S(10|200)$.
        \textbf{Bedeutung der Koordinaten:} $x_S=10\,$m ist die Länge der senkrechten Seite, bei der die Fläche maximal wird. $y_S=200\,$m$^2$ ist diese maximale Fläche. Da $a=-2 < 0$, handelt es sich tatsächlich um ein Maximum.

        \item \textbf{Optimale Abmessungen und maximale Fläche angeben:}
        \begin{itemize}
            \item Länge der Seiten senkrecht zur Mauer: $x = 10\,$m.
            \item Länge der Seite parallel zur Mauer: $40 - 2x = 40 - 2(10) = 20\,$m.
            \item Maximale Fläche: $A_{max} = 200\,$m$^2$.
        \end{itemize}
        Die optimalen Abmessungen sind $10\,$m (Tiefe von der Mauer weg) und $20\,$m (Länge entlang der Mauer).
    \end{itemize}

    \item \textbf{Funktion aus Scheitelpunkt und Punkt finden:}
    Eine Parabel hat ihren Scheitelpunkt bei $S(2|-1)$ und geht durch den Punkt $P(4|7)$.
    Wir nutzen die Scheitelpunktform $f(x) = a(x-x_S)^2 + y_S$.
    Mit $x_S=2$ und $y_S=-1$:
    $$ f(x) = a(x-2)^2 - 1 $$
    Nun setzen wir den Punkt $P(4|7)$ ein ($x=4, f(x)=7$), um $a$ zu bestimmen:
    $$ 7 = a(4-2)^2 - 1 $$
    $$ 7 = a(2)^2 - 1 $$
    $$ 7 = 4a - 1 $$
    Umformungsschritte:
    $$
    \begin{array}{r c l c l}
    \umformung{4a - 1}{7}{+}{1}
    \umformung{4a}{8}{\div}{4}
    \umformungend{a}{2}
    \end{array}
    $$
    Also ist $a=2$. Die Funktionsgleichung lautet:
    $$ f(x) = 2(x-2)^2 - 1 $$
    (Ausmultipliziert: $f(x) = 2(x^2-4x+4)-1 = 2x^2-8x+8-1 = 2x^2-8x+7$).

    \item \textbf{Funktion aus Nullstellen und Punkt finden:}
    Eine Parabel schneidet die x-Achse bei $x_1=-1$ und $x_2=3$ und geht durch $P(1|-4)$.
    Wir nutzen die faktorisierte Form $f(x) = a(x-x_1)(x-x_2)$.
    Mit $x_1=-1$ und $x_2=3$:
    $$ f(x) = a(x-(-1))(x-3) = a(x+1)(x-3) $$
    Nun setzen wir den Punkt $P(1|-4)$ ein ($x=1, f(x)=-4$), um $a$ zu bestimmen:
    $$ -4 = a(1+1)(1-3) $$
    $$ -4 = a(2)(-2) $$
    $$ -4 = -4a $$
    Umformungsschritte:
    $$
    \begin{array}{r c l c l}
    \umformung{-4a}{-4}{\div}{(-4)}
    \umformungend{a}{1}
    \end{array}
    $$
    Also ist $a=1$. Die Funktionsgleichung lautet:
    $$ f(x) = 1(x+1)(x-3) = (x+1)(x-3) $$
    (Ausmultipliziert: $f(x) = x^2-3x+x-3 = x^2-2x-3$).

    \item \textbf{Parameter untersuchen:} Gegeben ist die Funktionenschar $f_k(x) = x^2 - 2kx + 4$.
    Koeffizienten: $A=1, B=-2k, C=4$ (Großbuchstaben zur Unterscheidung vom Parameter $k$).
    \begin{itemize}
        \item \textbf{Koordinaten des Scheitelpunkts in Abhängigkeit von $k$:} \\
        $x_S(k) = -\frac{B}{2A} = -\frac{-2k}{2(1)} = \frac{2k}{2} = k$. \\
        $y_S(k) = f_k(x_S(k)) = f_k(k) = (k)^2 - 2k(k) + 4 = k^2 - 2k^2 + 4 = -k^2 + 4$. \\
        Der Scheitelpunkt ist $S_k(k | -k^2+4)$.

        \item \textbf{Anzahl der Nullstellen in Abhängigkeit von $k$:} \\
        Wir untersuchen die Diskriminante $D = B^2 - 4AC$:
        $D = (-2k)^2 - 4(1)(4) = 4k^2 - 16$.
        \begin{itemize}
            \item \textit{Genau eine Nullstelle:} $D=0$.
            $4k^2 - 16 = 0$
            $$
            \begin{array}{r c l c l}
            \umformung{4k^2 - 16}{0}{+}{16}
            \umformung{4k^2}{16}{\div}{4}
            \umformungend{k^2}{4}
            \end{array}
            $$
            Aus $k^2=4$ folgt $k = \pm\sqrt{4}$, also $k_1=2$ und $k_2=-2$.
            Für $k=2$ oder $k=-2$ gibt es genau eine Nullstelle.
            \item \textit{Zwei verschiedene Nullstellen:} $D>0$.
            $4k^2 - 16 > 0 \implies 4k^2 > 16 \implies k^2 > 4$.
            Dies ist erfüllt für $k < -2$ oder $k > 2$. (Intervall: $(-\infty, -2) \cup (2, \infty)$).
            \item \textit{Keine Nullstellen:} $D<0$.
            $4k^2 - 16 < 0 \implies 4k^2 < 16 \implies k^2 < 4$.
            Dies ist erfüllt für $-2 < k < 2$. (Intervall: $(-2, 2)$).
        \end{itemize}

        \item \textbf{Ortskurve der Scheitelpunkte:} \\
        Wir haben die Koordinaten des Scheitelpunkts in Abhängigkeit von $k$:
        $x_S = k$
        $y_S = -k^2 + 4$
        Um $y_S$ durch $x_S$ auszudrücken, setzen wir $k=x_S$ in die Gleichung für $y_S$ ein:
        $$ y_S = -(x_S)^2 + 4 $$
        Die Ortskurve der Scheitelpunkte ist eine Parabel mit der Gleichung $y = -x^2 + 4$.
    \end{itemize}
\end{enumerate}

\end{loesungsumgebung}

\begin{aufgabenumgebung}{Checkliste: Parabeln verstehen – Mehr als nur Formeln}
Diese Aufgabe soll dir helfen, dein qualitatives Verständnis für quadratische Funktionen und ihre Graphen (Parabeln) zu vertiefen. Oft kann man schon viel über eine Parabel aussagen, ohne gleich jede Formel parat haben zu müssen! Nutze für deine Argumentationen auch immer kleine Skizzen.

\begin{enumerate}[label=\textbf{Teil \arabic*:}]
    \item \textbf{Argumentieren mit Nullstellen und dem Öffnungsfaktor $a$} \\
    Stell dir vor, du kennst von einer quadratischen Funktion $f(x)=ax^2+bx+c$ die Nullstellen (also die $x$-Werte, an denen $f(x)=0$ ist) und das Vorzeichen des Parameters $a$.

    \begin{enumerate}[label=(\alph*)]
        \item Eine Parabel hat die Nullstellen $x_1 = -1$ und $x_2 = 5$. Der Öffnungsfaktor ist $a > 0$.
        \begin{itemize}
            \item Skizziere diese Parabel grob. Ist sie nach oben oder nach unten geöffnet?
            \item In welchen $x$-Bereichen (Intervallen) verlaufen die Funktionswerte $f(x)$ oberhalb der x-Achse (sind also positiv)? In welchen Bereichen verlaufen sie unterhalb (sind also negativ)? Begründe mit deiner Skizze.
            \item Ohne den Scheitelpunkt genau zu berechnen: Was kannst du über die Lage seiner x-Koordinate $x_S$ sagen? (Tipp: Symmetrie!)
        \end{itemize}
        \item Betrachte nun eine andere Parabel mit denselben Nullstellen $x_1 = -1$ und $x_2 = 5$, aber diesmal mit einem Öffnungsfaktor $a < 0$.
        \begin{itemize}
            \item Skizziere auch diesen Fall. Wie ändern sich die Antworten auf die Fragen aus (a) bezüglich Öffnung und Vorzeichen der Funktionswerte?
        \end{itemize}
    \end{enumerate}

    \item \textbf{Argumentieren mit dem y-Achsenabschnitt $c$ und dem Öffnungsfaktor $a$ (Nullstellen sind unbekannt)} \\
    Stell dir vor, du kennst von einer quadratischen Funktion $f(x)=ax^2+bx+c$ nur den y-Achsenabschnitt $c$ (also den Wert $f(0)$) und das Vorzeichen des Öffnungsfaktors $a$.

    \begin{enumerate}[label=(\alph*)]
        \item Eine Parabel ist nach oben geöffnet ($a > 0$) und schneidet die y-Achse bei $c = -2$.
        \begin{itemize}
            \item Skizziere zwei \textit{mögliche} Verläufe für diese Parabel.
            \item Muss diese Parabel zwangsläufig Nullstellen besitzen? Begründe deine Antwort (denke an die Lage des Scheitelpunkts).
            \item Was kannst du über den y-Wert des Scheitelpunkts $y_S$ im Vergleich zum y-Achsenabschnitt $c$ sagen? Ist $y_S \leq c$, $y_S \geq c$ oder kann das variieren? Begründe.
        \end{itemize}
        \item Eine Parabel ist nach unten geöffnet ($a < 0$) und schneidet die y-Achse bei $c = 3$.
        \begin{itemize}
            \item Skizziere auch hier zwei \textit{mögliche} Verläufe.
            \item Kannst du mit Sicherheit sagen, ob diese Parabel Nullstellen hat? Oder ob sie keine hat? Oder ist beides möglich? Erkläre deine Überlegungen.
        \end{itemize}
        \item Eine Parabel ist nach oben geöffnet ($a > 0$) und ihr y-Achsenabschnitt $c$ ist ebenfalls positiv ($c > 0$, z.B. $c=4$).
        \begin{itemize}
            \item Beschreibe und skizziere die drei Möglichkeiten für die Anzahl der Nullstellen (keine, eine, zwei).
            \item Welche Bedingung muss für den Scheitelpunkt (insbesondere dessen y-Koordinate $y_S$) erfüllt sein, damit die Parabel in diesem Fall
            \begin{itemize}
                \item keine Nullstellen hat?
                \item genau eine Nullstelle hat?
                \item zwei Nullstellen hat?
            \end{itemize}
        \end{itemize}
    \end{enumerate}
\end{enumerate}
Versuche, deine Antworten immer auch mit kleinen, beschrifteten Skizzen zu untermauern!
\end{aufgabenumgebung}



\begin{loesungsumgebung}[loes:parabeln-verstehen-qualitativ-konsolidiert]{Checkliste: Parabeln verstehen – Mehr als nur Formeln}

\begin{enumerate}[label=\textbf{Teil \arabic*:}]
    \item \textbf{Argumentieren mit Nullstellen und dem Öffnungsfaktor $a$} \\
    Gegeben ist eine quadratische Funktion $f(x)=ax^2+bx+c$.
    Die Skizzen zu diesem Teil sind in Abbildung \ref{fig:parabeln_teil1_vergleich_a} zusammengefasst.

    \begin{enumerate}[label=(\alph*)]
        \item Eine Parabel hat die Nullstellen $x_1 = -1$ und $x_2 = 5$. Der Öffnungsfaktor ist $a > 0$.
        \begin{itemize}
            \item \textbf{Skizze und Öffnung:} \\
            Da $a > 0$, ist die Parabel \textbf{nach oben geöffnet}. Sie schneidet die x-Achse an den Stellen $x=-1$ und $x=5$. (Siehe Abb. \ref{fig:parabeln_teil1_vergleich_a}, Fall 1)

            \item \textbf{Vorzeichen der Funktionswerte $f(x)$:}
            \begin{itemize}
                \item $f(x)$ ist \textbf{positiv} ($f(x)>0$) für $x < -1$ oder $x > 5$. (Bereiche außerhalb der Nullstellen).
                \item $f(x)$ ist \textbf{negativ} ($f(x)<0$) für $-1 < x < 5$. (Bereich zwischen den Nullstellen).
            \end{itemize}
            \textit{Begründung:} Die nach oben geöffnete Parabel kommt von $+\infty$, fällt bis zu ihrem Scheitelpunkt (zwischen den Nullstellen, unterhalb der x-Achse) und steigt dann wieder zu $+\infty$.

            \item \textbf{Lage der x-Koordinate des Scheitelpunkts $x_S$:} \\
            Aufgrund der Symmetrie liegt $x_S$ genau in der Mitte zwischen den Nullstellen:
            $ x_S = \frac{-1 + 5}{2} = 2 $.
        \end{itemize}

        \item Eine andere Parabel mit denselben Nullstellen $x_1 = -1$ und $x_2 = 5$, aber mit $a < 0$.
        \begin{itemize}
            \item \textbf{Skizze und Änderungen:} \\
            Da $a < 0$, ist diese Parabel \textbf{nach unten geöffnet}. Sie schneidet die x-Achse ebenfalls bei $x=-1$ und $x=5$. (Siehe Abb. \ref{fig:parabeln_teil1_vergleich_a}, Fall 2)
            \textbf{Änderungen im Vergleich zu (a):}
            \begin{itemize}
                \item \textbf{Öffnung:} Die Parabel ist nach unten geöffnet.
                \item \textbf{Vorzeichen der Funktionswerte $f(x)$:} Das Verhalten kehrt sich um.
                    \begin{itemize}
                        \item $f(x)$ ist \textbf{positiv} für $-1 < x < 5$.
                        \item $f(x)$ ist \textbf{negativ} für $x < -1$ oder $x > 5$.
                    \end{itemize}
                \item Die x-Koordinate des Scheitelpunkts $x_S=2$ bleibt gleich. Der Scheitelpunkt ist nun ein Hochpunkt (oberhalb der x-Achse).
            \end{itemize}
        \end{itemize}
    \end{enumerate}
    \begin{center}
    \includegraphics[width=0.8\textwidth]{grafiken/parabeln_teil1_vergleich_a.png}
    % --- Beschreibung der zusammengefassten Skizze für Teil 1 ---
    % Die Abbildung zeigt ein einzelnes Koordinatensystem.
    % Darin sind ZWEI Parabeln skizziert:
    % 1. Fall 1 (aus 1a): Nach oben geöffnete Parabel (a > 0), schneidet die x-Achse bei x = -1 und x = 5. Scheitelpunkt bei x=2 unterhalb der x-Achse.
    % 2. Fall 2 (aus 1b): Nach unten geöffnete Parabel (a < 0), schneidet die x-Achse ebenfalls bei x = -1 und x = 5. Scheitelpunkt bei x=2 oberhalb der x-Achse.
    % Die Parabeln sollten klar unterscheidbar sein (z.B. Farbe, Linienstil).
    \captionof{figure}{Skizzenvergleich für Teil 1: Parabeln mit Nullstellen bei -1 und 5, für $a>0$ (Fall 1) und $a<0$ (Fall 2).}
    \label{fig:parabeln_teil1_vergleich_a}
    \end{center}

    \item \textbf{Argumentieren mit dem y-Achsenabschnitt $c$ und dem Öffnungsfaktor $a$ (Nullstellen sind unbekannt)} \\
    Gegeben ist eine quadratische Funktion $f(x)=ax^2+bx+c$. Der y-Achsenabschnitt ist $f(0)=c$.
    Die Skizzen zu diesem Teil sind in Abbildung \ref{fig:parabeln_teil2_szenarien_ac} zusammengefasst, die aus drei Panels (A, B, C) bestehen könnte.

    \begin{enumerate}[label=(\alph*)]
        \item Eine Parabel ist nach oben geöffnet ($a > 0$) und schneidet die y-Achse bei $c = -2$.
        \begin{itemize}
            \item \textbf{Skizze zweier möglicher Verläufe:} Siehe Abbildung \ref{fig:parabeln_teil2_szenarien_ac}, Panel A.
            \textit{Möglichkeit 1:} Scheitelpunkt auf der y-Achse bei $S(0|-2)$. Parabel schneidet die x-Achse symmetrisch.
            \textit{Möglichkeit 2:} Scheitelpunkt z.B. bei $S(1|-3)$ (rechts von der y-Achse und tiefer). Parabel schneidet y-Achse bei $(0|-2)$.

            \item \textbf{Muss diese Parabel zwangsläufig Nullstellen besitzen?} \\
            \textbf{Ja}, sie muss zwei verschiedene Nullstellen besitzen.
            \textit{Begründung:} Da $a>0$ (nach oben geöffnet) und $f(0)=c=-2$ (negativ) ist, liegt der y-Achsenabschnitt unterhalb der x-Achse. Der Scheitelpunkt $S(x_S|y_S)$ als tiefster Punkt muss $y_S \le c = -2$ erfüllen. Da $y_S$ somit negativ ist, muss die nach oben geöffnete Parabel die x-Achse zweimal schneiden.

            \item \textbf{Was kannst du über $y_S$ im Vergleich zu $c$ sagen?} \\
            Da $a>0$, ist der Scheitelpunkt $S(x_S|y_S)$ der tiefste Punkt. Der Punkt $(0|c)$ liegt auf der Parabel. Also $y_S \le c$. Gleichheit gilt, wenn $x_S=0$.
        \end{itemize}

        \item Eine Parabel ist nach unten geöffnet ($a < 0$) und schneidet die y-Achse bei $c = 3$.
        \begin{itemize}
            \item \textbf{Skizze zweier möglicher Verläufe:} Siehe Abbildung \ref{fig:parabeln_teil2_szenarien_ac}, Panel B.
            \textit{Möglichkeit 1:} Scheitelpunkt auf der y-Achse bei $S(0|3)$. Parabel schneidet die x-Achse symmetrisch.
            \textit{Möglichkeit 2:} Scheitelpunkt z.B. bei $S(1|4)$ (rechts von der y-Achse und höher). Parabel schneidet y-Achse bei $(0|3)$.

            \item \textbf{Kannst du mit Sicherheit sagen, ob diese Parabel Nullstellen hat?} \\
            \textbf{Ja}, sie muss zwei verschiedene Nullstellen besitzen.
            \textit{Begründung:} Da $a<0$ (nach unten geöffnet) und $f(0)=c=3$ (positiv) ist, liegt der y-Achsenabschnitt oberhalb der x-Achse. Der Scheitelpunkt $S(x_S|y_S)$ als höchster Punkt muss $y_S \ge c = 3$ erfüllen. Da $y_S$ somit positiv ist, muss die nach unten geöffnete Parabel die x-Achse zweimal schneiden.
        \end{itemize}

        \item Eine Parabel ist nach oben geöffnet ($a > 0$) und ihr y-Achsenabschnitt $c$ ist ebenfalls positiv ($c > 0$, z.B. $c=4$).
        \begin{itemize}
            \item \textbf{Beschreibe und skizziere die drei Möglichkeiten für die Anzahl der Nullstellen:} Siehe Abbildung \ref{fig:parabeln_teil2_szenarien_ac}, Panel C.
            \begin{enumerate}
                \item \textit{Keine Nullstellen:} Der Scheitelpunkt $S(x_S|y_S)$ liegt oberhalb der x-Achse ($y_S > 0$). Die gesamte Parabel verläuft oberhalb der x-Achse. (Panel C, Fall 1)
                \item \textit{Genau eine Nullstelle:} Der Scheitelpunkt $S(x_S|y_S)$ liegt genau auf der x-Achse ($y_S = 0$). Die Parabel berührt die x-Achse. (Panel C, Fall 2)
                \item \textit{Zwei Nullstellen:} Der Scheitelpunkt $S(x_S|y_S)$ liegt unterhalb der x-Achse ($y_S < 0$). Die Parabel schneidet die x-Achse zweimal. (Panel C, Fall 3)
            \end{enumerate}

            \item \textbf{Bedingung für $y_S$ für Anzahl der Nullstellen ($a>0, c>0$):}
            \begin{itemize}
                \item \textbf{keine Nullstellen hat?} $y_S > 0$.
                \item \textbf{genau eine Nullstelle hat?} $y_S = 0$.
                \item \textbf{zwei Nullstellen hat?} $y_S < 0$.
            \end{itemize}
        \end{itemize}
    \end{enumerate}
    \begin{center}
    \includegraphics[width=0.9\textwidth]{grafiken/parabeln_teil2_szenarien_ac.png}
    % --- Beschreibung der zusammengefassten Skizze für Teil 2 ---
    % Die Abbildung ist in drei Panels (A, B, C) unterteilt oder stellt die Szenarien klar getrennt dar.
    % Panel A (für 2a): a > 0, c = -2. Zeigt zwei mögliche nach oben geöffnete Parabeln, die die y-Achse bei -2 schneiden und jeweils zwei Nullstellen haben. Eine mit S(0|-2), eine mit S z.B. S(1|-3).
    % Panel B (für 2b): a < 0, c = 3. Zeigt zwei mögliche nach unten geöffnete Parabeln, die die y-Achse bei 3 schneiden und jeweils zwei Nullstellen haben. Eine mit S(0|3), eine mit S z.B. S(1|4).
    % Panel C (für 2c): a > 0, c > 0 (z.B. c=4). Zeigt DREI mögliche nach oben geöffnete Parabeln, die die y-Achse bei c>0 schneiden:
    %    C1: Keine Nullstellen (Scheitelpunkt y_S > 0).
    %    C2: Eine Nullstelle (Scheitelpunkt y_S = 0, berührt x-Achse).
    %    C3: Zwei Nullstellen (Scheitelpunkt y_S < 0).
    % Alle Parabeln sind entsprechend ihrer Parameter (Öffnung, y-Achsenabschnitt) gezeichnet.
    \captionof{figure}{Skizzensammlung für Teil 2: Szenarien basierend auf Öffnungsfaktor $a$ und y-Achsenabschnitt $c$.}
    \label{fig:parabeln_teil2_szenarien_ac}
    \end{center}

\end{enumerate}

\end{loesungsumgebung}


\begin{aufgabenumgebung}{Grenzwerte im Unendlichen}
Bestimme das Verhalten der folgenden Polynomfunktionen für $x \to \infty$ und $x \to -\infty$:
\begin{enumerate}
    \item $f(x) = -x^5 + 3x^2 - 7$
    \item $g(x) = 0.1x^6 - 1000x + 200$
    \item $h(x) = (2-x)(x+1)(x-3)$ (Tipp: Multipliziere die Klammern nicht vollständig aus. Überlege dir, was der Term mit der höchsten Potenz sein wird und welches Vorzeichen sein Koeffizient hat.)
\end{enumerate}
\end{aufgabenumgebung}

\begin{loesungsumgebung}[loes:grenzwerte-unendlich]{Grenzwerte im Unendlichen}
Das Verhalten einer Polynomfunktion für $x \to \pm\infty$ wird durch den Term mit der höchsten Potenz von $x$ (dem Leitterm) bestimmt.

\begin{enumerate}[label=(\alph*)]
    \item \textbf{Funktion $f(x) = -x^5 + 3x^2 - 7$} \\
    Der Leitterm ist $-x^5$.
    \begin{itemize}
        \item Für $x \to \infty$: \\
        $x^5 \to \infty$. Somit $-x^5 \to -\infty$.
        Also gilt: $\lim_{x \to \infty} f(x) = -\infty$.
        \item Für $x \to -\infty$: \\
        $x^5 \to -\infty$ (da der Exponent 5 ungerade ist). Somit $-x^5 \to -(-\infty) = \infty$.
        Also gilt: $\lim_{x \to -\infty} f(x) = \infty$.
    \end{itemize}

    \item \textbf{Funktion $g(x) = 0.1x^6 - 1000x + 200$} \\
    Der Leitterm ist $0.1x^6$. Der Koeffizient $0.1$ ist positiv.
    \begin{itemize}
        \item Für $x \to \infty$: \\
        $x^6 \to \infty$. Da $0.1 > 0$, ist $0.1x^6 \to \infty$.
        Also gilt: $\lim_{x \to \infty} g(x) = \infty$.
        \item Für $x \to -\infty$: \\
        $x^6 \to \infty$ (da der Exponent 6 gerade ist). Da $0.1 > 0$, ist $0.1x^6 \to \infty$.
        Also gilt: $\lim_{x \to -\infty} g(x) = \infty$.
    \end{itemize}

    \item \textbf{Funktion $h(x) = (2-x)(x+1)(x-3)$} \\
    Um den Leitterm zu bestimmen, ohne vollständig auszumultiplizieren, betrachten wir die Terme mit der höchsten Potenz von $x$ in jeder Klammer: $(-x)$, $(x)$ und $(x)$.
    Das Produkt dieser Terme ist $(-x) \cdot (x) \cdot (x) = -x^3$.
    Der Leitterm ist also $-x^3$.
    \begin{itemize}
        \item Für $x \to \infty$: \\
        $x^3 \to \infty$. Somit $-x^3 \to -\infty$.
        Also gilt: $\lim_{x \to \infty} h(x) = -\infty$.
        \item Für $x \to -\infty$: \\
        $x^3 \to -\infty$ (da der Exponent 3 ungerade ist). Somit $-x^3 \to -(-\infty) = \infty$.
        Also gilt: $\lim_{x \to -\infty} h(x) = \infty$.
    \end{itemize}
\end{enumerate}

\begin{merksatzumgebung}{Globalverhalten von Polynomen}
Das Verhalten eines Polynoms $P(x) = a_n x^n + a_{n-1}x^{n-1} + \dots + a_1 x + a_0$ für $x \to \pm\infty$ hängt nur vom Leitterm $a_n x^n$ ab:
\begin{itemize}
    \item Ist $n$ \textbf{gerade}:
    \begin{itemize}
        \item Wenn $a_n > 0$: $P(x) \to \infty$ für $x \to \infty$ und $P(x) \to \infty$ für $x \to -\infty$ (kommt von links oben, geht nach rechts oben).
        \item Wenn $a_n < 0$: $P(x) \to -\infty$ für $x \to \infty$ und $P(x) \to -\infty$ für $x \to -\infty$ (kommt von links unten, geht nach rechts unten).
    \end{itemize}
    \item Ist $n$ \textbf{ungerade}:
    \begin{itemize}
        \item Wenn $a_n > 0$: $P(x) \to \infty$ für $x \to \infty$ und $P(x) \to -\infty$ für $x \to -\infty$ (kommt von links unten, geht nach rechts oben).
        \item Wenn $a_n < 0$: $P(x) \to -\infty$ für $x \to \infty$ und $P(x) \to \infty$ für $x \to -\infty$ (kommt von links oben, geht nach rechts unten).
    \end{itemize}
\end{itemize}
\end{merksatzumgebung}

\end{loesungsumgebung}

\begin{aufgabenumgebung}{Polynom 3. Grades: Wertetabelle, Graph und Verhalten}
Betrachten wir die Polynomfunktion 3. Grades:
\[ f(x) = x^3 - 3x^2 + 4 \]
\begin{enumerate}[label=(\alph*)]
    \item \textbf{Wertetabelle erstellen:} Erstelle eine Wertetabelle für $f(x)$ für die ganzzahligen $x$-Werte von $-2$ bis $3$.
    \begin{center}
    \begin{tabular}{c||c|c|c|c|c|c}
    $x$ & -2 & -1 & 0 & 1 & 2 & 3 \\
    \hline
    $f(x)$ &    &    &   &   &   &   \\
    \end{tabular}
    \end{center}
    \item \textbf{Graph skizzieren:} Zeichne den Graphen der Funktion $f(x)$ in ein Koordinatensystem, indem du die Punkte aus deiner Wertetabelle verbindest. Wähle die Achsenskalierung so, dass alle berechneten Punkte gut sichtbar sind.
    \item \textbf{Besondere Punkte identifizieren:}
    \begin{itemize}
        \item Kannst du anhand deiner Wertetabelle und/oder deiner Skizze vermutliche \textbf{Nullstellen} der Funktion erkennen? Notiere die $x$-Werte.
        \item Fallen dir in deiner Skizze Bereiche auf, die lokale \textbf{Hochpunkte} (Berggipfel) oder \textbf{Tiefpunkte} (Talsohlen) sein könnten? Markiere diese im Graphen und notiere die ungefähren Koordinaten $(x|y)$ dieser Punkte, soweit du sie aus deiner Tabelle oder Zeichnung ablesen kannst.
    \end{itemize}
    \item \textbf{Verhalten im Unendlichen (Globalverhalten):}
    Du kennst bereits das Konzept des Grenzwerts (Limes). Überlege dir, wie sich die Funktion $f(x) = x^3 - 3x^2 + 4$ für sehr große positive und sehr große negative $x$-Werte verhält. Welcher Term in der Funktionsgleichung dominiert das Verhalten für $x \to \infty$ und $x \to -\infty$?
    \begin{itemize}
        \item Was erwartest du für $f(x)$, wenn $x \to \infty$ (d.h. $x$ wird beliebig groß positiv)?
        \item Was erwartest du für $f(x)$, wenn $x \to -\infty$ (d.h. $x$ wird beliebig groß negativ)?
    \end{itemize}
    Passt dieses Verhalten zu deiner Skizze?
\end{enumerate}
\end{aufgabenumgebung}


\begin{loesungsumgebung}[loes:polynom-3-grad-analyse]{Polynom 3. Grades: Wertetabelle, Graph und Verhalten}
Wir untersuchen die Polynomfunktion $f(x) = x^3 - 3x^2 + 4$.

\begin{enumerate}[label=(\alph*)]
    \item \textbf{Wertetabelle erstellen:}
    Wir berechnen die Funktionswerte $f(x)$ für die ganzzahligen $x$-Werte von $-2$ bis $3$:
    \begin{itemize}
        \item $f(-2) = (-2)^3 - 3(-2)^2 + 4 = -8 - 3(4) + 4 = -8 - 12 + 4 = -16$
        \item $f(-1) = (-1)^3 - 3(-1)^2 + 4 = -1 - 3(1) + 4 = -1 - 3 + 4 = 0$
        \item $f(0) = (0)^3 - 3(0)^2 + 4 = 0 - 0 + 4 = 4$
        \item $f(1) = (1)^3 - 3(1)^2 + 4 = 1 - 3(1) + 4 = 1 - 3 + 4 = 2$
        \item $f(2) = (2)^3 - 3(2)^2 + 4 = 8 - 3(4) + 4 = 8 - 12 + 4 = 0$
        \item $f(3) = (3)^3 - 3(3)^2 + 4 = 27 - 3(9) + 4 = 27 - 27 + 4 = 4$
    \end{itemize}
    Die ausgefüllte Wertetabelle sieht wie folgt aus:
    \begin{center}
    \begin{tabular}{c||c|c|c|c|c|c}
    $x$ & -2 & -1 & 0 & 1 & 2 & 3 \\
    \hline
    $f(x)$ & -16 & 0 & 4 & 2 & 0 & 4 \\
    \end{tabular}
    \end{center}

    \item \textbf{Graph skizzieren:}
    Der Graph wird gezeichnet, indem die Punkte $(-2|-16)$, $(-1|0)$, $(0|4)$, $(1|2)$, $(2|0)$ und $(3|4)$ in ein Koordinatensystem eingetragen und sinnvoll verbunden werden.
    \begin{center}
    \includegraphics[width=0.8\textwidth]{grafiken/polynom_f_graph.png}
    % --- Beschreibung der Grafik ---
    % Die Grafik zeigt ein kartesisches Koordinatensystem.
    % Die x-Achse ist beschriftet und skaliert, um den Bereich von ca. -2.5 bis 3.5 abzudecken.
    % Die y-Achse ist beschriftet und skaliert, um den Bereich von ca. -17 bis 5 abzudecken.
    % Der Graph der Funktion f(x) = x^3 - 3x^2 + 4 ist als glatte Kurve durch die berechneten Punkte gezeichnet.
    % Die Punkte (-2|-16), (-1|0), (0|4), (1|2), (2|0), (3|4) sind auf der Kurve erkennbar.
    % Die Kurve zeigt einen Anstieg bis zu einem lokalen Maximum, fällt dann zu einem lokalen Minimum und steigt danach wieder an.
    \captionof{figure}{Skizze des Graphen der Funktion $f(x) = x^3 - 3x^2 + 4$ basierend auf der Wertetabelle.}
    \label{fig:polynom_f_graph}
    \end{center}

    \item \textbf{Besondere Punkte identifizieren:}
    \begin{itemize}
        \item \textbf{Vermutliche Nullstellen:}
        Anhand der Wertetabelle sehen wir, dass $f(-1)=0$ und $f(2)=0$. Daher sind $x_1 = -1$ und $x_2 = 2$ vermutliche Nullstellen der Funktion. Die Skizze bestätigt, dass der Graph an diesen Stellen die x-Achse schneidet bzw. berührt.
        \item \textbf{Lokale Hochpunkte und Tiefpunkte:}
        Betrachtet man die Wertetabelle und die Skizze:
        \begin{itemize}
            \item Zwischen $x=-1$ ($f(-1)=0$) und $x=1$ ($f(1)=2$) steigt der Graph bis zu $f(0)=4$ an und fällt dann wieder. Daher liegt ein vermutlicher \textbf{lokaler Hochpunkt (Berggipfel)} bei oder nahe $(0|4)$.
            \item Zwischen $x=1$ ($f(1)=2$) und $x=3$ ($f(3)=4$) fällt der Graph bis zu $f(2)=0$ und steigt dann wieder an. Daher liegt ein vermutlicher \textbf{lokaler Tiefpunkt (Talsohle)} bei oder nahe $(2|0)$. Die Tatsache, dass $x=2$ auch eine Nullstelle ist, deutet darauf hin, dass der Graph die x-Achse hier berührt.
        \end{itemize}
        Die ungefähren Koordinaten sind: Lokaler Hochpunkt $\approx (0|4)$ und lokaler Tiefpunkt $\approx (2|0)$.
    \end{itemize}

    \item \textbf{Verhalten im Unendlichen (Globalverhalten):}
    Die Funktion ist $f(x) = x^3 - 3x^2 + 4$. Der Term, der das Verhalten für $x \to \pm\infty$ dominiert, ist der Term mit der höchsten Potenz von $x$, also $x^3$.
    \begin{itemize}
        \item \textbf{Was erwartest du für $f(x)$, wenn $x \to \infty$?} \\
        Wenn $x$ sehr groß positiv wird ($x \to \infty$), wird $x^3$ ebenfalls sehr groß positiv ($x^3 \to \infty$).
        Daher erwarten wir $\lim_{x \to \infty} f(x) = \infty$.
        \item \textbf{Was erwartest du für $f(x)$, wenn $x \to -\infty$?} \\
        Wenn $x$ sehr groß negativ wird ($x \to -\infty$), wird $x^3$ ebenfalls sehr groß negativ ($x^3 \to -\infty$, da der Exponent ungerade ist).
        Daher erwarten wir $\lim_{x \to -\infty} f(x) = -\infty$.
    \end{itemize}
    \textbf{Passt dieses Verhalten zu deiner Skizze?} \\
    Ja, dieses Verhalten passt zur Skizze. Die Skizze deutet an, dass der Graph von links unten kommt ($x \to -\infty, f(x) \to -\infty$), dann ansteigt, lokale Extrema bildet und schließlich nach rechts oben weiter verläuft ($x \to \infty, f(x) \to \infty$).
\end{enumerate}

\end{loesungsumgebung}
\section{Einführung in die Differentialrechnung}
\label{sec:differentialrechnung}

Stell dir vor, du fährst mit einem Fahrrad einen Hügel hinauf und dann wieder hinunter. Manchmal ist der Anstieg steil, manchmal flach, und bei der Abfahrt wirst du immer schneller. Die \textbf{Differentialrechnung} gibt uns die mathematischen Werkzeuge an die Hand, um genau solche \textit{Veränderungen} zu beschreiben und zu analysieren. Es geht darum, wie sich eine Größe (z.B. deine Höhe am Berg) ändert, wenn sich eine andere Größe (z.B. die zurückgelegte Strecke) ändert – und das nicht nur im Durchschnitt über eine längere Strecke, sondern in jedem einzelnen Augenblick!

\begin{tcolorbox}[colback=blue!5!white, colframe=blue!75!black, title=Was du in diesem Kapitel lernen wirst:]
Nachdem du dieses Kapitel durchgearbeitet hast, wirst du in der Lage sein:
\begin{itemize}[noitemsep, topsep=0pt, leftmargin=*, itemsep=2pt]
    \item das \textbf{Konzept der Ableitung} als Grenzwert des Differenzenquotienten (h-Methode), als lokale (momentane) Änderungsrate und als Steigung der Tangente an einen Funktionsgraphen zu verstehen und zu erklären.
    \item die grundlegenden \textbf{Ableitungsregeln} – Konstanten-, Potenz- (auch für negative und gebrochene Exponenten), Faktor- und Summenregel – sicher anzuwenden, um Polynomfunktionen und verwandte Terme zu differenzieren.
    \item die erweiterten Ableitungsregeln – \textbf{Produkt-, Quotienten- und Kettenregel} – zu verstehen und korrekt auf komplexere Funktionen anzuwenden.
    \item die \textbf{erste, zweite und gegebenenfalls höhere Ableitungen} einer Funktion zu berechnen und ihre jeweilige Bedeutung (z.B. Steigung, Krümmung, Änderungsrate der Steigung) zu interpretieren.
    \item das \textbf{Monotonieverhalten} (steigend, fallend) und das \textbf{Krümmungsverhalten} (links-/rechtsgekrümmt) von Funktionsgraphen mithilfe der ersten bzw. zweiten Ableitung systematisch zu analysieren.
    \item \textbf{lokale Extrempunkte} (Hoch- und Tiefpunkte), \textbf{Wendepunkte} (inklusive der Unterscheidung zu Sattelpunkten) von Funktionen rechnerisch zu bestimmen und ihre Art zu klassifizieren.
    \item Gleichungen von \textbf{Tangenten und Normalen} an einen Funktionsgraphen an gegebenen Stellen oder mit gegebenen Eigenschaften aufzustellen.
    \item das \textbf{Verhalten von Funktionen im Unendlichen} (Grenzwerte für $x \to \pm\infty$) und an \textbf{Definitionslücken} (insbesondere Polstellen bei Funktionen wie $f(x) = a/x^n$ und deren Transformationen) zu untersuchen.
    \item eine vollständige \textbf{Kurvendiskussion} für Polynomfunktionen (und einfache gebrochen-rationale Funktionen) durchzuführen, alle relevanten Eigenschaften zu ermitteln und den Graphen präzise zu skizzieren.
    \item die erlernten Methoden der Differentialrechnung zur Lösung von \textbf{Anwendungs- und Optimierungsproblemen} (z.B. aus Geometrie, Physik oder wirtschaftlichen Kontexten) sowie zur \textbf{Rekonstruktion von Funktionen} aus gegebenen Bedingungen einzusetzen.
\end{itemize}
Du wirst somit ein fundamentales und vielseitiges Werkzeug der Analysis meistern, um Funktionen tiefgreifend zu verstehen, zu beschreiben und ihre Eigenschaften in verschiedensten Kontexten nutzbar zu machen!
\end{tcolorbox}
\bigskip

Wir haben im Kapitel über lineare Funktionen bereits die \textbf{durchschnittliche Änderungsrate} kennengelernt, die der Steigung einer Sekante durch zwei Punkte des Graphen entspricht. Die Differentialrechnung geht nun einen entscheidenden Schritt weiter: Wir wollen die \textbf{momentane Änderungsrate} bestimmen, also die Steigung der Funktion in einem \textit{einzigen Punkt}. Diese momentane Änderungsrate wird durch die \textbf{Ableitung} der Funktion beschrieben.

\begin{infoboxumgebung}{Von der Sekante zur Tangente – Die Idee der Ableitung}
Erinnerst du dich an die durchschnittliche Änderungsrate $\frac{\Delta y}{\Delta x} = \frac{f(x_2)-f(x_1)}{x_2-x_1}$? Das war die Steigung der Geraden (Sekante) durch die Punkte $P_1(x_1|f(x_1))$ und $P_2(x_2|f(x_2))$ auf dem Graphen.

Um die Steigung in \textit{genau einem Punkt} $P_1(x_0|f(x_0))$ zu finden, lassen wir den zweiten Punkt $P_2(x|f(x))$ immer näher an $P_1$ heranwandern. Das Intervall $[x_0, x]$ wird also immer kleiner. Die Sekante durch $P_1$ und $P_2$ nähert sich dabei immer mehr einer speziellen Geraden an, die den Graphen im Punkt $P_1$ nur noch 'berührt' – diese Gerade nennt man die \textbf{Tangente} an den Graphen im Punkt $P_1$.
Die Steigung dieser Tangente ist dann die \textbf{momentane Änderungsrate} der Funktion an der Stelle $x_0$ und wird als \textbf{Ableitung} $f'(x_0)$ bezeichnet.

\begin{center}
    \includegraphics[scale=0.5]{grafiken/Differentialrechnung_Sekante_Tangente.png}
    \captionof{figure}{Von der Sekantensteigung zur Tangentensteigung}
    \label{fig:sek_zu_tan}
\end{center}
% Der Text geht hier direkt weiter

Dieser Prozess des 'Heranwanderns' wird mathematisch durch den \textbf{Grenzwert} (Limes) beschrieben. Die formale Definition der Ableitung lautet daher:
\[ f'(x_0) = \lim_{x \to x_0} \frac{f(x) - f(x_0)}{x - x_0}\, \text{ oder alternativ mit } h = x-x_0;  \, f'(x_0) = \lim_{h \to 0} \frac{f(x_0+h) - f(x_0)}{h} \]
Die zweite Form mit $h$ (die sogenannte \textbf{h-Methode}) ist oft praktischer für Berechnungen. Das Berechnen von Ableitungen über diesen Grenzwert kann aufwendig sein. Glücklicherweise gibt es für viele Funktionstypen feste Regeln, die uns das Ableiten erleichtern! Aber es ist wichtig, die Idee dahinter einmal verstanden zu haben.
\end{infoboxumgebung}

\begin{beispielumgebung}[Ableitung mit der h-Methode]{Ableitung von $f(x) = x^2 + 1$}
Wir wollen die Ableitung der Funktion $f(x) = x^2 + 1$ an einer beliebigen Stelle $x_0$ mit der h-Methode bestimmen.
Die Formel lautet: $f'(x_0) = \lim_{h \to 0} \frac{f(x_0+h) - f(x_0)}{h}$.

\textbf{Schritt 1: $f(x_0+h)$ und $f(x_0)$ bestimmen.}
$f(x_0) = x_0^2 + 1$
$f(x_0+h) = (x_0+h)^2 + 1 = (x_0^2 + 2x_0h + h^2) + 1 = x_0^2 + 2x_0h + h^2 + 1$.

\textbf{Schritt 2: Differenz $f(x_0+h) - f(x_0)$ bilden.}
$f(x_0+h) - f(x_0) = (x_0^2 + 2x_0h + h^2 + 1) - (x_0^2 + 1)$
$f(x_0+h) - f(x_0) = x_0^2 + 2x_0h + h^2 + 1 - x_0^2 - 1$
$f(x_0+h) - f(x_0) = 2x_0h + h^2$.

\textbf{Schritt 3: Differenzenquotient $\frac{f(x_0+h) - f(x_0)}{h}$ bilden und vereinfachen.}
$\frac{f(x_0+h) - f(x_0)}{h} = \frac{2x_0h + h^2}{h}$
Hier können wir $h$ aus dem Zähler ausklammern (solange $h \neq 0$, was für den Grenzwertprozess der Fall ist, da $h$ sich nur Null nähert):
$\frac{h(2x_0 + h)}{h} = 2x_0 + h$.

\textbf{Schritt 4: Grenzwert für $h \to 0$ bilden.}
$f'(x_0) = \lim_{h \to 0} (2x_0 + h)$.
Wenn $h$ gegen Null geht, wird der Term $2x_0+h$ zu $2x_0+0 = 2x_0$.
Also: $f'(x_0) = 2x_0$.

Da $x_0$ eine beliebige Stelle war, können wir auch schreiben: Die Ableitungsfunktion von $f(x)=x^2+1$ ist $f'(x)=2x$.
Das bedeutet, die Steigung der Tangente an die Parabel $y=x^2+1$ an der Stelle $x$ ist immer $2x$. An der Stelle $x=1$ ist die Steigung $2 \cdot 1 = 2$, an der Stelle $x=3$ ist sie $2 \cdot 3 = 6$.
\end{beispielumgebung}

\begin{aufgabenumgebung}{Experiment zur Sekantensteigung und h-Methode}
Gegeben ist die Funktion $f(x) = x^2$. Wir wollen die Steigung der Tangente im Punkt $P(1|1)$ untersuchen.
\begin{enumerate}
    \item \textbf{Experiment mit Sekantensteigungen:}
        Wähle einen festen Punkt $P_1(1|f(1))$. Berechne $f(1)$.
        Wähle nun verschiedene zweite Punkte $P_2(x|f(x))$, wobei $x$ immer näher an $1$ rückt. Berechne jeweils die Steigung der Sekante $m_{Sek} = \frac{f(x)-f(1)}{x-1}$.
        \begin{itemize}
            \item $x = 2$
            \item $x = 1.5$
            \item $x = 1.1$
            \item $x = 1.01$
            \item $x = 1.001$
        \end{itemize}
        Was beobachtest du bei den Werten für die Sekantensteigung? Welchem Wert scheinen sie sich anzunähern?
    \item \textbf{Exakte Berechnung mit der h-Methode:}
        Bestimme die Ableitung $f'(x_0)$ für $f(x)=x^2$ an der Stelle $x_0=1$ mit der h-Methode:
        $f'(1) = \lim_{h \to 0} \frac{f(1+h) - f(1)}{h}$.
        Vergleiche dein Ergebnis mit deiner Beobachtung aus Teil a).
    \item Bestimme die allgemeine Ableitungsfunktion $f'(x)$ für $f(x)=x^2$ mit der h-Methode.
\end{enumerate}
\end{aufgabenumgebung}

\begin{aufgabenumgebung}[h_methode_vertiefung]{Weitere Übungen zur h-Methode}
Bestimme die Ableitungsfunktion $f'(x)$ für die folgenden Funktionen mithilfe der h-Methode. Zeige alle algebraischen Umformungsschritte.
\begin{enumerate}
    \item $f(x) = 3x + 2$ 
        \begin{tippumgebung}{Lineare Funktion}
        Welche Steigung erwartest du bei einer linearen Funktion? Das Ergebnis der h-Methode sollte dies bestätigen.
        \end{tippumgebung}
    \item $f(x) = x^2 - x$
    \item $f(x) = c$ (wobei $c$ eine beliebige Konstante ist)
        \begin{tippumgebung}{Konstante Funktion}
        Wie sieht der Graph einer konstanten Funktion aus? Welche Steigung hat er überall?
        \end{tippumgebung}
    \item $f(x) = ax^2$ (wobei $a$ eine Konstante ist)
        \begin{tippumgebung}{Parameter $a$}
        Behandle $a$ während der Rechnung wie eine feste Zahl. Das Ergebnis wird $a$ enthalten.
        \end{tippumgebung}
\end{enumerate}
Diese Übungen helfen dir, das Prinzip der h-Methode zu verinnerlichen und algebraische Termumformungen zu trainieren.
\end{aufgabenumgebung}


\begin{erinnerungsboxumgebung}{Potenzgesetze – Fit im Umgang mit Exponenten}
Potenzen begegnen uns ständig, und ein sicherer Umgang mit den Potenzgesetzen ist Gold wert, nicht nur beim Ableiten! Hier eine Auffrischung der wichtigsten Regeln:

\paragraph{1. Grundlagen und Definitionen}
\begin{itemize}[nosep, leftmargin=2em]
    \item \textbf{Positive ganzzahlige Exponenten:} $a^n = \underbrace{a \cdot a \cdot \dots \cdot a}_{n \text{ Faktoren}}$ (für $n \in \mathbb{N}, n \ge 1$)
    \item \textbf{Exponent Null:} $a^0 = 1$ (für $a \neq 0$) \\ \textit{Beispiel:} $5^0 = 1$; $(-2)^0 = 1$
    \item \textbf{Exponent Eins:} $a^1 = a$ \\ \textit{Beispiel:} $x^1 = x$; $10^1 = 10$
\end{itemize}

\paragraph{2. Multiplikation und Division von Potenzen}
\begin{itemize}[nosep, leftmargin=2em]
    \item \textbf{Gleiche Basis, verschiedene Exponenten (Multiplikation):} $a^m \cdot a^n = a^{m+n}$ \\ \textit{Beispiel:} $x^3 \cdot x^2 = x^{3+2} = x^5$
    \item \textbf{Gleiche Basis, verschiedene Exponenten (Division):} $\frac{a^m}{a^n} = a^{m-n}$ (für $a \neq 0$) \\ \textit{Beispiel:} $\frac{x^7}{x^4} = x^{7-4} = x^3$
    \item \textbf{Gleicher Exponent, verschiedene Basen (Multiplikation):} $(a \cdot b)^n = a^n \cdot b^n$ \\ \textit{Beispiel:} $(2x)^3 = 2^3 \cdot x^3 = 8x^3$
    \item \textbf{Gleicher Exponent, verschiedene Basen (Division):} $\left(\frac{a}{b}\right)^n = \frac{a^n}{b^n}$ (für $b \neq 0$) \\ \textit{Beispiel:} $\left(\frac{x}{2}\right)^4 = \frac{x^4}{2^4} = \frac{x^4}{16}$
\end{itemize}

\paragraph{3. Potenzieren von Potenzen (Potenz einer Potenz)}
\begin{itemize}[nosep, leftmargin=2em]
    \item \textbf{Regel:} $(a^m)^n = a^{m \cdot n}$ \\ \textit{Beispiel:} $(x^3)^4 = x^{3 \cdot 4} = x^{12}$
\end{itemize}

\paragraph{4. Negative Exponenten}
Negative Exponenten bedeuten, dass die Potenz im Nenner eines Bruchs steht (oder umgekehrt).
\begin{itemize}[nosep, leftmargin=2em]
    \item \textbf{Definition:} $a^{-n} = \frac{1}{a^n}$ (für $a \neq 0$) \\ \textit{Beispiel:} $x^{-3} = \frac{1}{x^3}$; $2^{-4} = \frac{1}{2^4} = \frac{1}{16}$
    \item \textbf{Kehrwert eines Bruchs:} $\left(\frac{a}{b}\right)^{-n} = \left(\frac{b}{a}\right)^n = \frac{b^n}{a^n}$ (für $a,b \neq 0$) \\ \textit{Beispiel:} $\left(\frac{2}{5}\right)^{-2} = \left(\frac{5}{2}\right)^2 = \frac{5^2}{2^2} = \frac{25}{4}$
    \item \textbf{Im Nenner:} $\frac{1}{a^{-n}} = a^n$ (für $a \neq 0$) \\ \textit{Beispiel:} $\frac{1}{x^{-5}} = x^5$
\end{itemize}

\paragraph{5. Gebrochen rationale Exponenten (Wurzeln)}
Gebrochene Exponenten stehen für Wurzeln. Der Nenner des Exponenten ist der Wurzelexponent.
\begin{itemize}[nosep, leftmargin=2em]
    \item \textbf{Quadratwurzel:} $\sqrt{a} = a^{\frac{1}{2}}$ (für $a \ge 0$) \\ \textit{Beispiel:} $\sqrt{9} = 9^{\frac{1}{2}} = 3$
    \item \textbf{n-te Wurzel:} $\sqrt[n]{a} = a^{\frac{1}{n}}$ (für $a \ge 0$, wenn $n$ gerade ist) \\ \textit{Beispiel:} $\sqrt[3]{8} = 8^{\frac{1}{3}} = 2$; $\sqrt[4]{16} = 16^{\frac{1}{4}} = 2$
    \item \textbf{Allgemeiner gebrochener Exponent:} $a^{\frac{m}{n}} = \sqrt[n]{a^m} = (\sqrt[n]{a})^m$ \\ \textit{Beispiel:} $4^{\frac{3}{2}} = (\sqrt{4})^3 = 2^3 = 8$
    \textit{Beispiel mit negativem Bruch im Exponenten:} $8^{-\frac{2}{3}} = \frac{1}{8^{\frac{2}{3}}} = \frac{1}{(\sqrt[3]{8})^2} = \frac{1}{2^2} = \frac{1}{4}$
\end{itemize}

\vspace{0.5em}
\textbf{Kurze Übungen dazu:}
Vereinfache die folgenden Terme so weit wie möglich oder berechne den Wert:
\begin{multicols}{3}
\begin{enumerate}[label=(\alph*)]
    \item $x^5 \cdot x^{-3} = ?$
    \item $\frac{a^2}{a^6} = ?$
    \item $(y^4)^2 = ?$
    \item $(3b)^3 = ?$
    \item $\left(\frac{c}{4}\right)^2 = ?$
    \item $z^{-5} = ?$
    \item $\left(\frac{2}{x}\right)^{-3} = ?$
    \item $64^{\frac{1}{3}} = ?$
    \item $25^{\frac{3}{2}} = ?$
    \item $16^{-\frac{1}{4}} = ?$
    \item $\sqrt{x^8} = ?$
    \item $\frac{1}{y^{-2}} = ?$
    \item $(2x^2y^{-1})^3 = ?$
    \item $\frac{a^{\frac{1}{2}}}{a^{-\frac{1}{2}}} = ?$
    \item $(b^0 \cdot b^3)^{-1} = ?$
    \item $(x+2)^3 = ?$ \textit{(Tipp: Schreibe als $(x+2)^2 \cdot (x+2)$)}
\end{enumerate}
\end{multicols}
Diese Regeln sind sehr mächtig und helfen dir, auch komplizierte Terme zu zähmen!
\end{erinnerungsboxumgebung}


\begin{merksatzumgebung}[Ableitung – Das Wichtigste]{Was ist die Ableitung $f'(x)$?}
Die Ableitung $f'(x)$ einer Funktion $f(x)$ an einer Stelle $x$ (oft auch $x_0$ geschrieben) hat mehrere wichtige Bedeutungen:
\begin{itemize}
    \item Sie ist die \textbf{momentane Änderungsrate} der Funktion $f$ an der Stelle $x$.
    \item Sie ist die \textbf{Steigung der Tangente} an den Graphen von $f$ im Punkt $(x|f(x))$.
    \item Sie ist der \textbf{Grenzwert des Differenzenquotienten} (Steigung der Sekante), wenn das Intervall $\Delta x$ gegen Null geht.
\end{itemize}
Die Funktion $f'(x)$, die jeder Stelle $x$ ihre Ableitung zuordnet, heißt \textbf{Ableitungsfunktion} oder kurz Ableitung von $f$. Den Vorgang des Bestimmens der Ableitung nennt man \textbf{Differenzieren} oder \textbf{Ableiten}.
\end{merksatzumgebung}


\begin{warumwichtigumgebung}{Anwendungen der Ableitung}
Warum ist die Ableitung so ein mächtiges Werkzeug? Mit ihr können wir:
\begin{itemize}
    \item \textbf{Extremstellen} (Hoch- und Tiefpunkte) von Funktionen finden: Dort ist die Tangentensteigung (also die Ableitung) oft Null.
    \item Das \textbf{Monotonieverhalten} von Funktionen untersuchen: Wo steigt oder fällt ein Graph? (Positives Vorzeichen von $f'$ $\implies$ Graph steigt; negatives Vorzeichen von $f'$ $\implies$ Graph fällt).
    \item \textbf{Wendepunkte} finden: Punkte, an denen sich das Krümmungsverhalten eines Graphen ändert (z.B. von einer Rechts- in eine Linkskurve). Hier spielt die zweite Ableitung $f''(x)$ eine Rolle.
    \item \textbf{Optimierungsprobleme} lösen: Wo wird ein Wert maximal oder minimal (z.B. maximale Fläche, minimale Kosten)?
    \item \textbf{Physikalische Prozesse} beschreiben: Geschwindigkeit ist die Ableitung des Weges nach der Zeit, Beschleunigung ist die Ableitung der Geschwindigkeit nach der Zeit.
\end{itemize}
Die Differentialrechnung ist somit ein Schlüsselwerkzeug in vielen Naturwissenschaften, Ingenieurwissenschaften, Wirtschaftswissenschaften und natürlich in der Mathematik selbst.
\end{warumwichtigumgebung}

\begin{funfactbox}{Die Natur als schlaue Optimiererin}
Hast du dich jemals gefragt, warum Seifenblasen immer perfekt rund sind oder warum Bienen ihre Waben in exakten Sechsecken bauen? Es scheint, als ob die Natur eine eingebaute Mathematikerin ist, die ständig nach den besten, effizientesten Lösungen sucht!

\begin{itemize}
    \item \textbf{Seifenblasen:} Eine Seifenblase umschließt mit einer gegebenen Menge Seifenlösung ein maximales Volumen Luft. Die Kugelform ist dabei die geometrische Form, die bei gegebenem Volumen die kleinste Oberfläche hat. So minimiert die Seifenblase ihre Oberflächenspannung.
    \item \textbf{Bienenwaben:} Die sechseckige Struktur der Bienenwaben ist extrem stabil und materialsparend. Mit einer minimalen Menge Wachs wird maximaler Raum für Honig und Brut geschaffen.
    \item \textbf{Lichtstrahlen:} Licht nimmt immer den Weg der schnellsten Zeit (Fermatsches Prinzip). Daraus lassen sich die Gesetze der Reflexion und Brechung herleiten, die erklären, warum ein Strohhalm im Wasserglas geknickt aussieht.
\end{itemize}
Viele dieser 'optimalen' Formen und Wege in der Natur lassen sich mit den Werkzeugen der Differentialrechnung beschreiben und verstehen. Wenn wir zum Beispiel das Maximum oder Minimum einer Größe suchen (maximale Fläche, minimaler Materialverbrauch, schnellster Weg), suchen wir oft nach Stellen, an denen die Änderungsrate (also die Ableitung) Null wird. Es ist faszinierend, wie die Mathematik uns hilft, diese natürlichen 'Optimierungsstrategien' zu entschlüsseln!

\begin{center}
    \includegraphics[width=0.6\textwidth]{grafiken/Natur_Optimierung.png}
    % Beschreibung für die Grafik 'Natur_Optimierung.png':
    % Die Grafik könnte eine kleine Collage zeigen: eine perfekte Seifenblase,
    % die sechseckige Struktur einer Bienenwabe und vielleicht einen Lichtstrahl,
    % der an einer Wasseroberfläche gebrochen wird (Brechungsgesetz).
    % Alternativ: Ein einzelnes starkes Bild, z.B. eine Nahaufnahme einer Schneeflocke (auch oft optimal geformt)
    % oder eine Sonnenblume mit ihren spiralförmig angeordneten Kernen.
    \captionof{figure}{Optimale Formen und Wege finden sich überall in der Natur.}
    \label{fig:natur_optimierung_funfact} % Eigenes Label für diese Grafik
\end{center}
\end{funfactbox}

\subsection{Ableitungsregeln – Dein Werkzeugkasten zum Differenzieren}

Glücklicherweise müssen wir nicht für jede Funktion mühsam den Grenzwert des Differenzenquotienten berechnen. Es gibt eine Reihe von \textbf{Ableitungsregeln}, die uns das Leben sehr erleichtern. Diese Regeln sind wie ein Werkzeugkasten – für jede Art von Funktion gibt es das passende Werkzeug. Wir werden diese Regeln Schritt für Schritt einführen und üben.

\subsubsection{Die Basis: Ableiten von Konstanten und Potenzen von $x$}

Fangen wir mit den grundlegendsten Bausteinen an.

\begin{merksatzumgebung}[Konstantenregel]{Ableitung einer konstanten Funktion}
Die Ableitung einer konstanten Funktion $f(x)=c$ (wobei $c$ eine beliebige reelle Zahl ist) ist immer Null.
\[ (c)' = 0 \]
\textbf{Beispiele:}
\begin{itemize}
    \item $f(x) = 5 \implies f'(x) = 0$
    \item $g(x) = -17.3 \implies g'(x) = 0$
    \item $h(x) = \pi \implies h'(x) = 0$ (denn $\pi$ ist auch nur eine Zahl/Konstante)
\end{itemize}
\textit{Warum ist das so?} Eine konstante Funktion hat einen waagerechten Graphen. Die Steigung einer waagerechten Geraden ist überall Null. Eine Konstante ändert sich nicht, ihre momentane Änderungsrate ist also Null.
\end{merksatzumgebung}

Die nächste wichtige Regel betrifft Potenzen von $x$.

\begin{merksatzumgebung}[Potenzregel]{Ableitung von $f(x) = x^n$}
Für Funktionen der Form $f(x) = x^n$ (wobei $n$ eine beliebige reelle Zahl sein kann, also auch Brüche oder negative Zahlen) ist die Ableitung:
\[ (x^n)' = n \cdot x^{n-1} \]
\textbf{Die Regel in Worten:} 'Ziehe den alten Exponenten als Faktor nach vorne und verringere dann den Exponenten um 1.'

\textbf{Beispiele:}
\begin{itemize}
    \item $f(x) = x^3 \implies n=3 \implies f'(x) = 3 \cdot x^{3-1} = 3x^2$
    \item $g(x) = x^7 \implies n=7 \implies g'(x) = 7 \cdot x^{7-1} = 7x^6$
    \item $h(x) = x = x^1 \implies n=1 \implies h'(x) = 1 \cdot x^{1-1} = 1 \cdot x^0 = 1 \cdot 1 = 1$
    (Die Ableitung von $f(x)=x$ ist also $1$. Das ist logisch, denn $y=x$ ist die Ursprungsgerade mit der Steigung 1.)
    \item $k(x) = \sqrt{x}$. Hier müssen wir erst umschreiben: $\sqrt{x} = x^{\frac{1}{2}}$. Also $n=\frac{1}{2}$.
    $k'(x) = \frac{1}{2} \cdot x^{\frac{1}{2}-1} = \frac{1}{2} \cdot x^{-\frac{1}{2}}$.
    Das kann man auch wieder umschreiben: $x^{-\frac{1}{2}} = \frac{1}{x^{\frac{1}{2}}} = \frac{1}{\sqrt{x}}$.
    Also: $k'(x) = \frac{1}{2\sqrt{x}}$.
    \item $m(x) = \frac{1}{x^2}$. Umschreiben: $\frac{1}{x^2} = x^{-2}$. Also $n=-2$.
    $m'(x) = -2 \cdot x^{-2-1} = -2 \cdot x^{-3}$.
    Umschreiben: $-2x^{-3} = -\frac{2}{x^3}$.
\end{itemize}
\end{merksatzumgebung}

\begin{tippumgebung}{Umgang mit Wurzeln und Brüchen beim Ableiten}
Um die Potenzregel auch für Wurzeln und Brüche mit $x$ im Nenner anwenden zu können, ist es sehr hilfreich, diese zuerst in Potenzschreibweise umzuwandeln:
\begin{itemize}
    \item $\sqrt[n]{x^m} = x^{\frac{m}{n}}$ (speziell: $\sqrt{x} = x^{\frac{1}{2}}$)
    \item $\frac{1}{x^k} = x^{-k}$ (speziell: $\frac{1}{x} = x^{-1}$)
\end{itemize}
Nach dem Ableiten kannst du das Ergebnis oft wieder in die ursprüngliche Schreibweise zurückführen, wenn das übersichtlicher ist.
\end{tippumgebung}

\subsubsection{Kombinationen: Faktor- und Summenregel}

Selten bestehen Funktionen nur aus einem einzigen $x^n$-Term. Meistens haben wir Vielfache davon oder Summen und Differenzen.

\begin{merksatzumgebung}[Faktorregel]{Ableitung von $c \cdot g(x)$}
Ein konstanter Faktor $c$ (also eine Zahl), der mit einer Funktion $g(x)$ multipliziert wird, bleibt beim Ableiten einfach erhalten und wird mit der Ableitung von $g(x)$ multipliziert:
\[ (c \cdot g(x))' = c \cdot g'(x) \]
\textbf{Beispiele:}
\begin{itemize}
    \item $f(x) = 5x^3$. Hier ist $c=5$ und $g(x)=x^3$. Wir wissen $(x^3)'=3x^2$.
    $f'(x) = 5 \cdot (x^3)' = 5 \cdot 3x^2 = 15x^2$.
    \item $g(x) = -2x^4 \implies g'(x) = -2 \cdot (x^4)' = -2 \cdot 4x^3 = -8x^3$.
    \item $h(x) = \frac{3}{x} = 3 \cdot \frac{1}{x} = 3 \cdot x^{-1}$.
    $h'(x) = 3 \cdot (x^{-1})' = 3 \cdot (-1x^{-2}) = -3x^{-2} = -\frac{3}{x^2}$.
\end{itemize}
\end{merksatzumgebung}

Die letzte Grundregel, die wir für Polynome brauchen, ist die Summenregel.

\begin{merksatzumgebung}[Summen- und Differenzregel]{Ableitung von $g(x) \pm h(x)$}
Die Ableitung einer Summe (oder Differenz) von zwei (oder mehr) Funktionen ist einfach die Summe (oder Differenz) ihrer einzelnen Ableitungen:
\[ (g(x) + h(x))' = g'(x) + h'(x) \]
\[ (g(x) - h(x))' = g'(x) - h'(x) \]
\textbf{In Worten:} 'Jeder Summand wird für sich abgeleitet, und die Ergebnisse werden dann addiert bzw. subtrahiert.'
\end{merksatzumgebung}

Mit diesen vier Regeln (Konstanten-, Potenz-, Faktor- und Summenregel) können wir nun jede beliebige ganzrationale Funktion (Polynomfunktion) ableiten!

\begin{beispielumgebung}[Ableiten einer Polynomfunktion]{Anwendung aller Grundregeln}
Leite die Funktion $f(x) = 4x^5 - \frac{2}{3}x^3 + 5x - \sqrt{2}$ ab.

Wir leiten jeden Summanden einzeln ab und nutzen dabei die Faktor- und Potenzregel:
\begin{itemize}
    \item $(4x^5)' = 4 \cdot (x^5)' = 4 \cdot (5x^4) = 20x^4$
    \item $(-\frac{2}{3}x^3)' = -\frac{2}{3} \cdot (x^3)' = -\frac{2}{3} \cdot (3x^2) = -2x^2$
    \item $(5x)' = 5 \cdot (x)' = 5 \cdot 1 = 5$
    \item $(-\sqrt{2})'$: Da $\sqrt{2}$ eine Konstante ist, ist ihre Ableitung $0$.
\end{itemize}
Zusammengesetzt ergibt das die Ableitungsfunktion:
\[ f'(x) = 20x^4 - 2x^2 + 5 \]
\end{beispielumgebung}

\begin{aufgabenumgebung}[A:EinfacheAbleitungen]{Grundregeln üben}
Leite die folgenden Funktionen ab. Notiere dir, welche Regeln du benutzt.
\begin{enumerate}
    \item $f_1(x) = 6x^4 - 3x^3 + 0.5x^2 - x + 12$
    \item $f_2(x) = -2x^5 + \frac{1}{4}x^4 - x^2 + 9x$
    \item $f_3(x) = (x-2)(x+3)$ (Tipp: Erst ausmultiplizieren!)
    \item $f_4(x) = 4\sqrt{x} - \frac{3}{x^2} + 2x^{-1}$ (Tipp: Erst in Potenzschreibweise umwandeln!)
    \item $f_5(x) = ax^2 + bx + c$ (Hier sind $a,b,c$ Konstanten/Parameter. Was ist die Ableitung?)
\end{enumerate}
\end{aufgabenumgebung}


\begin{tippumgebung}{Nach welcher Variable wird abgeleitet?}
In der Mathematik ist es üblich, dass Funktionen mit $f(x)$ bezeichnet werden und $x$ die Variable ist, nach der abgeleitet wird. Alle anderen Buchstaben in der Funktion (wie $a, b, c, k, \pi, \dots$) werden dann als \textbf{Konstanten} behandelt, es sei denn, es ist ausdrücklich etwas anderes gesagt (z.B. bei Funktionen mit mehreren Variablen, was aber erst viel später kommt).
Wenn eine Funktion z.B. $f(t) = at^2 + v_0t$ heißt, ist $t$ die Variable und $a$ sowie $v_0$ sind Konstanten. Die Ableitung nach $t$ wäre dann $f'(t) = 2at + v_0$.
Achte also immer genau darauf, welcher Buchstabe die Variable ist, nach der du ableiten sollst! Oft wird das durch die Schreibweise $f(x)$, $g(t)$, $A(r)$ etc. schon angedeutet.
Bei der h-Methode $f'(x_0) = \lim_{h \to 0} \frac{f(x_0+h) - f(x_0)}{h}$ ist $x_0$ der feste Punkt, an dem wir die Steigung suchen (wird also wie eine Konstante behandelt), und $h$ ist die Variable, die gegen Null geht. Die Ableitung $f'(x_0)$ ist dann die momentane Änderungsrate von $f$ bezüglich ihrer ursprünglichen Variablen (z.B. $x$), ausgewertet an der Stelle $x_0$.
\end{tippumgebung}

\begin{aufgabenumgebung}{Variable und Konstanten unterscheiden}
Leite die folgenden Funktionen nach der jeweils angegebenen Variablen ab. Behandle alle anderen Buchstaben als Konstanten.
\begin{enumerate}
    \item $f(t) = 5t^2 - at + b$. Leite nach $t$ ab. ($f'(t) = ?$)
    \item $g(a) = 3a^2x - 2at + 5x^2$. Leite nach $a$ ab. ($g'(a) = ?$)
    \item $s(t) = \frac{1}{2}gt^2 + v_0 t + s_0$. Leite nach $t$ ab. ($s'(t) = ?$) (Dies ist die Formel für den Weg bei gleichmäßiger Beschleunigung $g$ mit Anfangsgeschwindigkeit $v_0$ und Anfangsweg $s_0$.)
    \item $U(r) = 2\pi r$. Leite nach $r$ ab. ($U'(r) = ?$) (Umfang eines Kreises)
    \item $A(x) = k \cdot x^3 - m \cdot x$. Leite nach $x$ ab. ($A'(x) = ?$)
\end{enumerate}
\end{aufgabenumgebung}

\begin{warumwichtigumgebung}{Ableitung von Polynomen}
Das Ableiten von Polynomfunktionen ist eine fundamentale Fähigkeit. Viele komplexere Funktionen werden in der höheren Mathematik durch Polynome angenähert (z.B. durch Taylorreihen – ein Ausblick für später!). Wenn du Polynome sicher ableiten kannst, hast du eine wichtige Grundlage für viele weitere Themen der Analysis geschaffen, insbesondere für Kurvendiskussionen, bei denen Nullstellen der Ableitung (Extremstellen) und Nullstellen der zweiten Ableitung (Wendestellen) gesucht werden.
\end{warumwichtigumgebung}




\subsection{Die erste Ableitung $f'(x)$ – Was sie uns verrät}
\label{subsec:erste_ableitung_bedeutung}

Wir wissen jetzt, dass die erste Ableitung $f'(x)$ die Steigung der Tangente an den Graphen von $f(x)$ an der Stelle $x$ angibt. Aber was können wir daraus über den Verlauf der Funktion $f(x)$ schließen? Eine ganze Menge!

\subsubsection{Monotonie – Wo steigt und fällt der Graph?}

Das \textbf{Monotonieverhalten} einer Funktion beschreibt, in welchen Intervallen der Graph der Funktion steigt, fällt oder konstant verläuft. Die erste Ableitung ist hier unser Detektiv!

\begin{merksatzumgebung}{Monotonie und erste Ableitung}
Sei $f$ eine in einem Intervall $I$ differenzierbare Funktion. Dann gilt für alle $x \in I$:
\begin{itemize}
    \item Wenn $f'(x) > 0$ für alle $x$ in $I$, dann ist $f(x)$ in diesem Intervall \textbf{streng monoton steigend}. (Die Tangenten haben eine positive Steigung $\implies$ es geht bergauf).
    \item Wenn $f'(x) < 0$ für alle $x$ in $I$, dann ist $f(x)$ in diesem Intervall \textbf{streng monoton fallend}. (Die Tangenten haben eine negative Steigung $\implies$ es geht bergab).
    \item Wenn $f'(x) = 0$ für alle $x$ in $I$, dann ist $f(x)$ in diesem Intervall \textbf{konstant}. (Die Tangenten sind waagerecht).
\end{itemize}
Um die Monotonieintervalle einer Funktion zu bestimmen, untersuchst du also das \textbf{Vorzeichen der ersten Ableitung $f'(x)$}.
\end{merksatzumgebung}

\begin{beispielumgebung}{Monotonieverhalten untersuchen}
Untersuche das Monotonieverhalten der Funktion $f(x) = x^3 - 3x^2 + 1$.

\textbf{Schritt 1: Erste Ableitung bilden.}
$f'(x) = (x^3)' - (3x^2)' + (1)' = 3x^2 - 6x + 0 = 3x^2 - 6x$.

\textbf{Schritt 2: Nullstellen der ersten Ableitung finden.}
Wir setzen $f'(x) = 0$, um die Stellen zu finden, an denen die Tangente waagerecht ist (mögliche Extremstellen, an denen sich das Monotonieverhalten ändern könnte):
$3x^2 - 6x = 0$
Wir klammern $3x$ aus (Sonderfall $c=0$ bei quadratischen Gleichungen):
$3x(x - 2) = 0$
Die Lösungen sind $3x=0 \implies x_1 = 0$ und $x-2=0 \implies x_2 = 2$.
An den Stellen $x=0$ und $x=2$ hat die Funktion waagerechte Tangenten. Diese Stellen teilen die x-Achse in Intervalle, in denen wir das Vorzeichen von $f'(x)$ untersuchen.

\textbf{Schritt 3: Vorzeichen von $f'(x)$ in den Intervallen untersuchen.}
Die Nullstellen $x_1=0$ und $x_2=2$ teilen die x-Achse in drei offene Intervalle:
\begin{itemize}
    \item Intervall 1: $(-\infty, 0)$ (links von 0)
    \item Intervall 2: $(0, 2)$ (zwischen 0 und 2)
    \item Intervall 3: $(2, \infty)$ (rechts von 2)
\end{itemize}
Wir wählen für jedes Intervall einen Testwert und setzen ihn in $f'(x) = 3x^2 - 6x = 3x(x-2)$ ein:
\begin{itemize}
    \item Intervall 1: Wähle $x=-1$.
    $f'(-1) = 3(-1)^2 - 6(-1) = 3(1) + 6 = 3+6=9$.
    Da $f'(-1) = 9 > 0$, ist $f(x)$ im Intervall $(-\infty, 0)$ streng monoton steigend.
    
    \item Intervall 2: Wähle $x=1$.
    $f'(1) = 3(1)^2 - 6(1) = 3 - 6 = -3$.
    Da $f'(1) = -3 < 0$, ist $f(x)$ im Intervall $(0, 2)$ streng monoton fallend.

    \item Intervall 3: Wähle $x=3$.
    $f'(3) = 3(3)^2 - 6(3) = 3(9) - 18 = 27 - 18 = 9$.
    Da $f'(3) = 9 > 0$, ist $f(x)$ im Intervall $(2, \infty)$ streng monoton steigend.
\end{itemize}

\textbf{Zusammenfassung des Monotonieverhaltens:}
Basierend auf der Untersuchung der Vorzeichen von $f'(x)$ in den offenen Intervallen und der Tatsache, dass $f(x)$ als Polynomfunktion überall stetig ist, können wir schließen:
\begin{itemize}
    \item $f(x)$ ist \textbf{streng monoton steigend} für $x \in (-\infty, 0]$.
    \item $f(x)$ ist \textbf{streng monoton fallend} für $x \in [0, 2]$.
    \item $f(x)$ ist \textbf{streng monoton steigend} für $x \in [2, \infty)$.
\end{itemize}
\textit{Anmerkung zur Präzision:} Wenn wir sagen, eine Funktion ist 'streng monoton steigend auf $[a,b]$', bedeutet das, dass für alle $x_1, x_2 \in [a,b]$ mit $x_1 < x_2$ gilt $f(x_1) < f(x_2)$. Dies ist hier erfüllt, da $f'(x)$ in den \textit{offenen} Intervallen $(-\infty,0)$, $(0,2)$ und $(2,\infty)$ jeweils ein eindeutiges Vorzeichen hat und die Funktion $f$ an den Stellen $x=0$ und $x=2$ stetig ist. An den Punkten $x=0$ und $x=2$ selbst ist die Steigung $f'(x)$ zwar Null, die Funktion setzt aber ihren Trend (bis zu diesem Punkt steigend/fallend und ab diesem Punkt fallend/steigend) fort. Manchmal kann die Angabe von Monotonieintervallen mit offenen oder geschlossenen Klammern zu Diskussionen führen. Wichtig ist das Verständnis, dass sich das Monotonieverhalten \textit{an den Stellen ändern kann, wo $f'(x)=0$ ist}.


\begin{center}
    \includegraphics[width=0.8\textwidth]{grafiken/Monotonie_Polynom3Grades.png}
    \captionof{figure}{Monotonieverhalten von $f(x)=x^3-3x^2+1$}
    \label{fig:monotonie_bsp}
\end{center}
% Der Text geht hier direkt weiter

\end{beispielumgebung}

\begin{aufgabenumgebung}{Monotonie untersuchen – Vielfältige Polynome}
Untersuche das Monotonieverhalten der folgenden Funktionen. Bestimme dazu die erste Ableitung, deren Nullstellen und das Vorzeichen der Ableitung in den entsprechenden Intervallen. Skizziere grob den Verlauf der ersten Ableitung und überlege, was das für die Steigung der Originalfunktion bedeutet.
\begin{enumerate}
    \item $f_1(x) = x^3 - 6x^2 + 5$ 
        \begin{tippumgebung}{Nullstellen von $f_1'(x)$}
        Die Ableitung ist eine quadratische Funktion. Ihre Nullstellen kannst du mit der p-q-Formel oder Mitternachtsformel finden.
        \end{tippumgebung}
    \item $f_2(x) = \frac{1}{4}x^4 - x^3 + x^2$
        \begin{tippumgebung}{Nullstellen von $f_2'(x)$}
        Die Ableitung ist ein Polynom 3. Grades. Versuche, $x$ oder $x^2$ auszuklammern, um die Nullstellen zu finden.
        \end{tippumgebung}
    \item $f_3(x) = x^3 + 6x - 1$
        \begin{tippumgebung}{Immer positiv?}
        Untersuche die Ableitung $f_3'(x)$. Kann dieser Term jemals Null oder negativ werden? Was bedeutet das für die Monotonie von $f_3(x)$?
        \end{tippumgebung}
    \item $f_4(x) = -x^3 + 3x^2 - 3x + 2$
        \begin{tippumgebung}{Immer negativ?}
        Untersuche die Ableitung $f_4'(x)$. Kann dieser Term jemals Null oder positiv werden? (Hinweis: Quadratische Ergänzung bei $f_4'(x)$ könnte helfen, das Vorzeichen zu bestimmen.)
        \end{tippumgebung}
    \item $f_5(x) = x^3 - 4x^2 + 4x - 1$
        \begin{tippumgebung}{Nicht-ganzzahlige Nullstellen von $f_5'(x)$}
        Die Nullstellen der Ableitung sind hier nicht unbedingt ganze Zahlen, aber mit der Lösungsformel für quadratische Gleichungen gut zu finden.
        \end{tippumgebung}
    \item \textbf{Anwendung im Kontext:} Die Temperatur $T$ in Grad Celsius während eines bestimmten Tagesabschnitts kann näherungsweise durch die Funktion $T(t) = -0.1t^3 + 1.2t^2 - 2.5t + 15$ für $0 \le t \le 8$ (Stunden nach Beobachtungsbeginn) beschrieben werden. 
        \begin{itemize}
            \item In welchen Zeiträumen steigt die Temperatur?
            \item In welchen Zeiträumen fällt die Temperatur?
            \item Gibt es Zeitpunkte, an denen sich die Temperatur kurzzeitig nicht ändert (waagerechte Tangente)? Fällt an diesen Zeitpunkten etwas auf bezüglich der Temperatur? Versuche den Graph anhand der Monotonie und des y-Achsenabschnittes schon mal grob zu skizzieren! 
        \end{itemize}
\end{enumerate}
\end{aufgabenumgebung}

\subsubsection{Extremstellen – Gipfel und Täler im Funktionsgraphen}

Stellen, an denen die Funktion von steigend zu fallend übergeht (oder umgekehrt), sind oft besonders interessant. Hier liegen lokale \textbf{Hochpunkte} (Maxima) oder \textbf{Tiefpunkte} (Minima), zusammenfassend \textbf{Extrempunkte} genannt.

\begin{merksatzumgebung}{Extremstellen finden mit der ersten Ableitung}
Um lokale Extremstellen einer differenzierbaren Funktion $f(x)$ zu finden, gehst du wie folgt vor:

\textbf{1. Notwendige Bedingung für Extremstellen:}
Wenn $f(x)$ an der Stelle $x_E$ eine lokale Extremstelle hat, dann muss die Tangente dort waagerecht sein, also gilt:
\[ f'(x_E) = 0 \]
Die Stellen $x_E$, an denen $f'(x_E)=0$ ist, nennt man \textbf{kritische Stellen} oder \textbf{potentielle Extremstellen}. Nicht jede kritische Stelle ist aber automatisch eine Extremstelle (es könnte auch ein Sattelpunkt sein, dazu später mehr).

\textbf{2. Hinreichende Bedingung für Extremstellen (Vorzeichenwechselkriterium von $f'$):}
Nachdem du eine kritische Stelle $x_E$ (also eine Nullstelle von $f'(x)$) gefunden hast, musst du prüfen, ob dort tatsächlich ein Extremum vorliegt. Das geht mit dem Vorzeichenwechsel (VZW) von $f'(x)$ an der Stelle $x_E$:
\begin{itemize}
    \item Wenn $f'(x)$ an der Stelle $x_E$ das Vorzeichen von \textbf{Plus nach Minus} wechselt (d.h. $f(x)$ ist links von $x_E$ steigend und rechts davon fallend), dann hat $f(x)$ an der Stelle $x_E$ einen \textbf{lokalen Hochpunkt (Maximum)}.
    \item Wenn $f'(x)$ an der Stelle $x_E$ das Vorzeichen von \textbf{Minus nach Plus} wechselt (d.h. $f(x)$ ist links von $x_E$ fallend und rechts davon steigend), dann hat $f(x)$ an der Stelle $x_E$ einen \textbf{lokalen Tiefpunkt (Minimum)}.
    \item Wenn $f'(x)$ an der Stelle $x_E$ \textbf{keinen Vorzeichenwechsel} hat (z.B. von Plus nach Plus), dann liegt bei $x_E$ \textbf{kein Extrempunkt}, sondern ein Sattelpunkt (Terrassenpunkt) vor.
\end{itemize}
Die y-Koordinate des Extrempunktes erhältst du, indem du $x_E$ in die Originalfunktion $f(x)$ einsetzt: $y_E = f(x_E)$.
\end{merksatzumgebung}

\begin{tippumgebung}{Extrempunkte und Monotonie – Das Bild im Kopf}
\begin{itemize}
    \item \textbf{Hochpunkt (Maximum $\land$):} Der Graph steigt erst an ($f'(x)>0$, wie ein Pfeil $\nearrow$), erreicht den Gipfel (dort ist $f'(x_E)=0$), und fällt dann wieder ab ($f'(x)<0$, wie ein Pfeil $\searrow$). Die Form ähnelt einem 'Dach' oder einem umgedrehten 'V'.
    \item \textbf{Tiefpunkt (Minimum $\lor$):} Der Graph fällt erst ab ($f'(x)<0$, Pfeil $\searrow$), erreicht den Talboden ($f'(x_E)=0$), und steigt dann wieder an ($f'(x)>0$, Pfeil $\nearrow$). Die Form ähnelt einem 'Tal' oder einem 'V'.
\end{itemize}
Diese Vorstellung hilft dir, den Vorzeichenwechsel der ersten Ableitung richtig zu interpretieren. Später werden wir sehen, dass die zweite Ableitung $f''(x)$ uns eine alternative (und oft schnellere) Möglichkeit bietet, die Art eines Extrempunktes zu bestimmen, indem sie uns sagt, ob der Graph an dieser Stelle 'lachend' (linksgekrümmt $\implies$ Tiefpunkt) oder 'traurig' (rechtsgekrümmt $\implies$ Hochpunkt) ist.
\end{tippumgebung}

\begin{beispielumgebung}{Extremstellen bestimmen}
Bestimme die lokalen Extremstellen der Funktion $f(x) = x^3 - 3x^2 + 1$ aus dem vorherigen Beispiel.

Wir hatten bereits berechnet:
$f'(x) = 3x^2 - 6x$.
Die Nullstellen von $f'(x)$ waren $x_1=0$ und $x_2=2$. Das sind unsere kritischen Stellen.

Nun untersuchen wir den Vorzeichenwechsel von $f'(x)$ an diesen Stellen:
\begin{itemize}
    \item \textbf{Untersuchung bei $x_1=0$:}
        Wir wissen:
        Links von $x=0$ (z.B. bei $x=-1$) war $f'(-1)=9 > 0$ (steigend).
        Rechts von $x=0$ (z.B. bei $x=1$) war $f'(1)=-3 < 0$ (fallend).
        Es gibt also einen Vorzeichenwechsel von $f'(x)$ von $+$ nach $-$ bei $x=0$.
        Somit liegt bei $x=0$ ein \textbf{lokaler Hochpunkt} vor.
        Die y-Koordinate ist $f(0) = 0^3 - 3(0)^2 + 1 = 1$.
        Der Hochpunkt ist $H(0|1)$.

    \item \textbf{Untersuchung bei $x_2=2$:}
        Wir wissen:
        Links von $x=2$ (z.B. bei $x=1$) war $f'(1)=-3 < 0$ (fallend).
        Rechts von $x=2$ (z.B. bei $x=3$) war $f'(3)=9 > 0$ (steigend).
        Es gibt also einen Vorzeichenwechsel von $f'(x)$ von $-$ nach $+$ bei $x=2$.
        Somit liegt bei $x=2$ ein \textbf{lokaler Tiefpunkt} vor.
        Die y-Koordinate ist $f(2) = 2^3 - 3(2)^2 + 1 = 8 - 3(4) + 1 = 8 - 12 + 1 = -3$.
        Der Tiefpunkt ist $T(2|-3)$.
\end{itemize}
Die Funktion hat also einen Hochpunkt bei $H(0|1)$ und einen Tiefpunkt bei $T(2|-3)$.
\begin{center}
    \includegraphics[width=0.8\textwidth]{grafiken/Differentialrechnung_Extrempunkte.png}
    \captionof{figure}{Hoch- und Tiefpunkt von $f(x)=x^3-3x^2+1$}
    \label{fig:extrempunkte_bsp}
\end{center}
\end{beispielumgebung}



% Dieser Block ersetzt den Inhalt der bestehenden 'aufgabenumgebung' 
% mit dem Titel 'Extrempunkte finden' im Kapitel 'Einführung in die Differentialrechnung',
% Unterabschnitt 'Extremstellen – Gipfel und Täler im Funktionsgraphen'.

\begin{aufgabenumgebung}{Extrempunkte finden – Vielfältige Herausforderungen}
Bestimme die lokalen Extrempunkte (Art und Koordinaten) der folgenden Funktionen. Verwende primär das \textbf{Vorzeichenwechselkriterium der ersten Ableitung ($f'(x)$)}. Du kannst deine Ergebnisse zusätzlich mit dem Kriterium der zweiten Ableitung ($f''(x)$) überprüfen, wo dies sinnvoll und einfach möglich ist.
\begin{enumerate}
    \item $f_1(x) = x^3 - 6x^2 + 9x + 1$
        \begin{tippumgebung}{Standardfall Polynom 3. Grades}
        Bestimme $f_1'(x)$. Finde die Nullstellen von $f_1'(x)$ (das sind deine Kandidaten für Extremstellen). Untersuche den Vorzeichenwechsel von $f_1'(x)$ an diesen Stellen, um die Art der Extrema (Hoch- oder Tiefpunkt) zu bestimmen. Zur Überprüfung kannst du auch $f_1''(x)$ bilden und die Kandidaten dort einsetzen. Vergiss nicht, auch die y-Koordinaten der Extrempunkte zu berechnen.
        \end{tippumgebung}

    \item $f_2(x) = \frac{1}{4}x^4 - x^3 - 2x^2 + 5$
        % \begin{tippumgebung}{Polynom 4. Grades}
        % Die erste Ableitung $f_2'(x)$ wird ein Polynom 3. Grades sein. Versuche, $x$ auszuklammern, um eine Nullstelle direkt zu finden. Die verbleibende quadratische Gleichung kannst du dann mit der p-q-Formel oder Mitternachtsformel lösen, um weitere Kandidaten für Extremstellen zu erhalten. Diese Funktion kann mehrere Extrempunkte haben.
        % \end{tippumgebung}

    \item $f_3(x) = x^4 - \frac{8}{3}x^3 + 2x^2$
        % \begin{tippumgebung}{Besonderer Fall bei $f_3''(x_E)=0$?}
        % Es kann vorkommen, dass für eine kritische Stelle $x_E$ (also $f_3'(x_E)=0$) auch die zweite Ableitung $f_3''(x_E)=0$ ist. In diesem Fall liefert das Kriterium mit der zweiten Ableitung keine Aussage über die Art des Extrempunkts. Dann musst du auf das Vorzeichenwechselkriterium der ersten Ableitung $f_3'(x)$ zurückgreifen, um zu entscheiden, ob ein Hochpunkt, Tiefpunkt oder vielleicht ein Sattelpunkt (kein Extremum) vorliegt.
        % \end{tippumgebung}

    \item \textbf{Schwer: Funktion mit Parameter $k$} \\
        Gegeben ist die Funktion $f_k(x) = x^3 - 3kx + 2$, wobei $k \in \mathbb{R}$ ein Parameter ist. Untersuche in Abhängigkeit von $k$:
        \begin{itemize}
            \item Für welche Werte von $k$ hat die Funktion $f_k(x)$ keine lokalen Extrempunkte?
            \item Für welche Werte von $k$ hat die Funktion $f_k(x)$ genau einen lokalen Extrempunkt? (Überlege, was das für die Ableitung bedeutet – kann ein Polynom 3. Grades nur einen Extrempunkt haben, wenn es nicht konstant ist?) Was liegt stattdessen an der kritischen Stelle vor, wenn es kein Extremum ist?
            \item Für welche Werte von $k$ hat die Funktion $f_k(x)$ genau zwei lokale Extrempunkte? Bestimme deren Art (Hoch-/Tiefpunkt) und ihre Lage (x-Koordinaten) in Abhängigkeit von $k$.
        \end{itemize}
        \begin{tippumgebung}{Fallunterscheidung für Parameter $k$}
        Die erste Ableitung ist $f_k'(x) = 3x^2 - 3k$. Setze $f_k'(x)=0$ und löse nach $x^2$. Die Anzahl der Lösungen für $x$ (und damit die Anzahl der kritischen Stellen) hängt vom Wert und Vorzeichen von $k$ ab. Unterscheide die Fälle $k<0$, $k=0$ und $k>0$. Nutze dann das Vorzeichenwechselkriterium von $f_k'(x)$ oder die zweite Ableitung $f_k''(x)$, um die Art der Extrema zu bestimmen.
        \end{tippumgebung}
    \item \textbf{Anwendung: Optimale Form}
        Ein oben offener quaderförmiger Behälter mit quadratischer Grundfläche soll ein Volumen von $V=32 \text{ cm}^3$ haben. Bestimme die Abmessungen (Seitenlänge der Grundfläche und Höhe), für die der Materialverbrauch (also die Oberfläche) minimal wird.
        \begin{enumerate}[label=(\alph*)]
            \item Sei $a$ die Seitenlänge der quadratischen Grundfläche und $h$ die Höhe des Quaders. Stelle die Formel für das Volumen $V$ und die Oberfläche $O$ (Grundfläche + 4 Seitenflächen) auf.
            \item Drücke $h$ mithilfe der Volumenformel durch $a$ aus (Nebenbedingung).
            \item Setze $h$ in die Oberflächenformel ein, um eine Zielfunktion $O(a)$ zu erhalten, die nur noch von $a$ abhängt.
            \item Bestimme die erste Ableitung $O'(a)$ und finde die kritischen Stellen.
            \item Überprüfe mit der zweiten Ableitung $O''(a)$, ob ein Minimum vorliegt. (Hier ist die zweite Ableitung oft der schnellste Weg zur Klassifizierung in Optimierungsaufgaben).
            \item Berechne die optimale Seitenlänge $a$ und die zugehörige Höhe $h$.
        \end{enumerate}
\end{enumerate}
\end{aufgabenumgebung}


\begin{tippumgebung}{Skizze hilft!}
Eine kleine Skizze des Vorzeichenverlaufs von $f'(x)$ (eine Art 'Zahlenstrahl' mit den Nullstellen von $f'$ und den Vorzeichen dazwischen) kann sehr helfen, um die Vorzeichenwechsel und damit die Art der Extrempunkte schnell zu erkennen. Es kann auch sehr helfen, alle bereits erlangten Kenntnisse über Funktionen auszunutzen, um $f'(x)$ zu skizzieren. Wenn die Nullstellen schon klar sind, dann kannst du schnell eine steigende/fallende lineare Funktion,welche durch den schnell ablesbaren y-Achsenabschnitt geht, skizzieren. Ähnliches gilt auch für Parabeln, du musst nur erkennen, ob die Parabel nach unten oder oben geöffnet ist.  
\end{tippumgebung}

\subsection{Höhere Ableitungen – Die Ableitung der Ableitung (und so weiter)}
\label{subsec:hoehere_ableitungen}

Wir haben gesehen, dass die erste Ableitung $f'(x)$ uns Informationen über die Steigung und das Monotonieverhalten der Funktion $f(x)$ gibt. Aber wir können auch die Ableitungsfunktion $f'(x)$ selbst wieder ableiten! Das Ergebnis nennen wir die \textbf{zweite Ableitung} von $f(x)$ und schreiben sie als $f''(x)$ (gelesen 'f zwei Strich von x').

\begin{merksatzumgebung}{Zweite und höhere Ableitungen}
\begin{itemize}
    \item Die \textbf{zweite Ableitung $f''(x)$} einer Funktion $f(x)$ ist die Ableitung ihrer ersten Ableitung $f'(x)$:
    \[ f''(x) = (f'(x))' \]
    \item Man kann diesen Prozess fortsetzen: Die Ableitung der zweiten Ableitung ist die \textbf{dritte Ableitung $f'''(x)$}, die Ableitung der dritten ist die \textbf{vierte Ableitung $f^{(4)}(x)$} (ab hier verwendet man oft römische Ziffern oder Zahlen in Klammern, um die Striche zu vermeiden), und so weiter.
    \[ f'''(x) = (f''(x))' \]
    \[ f^{(n)}(x) = (f^{(n-1)}(x))' \quad (\text{die n-te Ableitung}) \]
\end{itemize}
\end{merksatzumgebung}

\begin{warumwichtigumgebung}{Bedeutung höherer Ableitungen}
\begin{itemize}
    \item \textbf{Erste Ableitung $f'(x)$:} Gibt die Steigung von $f(x)$ an. In der Physik: Wenn $s(t)$ der Weg ist, ist $s'(t)$ die Geschwindigkeit $v(t)$.
    \item \textbf{Zweite Ableitung $f''(x)$:} Gibt die \textit{Änderung der Steigung} von $f(x)$ an. Sie beschreibt das \textbf{Krümmungsverhalten} des Graphen von $f(x)$. In der Physik: Wenn $v(t)$ die Geschwindigkeit ist, ist $v'(t) = s''(t)$ die Beschleunigung $a(t)$. Die zweite Ableitung sagt uns also, wie schnell sich die Geschwindigkeit ändert.
    \item \textbf{Dritte Ableitung $f'''(x)$ und höher:} Haben auch ihre Bedeutungen, z.B. in der Physik der 'Ruck' (Änderung der Beschleunigung), sind aber in der Schulmathematik für Kurvendiskussionen seltener direkt relevant als $f'$ und $f''$.
\end{itemize}
Die zweite Ableitung ist also die 'Ableitung der Steigung' oder die 'Steigung der Steigungsfunktion'.
\end{warumwichtigumgebung}

\begin{beispielumgebung}{Höhere Ableitungen bilden}
Bestimme die ersten drei Ableitungen der Funktion $f(x) = x^4 - 5x^3 + 2x^2 - 7x + 10$.

$f(x) = x^4 - 5x^3 + 2x^2 - 7x + 10$

$f'(x) = (x^4)' - (5x^3)' + (2x^2)' - (7x)' + (10)'$
$f'(x) = 4x^3 - 5 \cdot 3x^2 + 2 \cdot 2x - 7 \cdot 1 + 0$
$f'(x) = 4x^3 - 15x^2 + 4x - 7$

$f''(x) = (f'(x))' = (4x^3 - 15x^2 + 4x - 7)'$
$f''(x) = (4x^3)' - (15x^2)' + (4x)' - (7)'$
$f''(x) = 4 \cdot 3x^2 - 15 \cdot 2x + 4 \cdot 1 - 0$
$f''(x) = 12x^2 - 30x + 4$

$f'''(x) = (f''(x))' = (12x^2 - 30x + 4)'$
$f'''(x) = (12x^2)' - (30x)' + (4)'$
$f'''(x) = 12 \cdot 2x - 30 \cdot 1 + 0$
$f'''(x) = 24x - 30$

Man könnte so weitermachen: $f^{(4)}(x) = 24$, $f^{(5)}(x) = 0$, und alle weiteren Ableitungen wären ebenfalls null.
\end{beispielumgebung}

\begin{aufgabenumgebung}{Höhere Ableitungen berechnen}
Bestimme die erste, zweite und dritte Ableitung der folgenden Funktionen:
\begin{enumerate}
    \item $f(x) = 2x^3 - 9x^2 + 12x - 5$
    \item $g(x) = -0.1x^5 + x^3 - 7$
    \item $h(t) = 2t^2 - \frac{1}{t}$ (Tipp: $\frac{1}{t} = t^{-1}$)
\end{enumerate}
\end{aufgabenumgebung}

\subsection{Die zweite Ableitung $f''(x)$ – Krümmung und Wendepunkte}
\label{subsec:zweite_ableitung_neu} 

Nachdem wir wissen, was höhere Ableitungen sind, konzentrieren wir uns nun auf die Bedeutung der \textbf{zweiten Ableitung $f''(x)$}. Sie gibt uns Auskunft über das \textbf{Krümmungsverhalten} des Graphen von $f(x)$ und hilft uns, \textbf{Wendepunkte} zu finden.

\subsubsection{Krümmungsverhalten – Links- oder Rechtskurve?}
Die zweite Ableitung beschreibt, wie sich die Steigung der Funktion ändert.
\begin{itemize}
    \item Wenn $f''(x) > 0$, bedeutet das, dass die Steigung $f'(x)$ zunimmt. Stell dir vor, du fährst auf dem Graphen: Erst ist die Steigung vielleicht negativ (bergab), dann wird sie weniger negativ, dann Null, dann positiv (bergauf). Das entspricht einer \textbf{Linkskurve} (man sagt auch, der Graph ist \textbf{konvex} oder 'nach oben offen' in diesem Bereich).
    \item Wenn $f''(x) < 0$, bedeutet das, dass die Steigung $f'(x)$ abnimmt. Du fährst vielleicht erst steil bergauf, dann wird der Anstieg flacher, dann Null, dann geht es bergab. Das entspricht einer \textbf{Rechtskurve} (man sagt auch, der Graph ist \textbf{konkav} oder 'nach unten offen' in diesem Bereich).
\end{itemize}

\begin{merksatzumgebung}{Krümmung und zweite Ableitung}
Sei $f$ eine in einem Intervall $I$ zweimal differenzierbare Funktion. Dann gilt für alle $x \in I$:
\begin{itemize}
    \item Wenn $f''(x) > 0$ für alle $x$ in $I$, dann ist der Graph von $f(x)$ in diesem Intervall \textbf{linksgekrümmt} (konvex). (Der Graph macht eine 'Linkskurve', wie ein lachender Smiley $\smile$ bei positiver zweiten Ableitung).
    \item Wenn $f''(x) < 0$ für alle $x$ in $I$, dann ist der Graph von $f(x)$ in diesem Intervall \textbf{rechtsgekrümmt} (konkav). (Der Graph macht eine 'Rechtskurve', wie ein trauriger Smiley $\frown$ bei negativer zweiten Ableitung).
    \item Wenn $f''(x) = 0$ an einer Stelle $x_W$, dann \textbf{könnte} dort ein \textbf{Wendepunkt} vorliegen (ein Punkt, an dem sich das Krümmungsverhalten ändert).
\end{itemize}
Um die Krümmungsintervalle zu bestimmen, untersuchst du also das \textbf{Vorzeichen der zweiten Ableitung $f''(x)$}.
\end{merksatzumgebung}

\begin{beispielumgebung}{Krümmungsverhalten untersuchen}
Untersuche das Krümmungsverhalten von $f(x) = x^3 - 3x^2 + 1$.
Wir hatten $f'(x) = 3x^2 - 6x$ und $f''(x) = 6x - 6$.

\textbf{Schritt 1: Nullstellen der zweiten Ableitung finden.}
$f''(x) = 0 \implies 6x - 6 = 0$
\begin{center}
\begin{tabular}{r @{\,} c @{\,} l @{\quad\quad} l}
$6x - 6$ & $=$ & $0$ & $|+6$ \\
$6x$ & $=$ & $6$ & $|:6$ \\
$x$ & $=$ & $1$ & \\
\end{tabular}
\end{center}
An der Stelle $x=1$ könnte sich das Krümmungsverhalten ändern.

\textbf{Schritt 2: Vorzeichen von $f''(x)$ in den Intervallen untersuchen.}
Die Stelle $x=1$ teilt die x-Achse in zwei Intervalle: $(-\infty, 1)$ und $(1, \infty)$.
\begin{itemize}
    \item Intervall $(-\infty, 1)$: Wähle Testwert $x=0$. $f''(0) = 6(0) - 6 = -6$.
    Da $f''(0) < 0$, ist der Graph von $f(x)$ im Intervall $(-\infty, 1)$ rechtsgekrümmt.
    \item Intervall $(1, \infty)$: Wähle Testwert $x=2$. $f''(2) = 6(2) - 6 = 12 - 6 = 6$.
    Da $f''(2) > 0$, ist der Graph von $f(x)$ im Intervall $(1, \infty)$ linksgekrümmt.
\end{itemize}
\begin{center}
    \includegraphics[width=0.8\textwidth]{grafiken/Differentialrechnung_Kruemmung.png}
    \captionof{figure}{Krümmungsverhalten von $f(x)=x^3-3x^2+1$}
    \label{fig:kruemmung_bsp_neu}
\end{center}


\end{beispielumgebung}

\subsubsection{Wendepunkte – Wo die Kurve ihre Richtung ändert}
Ein \textbf{Wendepunkt} ist ein Punkt auf dem Graphen einer Funktion, an dem sich das Krümmungsverhalten ändert. Das bedeutet, der Graph wechselt dort von einer Linkskurve in eine Rechtskurve oder umgekehrt. Die Tangente an diesem Punkt nennt man \textbf{Wendetangente}.

\begin{merksatzumgebung}{Wendepunkte finden}
Um Wendepunkte einer Funktion $f(x)$ zu finden (die mindestens zweimal differenzierbar sein muss):

\textbf{1. Notwendige Bedingung für Wendepunkte:}
Wenn $f(x)$ an der Stelle $x_W$ einen Wendepunkt hat, dann muss gelten:
\[ f''(x_W) = 0 \]
Die Stellen $x_W$, an denen $f''(x_W)=0$ ist, sind \textbf{potentielle Wendestellen}.

\textbf{2. Hinreichende Bedingung für Wendepunkte:}
Es gibt zwei gängige hinreichende Bedingungen, um zu bestätigen, dass an einer potentiellen Wendestelle $x_W$ tatsächlich ein Wendepunkt vorliegt:
\begin{itemize}
    \item \textbf{Vorzeichenwechselkriterium von $f''(x)$:}
    Wenn $f''(x)$ an der Stelle $x_W$ einen Vorzeichenwechsel hat (von $+$ nach $-$ oder von $-$ nach $+$), dann liegt bei $x_W$ ein Wendepunkt vor.
    \item \textbf{Mit der dritten Ableitung $f'''(x)$:}
    Wenn $f''(x_W)=0$ und zusätzlich $f'''(x_W) \neq 0$ ist, dann liegt bei $x_W$ ein Wendepunkt vor. (Diese Bedingung ist oft einfacher zu prüfen, wenn die dritte Ableitung leicht zu bilden ist und nicht Null ist.)
\end{itemize}
Die y-Koordinate des Wendepunktes erhältst du, indem du $x_W$ in die Originalfunktion $f(x)$ einsetzt: $y_W = f(x_W)$. Der Wendepunkt ist dann $W(x_W|y_W)$.
\end{merksatzumgebung}
\begin{infoboxumgebung}{Wendepunkte – Mehr als nur ein Krümmungswechsel}
Wir haben gelernt, dass an einem Wendepunkt $x_W$ die zweite Ableitung $f''(x_W)=0$ ist (und $f'''(x_W) \neq 0$ oder ein Vorzeichenwechsel von $f''$ stattfindet). Das bedeutet, die Krümmung des Graphen wechselt dort ihr Vorzeichen (z.B. von einer Rechts- in eine Linkskurve). Aber was bedeutet das für die Steigung der Funktion, also für $f'(x)$?

\textbf{Der Wendepunkt als Extremum der Steigung:}
Da $f''(x)$ die Ableitung von $f'(x)$ ist, bedeutet $f''(x_W)=0$ und ein Vorzeichenwechsel von $f''(x)$ bei $x_W$ (oder $f'''(x_W) \neq 0$), dass die \textbf{erste Ableitung $f'(x)$ an der Stelle $x_W$ eine Extremstelle} hat!
\begin{itemize}
    \item Wechselt $f''(x)$ von $-$ nach $+$ (also der Graph von $f(x)$ von rechts- zu linkskrümmt), dann hat $f'(x)$ bei $x_W$ ein \textit{Minimum}. Die Steigung von $f(x)$ ist an dieser Stelle am geringsten (d.h. am stärksten negativ oder am wenigsten positiv).
    \item Wechselt $f''(x)$ von $+$ nach $-$ (also der Graph von $f(x)$ von links- zu rechtskrümmt), dann hat $f'(x)$ bei $x_W$ ein \textit{Maximum}. Die Steigung von $f(x)$ ist an dieser Stelle am größten.
\end{itemize}
Ein Wendepunkt ist also eine Stelle, an der die \textbf{Steigung des Graphen von $f(x)$ am extremsten} ist (entweder maximal oder minimal im lokalen Sinne).

\textbf{Intuition – Der Weg zwischen den 'Gipfeln und Tälern':}
Stell dir vor, ein Graph hat einen Hochpunkt (Steigung 0) und danach einen Tiefpunkt (Steigung 0). Dazwischen muss die Steigung von 0 erst negativ geworden sein (bergab) und dann wieder weniger negativ, um beim Tiefpunkt wieder 0 zu werden. Irgendwo auf diesem Weg muss die Steigung ihren negativsten (minimalen) Wert erreicht haben – genau das ist der Wendepunkt! Die Straße macht dort die 'schärfste Kurve' nach unten. Analoges gilt für den Übergang von einem Tief- zu einem Hochpunkt.

\textbf{Sattelpunkte als Spezialfall:}
Wenn an einem Wendepunkt $x_W$ zusätzlich gilt, dass $f'(x_W)=0$ ist (die Tangente also waagerecht ist), dann nennen wir diesen Punkt einen \textbf{Sattelpunkt}. Hier ist die Steigung nicht nur extrem (nämlich 0), sondern es findet auch ein Krümmungswechsel statt, aber kein Monotoniewechsel der Funktion $f(x)$ selbst.

Die Betrachtung des Wendepunkts als Extremum der ersten Ableitung kann helfen, sein Auftreten und seine Bedeutung besser zu verstehen. Es ist ein schönes Beispiel dafür, wie die verschiedenen Ableitungen zusammenwirken, um das Verhalten einer Funktion zu beschreiben.
\end{infoboxumgebung}
\begin{beispielumgebung}{Wendepunkt bestimmen}
Bestimme den Wendepunkt von $f(x) = x^3 - 3x^2 + 1$.
Wir hatten:
$f'(x) = 3x^2 - 6x$
$f''(x) = 6x - 6$

\textbf{Schritt 1: Notwendige Bedingung $f''(x_W)=0$.}
$6x_W - 6 = 0 \implies 6x_W = 6 \implies x_W = 1$.
Die potentielle Wendestelle ist bei $x_W=1$.

\textbf{Schritt 2: Hinreichende Bedingung prüfen.}
\textit{Möglichkeit a) Vorzeichenwechsel von $f''(x)$:}
Wir hatten bereits im vorherigen Beispiel gesehen:
Links von $x=1$ (z.B. $x=0$): $f''(0) = -6 < 0$ (rechtsgekrümmt).
Rechts von $x=1$ (z.B. $x=2$): $f''(2) = 6 > 0$ (linksgekrümmt).
Da ein Vorzeichenwechsel von $f''(x)$ von $-$ nach $+$ an der Stelle $x=1$ stattfindet, liegt dort ein Wendepunkt vor.

\textit{Möglichkeit b) Mit der dritten Ableitung $f'''(x)$:}
$f'''(x) = (f''(x))' = (6x-6)' = 6$.
Nun setzen wir $x_W=1$ in $f'''(x)$ ein:
$f'''(1) = 6$.
Da $f'''(1) = 6 \neq 0$ (und $f''(1)=0$ war), liegt bei $x_W=1$ ein Wendepunkt vor.

\textbf{Schritt 3: y-Koordinate des Wendepunkts berechnen.}
$y_W = f(x_W) = f(1) = (1)^3 - 3(1)^2 + 1 = 1 - 3 + 1 = -1$.
Der Wendepunkt ist $W(1|-1)$.

Am Wendepunkt $W(1|-1)$ wechselt der Graph von $f(x)=x^3-3x^2+1$ von einer Rechtskrümmung in eine Linkskrümmung.
\begin{center}
    \includegraphics[width=0.8\textwidth]{grafiken/Differentialrechnung_Wendepunkt.png}
    \captionof{figure}{Wendepunkt von $f(x)=x^3-3x^2+1$}
    \label{fig:wendepunkt_bsp}
\end{center}
\end{beispielumgebung}

\begin{tippumgebung}{Zweite Ableitung für Extremstellen}
Die zweite Ableitung kann auch als hinreichende Bedingung für Extremstellen dienen (anstelle des Vorzeichenwechsels von $f'$):
Sei $x_E$ eine Stelle mit $f'(x_E)=0$.
\begin{itemize}
    \item Wenn $f''(x_E) < 0 \implies$ Lokaler Hochpunkt bei $x_E$. (Der Graph ist dort rechtsgekrümmt, wie ein Bergipfel).
    \item Wenn $f''(x_E) > 0 \implies$ Lokaler Tiefpunkt bei $x_E$. (Der Graph ist dort linksgekrümmt, wie ein Talboden).
    \item Wenn $f''(x_E) = 0 \implies$ Keine Aussage möglich mit diesem Kriterium! Dann muss man das Vorzeichenwechselkriterium von $f'$ verwenden. Es könnte ein Sattelpunkt oder doch ein Extremum sein.
\end{itemize}
\end{tippumgebung}

\begin{aufgabenumgebung}{Krümmung und Wendepunkte – Vielfältige Untersuchungen}
Untersuche die folgenden Funktionen auf ihr Krümmungsverhalten (Intervalle für Links- und Rechtskrümmung) und bestimme gegebenenfalls die Koordinaten der Wendepunkte. Nutze dazu die zweite und, falls nötig, die dritte Ableitung.
\begin{enumerate}
    \item $f_1(x) = \frac{1}{12}x^4 - \frac{1}{2}x^3 + x^2 + 1$
        \begin{tippumgebung}{Polynom 4. Grades}
        Die zweite Ableitung $f_1''(x)$ wird eine quadratische Funktion sein. Deren Nullstellen (potentielle Wendestellen) findest du mit den bekannten Lösungsformeln. Überprüfe die hinreichende Bedingung für Wendepunkte.
        \end{tippumgebung}

    \item $f_2(x) = x^5 - 5x^4 + 3x - 2$
        \begin{tippumgebung}{Polynom 5. Grades}
        Die zweite Ableitung $f_2''(x)$ wird ein Polynom 3. Grades sein. Versuche, $x$ (oder eine höhere Potenz von $x$) auszuklammern, um die Nullstellen von $f_2''(x)$ zu finden.
        \end{tippumgebung}

    \item \textbf{Anwendung: Infektionsgeschehen} \\
        Die Funktion $N(t) = -t^3 + 12t^2 + 20t$ beschreibt die Anzahl der neu infizierten Personen pro Tag während einer Grippewelle ($t$ in Tagen, $t \ge 0$).
        \begin{itemize}
            \item Bestimme die Funktion $N'(t)$, welche die Änderungsrate der Neuinfektionen (also die 'Geschwindigkeit' der Ausbreitung) beschreibt.
            \item Bestimme die Funktion $N''(t)$, welche die Änderungsrate der Wachstumsrate der Neuinfektionen beschreibt.
            \item Zu welchem Zeitpunkt $t_W$ ist die Zunahme der täglichen Neuinfektionen am größten? (Das bedeutet, $N'(t)$ hat ein Maximum, also suche einen Wendepunkt von $N(t)$, an dem die Krümmung von links nach rechts wechselt, d.h. $N''(t_W)=0$ und $N'''(t_W)<0$).
            \item Interpretiere die Bedeutung dieses Zeitpunktes $t_W$ für den Verlauf der Grippewelle. Was passiert mit der Zunahme der Neuinfektionen nach diesem Zeitpunkt?
        \end{itemize}

    \item \textbf{Schwer: Funktion mit Parameter $a$} \\
        Gegeben ist die Funktionenschar $f_a(x) = x^4 + ax^3$ mit $a \in \mathbb{R}, a \neq 0$.
        \begin{itemize}
            \item Bestimme die zweite Ableitung $f_a''(x)$.
            \item Zeige, dass $x_1=0$ eine potentielle Wendestelle ist. Untersuche mit der dritten Ableitung $f_a'''(x)$, ob für $x_1=0$ tatsächlich ein Wendepunkt vorliegt.
            \item Bestimme die andere potentielle Wendestelle $x_2$ in Abhängigkeit von $a$.
            \item Für welche Werte von $a$ existiert dieser zweite Wendepunkt $x_2$? (Beachte, dass $x_2 \neq x_1$ sein sollte für einen \textit{anderen} Wendepunkt).
            \item Untersuche das Krümmungsverhalten für $a=2$ und $a=-2$ und skizziere grob die Verläufe (ohne vollständige Kurvendiskussion, Fokus auf Krümmung und Wendepunkte).
        \end{itemize}
        \begin{tippumgebung}{Parameter $a$}
        Behandle $a$ wie eine Konstante beim Ableiten. Die Ergebnisse für Wendestellen und Krümmungsintervalle werden dann von $a$ abhängen.
        \end{tippumgebung}
\end{enumerate}
\end{aufgabenumgebung}

% Dieser Block sollte im Kapitel 'Einführung in die Differentialrechnung',
% nach der aufgabenumgebung 'Krümmung und Wendepunkte – Vielfältige Untersuchungen'
% und vor dem \subsection 'Grenzwerte (Limes)...' eingefügt werden.

\subsubsection{Tangenten und Normalen – Geraden am Graphen}
\label{subsubsec:tangenten_normalen}

Wir wissen bereits, dass die erste Ableitung $f'(x_0)$ die Steigung der Tangente an den Graphen der Funktion $f(x)$ im Punkt $P(x_0|f(x_0))$ angibt. Mit diesem Wissen können wir die Gleichung dieser Tangente und auch die Gleichung der Normalen (die Senkrechte zur Tangente im selben Punkt) bestimmen.

\begin{merksatzumgebung}{Gleichung der Tangente}
Die Gleichung der Tangente $t(x)$ an den Graphen einer Funktion $f(x)$ im Punkt $P(x_0|y_0)$ mit $y_0=f(x_0)$ lautet:
\[ t(x) = f'(x_0) \cdot (x - x_0) + y_0 \]
oder
\[ t(x) = f'(x_0) \cdot (x - x_0) + f(x_0) \]
Dabei ist $m_T = f'(x_0)$ die Steigung der Tangente. Diese Formel ist die Punkt-Steigungs-Form einer Geraden.
\end{merksatzumgebung}

\begin{merksatzumgebung}{Gleichung der Normale}
Die Normale $n(x)$ ist die Gerade, die senkrecht zur Tangente $t(x)$ im Punkt $P(x_0|y_0)$ steht.
Für die Steigungen $m_T$ der Tangente und $m_N$ der Normalen gilt (wenn $m_T \neq 0$): $m_N = -\frac{1}{m_T}$.
Die Gleichung der Normalen $n(x)$ im Punkt $P(x_0|y_0)$ mit $y_0=f(x_0)$ lautet also, falls $f'(x_0) \neq 0$:
\[ n(x) = -\frac{1}{f'(x_0)} \cdot (x - x_0) + y_0 \]
Falls $f'(x_0) = 0$ (horizontale Tangente, z.B. an Extrempunkten), ist die Tangente $y=y_0$ und die Normale eine senkrechte Gerade $x=x_0$.
\end{merksatzumgebung}

\begin{tippumgebung}{Tangenten und Normalen an besonderen Punkten}
\begin{itemize}
    \item \textbf{An Extrempunkten ($f'(x_E)=0$):}
        Die Tangente ist waagerecht: $t(x) = f(x_E)$.
        Die Normale ist senkrecht: $x = x_E$ (keine Funktion der Form $y=mx+b$).
        Die Berechnung einer Normalen\textit{gleichung} in der Form $y=mx+b$ ist hier also nicht sinnvoll, aber die Normale existiert als senkrechte Linie.
    \item \textbf{An Wendepunkten ($W(x_W|f(x_W))$):}
        Die Tangente im Wendepunkt wird \textbf{Wendetangente} genannt. Ihre Steigung $f'(x_W)$ ist oft die größte oder kleinste Steigung in der Umgebung des Wendepunkts.
        Die Normale im Wendepunkt wird \textbf{Wendenormale} genannt.
    \item \textbf{An beliebigen Punkten:} Man kann die Tangente und Normale an \textit{jedem} Punkt des Graphen berechnen, an dem die Funktion differenzierbar ist, nicht nur an ausgezeichneten Punkten wie Extrema oder Wendepunkten.
\end{itemize}
\end{tippumgebung}

\begin{beispielumgebung}{Wendetangente und Wendenormale berechnen}
Wir betrachten wieder die Funktion $f(x) = x^3 - 3x^2 + 1$.
Aus einem früheren Beispiel wissen wir:
$f'(x) = 3x^2 - 6x$
$f''(x) = 6x - 6$
$f'''(x) = 6$
Der Wendepunkt liegt bei $x_W=1$, mit $f(1) = 1-3+1 = -1$. Also $W(1|-1)$.

\textbf{1. Wendetangente $t_W(x)$ im Punkt $W(1|-1)$:}
Steigung im Wendepunkt: $m_T = f'(1) = 3(1)^2 - 6(1) = 3 - 6 = -3$.
Punkt-Steigungs-Form: $t_W(x) = m_T (x - x_W) + f(x_W)$
$t_W(x) = -3 (x - 1) + (-1)$
$t_W(x) = -3x + 3 - 1$
\[ t_W(x) = -3x + 2 \]

\textbf{2. Wendenormale $n_W(x)$ im Punkt $W(1|-1)$:}
Steigung der Tangente war $m_T = -3$.
Steigung der Normalen: $m_N = -\frac{1}{m_T} = -\frac{1}{-3} = \frac{1}{3}$.
Punkt-Steigungs-Form: $n_W(x) = m_N (x - x_W) + f(x_W)$
$n_W(x) = \frac{1}{3} (x - 1) + (-1)$
$n_W(x) = \frac{1}{3}x - \frac{1}{3} - 1$
$n_W(x) = \frac{1}{3}x - \frac{1}{3} - \frac{3}{3}$
\[ n_W(x) = \frac{1}{3}x - \frac{4}{3} \]
\begin{center}
    \includegraphics[width=0.8\textwidth]{grafiken/Wendetangente_Normale.png}
    \captionof{figure}{Wendetangente und Wendenormale für $f(x)=x^3-3x^2+1$ im Punkt $W(1|-1)$}
    \label{fig:wendetangente_normale}
\end{center}
\end{beispielumgebung}

\begin{aufgabenumgebung}{Tangenten und Normalen bestimmen – Vielfältige Aufgaben}
\begin{enumerate}
    \item Gegeben ist die Funktion $f(x) = \frac{1}{4}x^4 - x^2 + 1$.
        \begin{itemize}
            \item Bestimme die Gleichung der Tangente und der Normalen an den Graphen von $f$ an der Stelle $x_0 = 2$.
            \item An welchen Stellen $x$ hat der Graph von $f$ eine Tangente mit der Steigung $m=0$? Was für Punkte sind das?
            \item (Schwer): Gibt es eine Tangente an den Graphen von $f$, die parallel zur Geraden $y = -2x+5$ ist? Wenn ja, bestimme die Berührpunkte und die Gleichungen dieser Tangenten.
        \end{itemize}
    \item Gegeben ist die Funktion $g(x) = x^3 - 3x$.
        \begin{itemize}
            \item Bestimme die Wendepunkt(e) von $g(x)$.
            \item Bestimme die Gleichung der Wendetangente(n).
            \item Zeige, dass die Wendetangente im Ursprung (falls vorhanden) die x-Achse nur im Ursprung schneidet.
        \end{itemize}
    \item \textbf{Orthogonale Tangenten (Für Experten):}
        Gegeben ist die Parabel $f(x) = x^2$. Gibt es zwei Punkte $P_1(x_1|f(x_1))$ und $P_2(x_2|f(x_2))$ auf der Parabel, deren Tangenten sich senkrecht schneiden und deren x-Koordinaten die Bedingung $x_1 \cdot x_2 = -1/4$ erfüllen?
        \begin{tippumgebung}{Orthogonalität}
        Zwei Geraden mit Steigungen $m_1$ und $m_2$ sind orthogonal (senkrecht), wenn $m_1 \cdot m_2 = -1$ (vorausgesetzt $m_1, m_2 \neq 0$).
        \end{tippumgebung}
    \item \textbf{Normale durch den Ursprung:}
        Für die Funktion $f(x) = \frac{1}{2}x^2 - 2x + 3$, bestimme den Punkt $P(x_0|f(x_0))$ auf dem Graphen, dessen Normale durch den Ursprung $(0|0)$ verläuft.

    \item \textbf{Wendetangente mit speziellen Eigenschaften (Schwer):}
        Gegeben ist die Funktion $f(x) = \frac{1}{6}x^3 - x^2 + 2x + 1$.
        \begin{itemize}
            \item Bestimme die Koordinaten des Wendepunktes $W$.
            \item Bestimme die Gleichung der Wendetangente $t_W(x)$ und der Wendenormalen $n_W(x)$.
            \item Die Wendetangente, die Wendenormale und die y-Achse bilden ein Dreieck. Berechne den Flächeninhalt dieses Dreiecks.
            \item Unter welchem Winkel schneidet die Wendetangente die x-Achse? (Tipp: Der Tangens des Steigungswinkels $\alpha$ einer Geraden ist gleich ihrer Steigung $m$, also $\tan(\alpha) = m$. Du suchst $\alpha = \arctan(m)$.)
        \end{itemize}

    \item \textbf{Berührbedingung und Parameter (Schwer):}
        Gegeben sind die Funktionen $f(x) = x^2 + 2x + 2$ und die Geradenschar $g_k(x) = kx - 2$ (wobei $k \in \mathbb{R}$ ein Parameter ist).
        \begin{itemize}
            \item Für welchen Wert von $k$ berührt die Gerade $g_k(x)$ die Parabel $f(x)$?
            \begin{tippumgebung}{Berührbedingung}
            Zwei Funktionen $f(x)$ und $g(x)$ berühren sich an einer Stelle $x_B$, wenn gilt:
            \begin{enumerate}
                \item $f(x_B) = g(x_B)$ (gleicher Funktionswert am Berührpunkt)
                \item $f'(x_B) = g'(x_B)$ (gleiche Steigung am Berührpunkt)
            \end{enumerate}
            Du erhältst ein Gleichungssystem für $x_B$ und $k$.
            \end{tippumgebung}
            \item Bestimme den Berührpunkt und die Gleichung der gemeinsamen Tangente für diesen Wert von $k$.
            \item (Für Experten): Gibt es einen Wert für $k$, sodass die Gerade $g_k(x)$ eine Normale zur Parabel $f(x)$ an einem Punkt $P(x_0|f(x_0))$ ist? Wenn ja, bestimme $k$ und den Punkt $P$.
        \end{itemize}
\end{enumerate}
\end{aufgabenumgebung}

% Hier geht es dann weiter mit dem nächsten großen Abschnitt, z.B. Grenzwerten oder weiteren Ableitungsregeln.



\subsubsection{Sattelpunkte – Besondere Wendepunkte mit horizontaler Tangente}
\label{subsubsec:sattelpunkte}

In der Kurvendiskussion haben wir bereits Extrempunkte (Hoch- und Tiefpunkte) und Wendepunkte kennengelernt. Sattelpunkte, auch Terrassenpunkte genannt, sind eine spezielle Art von Wendepunkten, die eine interessante Eigenschaft mit Extrempunkten teilen: Die Tangente an den Graphen ist an einem Sattelpunkt \textbf{waagerecht}, genau wie bei einem Hoch- oder Tiefpunkt. Der Unterschied ist jedoch, dass die Funktion an einem Sattelpunkt ihr Monotonieverhalten \textit{nicht} ändert.

Stell dir vor, du fährst auf einer kurvigen Bergstraße. Ein Hochpunkt ist ein Gipfel, ein Tiefpunkt ein Tal. Ein Wendepunkt ist eine Stelle, an der eine Rechtskurve in eine Linkskurve übergeht (oder umgekehrt). Ein Sattelpunkt ist nun ein Wendepunkt, an dem die Straße für einen kurzen Moment exakt horizontal verläuft, bevor sie ihre Krümmung ändert und in derselben Richtung (steigend oder fallend) weiterführt.

\begin{merksatzumgebung}[def_sattelpunkt]{Definition und Bedingungen für einen Sattelpunkt}
Ein Punkt $S(x_S|f(x_S))$ auf dem Graphen einer Funktion $f(x)$ ist ein \textbf{Sattelpunkt} (oder Terrassenpunkt), wenn folgende Bedingungen erfüllt sind:
\begin{enumerate}
    \item \textbf{Horizontale Tangente:} Die erste Ableitung ist an der Stelle $x_S$ Null.
    \[ f'(x_S) = 0 \]
    (Dies ist die notwendige Bedingung auch für Extremstellen.)
    \item \textbf{Kein Vorzeichenwechsel der ersten Ableitung:} Die erste Ableitung $f'(x)$ wechselt an der Stelle $x_S$ \textbf{nicht} das Vorzeichen. Das bedeutet, die Funktion ist links und rechts von $x_S$ entweder beidesmal steigend oder beidesmal fallend.
    \textit{Alternativ (und oft einfacher zu prüfen mit höheren Ableitungen):}
    \item \textbf{Zweite Ableitung ist Null:} Die zweite Ableitung ist an der Stelle $x_S$ Null.
    \[ f''(x_S) = 0 \]
    (Dies ist die notwendige Bedingung für einen Wendepunkt.)
    \item \textbf{Dritte Ableitung ist ungleich Null:} Die dritte Ableitung an der Stelle $x_S$ ist nicht Null.
    \[ f'''(x_S) \neq 0 \]
    (Dies ist eine hinreichende Bedingung dafür, dass bei $f''(x_S)=0$ tatsächlich ein Wendepunkt vorliegt und kein Extremum höherer Ordnung, bei dem $f''$ zufällig Null ist.)
\end{enumerate}
Zusammenfassend: Ein Sattelpunkt ist ein Wendepunkt mit einer horizontalen Tangente.
\end{merksatzumgebung}

\begin{warumwichtigumgebung}{Sattelpunkte erkennen}
Sattelpunkte sind wichtig, weil sie Stellen markieren, an denen die Steigung kurzzeitig Null wird, ohne dass ein lokales Maximum oder Minimum vorliegt. In Optimierungsprozessen könnten solche Punkte 'falsche Freunde' sein – man denkt, man hat ein Optimum erreicht, aber tatsächlich geht es danach in gleicher Richtung weiter. In der Physik können sie Übergangszustände oder instabile Gleichgewichtslagen repräsentieren.
\end{warumwichtigumgebung}

\begin{beispielumgebung}{Untersuchung auf Sattelpunkte bei \texorpdfstring{$f(x) = x^3$}{f(x) = x hoch 3}}
Die einfachste Funktion mit einem Sattelpunkt im Ursprung ist $f(x) = x^3$.
\begin{enumerate}
    \item \textbf{Ableitungen bilden:}
        $f(x) = x^3$
        $f'(x) = 3x^2$
        $f''(x) = 6x$
        $f'''(x) = 6$

    \item \textbf{Potentielle Stellen für horizontale Tangenten (kritische Stellen):}
        $f'(x) = 0 \implies 3x^2 = 0 \implies x = 0$.
        Einzige kritische Stelle ist $x_S = 0$.

    \item \textbf{Überprüfung mit Vorzeichenwechsel von $f'(x)$:}
        \begin{itemize}
            \item Links von $x=0$ (z.B. $x=-1$): $f'(-1) = 3(-1)^2 = 3 > 0$ (steigend).
            \item Rechts von $x=0$ (z.B. $x=1$): $f'(1) = 3(1)^2 = 3 > 0$ (steigend).
        \end{itemize}
        Da $f'(x)$ bei $x=0$ keinen Vorzeichenwechsel hat (von $+$ nach $+$), liegt hier kein Extrempunkt vor.

    \item \textbf{Überprüfung der Wendepunktbedingungen:}
        $f''(x_S) = f''(0) = 6 \cdot 0 = 0$. (Notwendige Bedingung für Wendepunkt erfüllt).
        $f'''(x_S) = f'''(0) = 6 \neq 0$. (Hinreichende Bedingung für Wendepunkt erfüllt).
        Da $f'(0)=0$ und bei $x=0$ ein Wendepunkt vorliegt, ist $P(0|f(0))$ ein Sattelpunkt.

    \item \textbf{Koordinaten des Sattelpunkts:}
        $f(0) = 0^3 = 0$.
        Der Sattelpunkt ist $S(0|0)$.
\end{enumerate}
\begin{center}
    \includegraphics[width=0.7\textwidth]{grafiken/Sattelpunkt_xhoch3.png}
    \captionof{figure}{Sattelpunkt der Funktion $f(x)=x^3$ im Ursprung}
    \label{fig:sattelpunkt_x3}
\end{center}
\end{beispielumgebung}

\begin{beispielumgebung}{Untersuchung auf Sattelpunkte bei \texorpdfstring{$f(x) = (x-1)^3 + 2$}{f(x) = (x-1) hoch 3 + 2}}
\begin{enumerate}
    \item \textbf{Ableitungen bilden (Kettenregel!):}
        $f(x) = (x-1)^3 + 2$
        $f'(x) = 3(x-1)^2 \cdot 1 + 0 = 3(x-1)^2$
        $f''(x) = 3 \cdot 2(x-1)^1 \cdot 1 = 6(x-1)$
        $f'''(x) = 6$

    \item \textbf{Potentielle Stellen für horizontale Tangenten:}
        $f'(x) = 0 \implies 3(x-1)^2 = 0 \implies (x-1)^2 = 0 \implies x-1=0 \implies x = 1$.
        Kritische Stelle ist $x_S = 1$.

    \item \textbf{Überprüfung mit Vorzeichenwechsel von $f'(x)$:}
        $f'(x) = 3(x-1)^2$. Da $(x-1)^2$ immer $\ge 0$ ist (und nur für $x=1$ gleich Null ist), ist $f'(x) \ge 0$ für alle $x$.
        Es gibt keinen Vorzeichenwechsel bei $x=1$. Also kein Extrempunkt.

    \item \textbf{Überprüfung der Wendepunktbedingungen:}
        $f''(x_S) = f''(1) = 6(1-1) = 6 \cdot 0 = 0$. (Notwendige Bedingung erfüllt).
        $f'''(x_S) = f'''(1) = 6 \neq 0$. (Hinreichende Bedingung erfüllt).
        Da $f'(1)=0$ und bei $x=1$ ein Wendepunkt vorliegt, ist $P(1|f(1))$ ein Sattelpunkt.

    \item \textbf{Koordinaten des Sattelpunkts:}
        $f(1) = (1-1)^3 + 2 = 0^3 + 2 = 2$.
        Der Sattelpunkt ist $S(1|2)$.
\end{enumerate}
\end{beispielumgebung}

\begin{tippumgebung}{Systematisches Vorgehen bei der Suche nach Sattelpunkten}
\begin{enumerate}
    \item Berechne $f'(x)$ und $f''(x)$ (und ggf. $f'''(x)$).
    \item Setze $f'(x)=0$ und löse nach $x$. Das sind die Kandidaten $x_E$ für Extrem- oder Sattelpunkte.
    \item Setze diese Kandidaten $x_E$ in $f''(x)$ ein:
        \begin{itemize}
            \item Wenn $f''(x_E) \neq 0$, dann liegt ein Extrempunkt vor (Hochpunkt bei $f''(x_E)<0$, Tiefpunkt bei $f''(x_E)>0$).
            \item Wenn $f''(x_E) = 0$, dann könnte ein Sattelpunkt vorliegen. Fahre fort mit Schritt 4.
        \end{itemize}
    \item Wenn $f'(x_E)=0$ und $f''(x_E)=0$:
        \begin{itemize}
            \item Prüfe $f'''(x_E)$. Wenn $f'''(x_E) \neq 0$, dann ist es ein Sattelpunkt.
            \item Alternativ (oder wenn $f'''(x_E)$ auch $0$ ist): Prüfe das Vorzeichenwechselkriterium für $f'(x)$ an der Stelle $x_E$. Gibt es keinen Vorzeichenwechsel, ist es ein Sattelpunkt (vorausgesetzt, es ist auch ein Wendepunkt, d.h. $f''(x)$ wechselt an $x_E$ das Vorzeichen).
        \end{itemize}
\end{enumerate}
\end{tippumgebung}

\begin{aufgabenumgebung}{Extrempunkte und Sattelpunkte finden und identifizieren}
Untersuche die folgenden Funktionen auf Extrempunkte und Sattelpunkte. Gib jeweils die Koordinaten und die Art des Punktes an. Nutze die notwendigen und hinreichenden Bedingungen.

\begin{enumerate}
    \item $f(x) = x^3 - 3x^2 + 3x + 1$
        \begin{tippumgebung}{Analyse dieser kubischen Funktion}
        Bestimme $f'(x)$ und $f''(x)$. Gibt es Stellen $x_S$, für die $f'(x_S)=0$ und $f''(x_S)=0$ gilt? Überprüfe dann $f'''(x_S)$ oder das Vorzeichenwechselverhalten von $f'(x)$ an diesen Stellen. Hat die Funktion auch 'normale' Extrempunkte?
        \end{tippumgebung}

    \item $g(x) = \frac{1}{4}x^4 - \frac{1}{2}x^2 + 1$
        \begin{tippumgebung}{Polynom 4. Grades}
        Diese Funktion ähnelt einer der ursprünglichen Aufgaben. Untersuche hier sorgfältig alle kritischen Stellen ($g'(x)=0$). Prüfe, ob bei $g''(x_E)=0$ eventuell ein Sattelpunkt vorliegt oder ob es sich trotz $g''(x_E)=0$ um ein Extremum handeln könnte (dann das VZW-Kriterium für $g'(x)$ nutzen!). Hat diese Funktion Sattelpunkte?
        \end{tippumgebung}

    \item $h(x) = x^5 - \frac{5}{3}x^3$
        \begin{tippumgebung}{Ein Polynom 5. Grades}
        Bestimme $h'(x)$, $h''(x)$ und $h'''(x)$.
        \begin{itemize}
            \item Wo ist $h'(x)=0$?
            \item Welche dieser Stellen sind auch Nullstellen von $h''(x)$?
            \item Was sagt $h'''(x)$ an diesen speziellen Stellen aus? Gibt es sowohl Extrempunkte als auch Sattelpunkte?
        \end{itemize}
        \end{tippumgebung}

    \item \textbf{Für Experten:} Konstruiere eine Polynomfunktion $k(x)$ vom Grad 5, die bei $x=0$ einen Sattelpunkt hat und bei $x=2$ einen Tiefpunkt. (Beginne mit den Bedingungen für die Ableitungen.)
\end{enumerate}
\end{aufgabenumgebung}

\begin{fehlerboxumgebung}{Sattelpunkte nicht übersehen}
\begin{itemize}
    \item Wenn $f'(x_E)=0$ und $f''(x_E)=0$ ist, ist die Untersuchung noch nicht abgeschlossen! Es könnte ein Sattelpunkt sein. Man darf nicht vorschnell auf 'kein Extremum' schließen, ohne die Wendepunkteigenschaft (z.B. mit $f'''(x_E) \neq 0$ oder VZW von $f''$) zu prüfen.
    \item Ein Sattelpunkt ist ein Wendepunkt, aber nicht jeder Wendepunkt ist ein Sattelpunkt (nur die mit horizontaler Tangente).
\end{itemize}
\end{fehlerboxumgebung}

\begin{kurzknappumgebung}{Sattelpunkte}
\begin{itemize}
    \item \textbf{Definition:} Ein Punkt auf dem Graphen, an dem die Tangente waagerecht ist ($f'(x_S)=0$), aber kein Extremum vorliegt (kein Vorzeichenwechsel von $f'(x)$).
    \item \textbf{Eigenschaft:} Sattelpunkte sind immer auch Wendepunkte (Krümmungswechsel).
    \item \textbf{Bedingungen (üblich):} $f'(x_S)=0$ UND $f''(x_S)=0$ UND $f'''(x_S) \neq 0$.
\end{itemize}
\end{kurzknappumgebung}








\subsubsection{Nullstellen von Polynomen höheren Grades – Die Polynomdivision}
\label{subsubsec:polynomdivision}

Für quadratische Funktionen (Grad 2) haben wir die p-q-Formel oder die Mitternachtsformel, um Nullstellen zu finden. Aber was machen wir bei Polynomen höheren Grades, z.B. $f(x) = x^3 - 2x^2 - 5x + 6$? Hierfür gibt es keine allgemeine, einfache Lösungsformel wie bei quadratischen Gleichungen.

Ein wichtiges Werkzeug, um die Nullstellen solcher Polynome zu finden, ist die \textbf{Polynomdivision}. Die Idee ist, wenn wir eine Nullstelle $x_1$ des Polynoms $f(x)$ kennen (z.B. durch Raten oder aus dem Kontext der Aufgabe), dann wissen wir, dass $(x-x_1)$ ein Faktor von $f(x)$ sein muss.
Stell dir vor, du multiplizierst Linearfaktoren: $(x-x_1)(x-x_2)(x-x_3)$ würde ein Polynom 3. Grades ergeben, dessen Nullstellen $x_1, x_2, x_3$ sind. Die Polynomdivision ist der umgekehrte Prozess: Wir teilen das gegebene Polynom durch einen bekannten Faktor $(x-x_1)$, um ein Restpolynom zu erhalten, dessen Grad um 1 niedriger ist.

\begin{merksatzumgebung}{Polynomdivision}
Wenn $x_1$ eine Nullstelle eines Polynoms $f(x)$ vom Grad $n$ ist, dann ist der Term $(x-x_1)$ ein Linearfaktor von $f(x)$. Man kann $f(x)$ durch $(x-x_1)$ ohne Rest teilen:
\[ f(x) : (x-x_1) = g(x) \]
wobei $g(x)$ ein Restpolynom vom Grad $n-1$ ist. Die weiteren Nullstellen von $f(x)$ sind dann die Nullstellen von $g(x)$.
Wenn $f(x)$ vom Grad 3 ist, ist $g(x)$ vom Grad 2, und dessen Nullstellen können wir mit der p-q-Formel oder Mitternachtsformel finden.
\end{merksatzumgebung}

Das Verfahren der Polynomdivision ähnelt der schriftlichen Division von Zahlen.

\begin{beispielumgebung}{Polynomdivision durchführen}
Gegeben ist das Polynom $f(x) = x^3 - 2x^2 - 5x + 6$. Wir haben durch Probieren (z.B. Teiler des konstanten Gliedes 6: $\pm 1, \pm 2, \pm 3, \pm 6$) herausgefunden, dass $x_1=1$ eine Nullstelle ist, denn $f(1) = 1^3 - 2(1)^2 - 5(1) + 6 = 1 - 2 - 5 + 6 = 0$.
Wir teilen nun $f(x)$ durch den Linearfaktor $(x-1)$:
\[ (x^3 - 2x^2 - 5x + 6) : (x-1) \]

\polyset{style=C, vars=x, div=:} % Setzt den Divisionsoperator auf ':'
\polylongdiv{x^3 - 2x^2 - 5x + 6}{x-1}

Das Ergebnis der Polynomdivision ist $x^2 - x - 6$.
Nun suchen wir die Nullstellen dieses quadratischen Restpolynoms $g(x) = x^2 - x - 6$:
$x^2 - x - 6 = 0$. Mit der p-q-Formel ($p=-1, q=-6$):
$x_{2,3} = - \frac{-1}{2} \pm \sqrt{(\frac{-1}{2})^2 - (-6)} = \frac{1}{2} \pm \sqrt{\frac{1}{4} + \frac{24}{4}} = \frac{1}{2} \pm \sqrt{\frac{25}{4}} = \frac{1}{2} \pm \frac{5}{2}$.
$x_2 = \frac{1+5}{2} = \frac{6}{2} = 3$.
$x_3 = \frac{1-5}{2} = \frac{-4}{2} = -2$.
Die Nullstellen der Funktion $f(x) = x^3 - 2x^2 - 5x + 6$ sind also $x_1=1$, $x_2=3$ und $x_3=-2$.
\end{beispielumgebung}

\begin{tippumgebung}{Nullstelle raten und Koeffizientenvergleich}
Das Raten einer ersten Nullstelle (oft ganzzahlige Teiler des konstanten Gliedes) ist ein üblicher erster Schritt. Wenn die Polynomdivision zu aufwendig erscheint oder nicht explizit gefordert ist, kann man nach dem Finden einer Nullstelle $x_1$ auch den \textbf{Koeffizientenvergleich} nutzen, wie im Tipp der Aufgabe $s(x)$ in der nächsten Aufgabensammlung gezeigt.
\end{tippumgebung}
% --- ENDE DES NEUEN ABSCHNITTS ZUR POLYNOMDIVISION ---


\begin{aufgabenumgebung}{Nullstellen finden – Übung und Vertiefung}
Berechne die Nullstellen der folgenden Funktionen. Entscheide selbst, welche Methode (Ausklammern, p-q-Formel, Mitternachtsformel, Substitution, Polynomdivision nach Raten einer Nullstelle) am besten geeignet ist. Überprüfe bei quadratischen Gleichungen immer zuerst die Diskriminante, um die Anzahl der erwarteten reellen Nullstellen zu bestimmen.
\begin{enumerate}
    \item $f(x) = x^2 - x - 6$
        \begin{tippumgebung}{Lösungsweg}
        Dies ist eine Standard-quadratische Gleichung. p-q-Formel oder Mitternachtsformel sind hier gut geeignet.
        \end{tippumgebung}

    \item $g(x) = -2x^2 + 12x - 18$ 
        \begin{tippumgebung}{Besondere Diskriminante}
        Was sagt $D=0$ über den Graphen und die Art der Nullstelle aus?
        \end{tippumgebung}

    \item $h(x) = x^2 + 2x + 5$
        \begin{tippumgebung}{Keine reellen Nullstellen?}
        Was bedeutet es für den Graphen, wenn die Diskriminante $D<0$ ist?
        \end{tippumgebung}

    \item $k(x) = 3x^2 - 12$ 
        \begin{tippumgebung}{Vereinfachung}
        Hier geht es auch ohne Mitternachtsformel! Denke an das direkte Auflösen nach $x^2$. (Siehe Infobox zu Sonderfällen).
        \end{tippumgebung}

    \item $m(x) = -0.5x^2 + 2x$
        \begin{tippumgebung}{Ausklammern}
        Auch hier ist Ausklammern der schnellste Weg! (Siehe Infobox zu Sonderfällen).
        \end{tippumgebung}

    \item \textbf{Polynom 3. Grades durch Ausklammern:}
        $p(x) = x^3 - 5x^2 + 6x$
        \begin{tippumgebung}{Strategie}
        Klammere zuerst den gemeinsamen Faktor $x$ aus. Übrig bleibt ein quadratischer Term, dessen Nullstellen du mit den bekannten Formeln finden kannst.
        \end{tippumgebung}

    \item \textbf{Biquadratische Funktion:}
        $q(x) = x^4 - 10x^2 + 9$
        \begin{tippumgebung}{Substitution}
        Ersetze (substituiere) $x^2$ durch eine neue Variable, z.B. $z = x^2$. Dadurch erhältst du eine quadratische Gleichung in $z$. Löse diese nach $z$ und substituiere dann zurück ($x^2 = z_1$, $x^2 = z_2$), um die Nullstellen für $x$ zu finden. Achtung: Nicht jede Lösung für $z$ führt zu reellen Lösungen für $x$!
        \end{tippumgebung}

    \item \textbf{Produkt aus Linearfaktoren (versteckt):}
        $r(x) = (x^2-4)(x^2+x-2)$
        \begin{tippumgebung}{Satz vom Nullprodukt und Faktorisieren}
        Ein Produkt ist Null, wenn einer der Faktoren Null ist. Setze also jeden Klammerausdruck gleich Null. Der erste Faktor lässt sich mit der 3. binomischen Formel zerlegen. Für den zweiten Faktor kannst du die p-q-Formel verwenden.
        \end{tippumgebung}

    \item \textbf{Funktion mit bekannter Nullstelle (Polynomdivision anwenden):}
        Gegeben ist die Funktion $s(x) = x^3 - 2x^2 - 5x + 6$. Es ist bekannt, dass $x_1=1$ eine Nullstelle ist. Finde die anderen Nullstellen mithilfe der Polynomdivision.
        \textit{(Vergleiche auch den Tipp mit dem Koeffizientenvergleich aus der vorherigen Version dieser Aufgabe.)}

    \item \textbf{Nullstellen und Parameter:}
        Für welche Werte des Parameters $k$ hat die Funktion $f_k(x) = x^2 - 2kx + (k+2)$ genau eine, zwei oder keine reelle(n) Nullstelle(n)?
        \begin{tippumgebung}{Diskriminante}
        Untersuche die Diskriminante $D = b^2-4ac$ der quadratischen Gleichung $f_k(x)=0$ in Abhängigkeit von $k$.
        Setze $D=0$ für eine Nullstelle, $D>0$ für zwei und $D<0$ für keine.
        \end{tippumgebung}
    \item \textbf{Anwendung des Satzes von Vieta (Kopfrechnen für Profis):}
        Versuche, die Nullstellen der folgenden quadratischen Funktionen (in Normalform $x^2+px+q=0$) durch 'scharfes Hinsehen' mit dem Satz von Vieta zu finden. Suche also zwei Zahlen $x_1, x_2$, für die gilt: $x_1+x_2 = -p$ und $x_1 \cdot x_2 = q$.
        \begin{enumerate}[label=(\alph*)]
            \item $f(x) = x^2 - 5x + 6$
            \item $g(x) = x^2 + 2x - 8$
            \item $h(x) = x^2 - 7x + 10$
        \end{enumerate}
        \begin{tippumgebung}{Satz von Vieta nutzen}
        Für $f(x) = x^2 - 5x + 6$: Hier ist $p=-5$ und $q=6$. Du suchst also zwei Zahlen, deren Summe $-p = -(-5) = 5$ ist und deren Produkt $q=6$ ist. Welche Zahlen könnten das sein? (Denke an die Teiler von 6).
        \end{tippumgebung}
\end{enumerate}
\end{aufgabenumgebung}





\subsection{Kurvendiskussion von Polynomfunktionen – Das Gesamtpaket}
\label{subsec:kurvendiskussion_polynome}

Mit unserem Wissen über Ableitungen und Grenzwerte können wir nun eine vollständige Kurvendiskussion für Polynomfunktionen durchführen. Das Ziel ist es, ein umfassendes Bild vom Verlauf des Graphen zu erhalten, ohne jeden einzelnen Punkt berechnen zu müssen.

\begin{merksatzumgebung}{Checkliste für die Kurvendiskussion eines Polynoms}
Für eine Polynomfunktion $f(x) = a_n x^n + \dots + a_0$ untersuchen wir typischerweise:
\begin{enumerate}
    \item \textbf{Definitionsbereich ($D_f$):} Für Polynome immer $D_f = \mathbb{R}$.
    \item \textbf{Symmetrie:}
        \begin{itemize}
            \item \textbf{Achsensymmetrie zur y-Achse?}: Gilt $f(-x) = f(x)$? (Tritt auf, wenn $f(x)$ nur gerade Exponenten von $x$ enthält, z.B. $f(x)=x^4-2x^2+1$).
            \item \textbf{Punktsymmetrie zum Ursprung?}: Gilt $f(-x) = -f(x)$? (Tritt auf, wenn $f(x)$ nur ungerade Exponenten von $x$ enthält, z.B. $f(x)=x^3-x$).
            \item Ansonsten ist keine einfache Symmetrie zum Koordinatensystem vorhanden (aber ggf. zu einem anderen Punkt oder einer anderen Achse, was hier seltener untersucht wird).
        \end{itemize}
    \item \textbf{Verhalten im Unendlichen (Grenzwerte):} $\lim_{x \to \pm \infty} f(x)$ (siehe Abschnitt \ref{subsec:grenzwerte}).
    \item \textbf{Schnittpunkt mit der y-Achse ($P_y$):} Berechne $f(0)$. Der Punkt ist $P_y(0|f(0))$. (Bei Polynomen ist $f(0)=a_0$, der konstante Term).
    \item \textbf{Nullstellen ($N_i$):} Setze $f(x)=0$ und löse die Gleichung.
        \begin{itemize}
            \item Bei Grad $n=2$: p-q-Formel oder Mitternachtsformel.
            \item Bei Grad $n>2$:
                \begin{itemize}
                    \item $x$ ausklammern, falls kein konstanter Term $a_0$ vorhanden ist ($a_0=0$).
                    \item Eine Nullstelle $x_1$ raten (oft $\pm 1, \pm 2, \dots$, Teiler von $a_0$) und dann Polynomdivision durch $(x-x_1)$ durchführen, um den Grad zu reduzieren. \textit{Hinweis: Polynomdivision wird in diesem Skript nicht explizit behandelt. Aufgaben werden so gestellt, dass sie ohne lösbar sind, z.B. durch Ausklammern oder Substitution.}
                    \item Bei biquadratischen Gleichungen (z.B. $ax^4+bx^2+c=0$): Substitution $z=x^2$ verwenden.
                \end{itemize}
        \end{itemize}
    \item \textbf{Erste Ableitung $f'(x)$ bilden.}
    \item \textbf{Extremstellen (Hoch-/Tiefpunkte):}
        \begin{itemize}
            \item Notwendige Bedingung: $f'(x_E)=0$. Lösungen sind potentielle Extremstellen $x_E$.
            \item Hinreichende Bedingung: Vorzeichenwechsel von $f'(x)$ an $x_E$ ODER $f''(x_E) \neq 0$.
                \begin{itemize}
                    \item $f'(x)$ VZW von $+$ nach $-$ ODER $f''(x_E) < 0 \implies$ Hochpunkt.
                    \item $f'(x)$ VZW von $-$ nach $+$ ODER $f''(x_E) > 0 \implies$ Tiefpunkt.
                \end{itemize}
            \item y-Koordinaten: $y_E = f(x_E)$. Punkte $H(x_E|y_E)$ oder $T(x_E|y_E)$.
        \end{itemize}
    \item \textbf{Monotonieverhalten:} Intervalle bestimmen, in denen $f'(x)>0$ (steigend) oder $f'(x)<0$ (fallend).
    \item \textbf{Zweite Ableitung $f''(x)$ bilden.}
    \item \textbf{Wendepunkte ($W$):}
        \begin{itemize}
            \item Notwendige Bedingung: $f''(x_W)=0$. Lösungen sind potentielle Wendestellen $x_W$.
            \item Hinreichende Bedingung: Vorzeichenwechsel von $f''(x)$ an $x_W$ ODER $f'''(x_W) \neq 0$.
            \item y-Koordinaten: $y_W = f(x_W)$. Punkt $W(x_W|y_W)$.
        \end{itemize}
    \item \textbf{Krümmungsverhalten:} Intervalle bestimmen, in denen $f''(x)>0$ (linksgekrümmt) oder $f''(x)<0$ (rechtsgekrümmt).
    \item \textbf{Wertetabelle (optional):} Für wichtige Punkte und zur Verfeinerung der Skizze.
    \item \textbf{Skizze des Graphen:} Alle berechneten Punkte und Informationen verwenden.
\end{enumerate}
\end{merksatzumgebung}

\begin{beispielumgebung}[Kurvendiskussion Polynom 3. Grades]{Untersuchung von $f(x) = \frac{1}{3}x^3 - x^2 - 3x$}
\begin{enumerate}
    \item \textbf{Definitionsbereich:} $D_f = \mathbb{R}$.
    \item \textbf{Symmetrie:}
        $f(-x) = \frac{1}{3}(-x)^3 - (-x)^2 - 3(-x) = -\frac{1}{3}x^3 - x^2 + 3x$.
        $f(-x) \neq f(x)$ und $f(-x) \neq -f(x)$. Keine einfache Symmetrie zum Koordinatensystem.
    \item \textbf{Verhalten im Unendlichen:} Höchste Potenz ist $\frac{1}{3}x^3$ ($n=3$ ungerade, $a_3=\frac{1}{3}>0$).
        $\lim_{x \to \infty} f(x) = +\infty$; $\lim_{x \to -\infty} f(x) = -\infty$. (Kommt von links unten, geht nach rechts oben).
    \item \textbf{y-Achsenabschnitt:} $f(0) = \frac{1}{3}(0)^3 - (0)^2 - 3(0) = 0$. Also $P_y(0|0)$.
    \item \textbf{Nullstellen:} $f(x)=0 \implies \frac{1}{3}x^3 - x^2 - 3x = 0$.
        Wir können $x$ ausklammern: $x(\frac{1}{3}x^2 - x - 3) = 0$.
        Eine Nullstelle ist $x_1 = 0$.
        Für die anderen lösen wir $\frac{1}{3}x^2 - x - 3 = 0$. Multiplizieren mit 3: $x^2 - 3x - 9 = 0$.
        p-q-Formel ($p=-3, q=-9$):
        $x_{2,3} = - \frac{-3}{2} \pm \sqrt{(\frac{-3}{2})^2 - (-9)} = \frac{3}{2} \pm \sqrt{\frac{9}{4} + \frac{36}{4}} = \frac{3}{2} \pm \sqrt{\frac{45}{4}} = \frac{3}{2} \pm \frac{\sqrt{45}}{2} = \frac{3 \pm \sqrt{9 \cdot 5}}{2} = \frac{3 \pm 3\sqrt{5}}{2}$.
        $x_2 = \frac{3 + 3\sqrt{5}}{2} \approx \frac{3 + 3 \cdot 2.236}{2} \approx \frac{3+6.708}{2} \approx \frac{9.708}{2} \approx 4.85$.
        $x_3 = \frac{3 - 3\sqrt{5}}{2} \approx \frac{3 - 6.708}{2} \approx \frac{-3.708}{2} \approx -1.85$.
        Nullstellen: $N_1(0|0)$, $N_2(\frac{3+3\sqrt{5}}{2}|0)$, $N_3(\frac{3-3\sqrt{5}}{2}|0)$.
    \item \textbf{Erste Ableitung:} $f'(x) = x^2 - 2x - 3$.
    \item \textbf{Extremstellen:} $f'(x)=0 \implies x^2 - 2x - 3 = 0$.
        p-q-Formel ($p=-2, q=-3$):
        $x_{E1,E2} = - \frac{-2}{2} \pm \sqrt{(\frac{-2}{2})^2 - (-3)} = 1 \pm \sqrt{1+3} = 1 \pm \sqrt{4} = 1 \pm 2$.
        $x_{E1} = 1+2 = 3$.
        $x_{E2} = 1-2 = -1$.
        Potentielle Extremstellen bei $x_{E_1}=3$ und $x_{E_2}=-1$.
    \item \textbf{Zweite Ableitung:} $f''(x) = (x^2 - 2x - 3)' = 2x - 2$.
    \item \textbf{Art der Extremstellen mit $f''$ prüfen:}
        $f''(3) = 2(3) - 2 = 6-2=4 > 0 \implies$ Tiefpunkt bei $x=3$.
        $y_T = f(3) = \frac{1}{3}(3)^3 - (3)^2 - 3(3) = 9 - 9 - 9 = -9$. Tiefpunkt $T(3|-9)$.
        $f''(-1) = 2(-1) - 2 = -2-2=-4 < 0 \implies$ Hochpunkt bei $x=-1$.
        $y_H = f(-1) = \frac{1}{3}(-1)^3 - (-1)^2 - 3(-1) = -\frac{1}{3} - 1 + 3 = -\frac{1}{3} + 2 = \frac{5}{3}$. Hochpunkt $H(-1|\frac{5}{3})$.
    \item \textbf{Monotonie:} Bestimmt durch Vorzeichen von $f'(x)=(x-3)(x+1)$.
        \begin{itemize}
            \item $x < -1$ (z.B. $x=-2$): $f'(-2)=(-)(-)=+ \implies$ steigend.
            \item $-1 < x < 3$ (z.B. $x=0$): $f'(0)=(+)(-)=- \implies$ fallend.
            \item $x > 3$ (z.B. $x=4$): $f'(4)=(+)(+)=+ \implies$ steigend.
        \end{itemize}
    \item \textbf{Wendepunkte:} $f''(x_W)=0 \implies 2x_W - 2 = 0 \implies 2x_W=2 \implies x_W=1$.
        Dritte Ableitung: $f'''(x) = (2x-2)' = 2$.
        $f'''(1) = 2 \neq 0 \implies$ Wendepunkt bei $x_W=1$.
        $y_W = f(1) = \frac{1}{3}(1)^3 - (1)^2 - 3(1) = \frac{1}{3} - 1 - 3 = \frac{1}{3} - 4 = -\frac{11}{3}$.
        Wendepunkt $W(1|-\frac{11}{3})$.
    \item \textbf{Krümmungsverhalten:} Bestimmt durch Vorzeichen von $f''(x)=2x-2$.
        \begin{itemize}
            \item $x < 1$: $f''(x) < 0 \implies$ rechtsgekrümmt.
            \item $x > 1$: $f''(x) > 0 \implies$ linksgekrümmt.
        \end{itemize}
    \item \textbf{Skizze:}
\begin{center}
    \includegraphics[width=0.9\textwidth]{grafiken/Kurvendiskussion_Polynom3.png}
    \captionof{figure}{Graph von $f(x) = \frac{1}{3}x^3 - x^2 - 3x$}
    \label{fig:kurvendisk_poly3}
\end{center}
\end{enumerate}
\end{beispielumgebung}

\begin{aufgabenumgebung}{Kurvendiskussionen von Polynomen}
Führe eine vollständige Kurvendiskussion (alle Punkte der Checkliste) für die folgenden Funktionen durch und skizziere jeweils den Graphen.
\begin{enumerate}
    \item \textbf{Quadratische Funktion als Wiederholung:} $f(x) = -x^2 + 4x - 3$. Vergleiche den gefundenen Extrempunkt mit dem Scheitelpunkt, den du mit der Formel $x_S = -b/(2a)$ oder quadratischer Ergänzung bestimmen kannst.
    \item \textbf{Kubische Funktion (einfache Nullstellen):} $g(x) = x^3 - 4x$. (Tipp: $x$ ausklammern für Nullstellen).
    \item \textbf{Biquadratische Funktion:} $h(x) = x^4 - 5x^2 + 4$. (Tipp für Nullstellen: Substituiere $z=x^2$, löse die quadratische Gleichung für $z$ und substituiere dann zurück. Beachte, dass diese Funktion achsensymmetrisch zur y-Achse ist!)
    \item \textbf{Für Experten (Polynom 3. Grades mit Raten):} $k(x) = x^3 - 7x - 6$. (Tipp: Eine ganzzahlige Nullstelle ist ein Teiler des konstanten Gliedes -6. Probiere $\pm 1, \pm 2, \pm 3, \pm 6$. Wenn du eine Nullstelle $x_1$ gefunden hast, kannst du den Term $(x-x_1)$ durch Polynomdivision (hier nicht erklärt, aber in Schulbüchern zu finden) oder durch einen anderen Trick (Koeffizientenvergleich) abspalten, um eine quadratische Restgleichung zu erhalten. Alternativ: Wenn du später die Produktregel kennst, kannst du versuchen, die Funktion geschickt zu faktorisieren, falls möglich, oder du nutzt einen Taschenrechner/Software, um die Nullstellen numerisch zu finden und konzentrierst dich auf die anderen Aspekte der Kurvendiskussion.)
        \item \textbf{Bewegung eines Objekts (Anwendung):}
        Die Höhe $h$ (in Metern) eines senkrecht nach oben geworfenen Steins nach $t$ Sekunden wird durch die Funktion $h(t) = -5t^2 + 20t + 1$ beschrieben (für $t \ge 0$ und solange $h(t) \ge 0$).
        \begin{enumerate}
            \item Bestimme die Geschwindigkeit $v(t) = h'(t)$ und die Beschleunigung $a(t) = h''(t)$ des Steins.
            \item Zu welchen Zeiten ist die Geschwindigkeit positiv (Stein steigt), negativ (Stein fällt) oder Null? Interpretiere diese Ergebnisse im Kontext der Bewegung.
            \item Wann erreicht der Stein seine maximale Höhe und wie hoch ist diese? (Tipp: Extrempunkt von $h(t)$)
            \item Wann kehrt der Stein zum Boden zurück (Annahme $h(t) \ge 0$)? (Tipp: Nullstelle von $h(t)$)
            \item Mit welcher Geschwindigkeit trifft der Stein auf dem Boden auf?
        \end{enumerate}
\end{enumerate}
\end{aufgabenumgebung}

\subsubsection{Nullstellen aus faktorisierter Form – Der Satz vom Nullprodukt}
\label{subsec:nullprodukt}

Manchmal liegen Polynomfunktionen bereits in einer \textbf{faktorisierten Form} vor, oder sie lassen sich leicht in eine solche überführen (z.B. durch Ausklammern). Diese Form ist besonders praktisch, um die Nullstellen direkt abzulesen. Das Zauberwort hierfür ist der \textbf{Satz vom Nullprodukt}.

\begin{merksatzumgebung}{Satz vom Nullprodukt}
Ein Produkt ist genau dann Null, wenn mindestens einer seiner Faktoren Null ist.
\[ A \cdot B = 0 \quad \Leftrightarrow \quad A=0 \text{ oder } B=0 \]
Das gilt natürlich auch für Produkte mit mehr als zwei Faktoren: $A \cdot B \cdot C = 0 \Leftrightarrow A=0 \text{ oder } B=0 \text{ oder } C=0$.
\end{merksatzumgebung}

Wenn eine Polynomfunktion in der Form $f(x) = k \cdot (x-x_1) \cdot (x-x_2) \cdot \dots \cdot (x-x_n)$ gegeben ist (wobei $k$ ein konstanter Faktor ist und $x_1, x_2, \dots, x_n$ die sogenannten Linearfaktoren sind), dann sind die Nullstellen der Funktion genau die Werte $x_1, x_2, \dots, x_n$. Denn wenn $x$ einen dieser Werte annimmt, wird einer der Klammerausdrücke Null, und damit das gesamte Produkt.

\begin{beispielumgebung}{Nullstellen aus faktorisierter Form ablesen}
\begin{enumerate}
    \item $f(x) = 2(x-1)(x+3)(x-5)$
    Die Funktion wird Null, wenn:
    \begin{itemize}
        \item $x-1=0 \implies x_1 = 1$
        \item $x+3=0 \implies x_2 = -3$
        \item $x-5=0 \implies x_3 = 5$
    \end{itemize}
    Die Nullstellen sind also $1, -3$ und $5$.

    \item $g(x) = -0.5x(x-2)^2(x+1)$
    Die Funktion wird Null, wenn:
    \begin{itemize}
        \item $x=0 \implies x_1 = 0$
        \item $(x-2)^2=0 \implies x-2=0 \implies x_2 = 2$ (Dies ist eine \textbf{doppelte Nullstelle}, da der Faktor $(x-2)$ zweimal vorkommt. An einer doppelten Nullstelle berührt der Graph die x-Achse, ohne sie zu schneiden.)
        \item $x+1=0 \implies x_3 = -1$
    \end{itemize}
    Die Nullstellen sind $0, 2$ (doppelt) und $-1$.

    \item $h(x) = x^3 - 4x$
    Hier müssen wir zuerst faktorisieren, indem wir $x$ ausklammern:
    $h(x) = x(x^2-4)$
    Den Term $x^2-4$ können wir mit der dritten binomischen Formel weiter faktorisieren: $x^2-4 = (x-2)(x+2)$.
    Also: $h(x) = x(x-2)(x+2)$.
    Die Nullstellen sind $x_1=0, x_2=2, x_3=-2$.
\end{enumerate}
\end{beispielumgebung}

\begin{aufgabenumgebung}{Nullstellen aus faktorisierter Form bestimmen}
Bestimme die Nullstellen der folgenden Funktionen. Gib auch an, ob es sich um einfache oder mehrfache Nullstellen handelt. Eine Kurvendiskussion zu diesen Funktionen kann natürlich auch nie schaden.
\begin{enumerate}
    \item $f(x) = (x+4)(x-2.5)(x+1)$
    \item $g(x) = -3x^2(x-1)(x+2)^3$
    \item $h(x) = (x^2-9)(x+1)$ (Tipp: $x^2-9$ weiter faktorisieren!)
    \item $k(x) = 2x^4 - 8x^2$ (Tipp: Erst ausklammern, dann weiter überlegen.)
\end{enumerate}
\end{aufgabenumgebung}

\begin{warumwichtigumgebung}{Faktorisierte Form und Nullstellen}
Die faktorisierte Form einer Polynomfunktion ist extrem nützlich, weil sie uns die Nullstellen quasi 'auf dem Silbertablett serviert'. Bei Kurvendiskussionen ist die Bestimmung der Nullstellen oft ein wichtiger Schritt. Wenn eine Funktion bereits faktorisiert ist oder sich leicht faktorisieren lässt, erspart uns das oft das aufwendige Raten von Nullstellen und die Polynomdivision.
Außerdem gibt die Vielfachheit einer Nullstelle (einfach, doppelt, dreifach etc.) Auskunft über das Verhalten des Graphen an dieser Stelle (schneiden oder berühren der x-Achse).
\end{warumwichtigumgebung}

\begin{infoboxumgebung}{Anwendungen von Kurvendiskussionen}
Kurvendiskussionen sind nicht nur eine mathematische Übung. Sie sind entscheidend, um reale Prozesse zu verstehen und zu optimieren:
\begin{itemize}
    \item \textbf{Wirtschaft:} Gewinnmaximierung, Kostenminimierung (Extremwertaufgaben).
    \item \textbf{Technik:} Optimale Formen für Bauteile, Stabilitätsanalysen.
    \item \textbf{Naturwissenschaften:} Modellierung von Wachstumsprozessen, Zerfallsprozessen, Bewegungen. Die Stellen, an denen sich Änderungsraten ändern (Wendepunkte), können wichtige Übergänge in Systemen markieren.
\end{itemize}
Das Verständnis, wie sich Funktionen verhalten, ist ein Kernstück angewandter Mathematik.
\end{infoboxumgebung}

\subsection{Exkurs: Grenzwerte von Funktionen mit negativen Exponenten}
\label{subsec:grenzwerte_neg_exp}

Bisher haben wir uns Polynomfunktionen angesehen, die für alle reellen Zahlen definiert sind ($D_f = \mathbb{R}$). Es gibt aber auch wichtige Funktionen, die nicht überall definiert sind, insbesondere solche, bei denen $x$ im Nenner steht. Das sind die einfachsten Formen von \textbf{gebrochen-rationalen Funktionen}. Ein typisches Beispiel ist $f(x) = \frac{1}{x}$ oder allgemeiner $f(x) = \frac{a}{x^n} = ax^{-n}$ mit $n > 0$.

Für diese Funktionen ist die Stelle $x=0$ besonders interessant. Wenn wir $x=0$ in den Nenner einsetzen würden, würde dieser Null werden, und \textbf{durch Null darf man nicht teilen}! Das bedeutet, die Funktion ist an der Stelle $x=0$ nicht definiert. Man sagt, die Funktion hat an der Stelle $x=0$ eine \textbf{Definitionslücke}. Der Definitionsbereich solcher Funktionen ist also $D_f = \mathbb{R} \setminus \{0\}$ (alle reellen Zahlen außer der Null). Wir untersuchen nun das Verhalten der Funktion, wenn sich $x$ dieser Lücke nähert.

\begin{merksatzumgebung}{Verhalten von $f(x) = \frac{a}{x^n}$ für $x \to 0$ (Polstellen)}
Wir betrachten Funktionen der Form $f(x) = \frac{a}{x^n}$ (oder $ax^{-n}$), wobei $a \neq 0$ eine Konstante ist und $n$ eine positive ganze Zahl ($n=1, 2, 3, \dots$).
Wie bereits erwähnt, ist $x=0$ nicht im Definitionsbereich dieser Funktionen. Diese spezielle Art von Definitionslücke, bei der die Funktionswerte gegen Unendlich ($+\infty$ oder $-\infty$) streben, wenn sich $x$ der Lücke nähert, nennt man eine \textbf{Polstelle} (oder kurz Pol).
Der Graph der Funktion hat an einer Polstelle eine \textbf{senkrechte Asymptote}. Für $f(x) = \frac{a}{x^n}$ ist dies die y-Achse (mit der Gleichung $x=0$). Eine Asymptote ist eine Gerade, der sich der Graph der Funktion beliebig annähert, sie aber nie erreicht oder schneidet.

Das Verhalten der Funktionswerte $f(x)$, wenn $x$ sich der Polstelle $x=0$ nähert, hängt davon ab, ob der Exponent $n$ im Nenner gerade oder ungerade ist und vom Vorzeichen des Zählers $a$:

\textbf{Fall 1: $n$ ist ungerade} (z.B. $f(x) = \frac{a}{x}$, $f(x) = \frac{a}{x^3}$)
Die Potenz $x^n$ behält das Vorzeichen von $x$.
\begin{itemize}
    \item Wenn $a > 0$:
        \begin{itemize}
            \item $\lim_{x \to 0^+} f(x) = +\infty$ (nähert man sich der 0 von rechts (positive $x$-Werte), wird $x^n$ positiv und klein $\implies \frac{a}{x^n}$ wird sehr groß positiv)
            \item $\lim_{x \to 0^-} f(x) = -\infty$ (nähert man sich der 0 von links (negative $x$-Werte), wird $x^n$ negativ und klein $\implies \frac{a}{x^n}$ wird sehr groß negativ)
        \end{itemize}
    \item Wenn $a < 0$: (Die Vorzeichen der Grenzwerte kehren sich um)
        \begin{itemize}
            \item $\lim_{x \to 0^+} f(x) = -\infty$
            \item $\lim_{x \to 0^-} f(x) = +\infty$
        \end{itemize}
\end{itemize}
Man spricht hier von einer \textbf{Polstelle mit Vorzeichenwechsel} (abgekürzt VZW). Der Graph 'springt' von $-\infty$ nach $+\infty$ (oder umgekehrt) an der Polstelle.

\textbf{Fall 2: $n$ ist gerade} (z.B. $f(x) = \frac{a}{x^2}$, $f(x) = \frac{a}{x^4}$)
Die Potenz $x^n$ ist immer positiv (oder Null), egal ob $x$ positiv oder negativ ist.
\begin{itemize}
    \item Wenn $a > 0$:
        \begin{itemize}
            \item $\lim_{x \to 0^+} f(x) = +\infty$ (nähert man sich der 0 von rechts, wird $x^n$ positiv und klein $\implies \frac{a}{x^n}$ wird sehr groß positiv)
            \item $\lim_{x \to 0^-} f(x) = +\infty$ (nähert man sich der 0 von links, wird $x^n$ ebenfalls positiv und klein $\implies \frac{a}{x^n}$ wird sehr groß positiv)
        \end{itemize}
    \item Wenn $a < 0$: (Die Vorzeichen der Grenzwerte kehren sich um)
        \begin{itemize}
            \item $\lim_{x \to 0^+} f(x) = -\infty$
            \item $\lim_{x \to 0^-} f(x) = -\infty$
        \end{itemize}
\end{itemize}
Man spricht hier von einer \textbf{Polstelle ohne Vorzeichenwechsel}. Der Graph geht auf beiden Seiten der Polstelle entweder nach $+\infty$ oder auf beiden Seiten nach $-\infty$.

Die Schreibweisen $x \to 0^+$ (lies: 'x geht von rechts gegen Null', d.h. $x$ nähert sich 0 mit Werten, die größer als 0 sind) und $x \to 0^-$ (lies: 'x geht von links gegen Null', d.h. $x$ nähert sich 0 mit Werten, die kleiner als 0 sind) bezeichnen die \textbf{einseitigen Grenzwerte}.
\end{merksatzumgebung}

\begin{beispielumgebung}{Grenzwerte an Polstellen}
\begin{enumerate}
    \item $f(x) = \frac{1}{x}$ ($a=1 > 0$, $n=1$ ungerade). Definitionsbereich $D_f = \mathbb{R} \setminus \{0\}$.
        \begin{itemize}
            \item $\lim_{x \to 0^+} \frac{1}{x} = +\infty$ (z.B. für $x=0.001$ ist $1/x = 1000$)
            \item $\lim_{x \to 0^-} \frac{1}{x} = -\infty$ (z.B. für $x=-0.001$ ist $1/x = -1000$)
        \end{itemize}
        Polstelle bei $x=0$ mit Vorzeichenwechsel.
        Symmetrie: $f(-x) = \frac{1}{-x} = -\frac{1}{x} = -f(x) \implies$ Punktsymmetrie zum Ursprung.

    \item $g(x) = \frac{-2}{x^2}$ ($a=-2 < 0$, $n=2$ gerade). Definitionsbereich $D_g = \mathbb{R} \setminus \{0\}$.
        \begin{itemize}
            \item $\lim_{x \to 0^+} \frac{-2}{x^2} = -\infty$ (z.B. für $x=0.01$ ist $x^2=0.0001$, $\frac{-2}{0.0001} = -20000$)
            \item $\lim_{x \to 0^-} \frac{-2}{x^2} = -\infty$ (z.B. für $x=-0.01$ ist $x^2=0.0001$, $\frac{-2}{0.0001} = -20000$)
        \end{itemize}
        Polstelle bei $x=0$ ohne Vorzeichenwechsel.
        Symmetrie: $g(-x) = \frac{-2}{(-x)^2} = \frac{-2}{x^2} = g(x) \implies$ Achsensymmetrie zur y-Achse.
\end{enumerate}
\begin{center}
    \includegraphics[width=0.8\textwidth]{grafiken/Gebrochen_Rational_Verschoben.png}
    \captionof{figure}{Graph von $f(x)=\frac{2}{(x-1)^2}$}
    \label{fig:gebr_rat_verschoben}
\end{center}
\end{beispielumgebung}

\begin{aufgabenumgebung}{Grenzwerte und Symmetrie gebrochen-rationaler Grundfunktionen}
\begin{enumerate}
    \item Bestimme den Definitionsbereich und das Verhalten für $x \to 0^+$ und $x \to 0^-$ für die folgenden Funktionen. Gib auch an, ob es sich um eine Polstelle mit oder ohne Vorzeichenwechsel handelt.
        \begin{itemize}
            \item $f_1(x) = \frac{3}{x^3}$
            \item $f_2(x) = -\frac{1}{x^4}$
            \item $f_3(x) = \frac{10}{x}$
        \end{itemize}
    \item Untersuche die Symmetrie der Funktionen aus Teilaufgabe 1 (Achsensymmetrie zur y-Achse oder Punktsymmetrie zum Ursprung).
    \item \textbf{Zuordnung Aufgabe:} Ordne den folgenden Funktionsgraphen $k_1(x) = \frac{1}{x^2}$, $k_2(x) = -\frac{1}{x}$, $k_3(x) = \frac{2}{x^3}$, $k_4(x) = \frac{1}{x^4}$ die passenden Funktionsgleichungen zu. Begründe deine Entscheidung anhand des Verhaltens an der Polstelle $x=0$ und der Symmetrie.
    \begin{center}
        \includegraphics[width=0.9\textwidth]{grafiken/Zuordnung_Polstellen.png}
        \captionof{figure}{Graphen zur Zuordnung von Polstellenverhalten}
        \label{fig:zuordnung_polstellen}
    \end{center}
\end{enumerate}
\end{aufgabenumgebung}

Das Verständnis des Verhaltens von solchen Grundfunktionen an ihren Definitionslücken ist wichtig, da viele komplexere gebrochen-rationale Funktionen solche Terme enthalten.

\begin{kurzknappumgebung}{Verhalten an Polstellen ($f(x)=a/x^n$ bei $x=0$)}
\begin{itemize}
    \item \textbf{Definitionsbereich:} $D_f = \mathbb{R} \setminus \{0\}$ (Null ist nicht erlaubt, da man nicht durch Null teilen darf).
    \item \textbf{Polstelle:} Bei $x=0$ liegt eine Polstelle mit senkrechter Asymptote (y-Achse) vor.
    \item \textbf{Verhalten für $x \to 0$:}
        \begin{itemize}
            \item $n$ ungerade: Polstelle mit Vorzeichenwechsel (VZW). Die Funktionswerte gehen auf einer Seite gegen $+\infty$ und auf der anderen gegen $-\infty$. Das Vorzeichen von $a$ bestimmt, auf welcher Seite was passiert.
            \item $n$ gerade: Polstelle ohne Vorzeichenwechsel. Die Funktionswerte gehen auf beiden Seiten entweder gegen $+\infty$ (wenn $a>0$) oder gegen $-\infty$ (wenn $a<0$).
        \end{itemize}
    \item \textbf{Einseitige Grenzwerte} ($x \to 0^+$ und $x \to 0^-$) sind wichtig, um das Verhalten genau zu beschreiben.
\end{itemize}
\end{kurzknappumgebung}

\begin{fehlerboxumgebung}{Polstellen und Definitionsbereich}
\begin{itemize}
    \item \textbf{Definitionsbereich vergessen:} Immer zuerst den Definitionsbereich bestimmen! Eine Funktion kann nur dort Eigenschaften haben, wo sie auch definiert ist. Bei $f(x)=a/x^n$ ist $x=0$ \textbf{nicht} im Definitionsbereich.
    \item \textbf{Einseitige Grenzwerte verwechseln:} Achte genau darauf, ob du dich $x=0$ von positiven Werten ($x \to 0^+$) oder von negativen Werten ($x \to 0^-$) näherst, besonders bei ungeraden Exponenten $n$.
    \item \textbf{Vorzeichen von $a$ übersehen:} Das Vorzeichen von $a$ im Zähler kehrt die Richtung der 'Unendlichkeiten' um. Ist $a$ negativ, geht es z.B. bei $1/x^2$ nicht nach $+\infty$, sondern nach $-\infty$.
    \item \textbf{Polstelle mit Nullstelle verwechseln:} Eine Polstelle ist eine Definitionslücke, an der die Funktion 'explodiert'. Eine Nullstelle ist ein Punkt, an dem der Graph die x-Achse schneidet ($f(x)=0$). Funktionen wie $1/x$ haben keine Nullstellen.
\end{itemize}
\end{fehlerboxumgebung}

% HIER FOLGEN DANN DIE WEITEREN ABLEITUNGSREGELN (PRODUKT, QUOTIENT, KETTE)



% Vorheriger Inhalt des Kapitels bis zur Fehlerbox 'Polstellen und Definitionsbereich'
% ... (siehe vorherige Canvas-Version) ...

\subsubsection{Verhalten von $f(x) = \frac{a}{x^n}$ im Unendlichen und horizontale Asymptoten}

Wir haben das Verhalten von $f(x) = \frac{a}{x^n}$ in der Nähe der Polstelle $x=0$ untersucht. Aber was passiert, wenn $x$ sehr groß positiv ($x \to \infty$) oder sehr groß negativ ($x \to -\infty$) wird?

Wenn $n$ eine positive ganze Zahl ist ($n \ge 1$), dann wird der Nenner $x^n$ für betragsmäßig große $x$ sehr groß.
\begin{itemize}
    \item Für $x \to \infty$ wird $x^n \to \infty$.
    \item Für $x \to -\infty$:
        \begin{itemize}
            \item Wenn $n$ gerade ist, wird $x^n \to \infty$.
            \item Wenn $n$ ungerade ist, wird $x^n \to -\infty$.
        \end{itemize}
\end{itemize}
In allen diesen Fällen wird der Betrag von $x^n$ unendlich groß. Wenn wir nun eine feste Zahl $a$ durch eine unendlich große Zahl teilen, nähert sich das Ergebnis immer mehr der Null.

\begin{merksatzumgebung}{Grenzwert von $f(x) = \frac{a}{x^n}$ für $x \to \pm\infty$}
Für jede Funktion der Form $f(x) = \frac{a}{x^n}$ mit $a \neq 0$ und $n \in \mathbb{N}, n \ge 1$ gilt:
\[ \lim_{x \to \infty} \frac{a}{x^n} = 0 \]
\[ \lim_{x \to -\infty} \frac{a}{x^n} = 0 \]
Der Graph der Funktion nähert sich also für sehr große positive und sehr große negative $x$-Werte der x-Achse (der Geraden $y=0$) an. Man sagt, die Funktion hat eine \textbf{waagerechte (horizontale) Asymptote} bei $y=0$.
\end{merksatzumgebung}

\begin{beispielumgebung}{Horizontale Asymptoten}
\begin{enumerate}
    \item $f(x) = \frac{1}{x}$:
        $\lim_{x \to \infty} \frac{1}{x} = 0$ und $\lim_{x \to -\infty} \frac{1}{x} = 0$.
        Die x-Achse ($y=0$) ist eine waagerechte Asymptote.
        Zusammen mit der senkrechten Asymptote $x=0$ (y-Achse) ergibt sich das typische Bild einer Hyperbel.

    \item $g(x) = \frac{-2}{x^2}$:
        $\lim_{x \to \infty} \frac{-2}{x^2} = 0$ und $\lim_{x \to -\infty} \frac{-2}{x^2} = 0$.
        Auch hier ist die x-Achse ($y=0$) eine waagerechte Asymptote.
\end{enumerate}
\end{beispielumgebung}

\subsubsection{Substitution – Ein mächtiges Werkzeug zum Verständnis}

Manchmal sehen Funktionen komplizierter aus, als sie sind. Die Idee der \textbf{Substitution} (Ersetzung) kann uns helfen, bekannte Muster in neuen Verkleidungen zu erkennen.

Stell dir vor, du kennst das Verhalten der Funktion $f(x) = \frac{1}{x}$ sehr gut. Was ist dann mit der Funktion $g(x) = \frac{1}{x-5}$?
Wenn wir $z = x-5$ setzen (das ist unsere Substitution), dann ist $g(x)$ eigentlich $f(z) = \frac{1}{z}$. Die Funktion $g(x)$ verhält sich also genauso wie $f(x)$, nur dass alles um 5 Einheiten auf der x-Achse nach rechts verschoben ist!
\begin{itemize}
    \item $f(x) = \frac{1}{x}$ hat eine Polstelle bei $x=0$.
    \item $g(x) = \frac{1}{x-5}$ hat eine Polstelle dort, wo der Nenner Null wird, also bei $x-5=0 \implies x=5$.
\end{itemize}
Die senkrechte Asymptote verschiebt sich also von $x=0$ nach $x=5$. Das Verhalten um die Polstelle (mit Vorzeichenwechsel) bleibt aber qualitativ gleich. Auch das Verhalten im Unendlichen ($\lim_{x \to \pm\infty} g(x) = 0$) bleibt gleich.

\begin{merksatzumgebung}{Substitution und Transformationen}
Wenn du eine Funktion $f(u)$ kennst und eine neue Funktion $g(x) = f(x-c)$ betrachtest, dann ist der Graph von $g(x)$ einfach der Graph von $f(u)$, der um $c$ Einheiten \textbf{entlang der x-Achse verschoben} ist:
\begin{itemize}
    \item um $c$ nach rechts, wenn $c>0$ (wie bei $x-c$, z.B. $x-5$)
    \item um $|c|$ nach links, wenn $c<0$ (wie bei $x-(-|c|) = x+|c|$, z.B. $x+2$)
\end{itemize}
Ähnlich bewirkt $g(x) = f(x) + d$ eine Verschiebung um $d$ Einheiten entlang der y-Achse.

Dieses Prinzip kennst du schon von der \textbf{Scheitelpunktform} einer Parabel:
$f(x) = a(x-x_S)^2 + y_S$.
Das ist im Grunde die Normalparabel $u^2$, die:
\begin{enumerate}
    \item mit $a$ gestreckt/gestaucht/gespiegelt wird ($a u^2$)
    \item um $x_S$ in x-Richtung verschoben wird (ersetze $u$ durch $x-x_S \implies a(x-x_S)^2$)
    \item um $y_S$ in y-Richtung verschoben wird ($\implies a(x-x_S)^2 + y_S$)
\end{enumerate}
Die Substitution hilft uns, die 'innere Struktur' von Funktionen zu erkennen und komplexe Funktionen auf einfachere, bekannte Grundfunktionen zurückzuführen. Dieses Denken wird später bei der Kettenregel der Ableitung und bei der Integration durch Substitution extrem wichtig!
\end{merksatzumgebung}

\begin{beispielumgebung}{Verhalten von $f(x) = \frac{2}{(x-1)^2}$}
\begin{itemize}
    \item \textbf{Grundfunktion:} Wir erkennen die Struktur von $\frac{a}{u^n}$ mit $a=2$ und $n=2$ (gerade). Die Grundfunktion wäre $h(u) = \frac{2}{u^2}$.
    \item \textbf{Substitution/Verschiebung:} Hier ist $u = x-1$. Das bedeutet, der Graph von $h(u)$ ist um $1$ Einheit nach rechts verschoben.
    \item \textbf{Definitionsbereich:} Der Nenner $(x-1)^2$ wird Null, wenn $x-1=0 \implies x=1$. Also $D_f = \mathbb{R} \setminus \{1\}$.
    \item \textbf{Polstelle:} Bei $x=1$ liegt eine Polstelle mit senkrechter Asymptote $x=1$.
    \item \textbf{Verhalten an der Polstelle $x=1$:} Da $n=2$ (gerade) und $a=2$ (positiv) ist, haben wir eine Polstelle ohne Vorzeichenwechsel, und die Funktion geht gegen $+\infty$:
        $\lim_{x \to 1^+} f(x) = +\infty$ und $\lim_{x \to 1^-} f(x) = +\infty$.
    \item \textbf{Verhalten im Unendlichen:} $\lim_{x \to \pm\infty} \frac{2}{(x-1)^2} = 0$. Horizontale Asymptote $y=0$.
    \item \textbf{Symmetrie:} Die Grundfunktion $h(u)=\frac{2}{u^2}$ ist achsensymmetrisch zur u-Achse ($u=0$). Da unsere Funktion um $x=1$ verschoben ist, ist $f(x)=\frac{2}{(x-1)^2}$ achsensymmetrisch zur Geraden $x=1$.
\end{itemize}
\begin{center}
    \includegraphics[width=0.8\textwidth]{grafiken/Gebrochen_Rational_Verschoben.png}
    \captionof{figure}{Graph von $f(x)=\frac{2}{(x-1)^2}$}
    \label{fig:gebr_rat_verschoben}
\end{center}
\end{beispielumgebung}

\begin{aufgabenumgebung}{Funktionen mit negativen Exponenten und Substitution}
\begin{enumerate}
    \item Bestimme für die folgenden Funktionen den Definitionsbereich, die Gleichung der senkrechten Asymptote(n) und das Verhalten der Funktion für $x$ gegen die Polstelle(n) (einseitige Grenzwerte) sowie für $x \to \pm\infty$. Untersuche auch das Symmetrieverhalten bezüglich der senkrechten Asymptote oder eines Punktes.
        \begin{itemize}
            \item $f_1(x) = \frac{-1}{x+2}$
            \item $f_2(x) = \frac{3}{(x-3)^2}$
            \item $f_3(x) = 1 - \frac{1}{x^2}$ (Tipp: Was ist hier die horizontale Asymptote?)
        \end{itemize}
    \item Skizziere die Graphen der Funktionen aus Teilaufgabe 1.
    \item \textbf{Transformationskette verstehen:}
        Beschreibe, wie der Graph der Funktion $g(x) = \frac{-2}{(x+3)^2} - 4$ aus dem Graphen der Grundfunktion $h(u) = \frac{1}{u^2}$ durch Streckung/Stauchung, Spiegelung und Verschiebungen hervorgeht. Gib den Definitionsbereich und die Gleichungen der Asymptoten von $g(x)$ an.
\end{enumerate}
\end{aufgabenumgebung}

\begin{kurzknappumgebung}{Funktionen $f(x) = \frac{a}{(x-c)^n} + d$}
\begin{itemize}
    \item \textbf{Definitionsbereich:} $D_f = \mathbb{R} \setminus \{c\}$.
    \item \textbf{Senkrechte Asymptote (Polstelle):} Bei $x=c$. Verhalten wie bei $\frac{a}{u^n}$ für $u \to 0$.
    \item \textbf{Waagerechte Asymptote:} Bei $y=d$. $\lim_{x \to \pm\infty} f(x) = d$. (Wenn $d=0$, ist es die x-Achse).
    \item \textbf{Symmetrie:} Wenn die Grundfunktion $\frac{a}{u^n}$ symmetrisch zum Ursprung (n ungerade) oder zur y-Achse (n gerade) ist, dann ist $f(x)$ symmetrisch zum Punkt $(c|d)$ bzw. zur Achse $x=c$.
\end{itemize}
\end{kurzknappumgebung}

\begin{fehlerboxumgebung}{Grenzwerte und Asymptoten}
\begin{itemize}
    \item \textbf{Verschiebung nicht erkannt:} Bei Termen wie $\frac{1}{x-c}$ liegt die Polstelle bei $x=c$, nicht bei $x=0$.
    \item \textbf{Horizontale Asymptote bei Summen/Differenzen:} Bei $f(x) = \frac{a}{x^n} + d$ ist die horizontale Asymptote $y=d$, nicht $y=0$ (außer $d=0$). Der Term $\frac{a}{x^n}$ geht gegen Null, aber das $d$ bleibt!
    \item \textbf{Definitionsbereich und Polstellen:} Eine Polstelle ist immer außerhalb des Definitionsbereichs.
\end{itemize}
\end{fehlerboxumgebung}

Wir werden diese Ideen zu Grenzwerten und Asymptoten später bei der Diskussion komplexerer gebrochen-rationaler Funktionen wieder aufgreifen. Jetzt, da wir ein solides Fundament für Polynome und einfache gebrochen-rationale Funktionen gelegt haben, können wir unseren Werkzeugkasten der Ableitungsregeln erweitern.
\subsection{Anwendung und Vertiefung der bisherigen Differentialrechnung}
\label{subsec:anwendung_vertiefung_diff_neu}

Wir haben nun die Grundlagen der Differentialrechnung kennengelernt: die Idee der Ableitung als momentane Änderungsrate und Tangentensteigung, die h-Methode zur Herleitung von Ableitungen sowie die ersten wichtigen Ableitungsregeln (Konstanten-, Potenz-, Faktor- und Summenregel). Wir haben auch gesehen, wie uns die erste und zweite Ableitung helfen, das Verhalten von Funktionen (Monotonie, Extrempunkte, Krümmung, Wendepunkte) zu analysieren und wie Grenzwerte das Verhalten im Unendlichen und an Polstellen beschreiben.

Bevor wir uns weiteren, komplexeren Ableitungsregeln zuwenden, wollen wir dieses Wissen festigen und in anspruchsvolleren Aufgaben anwenden.

\begin{kurzknappumgebung}{Differentialrechnung – Die Grundlagen im Überblick}
\begin{itemize}
    \item \textbf{Ableitung $f'(x)$:} Momentane Änderungsrate von $f(x)$; Steigung der Tangente an den Graphen von $f(x)$.
    \item \textbf{h-Methode:} Grundlegendes Verfahren zur Bestimmung der Ableitung über den Grenzwert des Differenzenquotienten: $f'(x_0) = \lim_{h \to 0} \frac{f(x_0+h) - f(x_0)}{h}$.
    \item \textbf{Wichtige Ableitungsregeln (bisher):}
        \begin{itemize}
            \item Konstantenregel: $(c)' = 0$.
            \item Potenzregel: $(x^n)' = nx^{n-1}$. (Gilt auch für negative/gebrochene Exponenten!)
            \item Faktorregel: $(c \cdot g(x))' = c \cdot g'(x)$.
            \item Summenregel: $(g(x) \pm h(x))' = g'(x) \pm h'(x)$.
        \end{itemize}
    \item \textbf{Bedeutung von $f'(x)$:}
        \begin{itemize}
            \item $f'(x) > 0 \implies f(x)$ ist streng monoton steigend.
            \item $f'(x) < 0 \implies f(x)$ ist streng monoton fallend.
            \item $f'(x_E) = 0$: Notwendige Bedingung für eine Extremstelle bei $x_E$.
            \item VZW von $f'$ an $x_E$: Hinreichende Bedingung für Extremstellen (Hoch-/Tiefpunkt).
        \end{itemize}
    \item \textbf{Bedeutung von $f''(x)$ (zweite Ableitung):}
        \begin{itemize}
            \item $f''(x) > 0 \implies$ Graph von $f(x)$ ist linksgekrümmt (konvex).
            \item $f''(x) < 0 \implies$ Graph von $f(x)$ ist rechtsgekrümmt (konkav).
            \item $f''(x_W) = 0$: Notwendige Bedingung für eine Wendestelle bei $x_W$.
            \item VZW von $f''$ an $x_W$ (oder $f'''(x_W) \neq 0$): Hinreichende Bedingung für Wendestelle.
            \item $f'(x_E)=0$ und $f''(x_E) > 0 \implies$ Tiefpunkt; $f'(x_E)=0$ und $f''(x_E) < 0 \implies$ Hochpunkt.
        \end{itemize}
    \item \textbf{Grenzwerte ($\lim$):} Untersuchen das Verhalten von Funktionen für $x \to \pm\infty$ oder an Definitionslücken (z.B. Polstellen bei $f(x)=a/x^n$).
    \item \textbf{Kurvendiskussion:} Systematische Untersuchung einer Funktion auf ihre Eigenschaften (Definitionsbereich, Symmetrie, Grenzwerte, Achsenschnittpunkte, Extrempunkte, Wendepunkte, Monotonie, Krümmung) zur Erstellung einer Graphenskizze.
    \item \textbf{Substitution als Denkwerkzeug:} Erkennen von Grundfunktionen in transformierter Form (z.B. $g(x) = \frac{a}{(x-c)^n}+d$ als Transformation von $f(u)=\frac{a}{u^n}$).
\end{itemize}
\end{kurzknappumgebung}

\begin{warumwichtigumgebung}{Was du jetzt können solltest}
Nachdem du diesen ersten Teil des Kapitels Differentialrechnung durchgearbeitet hast, solltest du in der Lage sein:
\begin{itemize}
    \item Den Begriff der Ableitung als momentane Änderungsrate und Tangentensteigung zu erklären.
    \item Die grundlegenden Ableitungsregeln (Konstanten-, Potenz-, Faktor-, Summenregel) sicher anzuwenden, auch auf Terme mit Wurzeln oder $x$ im Nenner (nach Umformung in Potenzschreibweise).
    \item Höhere Ableitungen zu bilden.
    \item Die Bedeutung der ersten und zweiten Ableitung für Monotonie, Extrempunkte, Krümmungsverhalten und Wendepunkte zu verstehen und anzuwenden.
    \item Das Grenzwertverhalten von Polynomfunktionen und einfachen gebrochen-rationalen Funktionen (wie $a/x^n$ und deren Verschiebungen) für $x \to \pm\infty$ und an Polstellen zu bestimmen.
    \item Eine vollständige Kurvendiskussion für Polynomfunktionen bis zum Grad 4 (mit lösbaren Nullstellenproblemen) und für einfache transformierte gebrochen-rationale Funktionen durchzuführen.
    \item Die Idee der Substitution zu nutzen, um das Verhalten transformierter Funktionen zu verstehen.
    \item Einfache Anwendungsaufgaben zu lösen, bei denen Änderungsraten oder Optimierungsprobleme eine Rolle spielen.
\end{itemize}
Das ist schon eine ganze Menge! Sei stolz auf das, was du gelernt hast. Die folgenden Aufgaben helfen dir, dein Wissen zu festigen und zu vertiefen.
\end{warumwichtigumgebung}

\begin{aufgabenumgebung}[A:DiffUebergreifend]{Übergreifende Übungsaufgaben zum bisherigen Kapitel Differentialrechnung}
\begin{enumerate}
    \item \textbf{Polynom-Analyse (Grad 3):}
        Gegeben ist die Funktion $f(x) = -\frac{1}{3}x^3 + x^2 + 3x - \frac{7}{3}$.
        \begin{enumerate}
            \item Führe eine vollständige Kurvendiskussion für $f(x)$ durch (Definitionsbereich, Symmetrie, Verhalten im Unendlichen, Achsenschnittpunkte, Extrempunkte, Wendepunkte, Monotonie, Krümmung).
            \begin{tippumgebung}{Nullstellen}
            Eine Nullstelle dieser Funktion ist $x_1=1$. Nutze diese Information, um die weiteren Nullstellen zu finden (z.B. durch Faktorisieren, nachdem du $(x-1)$ als Faktor erkannt hast, oder indem du die verbleibende quadratische Gleichung löst).
            \end{tippumgebung}
            \item Zeichne den Graphen von $f(x)$ im Intervall $[-4, 6]$ unter Verwendung deiner Ergebnisse.
            \item Bestimme die Gleichung der Tangente an den Graphen von $f(x)$ im Punkt $P(0|f(0))$.
            \item In welchem Punkt hat die Tangente an den Graphen von $f(x)$ die Steigung $m=-5$?
        \end{enumerate}
    \item \textbf{Optimierungsproblem – Die optimale Dose:}
        Eine zylinderförmige Konservendose soll ein Volumen von $V = 500 \text{ cm}^3$ haben. Die Materialkosten sollen minimiert werden, d.h. die Oberfläche $O$ der Dose soll minimal werden.
        Die Formeln für einen Zylinder mit Radius $r$ und Höhe $h$ sind:
        Volumen: $V = \pi r^2 h$
        Oberfläche (Mantel + 2 Deckel): $O = 2\pi r^2 + 2\pi r h$
        \begin{enumerate}
            \item \textbf{Zielfunktion aufstellen:} Drücke die Oberfläche $O$ als Funktion nur einer Variablen (z.B. des Radius $r$) aus. Nutze dazu die Nebenbedingung für das Volumen $V=500 \text{ cm}^3$, um $h$ durch $r$ auszudrücken und in die Oberflächenformel einzusetzen. Du erhältst $O(r)$.
            \begin{tippumgebung}{Umgang mit $\pi$}
            Behandle $\pi$ wie eine Konstante.
            \end{tippumgebung}
            \item \textbf{Ableitung bilden:} Bilde die erste Ableitung $O'(r)$. (Hinweis: $O(r)$ wird einen Term der Form $\frac{k}{r}$ enthalten, was du als $kr^{-1}$ schreiben kannst.)
            \item \textbf{Extremstelle finden:} Setze $O'(r)=0$ und löse nach $r$ auf, um den Radius zu finden, der die Oberfläche minimiert.
            \item \textbf{Überprüfung (optional für Experten):} Überprüfe mit der zweiten Ableitung $O''(r)$, ob es sich tatsächlich um ein Minimum handelt.
            \item \textbf{Optimale Abmessungen:} Berechne die zugehörige Höhe $h$ und das minimale Oberflächenmaterial. Welcher Zusammenhang besteht zwischen $r$ und $h$ bei minimaler Oberfläche?
        \end{enumerate}
    \item \textbf{Analyse einer biquadratischen Funktion:}
        Gegeben ist die Funktion $f(x) = x^4 - 8x^2 + 7$.
        \begin{enumerate}
            \item Untersuche die Funktion auf Symmetrie.
            \item Bestimme die Nullstellen der Funktion. (Tipp: Substitution $z=x^2$).
            \item Bestimme die lokalen Extrempunkte von $f(x)$.
            \item Bestimme die Wendepunkte von $f(x)$.
            \item Skizziere den Graphen von $f(x)$ basierend auf deinen Ergebnissen.
        \end{enumerate}
    \item \textbf{Transformationen und Grenzwerte verstehen:}
        Betrachte die Funktion $g(x) = \frac{-2}{(x+1)^3} + 1$.
        \begin{enumerate}
            \item \textbf{Grundfunktion:} Von welcher einfachen Grundfunktion $h(u) = \frac{a}{u^n}$ lässt sich $g(x)$ ableiten?
            \item \textbf{Transformationsschritte:} Beschreibe, durch welche Verschiebungen, Streckungen oder Spiegelungen der Graph von $g(x)$ aus dem Graphen von $h(u)$ entsteht.
            \item \textbf{Definitionsbereich und Asymptoten:} Bestimme den Definitionsbereich von $g(x)$ sowie die Gleichungen der senkrechten und waagerechten Asymptoten.
            \item \textbf{Grenzwerte an der Polstelle:} Untersuche $\lim_{x \to -1^+} g(x)$ und $\lim_{x \to -1^-} g(x)$. Handelt es sich um eine Polstelle mit oder ohne Vorzeichenwechsel?
            \item \textbf{Skizze:} Skizziere den Graphen von $g(x)$ mit seinen Asymptoten.
        \end{enumerate}
    \item \textbf{Bewegung eines Objekts (Anwendung):}
        Die Höhe $h$ (in Metern) eines senkrecht nach oben geworfenen Steins nach $t$ Sekunden wird durch die Funktion $h(t) = -5t^2 + 20t + 1$ beschrieben (für $t \ge 0$ und solange $h(t) \ge 0$).
        \begin{enumerate}
            \item Bestimme die Funktion $v(t)$, die die Geschwindigkeit des Autos zum Zeitpunkt $t$ angibt.
            \item Bestimme die Funktion $a(t)$, die die Beschleunigung des Autos zum Zeitpunkt $t$ angibt.
            \item Zu welchen Zeitpunkten $t$ ist das Auto in Ruhe (Geschwindigkeit gleich Null)?
            \item In welchen Zeitintervallen fährt das Auto vorwärts ($v(t)>0$) und in welchen rückwärts ($v(t)<0$)?
            \item Wann ist die Beschleunigung Null? Was bedeutet das für die Geschwindigkeit zu diesem Zeitpunkt?
            \item (Für Experten): Wann ist die Geschwindigkeit des Autos am größten im Intervall $[0, 2]$? Wann ist sie am geringsten (d.h. am stärksten negativ) im Intervall $[0, 4]$?
        \end{enumerate}
\end{enumerate}
\end{aufgabenumgebung}

\begin{infoboxumgebung}{Ein kleiner Exkurs: Was ist eigentlich Polynomdivision?}
Du hast in einigen Aufgaben den Hinweis auf 'Polynomdivision' gesehen, wenn es darum ging, Nullstellen von Polynomen höheren Grades (Grad > 2) zu finden, nachdem eine Nullstelle $x_1$ bereits bekannt war (z.B. durch Raten). Aber was steckt dahinter?

Stell dir vor, du hast ein Polynom $f(x)$ und kennst eine Nullstelle $x_1$. Das bedeutet, $(x-x_1)$ ist ein Faktor von $f(x)$ (genau wie wenn 12 durch 3 teilbar ist, weil 3 ein Faktor von 12 ist). Die Polynomdivision ist nun ein Verfahren, ähnlich der schriftlichen Division von Zahlen, mit dem du $f(x)$ durch den Linearfaktor $(x-x_1)$ teilen kannst.
\[ f(x) : (x-x_1) = \text{Restpolynom} \]
Das Ergebnis ist ein 'Restpolynom', dessen Grad um 1 niedriger ist als der von $f(x)$. Wenn $f(x)$ also z.B. vom Grad 3 war, ist das Restpolynom vom Grad 2. Und die Nullstellen eines quadratischen Polynoms können wir ja mit der p-q-Formel oder Mitternachtsformel finden!

\textbf{Beispiel-Idee:}
Wenn $f(x) = x^3 - 7x - 6$ und wir wissen (durch Raten), dass $x_1 = -1$ eine Nullstelle ist (denn $f(-1) = (-1)^3 - 7(-1) - 6 = -1 + 7 - 6 = 0$), dann können wir $f(x)$ durch $(x - (-1)) = (x+1)$ teilen.
\[
\polyset{style=C, vars=x, div=:} % Setzt den Divisionsoperator auf ':'
\polylongdiv{x^3 - 7x - 6}{x+1}
\]
(Die $0x^2$ wird für das schriftliche Verfahren oft ergänzt.)
Das Restpolynom $x^2 - x - 6$ können wir nun mit der p-q-Formel lösen, um die weiteren Nullstellen $x_2=3$ und $x_3=-2$ zu finden.

Die Polynomdivision ist also ein nützliches Werkzeug, um Polynome höheren Grades in Faktoren zu zerlegen und so alle Nullstellen zu finden. Das genaue Verfahren der schriftlichen Polynomdivision findest du in vielen Schulbüchern oder Online-Quellen, falls es dich genauer interessiert!
\end{infoboxumgebung}

\begin{infoboxumgebung}{Noch ein faszinierender Grenzwert: Die Eulersche Zahl $e$}
Wir haben den Grenzwertbegriff im Zusammenhang mit der Ableitung ($h \to 0$) und dem Verhalten von Funktionen im Unendlichen ($x \to \pm\infty$) kennengelernt. Es gibt noch viele andere wichtige Grenzwerte in der Mathematik. Einer davon führt zu einer ganz besonderen Zahl, der \textbf{Eulerschen Zahl $e \approx 2,71828\dots$}.

Stell dir vor, du legst 1 Euro zu 100\% Zinsen pro Jahr an.
\begin{itemize}
    \item Bei jährlicher Verzinsung hast du nach einem Jahr: $1 \cdot (1 + \frac{1}{1})^1 = 2$ Euro.
    \item Bei halbjährlicher Verzinsung (also $2 \times 50\%$ Zinsen): $1 \cdot (1 + \frac{1}{2})^2 = (1.5)^2 = 2,25$ Euro.
    \item Bei vierteljährlicher Verzinsung ($4 \times 25\%$ Zinsen): $1 \cdot (1 + \frac{1}{4})^4 \approx 2,44$ Euro.
    \item Bei monatlicher Verzinsung ($12 \times \frac{100}{12}\%$ Zinsen): $1 \cdot (1 + \frac{1}{12})^{12} \approx 2,61$ Euro.
\end{itemize}
Was passiert, wenn man die Zinsperioden immer kürzer macht und $n$-mal pro Jahr verzinst (also mit $\frac{100}{n}\%$ Zinsen pro Periode)? Man betrachtet den Ausdruck:
\[ \left(1 + \frac{1}{n}\right)^n \]
Wenn $n$ nun unendlich groß wird ($n \to \infty$), also die Verzinsung quasi kontinuierlich (in jedem unendlich kleinen Augenblick) erfolgt, dann nähert sich dieser Ausdruck einem festen Wert:
\[ \lim_{n \to \infty} \left(1 + \frac{1}{n}\right)^n = e \]
Die Zahl $e$ ist die Basis des \textbf{natürlichen Logarithmus} ($\ln$) und spielt eine fundamentale Rolle bei Exponentialfunktionen, die natürliches Wachstum oder Zerfall beschreiben (z.B. $f(x) = e^x$). Diese Funktionen werden wir in einem späteren Kapitel genauer untersuchen.
\end{infoboxumgebung}

\begin{infoboxumgebung}{Ausblick auf weitere Ableitungsregeln}
Mit den bisher gelernten Regeln (Konstanten-, Potenz-, Faktor-, Summenregel) können wir schon viele Funktionen, insbesondere alle Polynomfunktionen, ableiten und analysieren. Für komplexere Funktionen, die durch Multiplikation, Division oder Verkettung anderer Funktionen entstehen (wie z.B. $f(x) = x^2 \cdot e^x$, $g(x) = \frac{\sin(x)}{x}$ oder $h(x) = \sqrt{x^2+1}$), benötigen wir weitere Werkzeuge: die Produktregel, die Quotientenregel und die Kettenregel. Diese werden wir im nächsten Abschnitt kennenlernen.
\end{infoboxumgebung}

\subsection{Ableitungsregeln - Fortsetzung}

\subsubsection{Die Produktregel – Ableiten von $f(x) = u(x) \cdot v(x)$}
\label{subsubsec:produktregel}

Bisher haben wir Funktionen betrachtet, die Summen, Differenzen oder Vielfache von Potenzfunktionen waren. Was aber, wenn eine Funktion selbst ein Produkt aus zwei Funktionen ist, die beide die Variable $x$ enthalten? 
Ein Beispiel hierfür wäre $f(x) = (x^2+1)(x^3-2x)$.

Man könnte nun denken, man leitet einfach jeden Faktor einzeln ab und multipliziert die Ergebnisse. \textbf{Aber Vorsicht, das ist im Allgemeinen falsch!}
Also: $(u(x) \cdot v(x))' \neq u'(x) \cdot v'(x)$.

Um solche Produkte korrekt ableiten zu können, benötigen wir die \textbf{Produktregel}.

\begin{merksatzumgebung}{Produktregel}
Ist eine Funktion $f(x)$ als Produkt zweier differenzierbarer Funktionen $u(x)$ und $v(x)$ gegeben, also $f(x) = u(x) \cdot v(x)$, dann lautet ihre Ableitung:
\[ f'(x) = u'(x) \cdot v(x) + u(x) \cdot v'(x) \]
In Kurzschreibweise: $(u \cdot v)' = u'v + uv'$.
\end{merksatzumgebung}

\begin{tippumgebung}{Merkspruch für die Produktregel}
Ein gängiger Merkspruch, um sich die Produktregel einzuprägen, lautet:
'Ableitung des ersten Faktors mal den zweiten Faktor (stehen lassen), plus den ersten Faktor (stehen lassen) mal die Ableitung des zweiten Faktors.'
Oder noch kürzer: 'Erste ableiten, Zweite stehen lassen, plus Erste stehen lassen, Zweite ableiten.'
\end{tippumgebung}

Schauen wir uns an, wie das funktioniert und warum es so wichtig ist, diese Regel zu verwenden.

\begin{beispielumgebung}{Anwendung der Produktregel bei Polynomen}
Wir wollen die Funktion $f(x) = (x^2+3x) \cdot (2x-1)$ ableiten.

\textbf{Methode 1: Mit der Produktregel}

\textbf{Schritt 1: Identifiziere $u(x)$ und $v(x)$.}
$u(x) = x^2+3x$
$v(x) = 2x-1$

\textbf{Schritt 2: Bilde die Ableitungen $u'(x)$ und $v'(x)$.}
Mit den uns bekannten Regeln (Potenz-, Faktor-, Summenregel):
$u'(x) = (x^2)' + (3x)' = 2x + 3$
$v'(x) = (2x)' - (1)' = 2 - 0 = 2$

\textbf{Schritt 3: Setze in die Produktregel $f'(x) = u'(x)v(x) + u(x)v'(x)$ ein.}
$f'(x) = (2x+3) \cdot (2x-1) + (x^2+3x) \cdot (2)$

\textbf{Schritt 4: Vereinfache den Term (ausmultiplizieren und zusammenfassen).}
$f'(x) = (2x \cdot 2x + 2x \cdot (-1) + 3 \cdot 2x + 3 \cdot (-1)) + (2 \cdot x^2 + 2 \cdot 3x)$
$f'(x) = (4x^2 - 2x + 6x - 3) + (2x^2 + 6x)$
$f'(x) = 4x^2 + 4x - 3 + 2x^2 + 6x$
$f'(x) = (4x^2 + 2x^2) + (4x + 6x) - 3$
\[ f'(x) = 6x^2 + 10x - 3 \]

\textbf{Methode 2: Probe durch vorheriges Ausmultiplizieren (ohne Produktregel)}
Wir können die Funktion $f(x)$ auch zuerst ausmultiplizieren und dann die Summen- und Potenzregel anwenden:
$f(x) = (x^2+3x)(2x-1)$
$f(x) = x^2 \cdot 2x + x^2 \cdot (-1) + 3x \cdot 2x + 3x \cdot (-1)$
$f(x) = 2x^3 - x^2 + 6x^2 - 3x$
$f(x) = 2x^3 + ( -x^2 + 6x^2) - 3x$
$f(x) = 2x^3 + 5x^2 - 3x$

Nun leiten wir diese Summe ab:
$f'(x) = (2x^3)' + (5x^2)' - (3x)'$
$f'(x) = 2 \cdot 3x^2 + 5 \cdot 2x - 3 \cdot 1$
\[ f'(x) = 6x^2 + 10x - 3 \]
Beide Methoden führen zum selben Ergebnis! Das zeigt, dass die Produktregel korrekt ist und uns bei komplizierteren Produkten, die sich nicht so leicht ausmultiplizieren lassen (z.B. wenn $e$-Funktionen oder trigonometrische Funktionen beteiligt sind), eine große Hilfe sein wird.
\end{beispielumgebung}

\textit{Selbst-Check:} Warum ist es bei $f(x) = 5x^3$ nicht notwendig, die Produktregel anzuwenden, obwohl man es als $u(x)=5$ und $v(x)=x^3$ auffassen könnte? (Antwort: Weil $u(x)=5$ eine Konstante ist. Die Faktorregel ist ein Spezialfall der Produktregel, wenn einer der Faktoren eine Konstante ist: $(c \cdot v(x))' = c' \cdot v(x) + c \cdot v'(x) = 0 \cdot v(x) + c \cdot v'(x) = c \cdot v'(x)$.)

\begin{aufgabenumgebung}{Produktregel trainieren}
Leite die folgenden Funktionen mit der Produktregel ab und vereinfache die Ergebnisse so weit wie möglich. Kontrolliere deine Ergebnisse für die ersten beiden Aufgaben, indem du die Terme zuerst ausmultiplizierst und dann ableitest.
\begin{enumerate}
    \item $f(x) = (x+2)(3x-4)$
    \item $g(x) = x^2(x^3+5x)$
    \item $h(t) = (t^2-1)(t^2+1)$ (Erkennst du hier eine binomische Formel, die das Ausmultiplizieren erleichtert?)
    \item $k(x) = (2x^3-x)(x^2+x+1)$
    \item $m(a) = (a^2+1)(a-1)$ (Leite nach $a$ ab.)
\end{enumerate}
\end{aufgabenumgebung}

\begin{warumwichtigumgebung}{Die Produktregel}
Die Produktregel ist unerlässlich, sobald Funktionen multiplikativ verknüpft sind und beide Faktoren von der Variablen abhängen. Viele reale Modelle entstehen durch Produkte von Funktionen (z.B. Umsatz = Preis $\times$ Menge, wobei Preis und Menge von einer anderen Größe abhängen können). Ohne die Produktregel könnten wir solche Modelle nicht korrekt analysieren. Sie ist ein Grundpfeiler der Differentialrechnung.
\end{warumwichtigumgebung}

% Vorheriger Inhalt des Kapitels bis zur warumwichtigumgebung 'Die Produktregel'
% ... (siehe vorherige Canvas-Version) ...

% Vorheriger Inhalt des Kapitels bis zur warumwichtigumgebung 'Die Produktregel'
% ... (siehe vorherige Canvas-Version) ...

\begin{aufgabenumgebung}[A:ProduktregelAnwendung]{Anwendung der Produktregel in einer Kurvendiskussion (vereinfacht)}
Gegeben sei die Funktion $f(x) = x \cdot (x-3)^2$.
\begin{enumerate}
    \item \textbf{Definitionsbereich und Verhalten im Unendlichen:} Bestimme $D_f$ und untersuche $\lim_{x \to \pm\infty} f(x)$.
    \item \textbf{Nullstellen:} Bestimme die Nullstellen von $f(x)$ direkt aus der faktorisierten Form. Welche Vielfachheit haben sie? Was bedeutet das für den Graphen?
    \item \textbf{Erste Ableitung mit Produktregel:}
        Identifiziere $u(x)=x$ und $v(x)=(x-3)^2$. 
        \begin{tippumgebung}{Ableitung von $v(x)=(x-3)^2$}
        Um $v(x)=(x-3)^2$ abzuleiten, kannst du es entweder ausmultiplizieren zu $x^2-6x+9$ und dann die Summen-/Potenzregel anwenden, oder du erkennst hier schon eine verkettete Funktion (die Kettenregel lernen wir später noch genauer kennen). Für jetzt: $(x-c)^2$ abgeleitet ist $2(x-c)\cdot 1 = 2(x-c)$. Also ist $v'(x) = 2(x-3)$.
        \end{tippumgebung}
        Bilde $f'(x)$ mit der Produktregel. Vereinfache $f'(x)$ so weit wie möglich (Tipp: $(x-3)$ ausklammern).
    \item \textbf{Extremstellen:} Bestimme die Nullstellen von $f'(x)$. Untersuche das Vorzeichen von $f'(x)$ (Monotonieintervalle), um die Art der Extremstellen zu bestimmen. Berechne die y-Koordinaten der Extrempunkte.
    \item \textbf{Skizze:} Skizziere den Graphen von $f(x)$ unter Verwendung deiner Ergebnisse.
\end{enumerate}
Diese Aufgabe zeigt, wie die Produktregel auch bei der Analyse von Polynomfunktionen nützlich sein kann, besonders wenn sie in faktorisierter Form vorliegen oder entstehen.
\end{aufgabenumgebung}

\subsubsection{Die Quotientenregel – Ableiten von $f(x) = \frac{u(x)}{v(x)}$}
\label{subsubsec:quotientenregel}

Nachdem wir Produkte ableiten können, stellt sich natürlich die Frage: Wie leitet man einen Bruch (Quotienten) von zwei Funktionen ab, bei dem die Variable $x$ sowohl im Zähler als auch im Nenner vorkommt? Ein Beispiel wäre $f(x) = \frac{x^2+1}{x-2}$.

Auch hier gilt: Man darf \textbf{nicht} einfach den Zähler und den Nenner getrennt ableiten und die Ergebnisse dividieren!
Also: $\left(\frac{u(x)}{v(x)}\right)' \neq \frac{u'(x)}{v'(x)}$.

Für Quotienten benötigen wir die \textbf{Quotientenregel}.

\begin{merksatzumgebung}{Quotientenregel}
Ist eine Funktion $f(x)$ als Quotient zweier differenzierbarer Funktionen $u(x)$ (Zähler) und $v(x)$ (Nenner) gegeben, also $f(x) = \frac{u(x)}{v(x)}$ (wobei $v(x) \neq 0$), dann lautet ihre Ableitung:
\[ f'(x) = \frac{u'(x) \cdot v(x) - u(x) \cdot v'(x)}{[v(x)]^2} \]
In Kurzschreibweise: $\left(\frac{u}{v}\right)' = \frac{u'v - uv'}{v^2}$.
\end{merksatzumgebung}

\textbf{Herleitung der Quotientenregel (für Interessierte):}
Wir können die Quotientenregel tatsächlich aus der Produktregel und der (später noch genauer behandelten) Kettenregel herleiten. Eine einfachere Herleitung für Polynome und Potenzfunktionen gelingt, wenn wir den Quotienten als Produkt umschreiben:
$f(x) = \frac{u(x)}{v(x)} = u(x) \cdot (v(x))^{-1}$.
Jetzt wenden wir die Produktregel $(fg)' = f'g + fg'$ an, wobei $g(x) = (v(x))^{-1}$.
Die Ableitung von $g(x)=(v(x))^{-1}$ ist nach der (verallgemeinerten) Potenzregel und Kettenregel $g'(x) = -1 \cdot (v(x))^{-2} \cdot v'(x) = -\frac{v'(x)}{(v(x))^2}$.
(Die Kettenregel besagt grob: äußere Ableitung mal innere Ableitung. Die äußere Funktion ist hier $(\dots)^{-1}$, ihre Ableitung ist $-1(\dots)^{-2}$. Die innere Funktion ist $v(x)$, ihre Ableitung ist $v'(x)$.)

Setzen wir dies in die Produktregel ein:
$f'(x) = u'(x) \cdot (v(x))^{-1} + u(x) \cdot \left( -1 \cdot (v(x))^{-2} \cdot v'(x) \right)$
$f'(x) = \frac{u'(x)}{v(x)} - \frac{u(x) \cdot v'(x)}{(v(x))^2}$
Um dies auf einen gemeinsamen Nenner zu bringen, erweitern wir den ersten Bruch mit $v(x)$:
$f'(x) = \frac{u'(x) \cdot v(x)}{v(x) \cdot v(x)} - \frac{u(x) \cdot v'(x)}{(v(x))^2}$
$f'(x) = \frac{u'(x) \cdot v(x) - u(x) \cdot v'(x)}{[v(x)]^2}$
Und das ist genau die Quotientenregel! Es ist schön zu sehen, wie die Regeln in der Mathematik zusammenhängen.

\begin{tippumgebung}{Merkspruch für die Quotientenregel}
Ein gängiger Merkspruch, der die Reihenfolge der Terme im Zähler betont (da hier ein Minuszeichen steht, ist die Reihenfolge wichtig!), lautet:
'Ableitung des Zählers mal Nenner (stehen lassen) \textbf{minus} Zähler (stehen lassen) mal Ableitung des Nenners, das Ganze geteilt durch den Nenner zum Quadrat.'
Oft abgekürzt als: \textbf{NAZ - ZAN} durch Nenner-Quadrat (Nenner mal Ableitung Zähler minus Zähler mal Ableitung Nenner).
\textbf{Achtung:} Der Merkspruch ist 'NAZ - ZAN', aber die Formel ist $u'v - uv'$. $u$ ist der Zähler, $v$ der Nenner. Also: $u'v - uv'$.
Ein anderer, vielleicht besserer Merkspruch ist: \textbf{'Nenner mal Ableitung Zähler minus Zähler mal Ableitung Nenner, durch Nenner hoch zwei.'} (N·AZ - Z·AN / N²)
Oder: 'Ableitung oben mal unten minus oben mal Ableitung unten, durch unten ins Quadrat.'
\end{tippumgebung}

Die Quotientenregel sieht auf den ersten Blick etwas komplizierter aus, aber mit Übung wird sie dir vertraut.

\begin{beispielumgebung}{Anwendung der Quotientenregel}
Wir wollen die Funktion $f(x) = \frac{x^2+1}{x-2}$ ableiten. (Definitionsbereich: $D_f = \mathbb{R} \setminus \{2\}$)

\textbf{Schritt 1: Identifiziere $u(x)$ (Zähler) und $v(x)$ (Nenner).}
$u(x) = x^2+1$
$v(x) = x-2$

\textbf{Schritt 2: Bilde die Ableitungen $u'(x)$ und $v'(x)$.}
$u'(x) = (x^2+1)' = 2x$
$v'(x) = (x-2)' = 1$

\textbf{Schritt 3: Setze in die Quotientenregel $f'(x) = \frac{u'(x)v(x) - u(x)v'(x)}{[v(x)]^2}$ ein.}
$f'(x) = \frac{(2x) \cdot (x-2) - (x^2+1) \cdot (1)}{(x-2)^2}$

\textbf{Schritt 4: Vereinfache den Zähler.}
$f'(x) = \frac{2x^2 - 4x - (x^2+1)}{(x-2)^2}$
$f'(x) = \frac{2x^2 - 4x - x^2 - 1}{(x-2)^2}$
\[ f'(x) = \frac{x^2 - 4x - 1}{(x-2)^2} \]
Den Nenner $(x-2)^2$ lässt man oft in dieser Form stehen und multipliziert ihn nicht aus, da er für die Bestimmung von Nullstellen der Ableitung (wenn der Zähler Null ist) oder für das Verhalten an der Polstelle ($x=2$) in dieser Form nützlicher ist.
\end{beispielumgebung}

\begin{fehlerboxumgebung}{Häufige Fehler bei der Quotientenregel}
\begin{itemize}
    \item \textbf{Reihenfolge im Zähler vertauscht:} Das Minuszeichen macht die Reihenfolge $u'v - uv'$ entscheidend! Ein Vertauschen führt zum falschen Vorzeichen.
    \item \textbf{Nenner nicht quadriert:} Der Nenner der Ableitung ist immer $[v(x)]^2$.
    \item \textbf{Klammern vergessen:} Besonders wenn $u(x)$ oder $v(x)$ Summen oder Differenzen sind, müssen beim Einsetzen in die Formel Klammern gesetzt werden, z.B. $u(x)v'(x)$ ist $(x^2+1) \cdot (1)$ und nicht $x^2+1 \cdot 1$.
    \item \textbf{Fehler beim Vereinfachen des Zählers:} Achte auf Vorzeichen, wenn du Klammern auflöst.
\end{itemize}
\end{fehlerboxumgebung}

\begin{aufgabenumgebung}{Quotientenregel trainieren}
Leite die folgenden Funktionen mit der Quotientenregel ab und vereinfache die Zähler der Ergebnisse so weit wie möglich. Gib auch den Definitionsbereich der ursprünglichen Funktion an.
\begin{enumerate}
    \item $f(x) = \frac{x}{x+1}$
    \item $g(x) = \frac{3x-2}{x^2}$ (Tipp: Dies könnte man auch als $g(x) = (3x-2)x^{-2}$ mit der Produktregel oder nach Aufteilen des Bruchs $\frac{3x}{x^2} - \frac{2}{x^2} = \frac{3}{x} - \frac{2}{x^2}$ mit Potenz-/Faktorregeln ableiten. Vergleiche die Ergebnisse!)
    \item $h(t) = \frac{t^2+2t}{t-1}$
    \item $k(x) = \frac{5}{2x+3}$ (Hier ist $u(x)=5$, also $u'(x)=0$. Was bedeutet das für die Formel?)
\end{enumerate}
\end{aufgabenumgebung}

\begin{warumwichtigumgebung}{Die Quotientenregel}
Die Quotientenregel ist notwendig, um gebrochen-rationale Funktionen korrekt ableiten zu können. Diese Funktionen spielen eine wichtige Rolle bei der Modellierung von Phänomenen, bei denen Verhältnisse oder Raten auftreten, die sich ändern, oder bei denen es Asymptoten gibt (z.B. Konzentrationen, Durchschnittskosten, bestimmte physikalische Gesetze). Ohne die Quotientenregel wären Kurvendiskussionen solcher Funktionen nicht möglich.
\end{warumwichtigumgebung}

\begin{kurzknappumgebung}{Produkt- und Quotientenregel}
\begin{itemize}
    \item \textbf{Produktregel $(u \cdot v)' = u'v + uv'$}: Anwenden, wenn zwei von $x$ abhängige Terme multipliziert werden.
    \item \textbf{Quotientenregel $(\frac{u}{v})' = \frac{u'v - uv'}{v^2}$}: Anwenden, wenn zwei von $x$ abhängige Terme dividiert werden (Bruch). Achte auf die Reihenfolge im Zähler!
    \item \textbf{Alternative bei Polynomen/einfachen Brüchen:} Manchmal ist Ausmultiplizieren (Produkt) oder Aufteilen des Bruchs (Quotient) und anschließendes Ableiten mit Summen-/Potenzregel einfacher oder eine gute Kontrollmöglichkeit.
\end{itemize}
\end{kurzknappumgebung}

\begin{aufgabenumgebung}{Anwendung der neuen Regeln in Kontexten}
\begin{enumerate}
    \item \textbf{Umsatzfunktion:} Ein Unternehmen verkauft ein Produkt. Die Preis-Absatz-Funktion (Nachfragefunktion) sei $p(x) = 100 - 0.5x$, wobei $x$ die verkaufte Menge und $p(x)$ der Preis pro Stück ist. Der Umsatz $U(x)$ ist Preis mal Menge, also $U(x) = p(x) \cdot x = (100-0.5x)x$.
        \begin{itemize}
            \item Bestimme die Umsatzfunktion $U(x)$.
            \item Bilde die erste Ableitung $U'(x)$ (Grenzerlös). Du kannst dies tun, indem du $U(x)$ zuerst ausmultiplizierst oder indem du die Produktregel auf $p(x) \cdot x$ anwendest. Vergleiche beide Wege.
            \item Bei welcher Verkaufsmenge $x$ wird der Grenzerlös Null? Was könnte das für den Gesamtumsatz bedeuten?
        \end{itemize}
    \item \textbf{Durchschnittskosten:} Die Kostenfunktion eines Unternehmens sei $K(x) = 0.1x^3 - 2x^2 + 50x + 100$. Die Durchschnittskosten (Stückkosten) sind $k(x) = \frac{K(x)}{x}$.
        \begin{itemize}
            \item Schreibe die Funktion für die Durchschnittskosten $k(x)$ auf.
            \item Bilde die erste Ableitung $k'(x)$ mit der Quotientenregel.
            \item (Für Experten): Versuche, die Stelle zu finden, an der die Durchschnittskosten minimal sind (also $k'(x)=0$ setzen und nach $x$ auflösen – das kann hier schwierig werden, aber der Ansatz ist wichtig).
        \end{itemize}
\end{enumerate}
\end{aufgabenumgebung}

% Vorheriger Inhalt des Kapitels bis zur aufgabenumgebung 'Anwendung der neuen Regeln in Kontexten'
% ... (siehe vorherige Canvas-Version) ...

% HIER FOLGT DANN DIE KETTENREGEL
% Dieser Kommentar wird durch den folgenden Inhalt ersetzt:

\subsubsection{Die Kettenregel – Ableiten von verketteten Funktionen $f(x) = g(h(x))$}
\label{subsubsec:kettenregel}

Wir haben gelernt, wie man Summen, Produkte und Quotienten von Funktionen ableitet. Aber was ist, wenn Funktionen ineinander 'verschachtelt' oder 'verkettet' sind? Stell dir eine Funktion vor wie eine Maschine: Du gibst $x$ hinein, die 'innere' Maschine $h$ macht etwas damit ($h(x)$), und dieses Ergebnis wird dann in eine 'äußere' Maschine $g$ gesteckt, die $g(h(x))$ produziert. Ein Beispiel wäre $f(x) = (2x+5)^3$. Hier ist die innere Funktion $h(x)=2x+5$ und die äußere Funktion $g(u)=u^3$.

Solche \textbf{verketteten Funktionen} (auch zusammengesetzte Funktionen genannt) können wir nicht einfach mit den bisherigen Regeln ableiten. Wir brauchen ein neues, sehr mächtiges Werkzeug: die \textbf{Kettenregel}.

\begin{merksatzumgebung}{Kettenregel}
Ist eine Funktion $f(x)$ als Verkettung zweier differenzierbarer Funktionen $g(u)$ (äußere Funktion) und $u=h(x)$ (innere Funktion) gegeben, also $f(x) = g(h(x))$, dann lautet ihre Ableitung:
\[ f'(x) = g'(h(x)) \cdot h'(x) \]
\textbf{In Worten:} 'Die Ableitung der äußeren Funktion (wobei die innere Funktion als Argument eingesetzt bleibt) multipliziert mit der Ableitung der inneren Funktion.'
Oder kurz: \textbf{Äußere Ableitung mal innere Ableitung.}
\end{merksatzumgebung}

\begin{infoboxumgebung}{Die Kettenregel verstehen – Wie eine Zwiebel oder Matrjoschka-Puppen}
Stell dir eine verkettete Funktion wie eine Zwiebel oder eine russische Matrjoschka-Puppe vor. Um zum Kern zu gelangen, musst du Schicht für Schicht 'ableiten':
\begin{enumerate}
    \item \textbf{Äußerste Schicht (äußere Funktion $g$):} Leite sie ab, aber behalte das, was 'innen' ist ($h(x)$), einfach bei. Das ist $g'(h(x))$.
    \item \textbf{Nächste Schicht (innere Funktion $h$):} Leite sie ab. Das ist $h'(x)$.
    \item \textbf{Multipliziere die Ergebnisse:} $g'(h(x)) \cdot h'(x)$.
\end{enumerate}
Wenn es mehrere Verschachtelungen gibt, z.B. $f(x) = a(b(c(x)))$, dann gilt die Regel entsprechend erweitert: $f'(x) = a'(b(c(x))) \cdot b'(c(x)) \cdot c'(x)$. Man leitet von außen nach innen ab und multipliziert die einzelnen Ableitungen.
\end{infoboxumgebung}

Schauen wir uns das an einem Beispiel an, das wir auch ohne Kettenregel (mit viel Mühe) lösen könnten.

\begin{beispielumgebung}{Anwendung der Kettenregel bei einer potenzierten Klammer}
Wir wollen die Funktion $f(x) = (2x+5)^3$ ableiten.

\textbf{Schritt 1: Identifiziere die äußere und innere Funktion.}
\begin{itemize}
    \item Die \textbf{innere Funktion} ist das, was in der Klammer potenziert wird: $h(x) = 2x+5$.
    \item Die \textbf{äußere Funktion} ist das Potenzieren mit 3: Wenn wir $u = h(x)$ setzen, ist $g(u) = u^3$.
\end{itemize}
Also $f(x) = g(h(x))$ mit $g(u)=u^3$ und $h(x)=2x+5$.

\textbf{Schritt 2: Bilde die Ableitungen der inneren und äußeren Funktion.}
\begin{itemize}
    \item Ableitung der inneren Funktion: $h'(x) = (2x+5)' = 2$.
    \item Ableitung der äußeren Funktion (nach ihrer Variablen $u$): $g'(u) = (u^3)' = 3u^2$.
\end{itemize}

\textbf{Schritt 3: Setze in die Kettenregel $f'(x) = g'(h(x)) \cdot h'(x)$ ein.}
\begin{itemize}
    \item $g'(h(x))$ bedeutet: Nimm die äußere Ableitung $g'(u)=3u^2$ und setze für $u$ wieder die innere Funktion $h(x)=2x+5$ ein.
    Also: $g'(h(x)) = 3(2x+5)^2$.
    \item $h'(x)$ hatten wir schon: $h'(x) = 2$.
\end{itemize}
Zusammengesetzt:
$f'(x) = \underbrace{3(2x+5)^2}_{g'(h(x))} \cdot \underbrace{2}_{h'(x)}$
\[ f'(x) = 6(2x+5)^2 \]

\textbf{Probe durch Ausmultiplizieren (optional und hier aufwendig):}
$(2x+5)^3 = (2x+5)(2x+5)(2x+5) = (4x^2+20x+25)(2x+5)$
$= 8x^3 + 20x^2 + 40x^2 + 100x + 50x + 125$
$= 8x^3 + 60x^2 + 150x + 125$.
Ableiten mit Summen-/Potenzregel:
$f'(x) = (8x^3)' + (60x^2)' + (150x)' + (125)'$
$f'(x) = 24x^2 + 120x + 150$.

Ist das dasselbe wie $6(2x+5)^2$?
$6(2x+5)^2 = 6((2x)^2 + 2 \cdot 2x \cdot 5 + 5^2) = 6(4x^2 + 20x + 25)$
$= 24x^2 + 120x + 150$. Ja, es stimmt!
Die Kettenregel war hier deutlich schneller als das Ausmultiplizieren.
\end{beispielumgebung}

\begin{tippumgebung}{Spezialfall der Kettenregel: Lineare innere Funktion}
Wenn die innere Funktion linear ist, also $h(x) = mx+b$, dann ist $h'(x)=m$.
Die Kettenregel für $f(x) = g(mx+b)$ lautet dann:
$f'(x) = g'(mx+b) \cdot m$.
Beispiel: $f(x) = (4x-7)^5$. Äußere Funktion $g(u)=u^5 \implies g'(u)=5u^4$. Innere Funktion $h(x)=4x-7 \implies h'(x)=4$.
$f'(x) = 5(4x-7)^4 \cdot 4 = 20(4x-7)^4$.
\end{tippumgebung}

\begin{aufgabenumgebung}{Kettenregel trainieren}
Leite die folgenden Funktionen mit der Kettenregel ab. Identifiziere zuerst sorgfältig die äußere und die innere Funktion.
\begin{enumerate}
    \item $f(x) = (x^2+1)^4$
    \item $g(x) = (5-3x)^7$
    \item $h(t) = \sqrt{t^2+3t}$ (Tipp: $\sqrt{u} = u^{1/2}$)
    \item $k(x) = \frac{1}{(x^3-2x)^2}$ (Tipp: $k(x) = (x^3-2x)^{-2}$)
    \item $m(x) = (ax^2+b)^n$ (wobei $a,b,n$ Konstanten sind. Was ist die Ableitung?)
\end{enumerate}
\end{aufgabenumgebung}

\begin{fehlerboxumgebung}{Häufige Fehler bei der Kettenregel}
\begin{itemize}
    \item \textbf{Die innere Ableitung $h'(x)$ wird vergessen!} Das ist der häufigste Fehler. Man leitet die äußere Funktion ab und vergisst, mit der Ableitung der inneren Funktion zu multiplizieren.
    \item \textbf{Falsche Identifikation von äußerer und innerer Funktion:} Überlege dir genau, welche Operation 'zuerst' auf $x$ wirkt (innere Funktion) und welche 'danach' auf das Ergebnis (äußere Funktion).
    \item \textbf{Äußere Funktion nicht korrekt abgeleitet:} Wenn $g(u)=u^n$, dann ist $g'(u)=nu^{n-1}$. In $g'(h(x))$ muss dann für $u$ der gesamte innere Term $h(x)$ eingesetzt werden, bevor mit $h'(x)$ multipliziert wird.
\end{itemize}
\end{fehlerboxumgebung}

\begin{warumwichtigumgebung}{Die Kettenregel}
Die Kettenregel ist eine der wichtigsten und am häufigsten angewendeten Ableitungsregeln. Sehr viele Funktionen, denen wir begegnen, sind Verkettungen. Ohne die Kettenregel könnten wir Funktionen wie $e^{x^2}$, $\sin(3x)$, $\ln(x^2+1)$ oder $\sqrt{4-x^2}$ nicht ableiten. Sie ist der Schlüssel zur Analyse einer riesigen Klasse von Funktionen und deren Verhalten.
\end{warumwichtigumgebung}

\begin{kurzknappumgebung}{Kettenregel}
\begin{itemize}
    \item \textbf{Anwendung:} Für verkettete Funktionen $f(x) = g(h(x))$ ('Funktion in Funktion').
    \item \textbf{Formel:} $f'(x) = g'(h(x)) \cdot h'(x)$ ('Äußere Ableitung mal innere Ableitung').
    \item \textbf{Vorgehen:}
        \begin{enumerate}
            \item Innere Funktion $h(x)$ und äußere Funktion $g(u)$ identifizieren.
            \item Beide getrennt ableiten: $h'(x)$ und $g'(u)$.
            \item In $g'(u)$ für $u$ wieder $h(x)$ einsetzen, um $g'(h(x))$ zu erhalten.
            \item $g'(h(x))$ mit $h'(x)$ multiplizieren.
        \end{enumerate}
\end{itemize}
\end{kurzknappumgebung}


\subsubsection{Zusammenfassung und kombinierte Anwendung aller Ableitungsregeln}
\label{subsubsec:anwendung_aller_regeln_neu} % Neues Label

Jetzt, da wir die Konstanten-, Potenz-, Faktor-, Summen-, Produkt-, Quotienten- und Kettenregel kennen, haben wir einen mächtigen Werkzeugkasten, um eine sehr große Vielfalt von Funktionen zu differenzieren. Oft müssen wir mehrere dieser Regeln in einer einzigen Aufgabe geschickt kombinieren.

\begin{kurzknappumgebung}{Alle bisherigen Ableitungsregeln im Überblick}
\begin{itemize}
    \item \textbf{Konstantenregel:} $(c)' = 0$
    \item \textbf{Potenzregel:} $(x^n)' = nx^{n-1}$
    \item \textbf{Faktorregel:} $(c \cdot g(x))' = c \cdot g'(x)$
    \item \textbf{Summenregel:} $(g(x) \pm h(x))' = g'(x) \pm h'(x)$
    \item \textbf{Produktregel:} $(u(x) \cdot v(x))' = u'(x)v(x) + u(x)v'(x)$
    \item \textbf{Quotientenregel:} $\left(\frac{u(x)}{v(x)}\right)' = \frac{u'(x)v(x) - u(x)v'(x)}{[v(x)]^2}$
    \item \textbf{Kettenregel:} $(g(h(x)))' = g'(h(x)) \cdot h'(x)$
\end{itemize}
Das Erkennen, welche Regel(n) in welcher Reihenfolge anzuwenden sind, ist eine Frage der Übung und des genauen Hinsehens auf die Struktur der Funktion.
\end{kurzknappumgebung}

\begin{aufgabenumgebung}[A:KombinierteAnwendung]{Kombinierte Anwendung der Ableitungsregeln}
Leite die folgenden Funktionen ab. Gib an, welche Regeln du in welcher Reihenfolge anwendest.
\begin{enumerate}
    \item $f(x) = x^2 \cdot (2x+1)^3$ (Produkt- und Kettenregel)
    \item $g(x) = \frac{(x^2-1)^2}{x}$ (Quotienten- und Kettenregel, oder erst Zähler ausmultiplizieren)
    \item $h(t) = t \cdot \sqrt{1-t^2}$ (Produkt- und Kettenregel)
    \item \textbf{Für Tüftler:} Untersuche die Funktion $f(x) = x \cdot (x-4)^3$ auf Nullstellen, Monotonie und Extrempunkte. Skizziere den Graphen. (Diese Funktion ähnelt der Aufgabe \ref{A:ProduktregelAnwendung}, erfordert aber nun die Kettenregel für $v'(x)$.)
\end{enumerate}
\end{aufgabenumgebung}

\begin{tippumgebung}{Struktur beim Ableiten komplexer Funktionen}
Wenn du eine komplizierte Funktion ableiten musst:
\begin{enumerate}
    \item \textbf{Analysiere die Gesamtstruktur:} Ist es eine Summe, ein Produkt, ein Quotient oder eine Verkettung als 'oberste' Operation?
    \item \textbf{Wende die entsprechende Hauptregel an.} Die Teile $u(x), v(x), g(u), h(x)$ können selbst wieder komplex sein.
    \item \textbf{Leite die Teilfunktionen ab:} Hierfür musst du eventuell erneut Ableitungsregeln anwenden. Arbeite dich 'von außen nach innen' oder 'von oben nach unten' durch die Struktur.
    \item \textbf{Setze alles zusammen und vereinfache (wenn nötig und sinnvoll).}
\end{enumerate}
Mit etwas Übung entwickelst du einen Blick dafür!
\end{tippumgebung}



\subsubsection{Weitere herausfordernde Anwendungsaufgaben}
\label{subsubsec:weitere_anwendungen_diff}

Die folgenden Aufgaben sind etwas komplexer und erfordern die sorgfältige Anwendung und Kombination der bisher gelernten Konzepte der Differentialrechnung. Sie sind eine gute Vorbereitung auf typische Problemstellungen, wie sie auch im Abitur vorkommen können.


\begin{aufgabenumgebung}{Optimierungsproblem – Die optimale Schachtel}
Aus einem quadratischen Stück Pappe der Seitenlänge $L=30\,$cm soll durch Ausschneiden von Quadraten an den Ecken und anschließendes Hochbiegen der entstehenden Seitenlaschen eine offene Schachtel (ohne Deckel) mit maximalem Volumen hergestellt werden.

\begin{center} % Zentriert den Inhalt
    \includegraphics[width=0.9\textwidth]{grafiken/Optimierung_Schachtel.png}
    \captionof{figure}{Von der Pappe zur Schachtel}
    \label{fig:optimierung_schachtel}
\end{center}
\begin{enumerate}
    \item \textbf{Variable festlegen:} Sei $x$ die Seitenlänge der Quadrate, die an den Ecken ausgeschnitten werden.
    \item \textbf{Maße der Schachtel:} Drücke die Länge $l$, die Breite $b$ und die Höhe $h$ der entstehenden Schachtel in Abhängigkeit von $x$ aus. Bedenke, dass von jeder Seite der Pappe $2x$ weggeschnitten wird.
    \item \textbf{Definitionsbereich für $x$:} Welche Werte für $x$ sind in diesem Sachzusammenhang sinnvoll? (Die Seitenlängen müssen positiv sein, und man kann nicht mehr wegschneiden, als Pappe da ist).
    \item \textbf{Zielfunktion für das Volumen:} Stelle die Funktion $V(x)$ auf, die das Volumen der Schachtel in Abhängigkeit von $x$ beschreibt ($V = l \cdot b \cdot h$).
    \item \textbf{Extremwertsuche:}
        \begin{itemize}
            \item Bilde die erste Ableitung $V'(x)$.
            \item Setze $V'(x)=0$ und löse nach $x$, um die kritischen Stellen zu finden.
            \item Überprüfe mit der zweiten Ableitung $V''(x)$ (oder dem Vorzeichenwechselkriterium von $V'(x)$), ob an den kritischen Stellen ein Maximum oder Minimum vorliegt.
            \item Berücksichtige den Definitionsbereich von $x$: Liegen die gefundenen Extremstellen im sinnvollen Bereich?
        \end{itemize}
    \item \textbf{Antwort:} Gib die Seitenlänge $x$ der auszuschneidenden Quadrate an, für die das Volumen der Schachtel maximal wird, sowie das maximale Volumen selbst.
\end{enumerate}
\end{aufgabenumgebung}

\begin{aufgabenumgebung}{Rekonstruktion einer Polynomfunktion}
Eine ganzrationale Funktion dritten Grades $f(x) = ax^3 + bx^2 + cx + d$ hat die folgenden Eigenschaften:
\begin{itemize}
    \item Der Graph der Funktion geht durch den Ursprung $P(0|0)$.
    \item Der Ursprung ist gleichzeitig ein Wendepunkt der Funktion.
    \item Die Tangente im Wendepunkt (die Wendetangente) hat die Gleichung $y_W(x) = -3x$.
    \item Der Graph der Funktion hat eine Nullstelle bei $x_N = 1$.
\end{itemize}
Bestimme die Funktionsgleichung $f(x)$.

\begin{tippumgebung}{Bedingungen übersetzen}
Übersetze jede der gegebenen Eigenschaften in eine mathematische Gleichung für die Funktion $f(x)$ oder ihre Ableitungen $f'(x)$ und $f''(x)$:
\begin{itemize}
    \item 'Graph geht durch $P(x_0|y_0)$' $\implies f(x_0) = y_0$.
    \item 'Wendepunkt bei $x_W$' $\implies f''(x_W) = 0$.
    \item 'Tangentensteigung im Punkt $P(x_P|f(x_P))$ ist $m$' $\implies f'(x_P) = m$. Die Wendetangente gibt dir also die Steigung im Wendepunkt.
    \item 'Nullstelle bei $x_N$' $\implies f(x_N) = 0$.
\end{itemize}
Du erhältst ein lineares Gleichungssystem mit den Unbekannten $a, b, c, d$. Löse dieses System.
\end{tippumgebung}


\end{aufgabenumgebung}

\begin{infoboxumgebung}{Lineare Gleichungssysteme (LGS) lösen – Ein kurzer Überblick}
Wenn du die Bedingungen aus der Aufgabe oben in Gleichungen übersetzt, wirst du ein System von mehreren linearen Gleichungen mit mehreren Unbekannten ($a,b,c,d$) erhalten. So ein System nennt man \textbf{Lineares Gleichungssystem (LGS)}.
Zum Beispiel könnte ein einfaches LGS mit zwei Gleichungen und zwei Unbekannten $x,y$ so aussehen:
\begin{align*}
    I: \quad 2x + 3y &= 7 \\
    II: \quad x - y &= 1
\end{align*}
Es gibt verschiedene Methoden, solche Systeme zu lösen:
\begin{itemize}
    \item \textbf{Einsetzungsverfahren:} Eine Gleichung nach einer Variablen auflösen und diesen Term in die andere(n) Gleichung(en) einsetzen.
    \item \textbf{Gleichsetzungsverfahren:} Zwei Gleichungen nach derselben Variablen auflösen und die entstehenden Terme gleichsetzen.
    \item \textbf{Additions-/Subtraktionsverfahren:} Gleichungen (oder Vielfache davon) so addieren oder subtrahieren, dass eine Variable wegfällt.
\end{itemize}
Für größere Systeme mit mehr Variablen (wie hier mit $a,b,c,d$) werden diese Verfahren schnell unübersichtlich. Es gibt aber systematischere Methoden, wie das \textbf{Gaußsche Eliminationsverfahren} (oft auch mit Matrizen dargestellt), die in der Schule meist ausführlich behandelt werden.

\textbf{Für diese Aufgabe:}
Versuche, die vier Gleichungen, die du aus den Bedingungen erhältst, geschickt zu nutzen. Oft sind einige Gleichungen sehr einfach (z.B. wenn $f(0)=0$ direkt $d=0$ liefert). Setze bekannte Werte direkt in die anderen Gleichungen ein, um das System zu vereinfachen.

\textit{Hinweis zum Selbstlernen:} Das Lösen von LGS ist ein eigenes wichtiges Thema. Wenn du hier Schwierigkeiten hast, ist das nicht schlimm! Du kannst diesen Teil der Aufgabe überspringen oder dich auf das Aufstellen der Gleichungen konzentrieren. Das Lösen von LGS ist aber eine sehr nützliche Fähigkeit für viele Bereiche der Mathematik und darüber hinaus – es lohnt sich, das bei Gelegenheit zu üben! Diese Aufgabe ist eine gute Herausforderung, um dein mathematisches Denken zu schulen.
\end{infoboxumgebung}

\begin{aufgabenumgebung}{Bewegungsanalyse – Zwei Läufer auf der Bahn}
Zwei Läufer, A und B, bewegen sich auf einer geraden Bahn.
Läufer A startet zum Zeitpunkt $t=0\,$s am Punkt $s_A(0)=0\,$m. Seine Position (in Metern) zum Zeitpunkt $t$ (in Sekunden) wird durch die Funktion $s_A(t) = t^2$ beschrieben.
Läufer B startet gleichzeitig am Punkt $s_B(0)=10\,$m. Seine Position wird durch $s_B(t) = -0.5t^2 + 7t + 10$ beschrieben.
Wir betrachten das Zeitintervall $[0, 5]$ Sekunden.

\begin{enumerate}
    \item \textbf{Geschwindigkeiten:} Bestimme die Geschwindigkeitsfunktionen $v_A(t) = s_A'(t)$ und $v_B(t) = s_B'(t)$ der beiden Läufer.
    \item \textbf{Gleiche Geschwindigkeit:} Zu welchem Zeitpunkt $t$ haben beide Läufer die gleiche Geschwindigkeit? Wie groß ist diese Geschwindigkeit?
    \item \textbf{Gleiche Position:} Haben die Läufer jemals die gleiche Position im betrachteten Zeitintervall $[0,5]$? Wenn ja, zu welchem Zeitpunkt/welchen Zeitpunkten?
        \begin{tippumgebung}{Gleichung lösen}
        Um herauszufinden, wann sie die gleiche Position haben, musst du die Gleichung $s_A(t) = s_B(t)$ lösen. Das wird auf eine quadratische Gleichung führen.
        \end{tippumgebung}
    \item \textbf{Abstand der Läufer:}
        \begin{itemize}
            \item Stelle eine Funktion $d(t)$ auf, die den Abstand zwischen den beiden Läufern zum Zeitpunkt $t$ beschreibt. 
            \textit{Hinweis:} Überlege zuerst, welcher Läufer im Intervall $[0,5]$ vorne liegt, um das Betragszeichen bei der Differenz $d(t) = |s_B(t) - s_A(t)|$ auflösen zu können. Du hast in Teil c) untersucht, ob sie sich treffen.
            \item Zu welchem Zeitpunkt im Intervall $[0,5]$ ist der Abstand zwischen den Läufern minimal? Wie groß ist dieser minimale Abstand?
            \item Zu welchem Zeitpunkt im Intervall $[0,5]$ ist der Abstand zwischen den Läufern maximal? Wie groß ist dieser maximale Abstand?
            \begin{tippumgebung}{Extremwerte im Intervall}
            Um Extremwerte einer Funktion in einem abgeschlossenen Intervall $[t_1, t_2]$ zu finden, musst du die Funktionswerte an den kritischen Stellen (wo $d'(t)=0$) \textbf{und} an den Rändern des Intervalls ($t_1$ und $t_2$) untersuchen und vergleichen.
            \end{tippumgebung}
        \end{itemize}
\end{enumerate}
\end{aufgabenumgebung}

\begin{aufgabenumgebung}{Tangentenprobleme an einer kubischen Funktion}
Gegeben ist die Funktion $f(x) = x^3 - 3x$.
\begin{enumerate}
    \item \textbf{Parallele Tangenten:}
        \begin{itemize}
            \item Bestimme die Steigung der Geraden $g(x) = 9x - 5$.
            \item Gibt es Punkte auf dem Graphen von $f(x)$, an denen die Tangente parallel zur Geraden $g(x)$ ist? Wenn ja, bestimme die Koordinaten dieser Punkte.
            \item Gib die Gleichungen der Tangenten an den Graphen von $f(x)$ in diesen Punkten an.
        \end{itemize}
    \item \textbf{Tangente von einem externen Punkt:}
        Von welchem Punkt $P_y(0|y_P)$ auf der y-Achse aus kann man eine Tangente an den Graphen von $f(x)$ legen, die den Graphen an der Stelle $x_B=2$ berührt?
        \begin{tippumgebung}{Schrittweise Lösung}
        \begin{enumerate}
            \item Bestimme die y-Koordinate des Berührpunkts $B(2|f(2))$.
            \item Bestimme die Steigung $m_T$ der Tangente an den Graphen von $f(x)$ an der Stelle $x_B=2$.
            \item Stelle die Gleichung der Tangente $t_B(x)$ im Punkt $B$ auf.
            \item Der gesuchte Punkt $P_y(0|y_P)$ muss auf dieser Tangente $t_B(x)$ liegen. Setze $x=0$ in die Tangentengleichung ein, um $y_P$ zu finden.
        \end{enumerate}
        \end{tippumgebung}
\end{enumerate}
\end{aufgabenumgebung}
% Hier geht es dann weiter mit der Tippumgebung 'Struktur beim Ableiten komplexer Funktionen'
% und dem Abschluss des Kapitels.


% Hier geht es dann weiter mit der Tippumgebung 'Struktur beim Ableiten komplexer Funktionen'
% und dem Abschluss des Kapitels.
% Vorheriger Inhalt des Kapitels bis zur letzten aufgabenumgebung
% ... (siehe vorherige Canvas-Version, die mit der Aufgabe 'Tangentenprobleme an einer kubischen Funktion' endet) ...

% Dieser Block ersetzt den Kommentar:
% % Hier geht es dann weiter mit der Tippumgebung 'Struktur beim Ableiten komplexer Funktionen',
% % der Quotientenregel etc.



\begin{infoboxumgebung}{Ausblick auf weitere Ableitungsregeln und Funktionen}
Wir haben nun die wichtigsten Ableitungsregeln (Konstanten-, Potenz-, Faktor-, Summen-, Produkt-, Quotienten- und Kettenregel) kennengelernt. Mit diesem Werkzeugkasten kannst du schon eine riesige Bandbreite an Funktionen ableiten!

In den folgenden Kapiteln (oder in weiterführenden Kursen) wirst du lernen, wie man auch andere wichtige Funktionstypen ableitet, wie zum Beispiel:
\begin{itemize}
    \item \textbf{Exponentialfunktionen} (z.B. $f(x) = e^x$ oder $f(x) = 2^x$)
    \item \textbf{Logarithmusfunktionen} (z.B. $f(x) = \ln(x)$ oder $f(x) = \log_{10}(x)$)
    \item \textbf{Trigonometrische Funktionen} (z.B. $f(x) = \sin(x)$, $f(x) = \cos(x)$, $f(x) = \tan(x)$)
\end{itemize}
Die hier gelernten Regeln, insbesondere die Produkt-, Quotienten- und Kettenregel, werden auch für diese Funktionstypen von zentraler Bedeutung sein, wenn sie in Kombinationen auftreten (z.B. $f(x) = x \cdot e^x$ oder $f(x) = \sin(x^2)$).
Das Fundament, das du dir hier erarbeitet hast, ist also sehr wertvoll für alles, was noch kommt!
\end{infoboxumgebung}



\section*{Abschluss des Kapitels zur Differentialrechnung}

Herzlichen Glückwunsch! Du hast dich nun intensiv mit den Grundlagen der Differentialrechnung auseinandergesetzt. Von der intuitiven Idee der Tangentensteigung über die formale Definition der Ableitung bis hin zu den wichtigen Ableitungsregeln und ihrer Anwendung in Kurvendiskussionen und Optimierungsproblemen hast du einen weiten Weg zurückgelegt.

\begin{merksatzumgebung}[Was du mitnehmen solltest]{Kernkompetenzen dieses Kapitels}
\begin{itemize}
    \item Du verstehst die \textbf{Ableitung} als Maß für die momentane Veränderung einer Funktion und als Steigung ihrer Tangente.
    \item Du kannst die \textbf{grundlegenden Ableitungsregeln} (Konstanten-, Potenz-, Faktor-, Summen-, Produkt-, Quotienten- und Kettenregel) sicher anwenden, um die Ableitungsfunktionen verschiedener Funktionstypen zu bestimmen.
    \item Du kannst die \textbf{erste und zweite Ableitung} nutzen, um das Verhalten von Funktionen detailliert zu untersuchen: Monotonie, Art und Lage von Extrempunkten, Krümmungsverhalten und Wendepunkte.
    \item Du kannst das \textbf{Grenzwertverhalten} von Polynomen und einfachen gebrochen-rationalen Funktionen analysieren.
    \item Du bist in der Lage, eine \textbf{vollständige Kurvendiskussion} für Polynomfunktionen und einfache gebrochen-rationale Funktionen durchzuführen und deren Graphen zu skizzieren.
    \item Du hast erste Einblicke gewonnen, wie die Differentialrechnung zur Lösung von \textbf{Anwendungsproblemen} (z.B. Optimierung, Bewegungsanalyse) eingesetzt werden kann.
\end{itemize}
Diese Fähigkeiten sind nicht nur für die Mathematik selbst von großer Bedeutung, sondern bilden auch die Grundlage für viele Anwendungen in den Naturwissenschaften, der Technik und den Wirtschaftswissenschaften.
\end{merksatzumgebung}

\begin{infoboxumgebung}{Der Weg geht weiter...}
Die Differentialrechnung ist nur ein Teil der Analysis. Ein ebenso wichtiges und eng damit verbundenes Gebiet ist die \textbf{Integralrechnung}, mit der wir uns im nächsten Kapitel beschäftigen werden. Dort geht es darum, den umgekehrten Prozess zur Ableitung zu finden (das 'Aufleiten' oder Integrieren) und damit zum Beispiel Flächen unter Kurven oder Volumina von Rotationskörpern zu berechnen. Du wirst sehen, dass viele der hier gelernten Konzepte auch dort wieder eine Rolle spielen werden.

Bleib neugierig und übe fleißig weiter – die Welt der Mathematik hat noch viele spannende Entdeckungen für dich bereit!
\end{infoboxumgebung}

\begin{aufgabenumgebung}{Checkliste: Von kubischen Funktionen zu linearen Ableitungen}
Diese Aufgabe zeigt dir, wie dein Wissen über lineare Funktionen dir hilft, das Verhalten von Polynomen 3. Grades zu verstehen.
Betrachte die Funktion $f(x) = x^3 - 6x^2 + 10x - 3$.

\begin{enumerate}[label=(\alph*)]
    \item \textbf{Ableitungen bilden:} Berechne die erste Ableitung $f'(x)$ und die zweite Ableitung $f''(x)$ der Funktion $f(x)$.
    \item \textbf{Analyse der zweiten Ableitung:}
    \begin{itemize}
        \item Welchen Funktionstyp erkennst du in $f''(x)$? Gib die Steigung $m_{f''}$ und den y-Achsenabschnitt $c_{f''}$ dieser Funktion an.
        \item Berechne die Nullstelle $x_W$ von $f''(x)$. Welche besondere Bedeutung hat diese Stelle $x_W$ für den Graphen der ursprünglichen Funktion $f(x)$? (Erinnere dich an die Definition von Wendepunkten).
    \end{itemize}
    \item \textbf{Krümmungsverhalten von $f(x)$ bestimmen:}
    Nutze dein Wissen über lineare Funktionen, um das Vorzeichen von $f''(x)$ zu bestimmen:
    \begin{itemize}
        \item Für welche $x$-Werte ist $f''(x) > 0$? (Tipp: Wann sind die Werte einer steigenden/fallenden linearen Funktion positiv?)
        \item Für welche $x$-Werte ist $f''(x) < 0$?
        \item Welche Schlussfolgerungen ziehst du daraus für das Krümmungsverhalten (links- oder rechtsgekrümmt) des Graphen von $f(x)$? Gib die Intervalle an.
    \end{itemize}
    \item \textbf{Reflexion:} Erkläre in eigenen Worten, warum das Verständnis der Eigenschaften einer linearen Funktion (insbesondere ihrer Nullstelle und des Vorzeichenverlaufs in Abhängigkeit von der Steigung) nützlich ist, um das Krümmungsverhalten einer kubischen Funktion zu analysieren.
\end{enumerate}
\end{aufgabenumgebung}

\begin{aufgabenumgebung}{Checkliste: Von Polynomen 4. Grades zu quadratischen Ableitungen}
Diese Aufgabe zeigt dir, wie dein Wissen über quadratische Funktionen dir hilft, das Verhalten von Polynomen 4. Grades zu verstehen.
Betrachte die Funktion $g(x) = \frac{1}{4}x^4 - x^3 + x^2 + 1$.

\begin{enumerate}[label=(\alph*)]
    \item \textbf{Ableitungen bilden:} Berechne die erste Ableitung $g'(x)$ und die zweite Ableitung $g''(x)$ der Funktion $g(x)$.
    \item \textbf{Analyse der zweiten Ableitung:}
    \begin{itemize}
        \item Welchen Funktionstyp erkennst du in $g''(x)$? Gib die Parameter dieser Funktion an (z.B. Öffnungsfaktor, etc.). Ist der Graph von $g''(x)$ nach oben oder unten geöffnet?
        \item Berechne die Nullstellen $x_{W1}, x_{W2}$ von $g''(x)$ (falls vorhanden). Welche Bedeutung haben diese Stellen für den Graphen von $g(x)$?
    \end{itemize}
    \item \textbf{Krümmungsverhalten von $g(x)$ bestimmen:}
    Nutze dein Wissen über quadratische Funktionen, um das Vorzeichen von $g''(x)$ zu bestimmen:
    \begin{itemize}
        \item Skizziere (oder stelle dir vor) den Graphen von $g''(x)$ basierend auf Öffnungsrichtung und Nullstellen.
        \item Für welche $x$-Werte ist $g''(x) > 0$?
        \item Für welche $x$-Werte ist $g''(x) < 0$?
        \item Welche Schlussfolgerungen ziehst du daraus für das Krümmungsverhalten des Graphen von $g(x)$? Gib die Intervalle an. Wie viele Wendepunkte hat $g(x)$?
    \end{itemize}
    \item \textbf{Reflexion:} Erkläre, wie das Verständnis der Eigenschaften einer quadratischen Funktion (Öffnung, Nullstellen, Vorzeichenverlauf) hilft, das Krümmungsverhalten eines Polynoms 4. Grades zu analysieren.
\end{enumerate}
\end{aufgabenumgebung}

\begin{aufgabenumgebung}{Checkliste: Die maximale Steigung finden (Anwendung)}
Oft ist nicht nur interessant, wo eine Funktion ihren höchsten oder tiefsten Wert hat, sondern auch, wo sie am stärksten steigt oder fällt. Das führt uns zur Untersuchung der Ableitung selbst.
Ein Unternehmen stellt fest, dass seine Produktionskosten $K(x)$ (in Euro) bei der Herstellung von $x$ Einheiten eines Produkts durch die Funktion $K(x) = \frac{1}{3}x^3 - 10x^2 + 150x + 500$ für $x \in [0, 25]$ beschrieben werden können.
Die \textit{Grenzkosten} geben an, um wie viel die Kosten ungefähr steigen, wenn eine Einheit mehr produziert wird. Mathematisch sind die Grenzkosten die Ableitung der Kostenfunktion, also $K'(x)$.
Das Unternehmen möchte wissen, bei welcher Produktionsmenge $x$ die Grenzkosten $K'(x)$ \textbf{minimal} sind (d.h., wann die Kosten pro zusätzlich produzierter Einheit am geringsten ansteigen).

\begin{enumerate}[label=(\alph*)]
    \item \textbf{Grenzkostenfunktion bestimmen:} Bilde die erste Ableitung $K'(x)$ der Kostenfunktion $K(x)$. Diese Funktion $K'(x)$ beschreibt die Steigung der Kostenfunktion.
    \item \textbf{Ziel verstehen:} Wir suchen das Minimum der Funktion $K'(x)$. Wie findet man normalerweise Minima einer Funktion? (Tipp: Denke an die Ableitung der zu untersuchenden Funktion!)
    \item \textbf{Ableitung der Grenzkostenfunktion bilden:} Bilde die Ableitung von $K'(x)$, also die zweite Ableitung der ursprünglichen Kostenfunktion, $K''(x)$.
    \item \textbf{Kritische Stelle für $K'(x)$ finden:} Setze $K''(x) = 0$ und löse nach $x$. Dies ist die potenzielle Stelle $x_W$, an der die Grenzkosten $K'(x)$ minimal (oder maximal) sein könnten.
    \item \textbf{Art des Extremums von $K'(x)$ prüfen:} Überprüfe mit der nächsten Ableitung, also $K'''(x)$, ob bei $x_W$ tatsächlich ein Minimum für $K'(x)$ vorliegt. (Wenn $K'''(x_W) \neq 0$ und $K''(x_W)=0$, dann ist $x_W$ ein Wendepunkt von $K(x)$ und ein Extremum von $K'(x)$). Alternativ: Untersuche den Vorzeichenwechsel von $K''(x)$ bei $x_W$.
    \item \textbf{Antwort formulieren:} Bei welcher Produktionsmenge $x$ sind die Grenzkosten minimal? Wie hoch sind die minimalen Grenzkosten $K'(x_W)$?
    \item \textbf{Reflexion:} Was für ein besonderer Punkt ist $x_W$ für die ursprüngliche Kostenfunktion $K(x)$? Warum ist es plausibel, dass die Steigung einer Funktion (hier $K(x)$) an einem Wendepunkt maximal oder minimal wird?
\end{enumerate}
\end{aufgabenumgebung}


\section{Einführung in die Integralrechnung}
\label{sec:integralrechnung} % Neues Label zur Unterscheidung

Willkommen zum nächsten großen Abenteuer in der Analysis: der \textbf{Integralrechnung}! Nachdem wir uns im vorherigen Kapitel intensiv damit beschäftigt haben, wie sich Funktionen verändern (Steigung, Ableitung), wollen wir uns nun oft dem umgekehrten Problem zuwenden oder eine ganz neue Frage stellen: Wie groß ist eigentlich die Fläche unter einer Kurve? \\

\begin{tcolorbox}[colback=blue!5!white, colframe=blue!75!black, title=Was du in diesem Kapitel lernen wirst:]
Nachdem du dieses Kapitel durchgearbeitet hast, wirst du in der Lage sein:
\begin{itemize}[noitemsep, topsep=0pt, leftmargin=*, itemsep=2pt]
    \item das Konzept des \textbf{bestimmten Integrals} als Grenzwert von Riemannsummen (Ober- und Untersummen) zu verstehen und es als (orientierten) \textbf{Flächeninhalt} unter einem Funktionsgraphen zu interpretieren.
    \item den Begriff der \textbf{Stammfunktion} $F(x)$ als Umkehrung der Ableitung zu definieren und die Menge aller Stammfunktionen als \textbf{unbestimmtes Integral} $\int f(x) \,dx = F(x)+C$ zu verstehen.
    \item die grundlegenden \textbf{Integrationsregeln} für Polynomfunktionen und einfache Potenzfunktionen (Potenz-, Faktor-, Summenregel, Integral einer Konstanten) sicher anzuwenden, um Stammfunktionen zu bilden.
    \item den \textbf{Hauptsatz der Differential- und Integralrechnung (HDI)} zu verstehen und anzuwenden, um bestimmte Integrale mithilfe von Stammfunktionen ($F(b)-F(a)$) exakt zu berechnen.
    \item geometrische \textbf{Flächeninhalte} mit bestimmten Integralen zu ermitteln, auch wenn Teile des Graphen unterhalb der x-Achse liegen oder die Fläche zwischen zwei Kurven eingeschlossen ist.
    \item \textbf{Symmetrieeigenschaften} von Funktionen zu nutzen, um die Berechnung bestimmter Integrale zu vereinfachen.
    \item den \textbf{Mittelwert einer Funktion} über einem Intervall mithilfe des bestimmten Integrals zu berechnen und zu interpretieren.
    \item die Integralrechnung als Werkzeug zur Rekonstruktion von Gesamtgrößen aus gegebenen \textbf{Änderungsraten} in einfachen Anwendungsbeispielen zu verstehen.
\end{itemize}
Du wirst somit die fundamentalen Ideen und Techniken der Integralrechnung für Polynomfunktionen beherrschen und ihre Bedeutung für Flächenberechnungen und andere Anwendungen erkennen.
\end{tcolorbox}
\bigskip

Stell dir vor, du hast den Graphen der Geschwindigkeit eines Autos über die Zeit aufgezeichnet. Die Fläche unter diesem Graphen entspricht der zurückgelegten Strecke. Oder denke an einen Stausee: Wenn du weißt, wie viel Wasser pro Sekunde zufließt (eine Zuflussrate, also eine Änderungsrate), wie kannst du die Gesamtmenge an Wasser im See nach einer bestimmten Zeit berechnen? Für solche und viele andere Probleme liefert die Integralrechnung die passenden Werkzeuge.

\begin{infoboxumgebung}{Wozu Integralrechnung? Ein Universalschlüssel!}
Die Integralrechnung ist nicht nur 'das Gegenteil vom Ableiten'. Sie ist ein unglaublich mächtiges und vielseitiges Konzept, das uns hilft:
\begin{itemize}
    \item \textbf{Flächeninhalte zu berechnen:} Flächen unter Kurven, zwischen Kurven oder von komplexeren Formen.
    \item \textbf{Volumina zu bestimmen:} Volumen von Rotationskörpern oder anderen dreidimensionalen Objekten.
    \item \textbf{Aus Änderungsraten Gesamtgrößen zu rekonstruieren:}
        \begin{itemize}
            \item Aus der Geschwindigkeit die zurückgelegte Strecke.
            \item Aus der Zuflussrate die Gesamtmenge an Wasser.
            \item Aus den Grenzkosten die Gesamtkostenfunktion (bis auf eine Konstante).
        \end{itemize}
    \item \textbf{Viele weitere Anwendungen} in Physik (Arbeit, Ladung), Wirtschaft, Wahrscheinlichkeitsrechnung und Biologie zu erschließen.
\end{itemize}
Du siehst, die Integralrechnung ist ein echter Universalschlüssel in vielen wissenschaftlichen und technischen Disziplinen!
\end{infoboxumgebung}


\begin{funfactbox}{Der kleine Gauß und die blitzschnelle Summe}
Carl Friedrich Gauß (1777--1855) gilt als einer der größten Mathematiker aller Zeiten. Schon als kleiner Junge zeigte sich seine außergewöhnliche Begabung. Eine berühmte Geschichte erzählt, wie sein Lehrer der Klasse die Aufgabe gab, alle Zahlen von 1 bis 100 zu addieren, vermutlich um die Schüler eine Weile zu beschäftigen. Doch der junge Carl Friedrich hatte die Lösung nach kürzester Zeit!

Wie hat er das gemacht? Statt mühsam $1+2+3+\dots+100$ zu rechnen, bemerkte er ein Muster:
\begin{itemize}
    \item Die erste Zahl (1) und die letzte Zahl (100) ergeben zusammen $1+100 = 101$.
    \item Die zweite Zahl (2) und die vorletzte Zahl (99) ergeben zusammen $2+99 = 101$.
    \item Die dritte Zahl (3) und die drittletzte Zahl (98) ergeben zusammen $3+98 = 101$.
\end{itemize}
Er erkannte, dass es genau 50 solcher Paare gibt, die jeweils die Summe 101 ergeben. Also rechnete er blitzschnell: $50 \cdot 101 = 5050$.

Diese Überlegung führt zur allgemeinen Formel für die Summe der ersten $n$ natürlichen Zahlen (die 'Gaußsche Summenformel'):
\[ 1 + 2 + 3 + \dots + n = \frac{n \cdot (n+1)}{2} \]
Für $n=100$ ist das $\frac{100 \cdot (100+1)}{2} = \frac{100 \cdot 101}{2} = 50 \cdot 101 = 5050$.

\textbf{Das Summenzeichen $\Sigma$:}
Um solch lange Summen nicht immer ausschreiben zu müssen, gibt es in der Mathematik ein praktisches Symbol: das griechische große Sigma $\Sigma$. Es bedeutet 'Bilde die Summe von...'.
Die Summe $1+2+3+\dots+100$ können wir damit kurz schreiben als:
\[ \sum_{i=1}^{100} i \]
Das liest sich so: 'Summe aller $i$, wobei $i$ von 1 bis 100 läuft.'
\begin{itemize}
    \item $\Sigma$: Das Summenzeichen selbst.
    \item $i$: Der \textbf{Laufindex} (oft auch $k$ oder $j$ genannt). Er nimmt nacheinander die Werte vom Startwert bis zum Endwert an.
    \item $i=1$: Der \textbf{Startwert} des Laufindexes (unter dem $\Sigma$).
    \item $100$: Der \textbf{Endwert} des Laufindexes (über dem $\Sigma$).
    \item $i$: Der Term, der für jeden Wert des Laufindexes gebildet und dann aufsummiert wird.
\end{itemize}
Allgemein schreibt man: $\sum_{i=k}^{n} a_i = a_k + a_{k+1} + \dots + a_n$.
Diese Schreibweise wird uns im nächsten Abschnitt sehr nützlich sein, wenn wir Flächen unter Kurven durch viele kleine Rechtecke annähern wollen!
\end{funfactbox}


Ein fundamentaler Zugang, um das Konzept der Fläche unter einer Kurve zu verstehen, ist die Annäherung durch einfache geometrische Figuren.

\subsection{Der Weg zur Fläche – Riemannsummen}
\label{sec:riemannsummen_integral}

Wie können wir den Flächeninhalt $A$ unter dem Graphen einer Funktion $f(x)$ über einem Intervall $[a,b]$ bestimmen, wenn der Graph krumm ist und wir keine einfache geometrische Formel dafür haben? Die Idee des Mathematikers Bernhard Riemann (1826–1866) war, diese Fläche durch eine Summe von schmalen Rechtecksflächen anzunähern.

\begin{merksatzumgebung}{Die Idee der Riemannsummen (Untersumme und Obersumme)}
Um den Flächeninhalt unter dem Graphen einer Funktion $f(x)$ im Intervall $[a,b]$ anzunähern (wir nehmen zunächst an, dass $f(x) \ge 0$ im Intervall ist), gehen wir wie folgt vor:
\begin{enumerate}
    \item \textbf{Unterteilung des Intervalls:} Wir zerlegen das Intervall $[a,b]$ in $n$ gleich breite Teilintervalle. Die Breite jedes Teilintervalls ist dann $\Delta x = \frac{b-a}{n}$. Die Teilungspunkte seien $x_0=a, x_1=a+\Delta x, x_2=a+2\Delta x, \dots, x_i=a+i\Delta x, \dots, x_n=a+n\Delta x=b$.
    \item \textbf{Rechtecke konstruieren:} Über jedem Teilintervall $[x_{i-1}, x_i]$ (für $i=1, \dots, n$) errichten wir ein Rechteck. Die Höhe dieses Rechtecks können wir auf verschiedene Weisen wählen:
        \begin{itemize}
            \item \textbf{Untersumme ($U_n$):} Wir wählen als Höhe des $i$-ten Rechtecks den \textit{kleinsten} Funktionswert $m_i = \min_{x \in [x_{i-1}, x_i]} \{f(x)\}$ in diesem Teilintervall. Die Summe der Flächen dieser Rechtecke, $U_n = \sum_{i=1}^{n} m_i \cdot \Delta x$, nennt man die Untersumme. Sie ist eine untere Schranke für den gesuchten Flächeninhalt (d.h. $U_n \le A$).
            \item \textbf{Obersumme ($O_n$):} Wir wählen als Höhe des $i$-ten Rechtecks den \textit{größten} Funktionswert $M_i = \max_{x \in [x_{i-1}, x_i]} \{f(x)\}$ in diesem Teilintervall. Die Summe der Flächen dieser Rechtecke, $O_n = \sum_{i=1}^{n} M_i \cdot \Delta x$, nennt man die Obersumme. Sie ist eine obere Schranke für den gesuchten Flächeninhalt (d.h. $A \le O_n$).
            \item \textbf{Riemannsumme mit linken/rechten Randpunkten oder Mittelpunkten:} Man kann als Höhe des $i$-ten Rechtecks auch einfach den Funktionswert am linken Rand $f(x_{i-1})$, am rechten Rand $f(x_i)$ oder in der Mitte des Teilintervalls $f(\frac{x_{i-1}+x_i}{2})$ wählen. Diese Summen nennt man dann allgemein Riemannsummen.
        \end{itemize}
    \item \textbf{Summe der Rechtecksflächen:} Die Fläche jedes einzelnen Rechtecks ist $\text{Breite} \cdot \text{Höhe} = \Delta x \cdot f(x_i^*)$ (wobei $x_i^*$ die Stelle ist, an der die Höhe im $i$-ten Intervall genommen wird). Die Riemannsumme ist die Summe all dieser kleinen Rechtecksflächen.
\end{enumerate}
\textbf{Die Kernidee:} Je größer wir die Anzahl $n$ der Teilintervalle wählen (und damit je schmaler die einzelnen Rechtecke werden, d.h. $\Delta x \to 0$), desto besser nähert die Riemannsumme (egal ob Unter-, Obersumme oder eine andere Wahl für $x_i^*$) den tatsächlichen Flächeninhalt unter der Kurve an. Der exakte Flächeninhalt $A$ ist dann der \textbf{Grenzwert} dieser Summen für $n \to \infty$.
\[ A = \lim_{n \to \infty} U_n = \lim_{n \to \infty} O_n = \lim_{n \to \infty} \sum_{i=1}^{n} f(x_i^*) \cdot \Delta x \]
Diesen Grenzwert nennen wir das \textbf{bestimmte Integral} von $f(x)$ über $[a,b]$ und schreiben $\int_a^b f(x) dx$.
\end{merksatzumgebung}

\begin{center}
    \includegraphics[width=0.9\textwidth]{grafiken/Riemannsummen_Untersumme_Obersumme.png}
    \captionof{figure}{Illustration von Untersumme und Obersumme (Riemannsummen)}
    \label{fig:riemannsummen_illustration}
\end{center}

Schauen wir uns ein konkretes Beispiel an, wie man eine Untersumme und eine Obersumme berechnet. Für einfache, monotone Funktionen ist die Bestimmung des Minimums/Maximums in einem Teilintervall einfach der Funktionswert am entsprechenden Rand.

\begin{beispielumgebung}{Untersumme und Obersumme für $f(x)=x^2$}
Wir wollen den Flächeninhalt unter der Funktion $f(x)=x^2$ im Intervall $[0,2]$ mit $n=4$ Teilintervallen durch Untersumme $U_4$ und Obersumme $O_4$ annähern.

\textbf{Schritt 1: Intervallbreite $\Delta x$ und Teilungspunkte $x_i$ bestimmen.}
Das Intervall ist $[a,b] = [0,2]$. Die Anzahl der Teilintervalle ist $n=4$.
Die Breite jedes Teilintervalls ist $\Delta x = \frac{b-a}{n} = \frac{2-0}{4} = \frac{2}{4} = 0.5$.
Die Teilungspunkte sind:
$x_0 = a = 0$
$x_1 = x_0 + \Delta x = 0 + 0.5 = 0.5$
$x_2 = x_1 + \Delta x = 0.5 + 0.5 = 1$
$x_3 = x_2 + \Delta x = 1 + 0.5 = 1.5$
$x_4 = x_3 + \Delta x = 1.5 + 0.5 = 2 (=b)$.

\textbf{Schritt 2: Untersumme $U_4$ berechnen.}
Die Funktion $f(x)=x^2$ ist im Intervall $[0,2]$ streng monoton steigend. Das bedeutet, der kleinste Funktionswert (Minimum) in jedem Teilintervall $[x_{i-1}, x_i]$ befindet sich am \textbf{linken Rand} $x_{i-1}$. Die Höhe des $i$-ten Rechtecks ist also $f(x_{i-1})$.
\begin{itemize}
    \item 1. Rechteck (Intervall $[x_0, x_1] = [0, 0.5]$): Höhe $f(x_0) = f(0) = 0^2 = 0$. Fläche $\Delta x \cdot f(0) = 0.5 \cdot 0 = 0$.
    \item 2. Rechteck (Intervall $[x_1, x_2] = [0.5, 1]$): Höhe $f(x_1) = f(0.5) = (0.5)^2 = 0.25$. Fläche $\Delta x \cdot f(0.5) = 0.5 \cdot 0.25 = 0.125$.
    \item 3. Rechteck (Intervall $[x_2, x_3] = [1, 1.5]$): Höhe $f(x_2) = f(1) = 1^2 = 1$. Fläche $\Delta x \cdot f(1) = 0.5 \cdot 1 = 0.5$.
    \item 4. Rechteck (Intervall $[x_3, x_4] = [1.5, 2]$): Höhe $f(x_3) = f(1.5) = (1.5)^2 = 2.25$. Fläche $\Delta x \cdot f(1.5) = 0.5 \cdot 2.25 = 1.125$.
\end{itemize}
Die Untersumme ist die Summe dieser Flächen:
$U_4 = 0 + 0.125 + 0.5 + 1.125 = 1.75$.

\textbf{Schritt 3: Obersumme $O_4$ berechnen.}
Da $f(x)=x^2$ im Intervall $[0,2]$ streng monoton steigend ist, liegt der größte Funktionswert (Maximum) in jedem Teilintervall $[x_{i-1}, x_i]$ am \textbf{rechten Rand} $x_i$. Die Höhe des $i$-ten Rechtecks ist also $f(x_i)$.
\begin{itemize}
    \item 1. Rechteck (Intervall $[x_0, x_1] = [0, 0.5]$): Höhe $f(x_1) = f(0.5) = (0.5)^2 = 0.25$. Fläche $\Delta x \cdot f(0.5) = 0.5 \cdot 0.25 = 0.125$.
    \item 2. Rechteck (Intervall $[x_1, x_2] = [0.5, 1]$): Höhe $f(x_2) = f(1) = 1^2 = 1$. Fläche $\Delta x \cdot f(1) = 0.5 \cdot 1 = 0.5$.
    \item 3. Rechteck (Intervall $[x_2, x_3] = [1, 1.5]$): Höhe $f(x_3) = f(1.5) = (1.5)^2 = 2.25$. Fläche $\Delta x \cdot f(1.5) = 0.5 \cdot 2.25 = 1.125$.
    \item 4. Rechteck (Intervall $[x_3, x_4] = [1.5, 2]$): Höhe $f(x_4) = f(2) = 2^2 = 4$. Fläche $\Delta x \cdot f(2) = 0.5 \cdot 4 = 2$.
\end{itemize}
Die Obersumme ist die Summe dieser Flächen:
$O_4 = 0.125 + 0.5 + 1.125 + 2 = 3.75$.

Der wahre Flächeninhalt $A$ unter $f(x)=x^2$ im Intervall $[0,2]$ liegt also zwischen $1.75$ und $3.75$:
$1.75 \le A \le 3.75$.
Wenn wir $n$ erhöhen, wird diese 'Schere' zwischen Unter- und Obersumme immer kleiner. Der exakte Wert ist übrigens $A = \frac{8}{3} \approx 2.667$. Unsere Näherungen sind also noch recht grob, aber sie zeigen das Prinzip.
\end{beispielumgebung}

\begin{tippumgebung}{Summenformel für Riemannsummen}
Allgemein lässt sich die Untersumme $U_n$ und Obersumme $O_n$ mit dem Summenzeichen $\sum$ schreiben:
Sei $m_i$ das Minimum von $f(x)$ im $i$-ten Teilintervall $[x_{i-1}, x_i]$ und $M_i$ das Maximum.
Dann ist:
\[ U_n = \sum_{i=1}^{n} m_i \cdot \Delta x \]
\[ O_n = \sum_{i=1}^{n} M_i \cdot \Delta x \]
Wenn $f(x)$ im Intervall $[a,b]$ monoton steigend ist, dann ist $m_i = f(x_{i-1})$ (linker Rand) und $M_i = f(x_i)$ (rechter Rand).
Wenn $f(x)$ im Intervall $[a,b]$ monoton fallend ist, dann ist $m_i = f(x_i)$ (rechter Rand) und $M_i = f(x_{i-1})$ (linker Rand).
\end{tippumgebung}

\begin{aufgabenumgebung}{Riemannsummen berechnen}
Gegeben ist die Funktion $f(x) = x+1$.
\begin{enumerate}
    \item Berechne die Untersumme $U_5$ und die Obersumme $O_5$ für das Intervall $[0,5]$ mit $n=5$ Teilintervallen.
    \begin{tippumgebung}{Monotonie}
    Ist $f(x)=x+1$ monoton steigend oder fallend? Wo liegt also das Minimum bzw. Maximum in jedem Teilintervall?
    \end{tippumgebung}
    \item Der Graph von $f(x)=x+1$ ist eine Gerade. Der Bereich unter dem Graphen im Intervall $[0,5]$ bildet ein Trapez. Berechne den exakten Flächeninhalt dieses Trapezes mit der geometrischen Formel $A_{Trapez} = \frac{(a+c)}{2} \cdot h$ (wobei $a$ und $c$ die parallelen Seiten sind und $h$ die Höhe).
    \item Vergleiche deine Ergebnisse für $U_5$ und $O_5$ mit dem exakten Flächeninhalt.
    \item Was würde passieren, wenn du $n=10$ oder $n=100$ Teilintervalle wählen würdest? Wie würden sich $U_n$ und $O_n$ verändern?
\end{enumerate}

\begin{center}
    \includegraphics[width=0.9\textwidth]{grafiken/Riemannsummen_lineareFunktion.png}
    \captionof{figure}{Fläche unter $f(x)=x+1$ im Intervall $[0,5]$ }
    \label{fig:riemannsummen_linear_illustration}
\end{center}
\end{aufgabenumgebung}

Die Idee der Riemannsummen führt uns direkt zum Begriff des \textbf{bestimmten Integrals}.

\subsection{Das bestimmte Integral – Der exakte Flächeninhalt}
\label{subsec:bestimmtes_integral_neu} % Neues Label

Wir haben gesehen, dass wir den Flächeninhalt unter einer Kurve durch Riemannsummen (z.B. Unter- und Obersummen) annähern können. Je mehr Rechtecke wir verwenden (je größer $n$ wird und damit je kleiner $\Delta x = \frac{b-a}{n}$ wird), desto genauer wird unsere Näherung.

Wenn wir diesen Prozess ins Unendliche treiben, also $n \to \infty$ (und damit $\Delta x \to 0$), dann konvergieren die Untersumme und die Obersumme (für 'nette', d.h. integrierbare Funktionen) gegen denselben Wert. Dieser gemeinsame Grenzwert ist der exakte Flächeninhalt unter der Kurve und wird als das \textbf{bestimmte Integral} der Funktion $f(x)$ von $a$ bis $b$ bezeichnet.

\begin{merksatzumgebung}{Das bestimmte Integral}
Das \textbf{bestimmte Integral} einer Funktion $f(x)$ im Intervall $[a,b]$ ist der Grenzwert der Riemannsummen, wenn die Anzahl der Teilintervalle $n$ gegen unendlich geht (und die Breite $\Delta x$ der Teilintervalle gegen Null geht):
\[ \int_{a}^{b} f(x) \,dx = \lim_{n \to \infty} \sum_{i=1}^{n} f(x_i^*) \cdot \Delta x \]
Dabei ist:
\begin{itemize}
    \item $\int$ das \textbf{Integrationszeichen} (ein stilisiertes S für 'Summe').
    \item $a$ die \textbf{untere Integrationsgrenze}.
    \item $b$ die \textbf{obere Integrationsgrenze}.
    \item $f(x)$ der \textbf{Integrand} (die zu integrierende Funktion).
    \item $dx$ das \textbf{Differential}, das anzeigt, nach welcher Variablen integriert wird (hier $x$) und symbolisiert die unendlich kleine Breite der Rechtecke.
\end{itemize}
Wenn $f(x) \ge 0$ im Intervall $[a,b]$ ist, dann gibt $\int_{a}^{b} f(x) \,dx$ den \textbf{Flächeninhalt} der Fläche an, die vom Graphen von $f(x)$, der x-Achse und den senkrechten Geraden $x=a$ und $x=b$ eingeschlossen wird.
\end{merksatzumgebung}

\begin{infoboxumgebung}{Was passiert, wenn $f(x)$ unterhalb der x-Achse liegt?}
Wenn der Graph von $f(x)$ in einem Intervall unterhalb der x-Achse verläuft (also $f(x) < 0$), dann liefert das bestimmte Integral in diesem Bereich einen \textbf{negativen Wert}. Dieser negative Wert entspricht dem Flächeninhalt zwischen dem Graphen und der x-Achse, aber eben mit negativem Vorzeichen.
Man spricht dann von einer \textbf{orientierten Fläche}. Flächenanteile oberhalb der x-Achse zählen positiv, Flächenanteile unterhalb der x-Achse zählen negativ.
Um den 'echten' geometrischen Flächeninhalt zu bekommen, wenn Teile unterhalb liegen, muss man die Beträge der entsprechenden Integrale addieren oder die Funktion an den entsprechenden Stellen spiegeln (also $|f(x)|$ integrieren).
\begin{center}
    \includegraphics[width=0.8\textwidth]{grafiken/Integral_Orientierte_Flaeche.png}
    \captionof{figure}{Orientierter Flächeninhalt beim bestimmten Integral}
    \label{fig:orientierte_flaeche}
\end{center}
\end{infoboxumgebung}

\begin{fehlerboxumgebung}{Orientierte vs. geometrische Fläche – Genau hinschauen!}
Das bestimmte Integral liefert die Flächenbilanz, nicht immer den rein geometrischen Flächeninhalt!
\begin{itemize}
    \item \textbf{Orientierte Fläche:} Das Ergebnis von $\int_a^b f(x)dx$ ist die \textbf{Summe der vorzeichenbehafteten Flächenstücke}. Flächen über der x-Achse gehen positiv ein, Flächen darunter negativ.
    \item \textbf{Geometrischer Gesamtflächeninhalt gesucht?} Wenn die Aufgabe nach dem tatsächlichen, sichtbaren Flächeninhalt zwischen Graph und x-Achse fragt, musst du aufpassen:
    \begin{itemize}
        \item \textbf{Nullstellen prüfen:} Bestimme immer zuerst die Nullstellen von $f(x)$ im Intervall $[a,b]$. Diese teilen das Gesamtintervall eventuell in Teilintervalle auf.
        \item \textbf{Teilintervalle betrachten:} Untersuche das Vorzeichen von $f(x)$ in jedem Teilintervall.
        \item \textbf{Beträge addieren:} Für Teilintervalle, in denen $f(x) < 0$ (Graph unter der x-Achse), ist das Integral negativ. Für den geometrischen Flächeninhalt musst du den \textbf{Betrag} dieses negativen Wertes nehmen und zu den positiven Flächenanteilen addieren.
    \end{itemize}
    \item \textbf{Negatives Integral $\neq$ Rechenfehler:} Ein negatives Ergebnis für ein bestimmtes Integral ist oft korrekt und bedeutet lediglich, dass der Flächenanteil unterhalb der x-Achse im betrachteten Intervall überwiegt (oder die gesamte Fläche unterhalb liegt).
\end{itemize}
\end{fehlerboxumgebung}

Die Berechnung von Integralen über den Grenzwert von Riemannsummen ist sehr aufwendig. Glücklicherweise gibt es einen viel eleganteren Weg, der die Integralrechnung mit der Differentialrechnung verbindet: den Hauptsatz der Differential- und Integralrechnung. Dafür benötigen wir aber zuerst das Konzept der Stammfunktion.

\subsection{Die Stammfunktion – Das 'Gegenteil' vom Ableiten (Aufleiten)}
\label{subsec:stammfunktion_integral} % Neues Label

In der Differentialrechnung haben wir gelernt, zu einer gegebenen Funktion $f(x)$ ihre Ableitungsfunktion $f'(x)$ zu finden, die uns die Steigung von $f(x)$ an jeder Stelle liefert.
Die Integralrechnung stellt nun oft die umgekehrte Frage:
\textit{Wenn wir eine Funktion $f(x)$ gegeben haben (die wir uns jetzt als Ableitung einer anderen Funktion vorstellen können), welche Funktion $F(x)$ müssen wir ableiten, um genau dieses $f(x)$ als Ergebnis zu erhalten?}
Eine solche Funktion $F(x)$ nennen wir eine \textbf{Stammfunktion} von $f(x)$.

\begin{merksatzumgebung}{Stammfunktion}
Eine Funktion $F(x)$ heißt \textbf{Stammfunktion} einer Funktion $f(x)$, wenn für alle $x$ im Definitionsbereich gilt:
\[ F'(x) = f(x) \]
Das bedeutet, die Ableitung der Stammfunktion $F(x)$ ergibt die ursprüngliche Funktion $f(x)$.
Den Vorgang des Findens einer Stammfunktion nennt man auch \textbf{Integrieren} oder umgangssprachlich (und sehr anschaulich) \textbf{Aufleiten}.
\end{merksatzumgebung}

Das Finden von Stammfunktionen ist also wie ein Rätsel: 'Welche Funktion wurde hier abgeleitet?'

\begin{beispielumgebung}{Stammfunktionen finden durch 'Rückwärts-Ableiten'}
\begin{enumerate}
    \item \textbf{Gegeben: $f(x) = 2x$.}
        Wir fragen uns: Welche Funktion $F(x)$ hat als Ableitung $2x$?
        Aus der Potenzregel der Ableitung wissen wir: $(x^2)' = 2x^1 = 2x$.
        Also ist $F(x) = x^2$ eine Stammfunktion von $f(x)=2x$.

        \textit{Aber Moment mal!} Was ist mit $F_1(x) = x^2 + 5$?
        $F_1'(x) = (x^2)' + (5)' = 2x + 0 = 2x$.
        Auch $F_1(x) = x^2+5$ ist eine Stammfunktion von $f(x)=2x$.

        Und was ist mit $F_2(x) = x^2 - 17$?
        $F_2'(x) = (x^2)' - (17)' = 2x - 0 = 2x$.
        Ebenfalls eine Stammfunktion!

        Es scheint unendlich viele Stammfunktionen zu geben, die sich nur durch eine additive Konstante unterscheiden.

    \item \textbf{Gegeben: $f(x) = x^2$.}
        Welche Funktion $F(x)$ ergibt abgeleitet $x^2$?
        Wir wissen, beim Ableiten wird der Exponent um 1 kleiner. Also muss der Exponent der Stammfunktion um 1 größer sein, also $x^3$.
        Probieren wir $G(x)=x^3$. Die Ableitung ist $G'(x)=3x^2$.
        Das ist noch nicht ganz $x^2$, sondern das Dreifache. Um das auszugleichen, müssen wir $x^3$ durch 3 teilen:
        $F(x) = \frac{1}{3}x^3$.
        Machen wir die Probe: $F'(x) = (\frac{1}{3}x^3)' = \frac{1}{3} \cdot (3x^2) = x^2$. Perfekt!
        Also ist $F(x) = \frac{1}{3}x^3$ eine Stammfunktion von $f(x)=x^2$.
        Und natürlich sind auch $F(x) = \frac{1}{3}x^3 + 7$ oder $F(x) = \frac{1}{3}x^3 - \pi$ Stammfunktionen.
\end{enumerate}
\end{beispielumgebung}

Das erste Beispiel hat uns eine wichtige Eigenschaft gezeigt:

\begin{merksatzumgebung}{Die Menge aller Stammfunktionen (Das unbestimmte Integral)}
Wenn $F(x)$ eine Stammfunktion einer Funktion $f(x)$ ist (d.h. $F'(x)=f(x)$), dann ist auch jede Funktion der Form $F(x)+C$, wobei $C$ eine beliebige reelle Konstante ist, eine Stammfunktion von $f(x)$.
Denn $(F(x)+C)' = F'(x) + (C)' = f(x) + 0 = f(x)$.

Die Menge aller dieser Stammfunktionen wird als das \textbf{unbestimmte Integral} von $f(x)$ bezeichnet und man schreibt dafür:
\[ \int f(x) \,dx = F(x) + C \]
Dabei ist:
\begin{itemize}
    \item $\int$ das \textbf{Integrationszeichen} (ein stilisiertes S, das an 'Summe' erinnert – ein Hinweis auf die Riemannsummen, die wir später kennenlernen).
    \item $f(x)$ der \textbf{Integrand} (die Funktion, die integriert/aufgeleitet wird).
    \item $dx$ das \textbf{Differential}, das anzeigt, nach welcher Variablen integriert wird (hier $x$). Es ist ein wichtiger Bestandteil der Notation.
    \item $F(x)$ irgendeine spezielle Stammfunktion von $f(x)$.
    \item $C$ die \textbf{Integrationskonstante} (eine beliebige reelle Zahl).
\end{itemize}
Das unbestimmte Integral liefert uns also nicht nur eine einzelne Funktion, sondern eine ganze \textbf{Schar von Funktionen}, die sich alle nur durch eine Verschiebung entlang der y-Achse unterscheiden.
\end{merksatzumgebung}

\textit{Selbst-Check:} Warum ist es wichtig, die Integrationskonstante $C$ beim unbestimmten Integral anzugeben? (Antwort: Weil es unendlich viele Funktionen gibt, deren Ableitung $f(x)$ ist, und $C$ repräsentiert all diese Möglichkeiten.)

\subsubsection{Grundlegende Integrationsregeln (Umkehrung der Ableitungsregeln)}
Ähnlich wie beim Ableiten gibt es auch beim Integrieren Regeln, die uns helfen, Stammfunktionen systematisch zu finden. Viele davon ergeben sich direkt durch Umkehrung der uns bekannten Ableitungsregeln. Wir konzentrieren uns hier zunächst auf Regeln für Polynomfunktionen.

\begin{merksatzumgebung}{Grundlegende Integrationsregeln für Polynome}
\begin{itemize}
    \item \textbf{Potenzregel der Integration:} Für $f(x) = x^n$ (mit $n \in \mathbb{R}, n \neq -1$) gilt:
    \[ \int x^n \,dx = \frac{1}{n+1}x^{n+1} + C \]
    \textit{Regel in Worten:} 'Erhöhe den Exponenten um 1 und teile dann durch diesen neuen Exponenten.'
    \textit{Beachte:} Diese Regel gilt nicht für $n=-1$, also für $f(x)=x^{-1}=\frac{1}{x}$. Die Stammfunktion von $\frac{1}{x}$ ist $\ln|x|+C$. Dies werden wir später bei den Logarithmusfunktionen genauer betrachten. Für Polynome tritt dieser Fall aber nicht auf.

    \item \textbf{Faktorregel der Integration:} Ein konstanter Faktor $k$ kann vor das Integral gezogen werden:
    \[ \int k \cdot f(x) \,dx = k \cdot \int f(x) \,dx \]
    Das bedeutet, wir können erst die Stammfunktion von $f(x)$ finden und diese dann mit $k$ multiplizieren.

    \item \textbf{Summenregel der Integration:} Das Integral einer Summe (oder Differenz) von Funktionen ist die Summe (oder Differenz) ihrer Integrale:
    \[ \int (f(x) \pm g(x)) \,dx = \int f(x) \,dx \pm \int g(x) \,dx \]
    \textit{Regel in Worten:} 'Jeder Summand wird für sich integriert/aufgeleitet, und die Ergebnisse werden dann addiert bzw. subtrahiert.'

    \item \textbf{Integral einer Konstanten:} Für $f(x)=k$ (eine Konstante) gilt:
    \[ \int k \,dx = kx + C \]
    (Denn die Ableitung von $kx+C$ ist $k$.)
\end{itemize}
\end{merksatzumgebung}

\begin{beispielumgebung}{Stammfunktionen mit Regeln bilden}
\begin{enumerate}
    \item \textbf{Bestimme $\int x^4 \,dx$.}
        Hier ist $n=4$. Nach der Potenzregel der Integration:
        $\int x^4 \,dx = \frac{1}{4+1}x^{4+1} + C = \frac{1}{5}x^5 + C$.
        \textit{Probe durch Ableiten:} $(\frac{1}{5}x^5+C)' = \frac{1}{5} \cdot 5x^4 + 0 = x^4$. Stimmt.

    \item \textbf{Bestimme $\int 6x^2 \,dx$.}
        Nach Faktor- und Potenzregel:
        $\int 6x^2 \,dx = 6 \cdot \int x^2 \,dx = 6 \cdot \left(\frac{1}{2+1}x^{2+1}\right) + C = 6 \cdot \frac{1}{3}x^3 + C = 2x^3 + C$.
        \textit{Probe:} $(2x^3+C)' = 2 \cdot 3x^2 + 0 = 6x^2$. Stimmt.

    \item \textbf{Bestimme $\int (3x^2 - 4x + 5) \,dx$.}
        Nach Summen-, Faktor- und Potenzregel:
        $\int (3x^2 - 4x + 5) \,dx = \int 3x^2 \,dx - \int 4x \,dx + \int 5 \,dx$
        $= 3 \cdot \int x^2 \,dx - 4 \cdot \int x^1 \,dx + \int 5x^0 \,dx$
        $= 3 \cdot \left(\frac{1}{3}x^3\right) - 4 \cdot \left(\frac{1}{2}x^2\right) + 5 \cdot \left(\frac{1}{1}x^1\right) + C$
        $= x^3 - 2x^2 + 5x + C$.
        \textit{Probe:} $(x^3 - 2x^2 + 5x + C)' = 3x^2 - 4x + 5$. Stimmt.
\end{enumerate}
\end{beispielumgebung}

\begin{aufgabenumgebung}{Stammfunktionen bilden üben}
Bestimme jeweils die Menge aller Stammfunktionen (das unbestimmte Integral) für die folgenden Funktionen:
\begin{enumerate}
    \item $f(x) = x^5$
    \item $g(x) = 12x^3$
    \item $h(x) = 2x^3 - 7x^2 + 4x - 1$
    \item $k(x) = \sqrt{x} + \frac{1}{x^3}$ (Tipp: Erst in Potenzschreibweise $x^n$ umwandeln! $\sqrt{x}=x^{1/2}$ und $\frac{1}{x^3}=x^{-3}$)
    \item $m(t) = at+b$ (wobei $a,b$ Konstanten sind; integriere nach $t$)
    \item $p(x) = (x+1)(x-2)$ (Tipp: Erst ausmultiplizieren!)
\end{enumerate}
Mache bei mindestens zwei Aufgaben die Probe durch Ableiten deiner Stammfunktion.
\end{aufgabenumgebung}

\begin{fehlerboxumgebung}{Integrationskonstante $C$ nicht vergessen!}
Ein sehr häufiger Fehler beim unbestimmten Integrieren ist das Vergessen der Integrationskonstante $+C$. Da die Ableitung einer Konstanten immer Null ist, gibt es zu jeder Funktion unendlich viele Stammfunktionen, die sich alle nur durch diese Konstante unterscheiden. Bei bestimmten Integralen (mit Grenzen) fällt diese Konstante später heraus, aber beim unbestimmten Integral ist sie wichtig! Stell dir vor, jede Stammfunktion ist wie ein Mitglied einer großen Familie von Kurven, die alle parallel zueinander verlaufen.
\end{fehlerboxumgebung}

\begin{warumwichtigumgebung}{Stammfunktionen und das unbestimmte Integral}
Das Konzept der Stammfunktion ist der Schlüssel, um den fundamentalen Zusammenhang zwischen Differential- und Integralrechnung zu verstehen. Es erlaubt uns, von einer gegebenen Änderungsrate auf die ursprüngliche Größe zurückzuschließen. Das unbestimmte Integral gibt uns die Gesamtheit aller möglichen Funktionen an, deren Ableitung die gegebene Funktion ist. Diese 'Familie' von Stammfunktionen wird entscheidend sein, wenn wir uns gleich dem bestimmten Integral und dem Hauptsatz zuwenden.
\end{warumwichtigumgebung}

Als Nächstes werden wir sehen, wie die Stammfunktion uns auf elegante Weise hilft, bestimmte Integrale und damit Flächeninhalte zu berechnen, ohne den mühsamen Weg über Riemannsummen gehen zu müssen. Das ist die Kernaussage des Hauptsatzes der Differential- und Integralrechnung.

% Vorheriger Inhalt des Kapitels bis zur warumwichtigumgebung 'Stammfunktionen und das unbestimmte Integral'
% ... (siehe vorherige Canvas-Version) ...

% Hier geht es dann weiter mit dem Hauptsatz der Differential- und Integralrechnung.
% Dieser Kommentar wird durch den folgenden Inhalt ersetzt:

\subsection{Der Hauptsatz der Differential- und Integralrechnung (HDI)}
\label{subsec:hdi}

Wir haben gesehen, dass das Berechnen von Flächeninhalten über den Grenzwert von Riemannsummen ziemlich mühsam sein kann. Gibt es einen einfacheren Weg, um das bestimmte Integral $\int_a^b f(x) \,dx$ exakt zu berechnen, ohne unendlich viele Rechtecke addieren zu müssen? Die Antwort ist ein klares Ja, und sie liegt in einem der wichtigsten Sätze der gesamten Mathematik: dem \textbf{Hauptsatz der Differential- und Integralrechnung} (oft abgekürzt als HDI).

Dieser Satz stellt eine fundamentale Verbindung zwischen der Differentialrechnung (dem Ableiten) und der Integralrechnung (dem Aufleiten bzw. Flächenberechnen) her. Er ist so etwas wie die 'magische Brücke' zwischen diesen beiden großen Gebieten der Analysis.

\begin{merksatzumgebung}{Der Hauptsatz der Differential- und Integralrechnung (HDI)}
Sei $f(x)$ eine im Intervall $[a,b]$ stetige Funktion und $F(x)$ eine beliebige Stammfunktion von $f(x)$ (d.h. $F'(x) = f(x)$). Dann gilt für das bestimmte Integral von $f(x)$ über dem Intervall $[a,b]$:
\[ \int_{a}^{b} f(x) \,dx = [F(x)]_{a}^{b} = F(b) - F(a) \]
\textbf{In Worten:} Um das bestimmte Integral von $f(x)$ in den Grenzen von $a$ bis $b$ zu berechnen, bilde eine Stammfunktion $F(x)$ von $f(x)$, setze die obere Grenze $b$ in $F(x)$ ein, setze die untere Grenze $a$ in $F(x)$ ein und subtrahiere den zweiten Wert vom ersten.

Die Schreibweise $[F(x)]_{a}^{b}$ ist eine Kurzform für $F(b) - F(a)$.
\end{merksatzumgebung}

\begin{warumwichtigumgebung}{Die Bedeutung des HDI}
Der Hauptsatz ist revolutionär, weil er uns sagt: Statt komplizierte Grenzwerte von Summen zu berechnen, um eine Fläche zu finden, können wir einfach eine Stammfunktion suchen (was oft viel einfacher ist) und deren Werte an den Rändern des Intervalls auswerten! Das Ableiten und Integrieren sind also tatsächlich Umkehroperationen zueinander.
\end{warumwichtigumgebung}

Schauen wir uns an, wie man den HDI anwendet.

\begin{beispielumgebung}{Bestimmtes Integral mit dem HDI berechnen}
\begin{enumerate}
    \item \textbf{Fläche unter $f(x)=x^2$ im Intervall $[0,2]$} (Vergleiche mit dem Riemannsummen-Beispiel!)
        Wir wollen $\int_{0}^{2} x^2 \,dx$ berechnen.
        \begin{itemize}
            \item \textbf{Schritt 1: Stammfunktion $F(x)$ von $f(x)=x^2$ finden.}
            Mit der Potenzregel der Integration: $F(x) = \frac{1}{2+1}x^{2+1} = \frac{1}{3}x^3$. (Die Integrationskonstante $C$ können wir hier weglassen, da sie sich bei der Differenz $F(b)-F(a)$ ohnehin aufheben würde: $(F(b)+C) - (F(a)+C) = F(b)-F(a)$.)
            \item \textbf{Schritt 2: Grenzen einsetzen $F(b)-F(a)$.}
            Hier ist $a=0$ und $b=2$.
            $\int_{0}^{2} x^2 \,dx = [ \frac{1}{3}x^3 ]_{0}^{2} = F(2) - F(0)$
            $= \left(\frac{1}{3}(2)^3\right) - \left(\frac{1}{3}(0)^3\right)$
            $= \frac{1}{3} \cdot 8 - \frac{1}{3} \cdot 0 = \frac{8}{3} - 0 = \frac{8}{3}$.
        \end{itemize}
        Der exakte Flächeninhalt ist $\frac{8}{3} \approx 2.667$. Das passt viel besser zu unseren Riemannsummen-Näherungen ($U_4=1.75, O_4=3.75$) und war viel einfacher zu berechnen!

    \item \textbf{Berechne $\int_{1}^{3} (3x^2 - 4x + 5) \,dx$.}
        Der Integrand ist $f(x) = 3x^2 - 4x + 5$.
        \begin{itemize}
            \item \textbf{Schritt 1: Stammfunktion $F(x)$ finden.}
            $F(x) = x^3 - 2x^2 + 5x$. (Wir lassen $C$ weg.)
            \item \textbf{Schritt 2: Grenzen einsetzen.}
            $\int_{1}^{3} (3x^2 - 4x + 5) \,dx = [x^3 - 2x^2 + 5x]_{1}^{3}$
            $= F(3) - F(1)$
            $= ((3)^3 - 2(3)^2 + 5(3)) - ((1)^3 - 2(1)^2 + 5(1))$
            $= (27 - 2 \cdot 9 + 15) - (1 - 2 \cdot 1 + 5)$
            $= (27 - 18 + 15) - (1 - 2 + 5)$
            $= (9 + 15) - (4) = 24 - 4 = 20$.
        \end{itemize}
        Der Wert des bestimmten Integrals ist 20.
\end{enumerate}
\end{beispielumgebung}

\begin{fehlerboxumgebung}{Hauptsatz-Anwendung – Typische Stolpersteine}
Der Hauptsatz der Differential- und Integralrechnung (HDI) ist mächtig, aber bei der Anwendung lauern ein paar typische Fehlerquellen:
\begin{itemize}
    \item \textbf{Grenzen vertauscht ($F(a)-F(b)$ statt $F(b)-F(a)$):} Achte penibel auf die Reihenfolge: Immer 'Stammfunktion an der oberen Grenze' minus 'Stammfunktion an der unteren Grenze', also $F(b)-F(a)$. Ein Vertauschen führt zum Vorzeichenfehler im Ergebnis!
    \item \textbf{Rechenfehler beim Einsetzen der Grenzen:} Das ist eine der häufigsten Fehlerquellen!
    \begin{itemize}
        \item Besonders bei negativen Zahlen als Grenzen oder in der Stammfunktion: Setze sorgfältig Klammern, z.B. $F(-2) = \frac{1}{3}(-2)^3 - (-2) = -\frac{8}{3} + 2$.
        \item Brüche und Potenzen korrekt berechnen. Nimm dir Zeit für diesen Schritt.
    \end{itemize}
    \item \textbf{Stammfunktion $F(x)$ falsch gebildet:} Der HDI funktioniert nur, wenn $F(x)$ auch wirklich eine korrekte Stammfunktion von $f(x)$ ist. Überprüfe deine Integrationsregeln (Potenzregel, Faktorregel, Summenregel etc.) sorgfältig. Im Zweifel: Leite deine gefundene Stammfunktion $F(x)$ zur Probe ab – es muss wieder $f(x)$ herauskommen!
    \item \textbf{Integrationskonstante $C$ beim bestimmten Integral:} Für die Berechnung des bestimmten Integrals $\int_a^b f(x)dx = F(b)-F(a)$ kannst du die Integrationskonstante $C$ weglassen (oder $C=0$ setzen), da sie sich ohnehin herauskürzen würde: $(F(b)+C) - (F(a)+C) = F(b)-F(a)$. Wenn du sie mitschleppst, achte darauf, dass sie sich korrekt aufhebt.
\end{itemize}
Sorgfältiges und schrittweises Rechnen hilft, diese Fehler zu vermeiden!
\end{fehlerboxumgebung}

\begin{aufgabenumgebung}{Bestimmte Integrale mit dem HDI berechnen}
Berechne die folgenden bestimmten Integrale mit dem Hauptsatz.
\begin{enumerate}
    \item $\int_{1}^{2} (4x^3 - 6x) \,dx$
    \item $\int_{-1}^{1} (x^2 + 2) \,dx$
    \item $\int_{0}^{4} (\sqrt{x} + 1) \,dx$ (Tipp: $\sqrt{x} = x^{1/2}$)
    \item \textbf{Fläche visualisieren:}
        Die Funktion $f(x) = -x^2 + 4x$ hast du vielleicht schon in früheren Aufgaben skizziert (eine nach unten geöffnete Parabel).
        \begin{itemize}
            \item Berechne die Nullstellen von $f(x)$.
            \item Berechne das bestimmte Integral $\int_{x_1}^{x_2} f(x) \,dx$, wobei $x_1$ und $x_2$ die Nullstellen sind ($x_1 < x_2$).
            \item Was stellt dieser Wert geometrisch dar? Markiere die entsprechende Fläche in einer Skizze des Graphen von $f(x)$.
\begin{center}
    \includegraphics[width=0.8\textwidth]{grafiken/Integral_Flaeche_Parabel.png}
    \captionof{figure}{Fläche unter $f(x)=-x^2+4x$ zwischen den Nullstellen}
    \label{fig:flaeche_parabel}
\end{center}
        \end{itemize}
\end{enumerate}
\end{aufgabenumgebung}


\begin{funfactbox}{Newton, unendliche Reihen und die Quadratur des Kreises (fast!)}
Die Zahl $\pi$ fasziniert Mathematiker seit Jahrtausenden. Die alten Methoden zur Annäherung von $\pi$ waren oft geometrisch und extrem aufwendig. Isaac Newton fand um 1666 einen revolutionär neuen Weg, der die gerade erst entwickelte Analysis nutzte.

Seine Idee war, die Fläche eines Viertel-Einheitskreises zu berechnen, denn diese Fläche ist genau $\frac{\pi}{4}$. Die Gleichung eines Kreises mit Radius 1 ist $x^2+y^2=1$, also ist die obere Hälfte $y = \sqrt{1-x^2}$. Die Fläche des Viertelkreises im ersten Quadranten ist dann das bestimmte Integral:
\[ \text{Fläche} = \int_0^1 \sqrt{1-x^2} \,dx = \frac{\pi}{4} \]
Das Problem: Wie integriert man $\sqrt{1-x^2}$? Newton hatte dafür einen Trick! Er nutzte seine verallgemeinerte Form des \textbf{Binomialtheorems}, um $\sqrt{1-x^2}$ als eine \textbf{unendliche Summe (Reihe)} von Potenzen von $x$ darzustellen. Für $|x|<1$ lauten die ersten Terme dieser Reihe:
\[ \sqrt{1-x^2} = (1-x^2)^{\frac{1}{2}} = 1 - \frac{1}{2}x^2 - \frac{1}{8}x^4 - \frac{1}{16}x^6 - \frac{5}{128}x^8 - \dots \]
(Die genaue Herleitung dieser Koeffizienten ist etwas für Fortgeschrittene, aber die Idee ist, dass man den Ausdruck in eine 'unendlich langes Polynom' umwandelt.)

Das Geniale: Diese unendliche Summe konnte Newton nun Glied für Glied integrieren, ganz ähnlich wie du es bei Polynomen gelernt hast (Potenzregel der Integration: $\int x^n dx = \frac{1}{n+1}x^{n+1}$)!
\begin{align*} \int_0^1 \sqrt{1-x^2} \,dx &= \int_0^1 \left(1 - \frac{1}{2}x^2 - \frac{1}{8}x^4 - \frac{1}{16}x^6 - \dots \right) \,dx \\ &= \left[ x - \frac{1}{2 \cdot 3}x^3 - \frac{1}{8 \cdot 5}x^5 - \frac{1}{16 \cdot 7}x^7 - \dots \right]_0^1 \\ &= \left(1 - \frac{1}{6} - \frac{1}{40} - \frac{1}{112} - \dots \right) - (0) \end{align*}
Also erhielt Newton für $\pi/4$ die unendliche Reihe:
\[ \frac{\pi}{4} = 1 - \frac{1}{6} - \frac{1}{40} - \frac{1}{112} - \dots \]
Durch Aufsummieren von nur wenigen Termen dieser Reihe konnte Newton $\pi$ wesentlich genauer und schneller berechnen, als es mit den alten geometrischen Methoden möglich war. Er nutzte später sogar eine geschicktere Wahl der Integrationsgrenzen (von $0$ bis $1/2$), um eine Reihe zu erhalten, die noch schneller den Wert von $\pi$ liefert.

Dieser Ansatz zeigt eindrucksvoll, wie die Verbindung von unendlichen Reihen (oft aus der Differentialrechnung über Taylorreihen gewonnen) und der Integralrechnung völlig neue Wege zur Lösung alter Probleme eröffnete!

\begin{center}
    \includegraphics[width=0.6\textwidth]{grafiken/Newton_Pi_Integral.png}
    % Beschreibung für die Grafik 'Newton_Pi_Integral.png':
    % Die Grafik könnte den Graphen von y = sqrt(1-x^2) im ersten Quadranten zeigen (Viertelkreis).
    % Darunter die Fläche von x=0 bis x=1 schraffiert, beschriftet mit A = Pi/4.
    % Daneben oder darunter die ersten Terme der Reihe für sqrt(1-x^2) und 
    % die ersten Terme der integrierten Reihe für Pi/4.
    % Optional ein kleines Portrait von Newton.
    \captionof{figure}{Konzept der $\pi$-Berechnung durch Integration der Binomialreihe für den Viertelkreis}
    \label{fig:newton_pi_integral_funfact}
\end{center}
\end{funfactbox}

\subsubsection{Anwendungen und Interpretationen des bestimmten Integrals}

Die wichtigste geometrische Interpretation des bestimmten Integrals $\int_a^b f(x)dx$ ist, wie erwähnt, der orientierte Flächeninhalt zwischen dem Graphen von $f(x)$ und der x-Achse im Intervall $[a,b]$.

\begin{infoboxumgebung}{Flächenberechnung bei Nullstellen und unterhalb der x-Achse}
Wenn eine Funktion $f(x)$ im Integrationsintervall $[a,b]$ Nullstellen besitzt und somit Teile des Graphen unterhalb der x-Achse liegen, liefert das bestimmte Integral $\int_a^b f(x)dx$ die \textbf{Flächenbilanz}. Das bedeutet, Flächenstücke oberhalb der x-Achse werden positiv gewertet, Flächenstücke unterhalb der x-Achse negativ.

Um den \textbf{tatsächlichen geometrischen Flächeninhalt} zu berechnen, der zwischen dem Graphen und der x-Achse eingeschlossen wird, musst du:
\begin{enumerate}
    \item Die Nullstellen $x_N$ der Funktion im Intervall $[a,b]$ finden.
    \item Das Integral über die Teilintervalle berechnen, die durch die Nullstellen entstehen.
    \item Die \textbf{Beträge} der Integralteile addieren, bei denen der Graph unterhalb der x-Achse verläuft (also wo das Integral negativ wäre).
\end{enumerate}
Mathematisch entspricht das der Berechnung von $\int_a^b |f(x)| \,dx$. In der Praxis ist es oft einfacher, die Teilintegrale zu berechnen und dann die Beträge zu addieren.

Beispiel: Wenn $f(x)$ eine Nullstelle $x_N$ zwischen $a$ und $b$ hat und $f(x) \ge 0$ für $x \in [a, x_N]$ und $f(x) \le 0$ für $x \in [x_N, b]$, dann ist der Gesamtflächeninhalt $A_{ges}$:
\[ A_{ges} = \int_a^{x_N} f(x) \,dx + \left| \int_{x_N}^b f(x) \,dx \right| = \int_a^{x_N} f(x) \,dx - \int_{x_N}^b f(x) \,dx \]
(Da $\int_{x_N}^b f(x) \,dx$ negativ wäre, wird durch das Minuszeichen der Betrag addiert).
\end{infoboxumgebung}

\begin{beispielumgebung}{Fläche mit Anteilen unter der x-Achse}
Berechne den Flächeninhalt, den der Graph der Funktion $f(x) = x^2 - 1$ mit der x-Achse im Intervall $[-2, 2]$ einschließt.

\textbf{Schritt 1: Nullstellen von $f(x)$ finden.}
$x^2 - 1 = 0 \implies x^2 = 1 \implies x_{N1} = -1, x_{N2} = 1$.
Beide Nullstellen liegen im Intervall $[-2,2]$.

\textbf{Schritt 2: Vorzeichen von $f(x)$ in den Teilintervallen bestimmen.}
Die Parabel $f(x)=x^2-1$ ist nach oben geöffnet.
\begin{itemize}
    \item Intervall $[-2, -1]$: z.B. $f(-1.5) = (-1.5)^2-1 = 2.25-1 = 1.25 > 0$. (Graph oberhalb)
    \item Intervall $[-1, 1]$: z.B. $f(0) = 0^2-1 = -1 < 0$. (Graph unterhalb)
    \item Intervall $[1, 2]$: z.B. $f(1.5) = (1.5)^2-1 = 2.25-1 = 1.25 > 0$. (Graph oberhalb)
\end{itemize}

\textbf{Schritt 3: Teilintegrale berechnen.}
Stammfunktion $F(x) = \frac{1}{3}x^3 - x$.
$A_1 = \int_{-2}^{-1} (x^2-1) \,dx = [\frac{1}{3}x^3-x]_{-2}^{-1} = (\frac{1}{3}(-1)^3 - (-1)) - (\frac{1}{3}(-2)^3 - (-2))$
$= (-\frac{1}{3}+1) - (-\frac{8}{3}+2) = \frac{2}{3} - (-\frac{8}{3}+\frac{6}{3}) = \frac{2}{3} - (-\frac{2}{3}) = \frac{2}{3} + \frac{2}{3} = \frac{4}{3}$.

$A_2 = \int_{-1}^{1} (x^2-1) \,dx = [\frac{1}{3}x^3-x]_{-1}^{1} = (\frac{1}{3}(1)^3 - 1) - (\frac{1}{3}(-1)^3 - (-1))$
$= (\frac{1}{3}-1) - (-\frac{1}{3}+1) = -\frac{2}{3} - \frac{2}{3} = -\frac{4}{3}$.
(Das Integral ist negativ, da die Fläche unter der x-Achse liegt. Der Flächen\textit{inhalt} ist $|-\frac{4}{3}| = \frac{4}{3}$).

$A_3 = \int_{1}^{2} (x^2-1) \,dx = [\frac{1}{3}x^3-x]_{1}^{2} = (\frac{1}{3}(2)^3 - 2) - (\frac{1}{3}(1)^3 - 1)$
$= (\frac{8}{3}-2) - (\frac{1}{3}-1) = (\frac{8}{3}-\frac{6}{3}) - (-\frac{2}{3}) = \frac{2}{3} + \frac{2}{3} = \frac{4}{3}$.

\textbf{Schritt 4: Gesamtflächeninhalt.}
$A_{ges} = A_1 + |A_2| + A_3 = \frac{4}{3} + \frac{4}{3} + \frac{4}{3} = \frac{12}{3} = 4$.
Der Gesamtflächeninhalt beträgt 4 Flächeneinheiten.
\begin{center}
    \includegraphics[width=0.8\textwidth]{grafiken/Integral_Flaeche_xhoch2minus1.png}
    \captionof{figure}{Fläche zwischen $f(x)=x^2-1$ und der x-Achse von -2 bis 2}
    \label{fig:flaeche_x2minus1}
\end{center}
\end{beispielumgebung}

\begin{aufgabenumgebung}{Flächenberechnung mit Nullstellen im Intervall}
Berechne den Inhalt der Fläche, die vom Graphen der Funktion $f(x) = x^3 - x$ und der x-Achse im Intervall $[-2, 2]$ eingeschlossen wird.
\begin{tippumgebung}{Vorgehen}
\begin{enumerate}
    \item Skizziere den Graphen (oder überlege dir den Verlauf anhand von Symmetrie und Nullstellen).
    \item Finde alle Nullstellen von $f(x)$.
    \item Bestimme, in welchen Teilintervallen $f(x) \ge 0$ und in welchen $f(x) \le 0$ ist.
    \item Berechne die bestimmten Integrale über die Teilintervalle und addiere deren Beträge.
\end{enumerate}
\end{tippumgebung}
\end{aufgabenumgebung}

\begin{infoboxumgebung}{Symmetrie und bestimmte Integrale – Rechnungen vereinfachen!}
Die Symmetrieeigenschaften von Funktionen können uns die Berechnung bestimmter Integrale oft erheblich erleichtern, besonders wenn das Integrationsintervall symmetrisch zum Ursprung ist (also von der Form $[-a, a]$).

\begin{itemize}
    \item \textbf{Punktsymmetrie zum Ursprung:}
        Wenn eine Funktion $f(x)$ punktsymmetrisch zum Ursprung ist (d.h. $f(-x) = -f(x)$, wie z.B. bei $x, x^3, x^5, \sin(x)$), dann gilt für jedes $a>0$:
        \[ \int_{-a}^{a} f(x) \,dx = 0 \]
        \textit{Warum?} Die Fläche links vom Ursprung (unterhalb der x-Achse, wenn $f(x)$ für $x>0$ oberhalb ist) ist genauso groß wie die Fläche rechts vom Ursprung (oberhalb der x-Achse), aber mit entgegengesetztem Vorzeichen. Sie heben sich also gegenseitig auf.
        \begin{center}
            \includegraphics[width=0.7\textwidth]{grafiken/Integral_Punktsymmetrie.png}
            \captionof{figure}{Integral einer punktsymmetrischen Funktion über $[-a,a]$}
            \label{fig:integral_punktsymmetrie}
        \end{center}

    \item \textbf{Achsensymmetrie zur y-Achse:}
        Wenn eine Funktion $f(x)$ achsensymmetrisch zur y-Achse ist (d.h. $f(-x) = f(x)$, wie z.B. bei $x^2, x^4, |x|, \cos(x)$), dann gilt für jedes $a>0$:
        \[ \int_{-a}^{a} f(x) \,dx = 2 \cdot \int_{0}^{a} f(x) \,dx \]
        \textit{Warum?} Die Fläche links von der y-Achse (von $-a$ bis $0$) ist genauso groß wie die Fläche rechts von der y-Achse (von $0$ bis $a$). Man kann also die Rechnung vereinfachen, indem man nur eine Hälfte berechnet und das Ergebnis verdoppelt.
        % \begin{center}
        %     % Platzhalter für die Grafik 'Integral Achsensymmetrie'
        %     \framebox{\begin{minipage}{0.7\textwidth}
        %         \centering \vspace{1.5cm}
        %         \textbf{Platzhalter: Integral\_Achsensymmetrie.png} \\
        %         Beschreibung: Graph einer achsensymmetrischen Funktion (z.B. $x^2$) über $[-a,a]$. Die Fläche $A_1$ von $-a$ bis $0$ und $A_2$ von $0$ bis $a$ sind gleich groß und haben das gleiche Vorzeichen.
        %         \vspace{1.5cm}
        %     \end{minipage}}
        %     \captionof{figure}{Integral einer achsensymmetrischen Funktion über $[-a,a]$ (Platzhalter).}
        % \end{center}
        \begin{center}
            \includegraphics[width=0.7\textwidth]{grafiken/Integral_Achsensymmetrie.png}
            \captionof{figure}{Integral einer achsensymmetrischen Funktion über $[-a,a]$}
            \label{fig:integral_achsensymmetrie}
        \end{center}
\end{itemize}
Diese Symmetrieüberlegungen können dir viel Rechenarbeit ersparen!
\end{infoboxumgebung}

\begin{beispielumgebung}{Symmetrie beim Integrieren nutzen}
\begin{enumerate}
    \item Berechne $\int_{-1}^{1} x^3 \,dx$.
        Die Funktion $f(x)=x^3$ ist punktsymmetrisch zum Ursprung, da $f(-x)=(-x)^3 = -x^3 = -f(x)$.
        Das Integrationsintervall $[-1,1]$ ist symmetrisch zum Ursprung.
        Daher gilt: $\int_{-1}^{1} x^3 \,dx = 0$.
        \textit{Probe (mit HDI):} $F(x) = \frac{1}{4}x^4$.
        $[ \frac{1}{4}x^4 ]_{-1}^{1} = (\frac{1}{4}(1)^4) - (\frac{1}{4}(-1)^4) = \frac{1}{4} - \frac{1}{4} = 0$. Stimmt!

    \item Berechne $\int_{-2}^{2} (3x^2 - 5) \,dx$.
        Die Funktion $f(x)=3x^2-5$ ist achsensymmetrisch zur y-Achse, da sie nur gerade Potenzen von $x$ enthält (und eine Konstante): $f(-x) = 3(-x)^2-5 = 3x^2-5 = f(x)$.
        Das Intervall $[-2,2]$ ist symmetrisch.
        Daher: $\int_{-2}^{2} (3x^2 - 5) \,dx = 2 \cdot \int_{0}^{2} (3x^2 - 5) \,dx$.
        Stammfunktion $F(x) = x^3 - 5x$.
        $2 \cdot [x^3 - 5x]_{0}^{2} = 2 \cdot ( ( (2)^3 - 5(2) ) - ( (0)^3 - 5(0) ) )$
        $= 2 \cdot ( (8 - 10) - (0) ) = 2 \cdot (-2) = -4$.
\end{enumerate}
\end{beispielumgebung}

\begin{aufgabenumgebung}{Symmetrie beim Integrieren anwenden}
Berechne die folgenden bestimmten Integrale. Nutze Symmetrieeigenschaften, wenn möglich, um die Rechnung zu vereinfachen.
\begin{enumerate}
    \item $\int_{-5}^{5} (x^5 - 2x^3 + x) \,dx$
    \item $\int_{-1}^{1} (x^4 + 3x^2 - 1) \,dx$
    \item $\int_{-\pi}^{\pi} \sin(x) \,dx$ (Du weißt vielleicht schon, dass $\sin(x)$ punktsymmetrisch ist. Die Stammfunktion von $\sin(x)$ ist $-\cos(x)$.)
\end{enumerate}
\end{aufgabenumgebung}

% Hier geht es dann weiter mit weiteren Anwendungen oder dem Abschluss des Kapitels.

% Vorheriger Inhalt des Kapitels bis zur aufgabenumgebung 'Symmetrie beim Integrieren anwenden'
% ... (siehe vorherige Canvas-Version) ...

\begin{aufgabenumgebung}{Fläche zwischen zwei Kurven}
Die Graphen der Funktionen $f(x) = -x^2 + 4x + 1$ und $g(x) = x^2 - 2x + 1$ schließen eine Fläche ein.
\begin{enumerate}
    \item \textbf{Schnittpunkte bestimmen:} Berechne die x-Koordinaten der Schnittpunkte der beiden Graphen, indem du $f(x) = g(x)$ setzt und die entstehende Gleichung löst. Diese Schnittpunkte sind deine Integrationsgrenzen $a$ und $b$.
    \item \textbf{Welche Funktion liegt oben?} Bestimme, welche der beiden Funktionen im Intervall $[a,b]$ die größeren Funktionswerte hat (also 'oben' liegt). Du kannst dies tun, indem du einen Testwert aus dem Intervall $(a,b)$ in beide Funktionen einsetzt oder die Graphen skizzierst.
    \item \textbf{Differenzfunktion bilden:} Bilde die Differenzfunktion $d(x) = \text{obere Funktion} - \text{untere Funktion}$.
    \item \textbf{Flächeninhalt berechnen:} Berechne den Flächeninhalt $A = \int_a^b d(x) \,dx$.
    \item \textbf{Skizze:} Skizziere beide Parabeln und die eingeschlossene Fläche in ein Koordinatensystem.
\end{enumerate}
\begin{merksatzumgebung}{Fläche zwischen zwei Graphen}
Den Flächeninhalt $A$ zwischen den Graphen zweier Funktionen $f(x)$ und $g(x)$ im Intervall $[a,b]$, wobei $f(x) \ge g(x)$ für alle $x \in [a,b]$ gilt (d.h. $f(x)$ ist die obere Funktion), berechnet man mit:
\[ A = \int_a^b (f(x) - g(x)) \,dx \]
Wenn nicht klar ist, welche Funktion oben liegt, oder wenn sich die obere Funktion ändert, muss man das Intervall entsprechend aufteilen und/oder den Betrag der Differenz integrieren: $A = \int_a^b |f(x) - g(x)| \,dx$.
\end{merksatzumgebung}
\begin{center}
    \includegraphics[width=0.8\textwidth]{grafiken/Integral_Flaeche_zwischen_Kurven.png}
    \captionof{figure}{Fläche zwischen den Graphen von $f(x)$ und $g(x)$}
    \label{fig:flaeche_zwischen_kurven}
\end{center}
\end{aufgabenumgebung}

\begin{aufgabenumgebung}{Der Mittelwert einer Funktion}
Manchmal möchte man nicht den Gesamtwert (wie die Gesamtfläche oder den Gesamtweg), sondern einen Durchschnittswert einer Größe über ein Intervall bestimmen. Die Integralrechnung hilft auch hier!

\begin{merksatzumgebung}{Mittelwert einer Funktion}
Der \textbf{Mittelwert $\mu$} einer stetigen Funktion $f(x)$ im Intervall $[a,b]$ (mit $a<b$) ist definiert als:
\[ \mu = \frac{1}{b-a} \int_a^b f(x) \,dx \]
\textit{Anschauliche Deutung:} Der Mittelwert $\mu$ ist die Höhe eines Rechtecks mit der Breite $(b-a)$, dessen Flächeninhalt gleich dem Flächeninhalt unter dem Graphen von $f(x)$ im Intervall $[a,b]$ ist. Also: $\mu \cdot (b-a) = \int_a^b f(x) \,dx$.
\end{merksatzumgebung}

\textbf{Aufgabe:}
Ein Tag hat vereinfacht 12 Stunden Helligkeit (von $t=0$ bis $t=12$). Die Temperatur $T$ (in °C) an diesem Tag kann durch die Funktion $T(t) = -0.5t^2 + 6t + 5$ modelliert werden.
\begin{enumerate}
    \item Skizziere den Graphen der Temperaturfunktion im Intervall $[0,12]$.
    \item Berechne die Durchschnittstemperatur während dieser 12 Stunden mit der Formel für den Mittelwert einer Funktion.
    \item Zeichne in deine Skizze ein Rechteck mit der Breite des Intervalls (12 Stunden) und der Höhe der Durchschnittstemperatur. Vergleiche die Fläche dieses Rechtecks mit der Fläche unter dem Temperatur-Graphen (visuell).
\end{enumerate}
\begin{center}
    \includegraphics[width=0.8\textwidth]{grafiken/Integral_Mittelwert_Funktion.png}
    \captionof{figure}{Geometrische Deutung des Mittelwerts einer Funktion}
    \label{fig:mittelwert_funktion}
\end{center}
\end{aufgabenumgebung}

\begin{kurzknappumgebung}{Bestimmtes Integral und Hauptsatz}
\begin{itemize}
    \item \textbf{Bestimmtes Integral $\int_a^b f(x)dx$:} Grenzwert der Riemannsummen; gibt den orientierten Flächeninhalt zwischen Graph von $f(x)$ und x-Achse im Intervall $[a,b]$ an.
    \item \textbf{Stammfunktion $F(x)$:} Eine Funktion, deren Ableitung $f(x)$ ist ($F'(x)=f(x)$). Es gibt unendlich viele, die sich durch eine Konstante $C$ unterscheiden: $\int f(x)dx = F(x)+C$ (unbestimmtes Integral).
    \item \textbf{Hauptsatz (HDI):} $\int_a^b f(x)dx = F(b) - F(a)$. Ermöglicht einfache Berechnung bestimmter Integrale über Stammfunktionen.
    \item \textbf{Flächenberechnung:} Bei Nullstellen im Intervall müssen Teilintegrale gebildet und Beträge addiert werden für den geometrischen Flächeninhalt.
    \item \textbf{Symmetrie nutzen:} Bei punktsymmetrischen Funktionen über $[-a,a]$ ist $\int_{-a}^a f(x)dx = 0$. Bei achsensymmetrischen ist $\int_{-a}^a f(x)dx = 2 \cdot \int_0^a f(x)dx$.
    \item \textbf{Fläche zwischen Kurven $f(x)$ und $g(x)$ (mit $f(x) \ge g(x)$):} $A = \int_a^b (f(x)-g(x))dx$.
    \item \textbf{Mittelwert $\mu$ von $f(x)$ auf $[a,b]$:} $\mu = \frac{1}{b-a}\int_a^b f(x)dx$.
\end{itemize}
\end{kurzknappumgebung}

% Vorheriger Inhalt des Kapitels bis zur aufgabenumgebung 'Der Mittelwert einer Funktion'
% ... (siehe vorherige Canvas-Version) ...

\subsection{Zusammenfassung und Ausblick zur Integralrechnung}
\label{subsec:zusammenfassung_ausblick_integral}

Wir haben nun die fundamentalen Ideen der Integralrechnung kennengelernt: von der anschaulichen Flächenapproximation durch Riemannsummen über das Konzept der Stammfunktion als 'Gegenstück' zur Ableitung bis hin zum mächtigen Hauptsatz der Differential- und Integralrechnung. Mit diesen Werkzeugen können wir bereits viele wichtige Probleme lösen, insbesondere Flächeninhalte unter und zwischen Kurven von Polynomfunktionen berechnen sowie Mittelwerte von Funktionen bestimmen.

\begin{kurzknappumgebung}{Integralrechnung – Das Wichtigste auf einen Blick}
\begin{itemize}
    \item \textbf{Riemannsummen (Unter-/Obersumme):} Annäherung von Flächen unter Kurven durch Summen von Rechtecksflächen.
    \item \textbf{Bestimmtes Integral $\int_a^b f(x)dx$:} Grenzwert der Riemannsummen; gibt den orientierten Flächeninhalt zwischen dem Graphen von $f(x)$ und der x-Achse im Intervall $[a,b]$ an.
    \item \textbf{Stammfunktion $F(x)$:} Eine Funktion, deren Ableitung $f(x)$ ist ($F'(x)=f(x)$). Es gibt unendlich viele Stammfunktionen, die sich durch eine additive Konstante $C$ unterscheiden: $\int f(x)dx = F(x)+C$ (unbestimmtes Integral).
    \item \textbf{Hauptsatz der Differential- und Integralrechnung (HDI):} Die Brücke zwischen Ableiten und Integrieren!
    \[ \int_a^b f(x)dx = [F(x)]_a^b = F(b) - F(a) \]
    Er ermöglicht die exakte Berechnung bestimmter Integrale über Stammfunktionen.
    \item \textbf{Flächenberechnung:}
        \begin{itemize}
            \item Liegt $f(x)$ im Intervall $[a,b]$ nicht unterhalb der x-Achse, ist $A = \int_a^b f(x)dx$.
            \item Bei Nullstellen im Intervall müssen Teilintegrale gebildet und deren Beträge addiert werden, um den geometrischen Gesamtflächeninhalt zu erhalten ($A = \int_a^b |f(x)|dx$).
            \item Fläche zwischen zwei Kurven $f(x)$ und $g(x)$ (mit $f(x) \ge g(x)$ auf $[a,b]$): $A = \int_a^b (f(x)-g(x))dx$.
        \end{itemize}
    \item \textbf{Symmetrie nutzen:} Bei punktsymmetrischen Funktionen über $[-a,a]$ ist $\int_{-a}^a f(x)dx = 0$. Bei achsensymmetrischen ist $\int_{-a}^a f(x)dx = 2 \cdot \int_0^a f(x)dx$.
    \item \textbf{Mittelwert $\mu$ von $f(x)$ auf $[a,b]$:} $\mu = \frac{1}{b-a}\int_a^b f(x)dx$.
\end{itemize}
\end{kurzknappumgebung}

\begin{infoboxumgebung}{Ausblick: Was kommt noch in der Integralrechnung?}
Die bisher gelernten Integrationsregeln (Potenz-, Faktor-, Summenregel als Umkehrung der Ableitungsregeln) reichen für Polynomfunktionen und einfache gebrochen-rationale Funktionen gut aus. Aber was ist mit komplexeren Funktionen, die wir bereits beim Ableiten kennengelernt haben?
\begin{itemize}
    \item Wie integriert man Produkte von Funktionen, z.B. $f(x) = x^2 \cdot e^x$?
    \item Wie integriert man Quotienten, z.B. $g(x) = \frac{2x}{x^2+1}$?
    \item Wie integriert man verkettete Funktionen, z.B. $h(x) = (3x+5)^4$ oder $k(x) = e^{x^2+x+1} \cdot (2x+1)$ oder $m(x) = x \cdot \sin(x^2)$?
\end{itemize}
Für solche Fälle gibt es weiterführende \textbf{Integrationstechniken}, die oft auf der Umkehrung der komplexeren Ableitungsregeln basieren:
\begin{itemize}
    \item Die \textbf{partielle Integration} (Umkehrung der Produktregel).
    \item Die \textbf{Integration durch Substitution} (Umkehrung der Kettenregel).
    % \item Die \textbf{Partialbruchzerlegung} für kompliziertere gebrochen-rationale Funktionen.
\end{itemize}
Diese Techniken erweitern unseren 'Integrations-Werkzeugkasten' erheblich und ermöglichen die Behandlung einer viel größeren Klasse von Funktionen. Sie sind oft Gegenstand weiterführender Kurse oder Vertiefungen in der Oberstufe.

Auch das Integrieren von Exponentialfunktionen (wie $e^x$), Logarithmusfunktionen (wie $\ln x$) und trigonometrischen Funktionen (wie $\sin x, \cos x$) erfordert eigene Stammfunktionen, die du noch kennenlernen wirst. Die Welt der Integrale ist groß und mächtig! Das Fundament, das du hier gelegt hast, ist aber entscheidend für alles Weitere.
\end{infoboxumgebung}



\begin{aufgabenumgebung}{Checkliste: Das bestimmte Integral – Von der Summe zur Fläche}
Das bestimmte Integral ist ein zentrales Konzept der Analysis. Diese Fragen helfen dir, die Idee dahinter besser zu greifen:

\begin{enumerate}[label=(\alph*)]
    \item \textbf{Riemannsummen als Annäherung:}
    \begin{itemize}
        \item Erkläre mit eigenen Worten, warum die Unter- und Obersumme sich dem tatsächlichen Flächeninhalt unter einer Kurve annähern, wenn man die Anzahl $n$ der Rechtecke immer weiter erhöht. Was passiert dabei mit der Breite $\Delta x$ der einzelnen Rechtecke?
        \item Skizziere eine Funktion, die in einem Intervall $[a,b]$ sowohl positive als auch negative Werte annimmt. Wie würdest du die Riemannsumme (z.B. mit linken Rändern) interpretieren? Was passiert mit den Rechtecksflächen, die unterhalb der x-Achse liegen?
    \end{itemize}
    \item \textbf{Das bestimmte Integral $\int_a^b f(x)dx$:}
    \begin{itemize}
        \item Was bedeuten die einzelnen Bestandteile der Notation: $\int$, $a$, $b$, $f(x)$ und $dx$? Welche Rolle spielt das $dx$ in Erinnerung an die Riemannsummen?
        \item Wenn $f(x)$ die Änderungsrate einer Größe beschreibt (z.B. die Geschwindigkeit $v(t)$ in m/s), welche Bedeutung und welche Einheit hat dann das bestimmte Integral $\int_{t_1}^{t_2} f(x)dx$ (bzw. $\int_{t_1}^{t_2} v(t)dt$)?
    \end{itemize}
    \item \textbf{Orientierte Fläche vs. Geometrische Fläche:}
    \begin{itemize}
        \item Angenommen, $\int_0^2 f(x)dx = 5$ und $\int_2^3 f(x)dx = -2$. Was ist der Wert von $\int_0^3 f(x)dx$? Welchen geometrischen Gesamtflächeninhalt schließt der Graph von $f(x)$ mit der x-Achse im Intervall $[0,3]$ ein? Erkläre den Unterschied.
        \item Wie würdest du vorgehen, um den \textit{geometrischen} Flächeninhalt zwischen dem Graphen von $f(x)=x^3-x$ und der x-Achse im Intervall $[-1,1]$ zu berechnen? Warum reicht hier nicht einfach $\int_{-1}^1 (x^3-x)dx$? (Tipp: Symmetrie und Nullstellen beachten).
    \end{itemize}
    \item \textbf{Eigenschaften des bestimmten Integrals:}
    \begin{itemize}
        \item Was ist der Wert von $\int_a^a f(x)dx$ und warum?
        \item Welcher Zusammenhang besteht zwischen $\int_a^b f(x)dx$ und $\int_b^a f(x)dx$? Wie lässt sich das mit $F(b)-F(a)$ erklären?
    \end{itemize}
\end{enumerate}
\end{aufgabenumgebung}

\begin{aufgabenumgebung}{Checkliste: Stammfunktion und Hauptsatz – Die große Verbindung}
Die Entdeckung des Zusammenhangs zwischen Ableitung und Integral durch den Hauptsatz ist revolutionär. Teste dein Verständnis:

\begin{enumerate}[label=(\alph*)]
    \item \textbf{Stammfunktion und unbestimmtes Integral:}
    \begin{itemize}
        \item Erkläre den Unterschied zwischen 'einer Stammfunktion $F(x)$ von $f(x)$' und 'dem unbestimmten Integral $\int f(x)dx$'. Warum ist die Integrationskonstante $C$ beim unbestimmten Integral so wichtig?
        \item Wenn $F(x)$ eine Stammfunktion von $f(x)$ ist und $G(x)$ eine Stammfunktion von $g(x)$ ist: Ist $F(x) \cdot G(x)$ dann automatisch eine Stammfunktion von $f(x) \cdot g(x)$? Überprüfe deine Vermutung mit einem einfachen Beispiel (z.B. $f(x)=1, g(x)=2x$). Was schließt du daraus für Integrationsregeln für Produkte?
    \end{itemize}
    \item \textbf{Der Hauptsatz der Differential- und Integralrechnung (HDI):}
    \begin{itemize}
        \item Formuliere den HDI mit eigenen Worten. Was ist die 'Brücke', die er zwischen der Differential- und Integralrechnung schlägt?
        \item Warum ist der HDI so praktisch für die Berechnung von bestimmten Integralen im Vergleich zur Methode mit den Riemannsummen?
        \item Angenommen, jemand behauptet, eine Stammfunktion von $f(x)=2x$ sei $F(x)=x^2+1000$, und eine andere Person sagt, es sei $G(x)=x^2-5$. Wer hat Recht? Und wie wirkt sich die Wahl von $F(x)$ oder $G(x)$ auf das Ergebnis von $\int_1^2 2x \,dx$ aus? Begründe.
    \end{itemize}
    \item \textbf{Anwendungen und Interpretationen:}
    \begin{itemize}
        \item Wenn $\int_a^b f(x)dx = 0$ ist, bedeutet das zwangsläufig, dass $f(x)=0$ für alle $x \in [a,b]$ gilt? Erkläre anhand einer Skizze oder eines Beispiels (nutze Symmetrie!).
        \item Erkläre die geometrische Bedeutung des Mittelwerts $\mu = \frac{1}{b-a} \int_a^b f(x)dx$ einer Funktion $f(x)$ im Intervall $[a,b]$ mithilfe eines flächengleichen Rechtecks.
    \end{itemize}
\end{enumerate}
\end{aufgabenumgebung}




\section{Exponentialfunktionen – Die Funktionen des Wachstums und Zerfalls}
\label{sec:exponentialfunktionen_intro}


\begin{aufgabenumgebung}{Erste Übungen mit $e^x$}
\begin{enumerate}
    \item Bilde die erste und zweite Ableitung der folgenden Funktionen:
        \begin{itemize}
            \item $f_1(x) = 3e^x - x^3 + 2x - 7$
            \item $f_2(x) = -0.5e^x + \frac{1}{x}$ (Tipp: $\frac{1}{x} = x^{-1}$)
        \end{itemize}
    \item Bestimme die Menge aller Stammfunktionen:
        \begin{itemize}
            \item $g_1(x) = 4e^x + 6x^2 - 1$
            \item $g_2(x) = \frac{e^x}{2} - \sqrt{x}$ (Tipp: $\frac{e^x}{2} = \frac{1}{2}e^x$)
        \end{itemize}
    \item Berechne das bestimmte Integral $\int_0^1 e^x \,dx$. Was stellt dieser Wert geometrisch dar?
\end{enumerate}
\end{aufgabenumgebung}

\begin{loesungsumgebung}[loes:erste-uebungen-ex]{Erste Übungen mit $e^x$}

\begin{enumerate}[label=(\alph*)]
    \item \textbf{Bilde die erste und zweite Ableitung der folgenden Funktionen:}
    \begin{itemize}
        \item \textbf{Funktion $f_1(x) = 3e^x - x^3 + 2x - 7$}
        \begin{itemize}
            \item Erste Ableitung $f_1'(x)$:
            $$ f_1'(x) = \frac{d}{dx}(3e^x - x^3 + 2x - 7) $$
            $$ f_1'(x) = 3\frac{d}{dx}(e^x) - \frac{d}{dx}(x^3) + \frac{d}{dx}(2x) - \frac{d}{dx}(7) $$
            $$ f_1'(x) = 3e^x - 3x^2 + 2 - 0 $$
            $$ \mathbf{f_1'(x) = 3e^x - 3x^2 + 2} $$
            \item Zweite Ableitung $f_1''(x)$:
            $$ f_1''(x) = \frac{d}{dx}(3e^x - 3x^2 + 2) $$
            $$ f_1''(x) = 3\frac{d}{dx}(e^x) - 3\frac{d}{dx}(x^2) + \frac{d}{dx}(2) $$
            $$ f_1''(x) = 3e^x - 3(2x) + 0 $$
            $$ \mathbf{f_1''(x) = 3e^x - 6x} $$
        \end{itemize}

        \item \textbf{Funktion $f_2(x) = -0.5e^x + \frac{1}{x}$} \\
        Wir schreiben $\frac{1}{x} = x^{-1}$. Also $f_2(x) = -0.5e^x + x^{-1}$.
        \begin{itemize}
            \item Erste Ableitung $f_2'(x)$:
            $$ f_2'(x) = \frac{d}{dx}(-0.5e^x + x^{-1}) $$
            $$ f_2'(x) = -0.5\frac{d}{dx}(e^x) + \frac{d}{dx}(x^{-1}) $$
            $$ f_2'(x) = -0.5e^x + (-1)x^{-1-1} $$
            $$ \mathbf{f_2'(x) = -0.5e^x - x^{-2}} \quad \text{oder} \quad \mathbf{-0.5e^x - \frac{1}{x^2}} $$
            \item Zweite Ableitung $f_2''(x)$:
            $$ f_2''(x) = \frac{d}{dx}(-0.5e^x - x^{-2}) $$
            $$ f_2''(x) = -0.5\frac{d}{dx}(e^x) - \frac{d}{dx}(x^{-2}) $$
            $$ f_2''(x) = -0.5e^x - (-2)x^{-2-1} $$
            $$ \mathbf{f_2''(x) = -0.5e^x + 2x^{-3}} \quad \text{oder} \quad \mathbf{-0.5e^x + \frac{2}{x^3}} $$
        \end{itemize}
    \end{itemize}

    \item \textbf{Bestimme die Menge aller Stammfunktionen:}
    \begin{itemize}
        \item \textbf{Funktion $g_1(x) = 4e^x + 6x^2 - 1$}
        \begin{align*} G_1(x) &= \int (4e^x + 6x^2 - 1) \,dx \\ &= 4\int e^x \,dx + 6\int x^2 \,dx - \int 1 \,dx \\ &= 4e^x + 6 \cdot \frac{x^3}{3} - x + C \\ &= \mathbf{4e^x + 2x^3 - x + C} \end{align*}

        \item \textbf{Funktion $g_2(x) = \frac{e^x}{2} - \sqrt{x}$} \\
        Wir schreiben $\frac{e^x}{2} = \frac{1}{2}e^x$ und $\sqrt{x} = x^{1/2}$. Also $g_2(x) = \frac{1}{2}e^x - x^{1/2}$.
        \begin{align*} G_2(x) &= \int \left(\frac{1}{2}e^x - x^{1/2}\right) \,dx \\ &= \frac{1}{2}\int e^x \,dx - \int x^{1/2} \,dx \\ &= \frac{1}{2}e^x - \frac{x^{\frac{1}{2}+1}}{\frac{1}{2}+1} + C \\ &= \frac{1}{2}e^x - \frac{x^{3/2}}{3/2} + C \\ &= \mathbf{\frac{1}{2}e^x - \frac{2}{3}x^{3/2} + C} \quad \text{oder} \quad \mathbf{\frac{1}{2}e^x - \frac{2}{3}\sqrt{x^3} + C} \end{align*}
    \end{itemize}

    \item \textbf{Berechne das bestimmte Integral $\int_0^1 e^x \,dx$. Was stellt dieser Wert geometrisch dar?}
    \begin{itemize}
        \item Eine Stammfunktion von $f(x)=e^x$ ist $F(x)=e^x$.
        \item Mit dem Hauptsatz der Differential- und Integralrechnung:
        $$ \int_0^1 e^x \,dx = [e^x]_0^1 = F(1) - F(0) = e^1 - e^0 $$
        Da $e^1=e$ und $e^0=1$, gilt:
        $$ \int_0^1 e^x \,dx = \mathbf{e - 1} $$
        (Numerischer Wert: $e-1 \approx 2.71828 - 1 = 1.71828$)
        \item \textbf{Geometrische Darstellung:} Der Wert $e-1$ stellt den \textbf{Flächeninhalt} dar, der vom Graphen der Funktion $f(x)=e^x$, der x-Achse, der y-Achse (Gerade $x=0$) und der Geraden $x=1$ eingeschlossen wird. Da die Funktion $e^x$ im Intervall $[0,1]$ stets positiv ist, entspricht das bestimmte Integral genau diesem Flächeninhalt.
    \end{itemize}
\end{enumerate}

\end{loesungsumgebung}

\begin{aufgabenumgebung}{Transformationen und Ableitungen/Stammfunktionen}
\begin{enumerate}
    \item Beschreibe, wie der Graph der Funktion $f(x) = 2e^{-0.5(x-1)} + 3$ aus dem Graphen der natürlichen Exponentialfunktion $y=e^x$ hervorgeht. Gib den Definitionsbereich, Wertebereich und die Gleichung der Asymptote an. Skizziere den Graphen.
    \item Bilde die erste Ableitung der folgenden Funktionen:
        \begin{itemize}
            \item $f_1(x) = 7e^{4x}$
            \item $f_2(x) = e^{-x} + 3x$
            \item $f_3(t) = A \cdot e^{-kt}$ ($A, k$ sind positive Konstanten; oft Modell für Zerfall)
        \end{itemize}
    \item Bestimme die Menge aller Stammfunktionen:
        \begin{itemize}
            \item $g_1(x) = 10e^{0.2x}$
            \item $g_2(x) = e^{-3x} - e^{2x}$
        \end{itemize}
\end{enumerate}
\end{aufgabenumgebung}


\begin{loesungsumgebung}[loes:trafo-ableitung-stammfunktion-ex]{Transformationen und Ableitungen/Stammfunktionen}

\begin{enumerate}[label=(\alph*)]
    \item \textbf{Beschreibung und Analyse von $f(x) = 2e^{-0.5(x-1)} + 3$}
    Der Graph der Funktion $f(x)$ geht aus dem Graphen der natürlichen Exponentialfunktion $y_0(u) = e^u$ durch folgende Transformationen hervor:
    \begin{enumerate}
        \item \textbf{Horizontale Skalierung/Spiegelung im Exponenten:} Der Exponent ist $-0.5(x-1) = -0.5x + 0.5$. Dies entspricht im Vergleich zu $e^x$:
        \begin{itemize}
            \item Einer Spiegelung an der y-Achse (wegen des negativen Vorzeichens vor dem $x$-Term).
            \item Einer horizontalen Streckung um den Faktor $1/0.5 = 2$.
            \item Einer horizontalen Verschiebung (die im Term $(x-1)$ bereits enthalten ist).
        \end{itemize}
        Alternativ kann man die Schritte so sehen:
        $e^x \xrightarrow{\text{Verschiebung um 1 nach rechts}} e^{x-1} \xrightarrow{\text{Horiz. Stauchung Faktor 0.5 + Spiegelung}} e^{-0.5(x-1)}$
        \item \textbf{Vertikale Streckung:} Multiplikation mit dem Faktor $2 \Rightarrow y_1(x) = 2e^{-0.5(x-1)}$. Der Graph wird in y-Richtung gestreckt.
        \item \textbf{Vertikale Verschiebung:} Addition von $3 \Rightarrow f(x) = 2e^{-0.5(x-1)} + 3$. Der Graph wird um 3 Einheiten nach oben verschoben.
    \end{enumerate}

    \begin{itemize}
        \item \textbf{Definitionsbereich ($D_f$):} Die Exponentialfunktion ist für alle reellen Zahlen definiert. Daher ist $\mathbf{D_f = \mathbb{R}}$.
        \item \textbf{Wertebereich ($W_f$):}
        Da $e^{\text{Exponent}}$ immer positiv ist ($e^{-0.5(x-1)} > 0$), ist $2e^{-0.5(x-1)} > 0$.
        Somit ist $f(x) = 2e^{-0.5(x-1)} + 3 > 3$.
        Der Wertebereich ist $\mathbf{W_f = (3, \infty)}$.
        \item \textbf{Gleichung der Asymptote:}
        Für $x \to \infty$: Der Exponent $-0.5(x-1)$ geht gegen $-\infty$. Daher $e^{-0.5(x-1)} \to 0$.
        Somit $\lim_{x \to \infty} f(x) = 2 \cdot 0 + 3 = 3$.
        Die Funktion hat eine \textbf{horizontale Asymptote} mit der Gleichung $\mathbf{y=3}$ (für $x \to \infty$).
        Für $x \to -\infty$: Der Exponent $-0.5(x-1)$ geht gegen $+\infty$. Daher $e^{-0.5(x-1)} \to \infty$.
        Somit $\lim_{x \to -\infty} f(x) = \infty$.
        \item \textbf{Skizze des Graphen:}
        Der Graph ist monoton fallend. Er nähert sich für $x \to \infty$ der Asymptote $y=3$ von oben an. Für $x \to -\infty$ wächst er unbegrenzt. Ein markanter Punkt ist $f(1) = 2e^0+3 = 2+3=5$, also $(1|5)$. Bei $x=0$ ist $f(0)=2e^{0.5}+3 \approx 2 \cdot 1.649 + 3 = 3.298+3 = 6.298$.
        \begin{center}
        \includegraphics[width=0.8\textwidth]{grafiken/trafo_ex_graph.png}
        % --- Beschreibung der Skizze ---
        % Die Skizze zeigt einen monoton fallenden Graphen.
        % Horizontale Asymptote bei y=3, der sich der Graph für x -> +unendlich von oben nähert.
        % Der Graph verläuft durch den Punkt (1|5). Für x -> -unendlich steigt der Graph ins Unendliche.
        % Der y-Achsenabschnitt liegt bei ca. (0|6.3).
        \captionof{figure}{Graph der Funktion $f(x) = 2e^{-0.5(x-1)} + 3$ mit Asymptote.}
        \label{fig:trafo_ex_graph}
        \end{center}
    \end{itemize}

    \item \textbf{Bilde die erste Ableitung der folgenden Funktionen:}
    Wir verwenden die Kettenregel $(e^{i(x)})' = e^{i(x)} \cdot i'(x)$ und die Faktorregel.
    \begin{itemize}
        \item \textbf{$f_1(x) = 7e^{4x}$} \\
        Innere Funktion $i(x)=4x \Rightarrow i'(x)=4$.
        $f_1'(x) = 7 \cdot e^{4x} \cdot 4 = \mathbf{28e^{4x}}$.

        \item \textbf{$f_2(x) = e^{-x} + 3x$} \\
        Für $e^{-x}$: Innere Funktion $i(x)=-x \Rightarrow i'(x)=-1$.
        $f_2'(x) = e^{-x} \cdot (-1) + 3 = \mathbf{-e^{-x} + 3}$.

        \item \textbf{$f_3(t) = A \cdot e^{-kt}$} ($A, k$ sind positive Konstanten) \\
        Ableitung nach $t$. Innere Funktion $i(t)=-kt \Rightarrow i'(t)=-k$.
        $f_3'(t) = A \cdot e^{-kt} \cdot (-k) = \mathbf{-Ake^{-kt}}$.
    \end{itemize}

    \item \textbf{Bestimme die Menge aller Stammfunktionen:}
    Wir verwenden die Regel $\int e^{ax} \,dx = \frac{1}{a}e^{ax} + C$ (für $a \neq 0$).
    \begin{itemize}
        \item \textbf{$g_1(x) = 10e^{0.2x}$}
        $$ G_1(x) = \int 10e^{0.2x} \,dx = 10 \cdot \frac{1}{0.2} e^{0.2x} + C $$
        Da $\frac{1}{0.2} = \frac{1}{1/5} = 5$:
        $$ G_1(x) = 10 \cdot 5 e^{0.2x} + C = \mathbf{50e^{0.2x} + C} $$

        \item \textbf{$g_2(x) = e^{-3x} - e^{2x}$}
        \begin{align*} G_2(x) &= \int (e^{-3x} - e^{2x}) \,dx \\ &= \int e^{-3x} \,dx - \int e^{2x} \,dx \\ &= \frac{1}{-3}e^{-3x} - \frac{1}{2}e^{2x} + C \\ &= \mathbf{-\frac{1}{3}e^{-3x} - \frac{1}{2}e^{2x} + C} \end{align*}
    \end{itemize}
\end{enumerate}

\end{loesungsumgebung}



\begin{aufgabenumgebung}{Umgang mit allgemeinen Exponentialfunktionen}
\begin{enumerate}
    \item Schreibe die folgenden Funktionen mit der Basis $e$ (d.h. in der Form $a \cdot e^{kx}$):
        \begin{itemize}
            \item $f_1(x) = 3^x$
            \item $f_2(x) = 10 \cdot (0.8)^x$
        \end{itemize}
    \item Bilde die erste Ableitung der folgenden Funktionen:
        \begin{itemize}
            \item $g_1(x) = 4^x + x^4$
            \item $g_2(x) = 7 \cdot (1.5)^x - e^x$
            \item $g_3(t) = P_0 \cdot a^t$ ($P_0$ und $a$ sind positive Konstanten. Dies ist ein typisches Modell für exponentielles Wachstum oder Zerfall, je nachdem ob $a>1$ oder $0<a<1$.)
        \end{itemize}
    \item Bestimme die Menge aller Stammfunktionen:
        \begin{itemize}
            \item $h_1(x) = 5^x$
            \item $h_2(x) = 3 \cdot (0.2)^x + e^{2x}$
        \end{itemize}
    \item \textbf{Vergleich $2^x$ und $e^x$:}
        \begin{itemize}
            \item Skizziere die Graphen von $f(x)=2^x$ und $g(x)=e^x$ in ein gemeinsames Koordinatensystem (z.B. für $x \in [-2, 3]$). Nutze dazu eine Wertetabelle.
            \item Vergleiche die Steigungen der beiden Funktionen an der Stelle $x=0$. Welche Funktion wächst dort schneller?
            \item Berechne $(2^x)'$ und $(e^x)'$.
        \end{itemize}
\end{enumerate}
\end{aufgabenumgebung}

\begin{loesungsumgebung}[loes:allgemeine-exponentialfunktionen]{Umgang mit allgemeinen Exponentialfunktionen}

\begin{enumerate}[label=(\alph*)]
    \item \textbf{Schreibe die folgenden Funktionen mit der Basis $e$ (Form $a \cdot e^{kx}$):}
    \begin{itemize}
        \item \textbf{$f_1(x) = 3^x$} \\
        Wir verwenden die Beziehung $b^x = e^{x \ln b}$.
        $$ f_1(x) = 3^x = e^{\ln(3^x)} = \mathbf{e^{x \ln 3}} $$
        Hier ist der Vorfaktor $a=1$ und $k = \ln 3 \approx 1.0986$.

        \item \textbf{$f_2(x) = 10 \cdot (0.8)^x$} \\
        $$ (0.8)^x = e^{\ln((0.8)^x)} = e^{x \ln 0.8} $$
        Somit:
        $$ f_2(x) = \mathbf{10 \cdot e^{x \ln 0.8}} $$
        Hier ist der Vorfaktor $a=10$ und $k = \ln 0.8 \approx -0.2231$.
    \end{itemize}

    \item \textbf{Bilde die erste Ableitung der folgenden Funktionen:}
    Wir verwenden die Regel $(b^x)' = b^x \ln b$ sowie die Summen- und Faktorregel.
    \begin{itemize}
        \item \textbf{$g_1(x) = 4^x + x^4$}
        $$ g_1'(x) = \frac{d}{dx}(4^x) + \frac{d}{dx}(x^4) = \mathbf{4^x \ln 4 + 4x^3} $$

        \item \textbf{$g_2(x) = 7 \cdot (1.5)^x - e^x$}
        $$ g_2'(x) = 7 \cdot \frac{d}{dx}((1.5)^x) - \frac{d}{dx}(e^x) = \mathbf{7 \cdot (1.5)^x \ln(1.5) - e^x} $$

        \item \textbf{$g_3(t) = P_0 \cdot a^t$} ($P_0$ und $a$ sind positive Konstanten) \\
        Ableitung nach $t$. $P_0$ und $a$ sind Konstanten bezüglich $t$.
        $$ g_3'(t) = P_0 \cdot \frac{d}{dt}(a^t) = \mathbf{P_0 \cdot a^t \ln a} $$
    \end{itemize}

    \item \textbf{Bestimme die Menge aller Stammfunktionen:}
    Wir verwenden die Regel $\int b^x \,dx = \frac{b^x}{\ln b} + C$ (für $b>0, b \neq 1$).
    \begin{itemize}
        \item \textbf{$h_1(x) = 5^x$}
        $$ H_1(x) = \int 5^x \,dx = \mathbf{\frac{5^x}{\ln 5} + C} $$

        \item \textbf{$h_2(x) = 3 \cdot (0.2)^x + e^{2x}$}
        \begin{align*} H_2(x) &= \int (3 \cdot (0.2)^x + e^{2x}) \,dx \\ &= 3 \int (0.2)^x \,dx + \int e^{2x} \,dx \\ &= 3 \cdot \frac{(0.2)^x}{\ln 0.2} + \frac{1}{2}e^{2x} + C \\ &= \mathbf{\frac{3 \cdot (0.2)^x}{\ln 0.2} + \frac{1}{2}e^{2x} + C} \end{align*}
        (Beachte: $\ln 0.2 = \ln(1/5) = -\ln 5$ ist negativ.)
    \end{itemize}

    \item \textbf{Vergleich $2^x$ und $e^x$:}
    \begin{itemize}
        \item \textbf{Skizziere die Graphen von $f(x)=2^x$ und $g(x)=e^x$ in ein gemeinsames Koordinatensystem (z.B. für $x \in [-2, 3]$). Nutze dazu eine Wertetabelle.} \\
        \textbf{Wertetabelle:}
        \begin{center}
        \begin{tabular}{c|c|c}
        $x$ & $f(x)=2^x$ & $g(x)=e^x$ (ca.) \\
        \hline
        -2 & $2^{-2} = 0.25$ & $e^{-2} \approx 0.135$ \\
        -1 & $2^{-1} = 0.5$ & $e^{-1} \approx 0.368$ \\
        0 & $2^0 = 1$ & $e^0 = 1$ \\
        1 & $2^1 = 2$ & $e^1 \approx 2.718$ \\
        2 & $2^2 = 4$ & $e^2 \approx 7.389$ \\
        3 & $2^3 = 8$ & $e^3 \approx 20.086$ \\
        \end{tabular}
        \end{center}
        \begin{center}
        \includegraphics[width=0.8\textwidth]{grafiken/vergleich_2x_ex.png}
        % --- Beschreibung der Skizze ---
        % Die Skizze zeigt ein Koordinatensystem mit x-Achse von -2 bis 3 und y-Achse von 0 bis ca. 21.
        % Zwei Graphen sind eingezeichnet:
        % 1. f(x)=2^x: Startet flach für negative x, geht durch (0|1), (1|2), (2|4), (3|8).
        % 2. g(x)=e^x: Startet noch flacher als 2^x für negative x, geht ebenfalls durch (0|1), dann (1|e approx 2.72), (2|e^2 approx 7.39), (3|e^3 approx 20.09).
        % Beide Graphen sind streng monoton steigend und schneiden sich bei (0|1).
        % Für x > 0 liegt der Graph von e^x oberhalb des Graphen von 2^x und steigt steiler an.
        % Für x < 0 liegt der Graph von e^x unterhalb des Graphen von 2^x und nähert sich schneller der x-Achse.
        \captionof{figure}{Vergleich der Graphen von $f(x)=2^x$ und $g(x)=e^x$.}
        \label{fig:vergleich_2x_ex}
        \end{center}

        \item \textbf{Vergleiche die Steigungen der beiden Funktionen an der Stelle $x=0$. Welche Funktion wächst dort schneller?}
        Die Steigung ist gegeben durch die erste Ableitung:
        Für $f(x)=2^x$: $f'(x) = 2^x \ln 2$.
        An der Stelle $x=0$: $f'(0) = 2^0 \ln 2 = 1 \cdot \ln 2 = \ln 2 \approx 0.693$.
        Für $g(x)=e^x$: $g'(x) = e^x$.
        An der Stelle $x=0$: $g'(0) = e^0 = 1$.
        Da $1 > \ln 2 \approx 0.693$, ist die Steigung von $g(x)=e^x$ an der Stelle $x=0$ größer als die von $f(x)=2^x$.
        Die Funktion $\mathbf{g(x)=e^x}$ wächst an der Stelle $x=0$ schneller.

        \item \textbf{Berechne $(2^x)'$ und $(e^x)'$.}
        $(2^x)' = \mathbf{2^x \ln 2}$.
        $(e^x)' = \mathbf{e^x}$.
    \end{itemize}
\end{enumerate}

\end{loesungsumgebung}


\begin{aufgabenumgebung}{Produktregel mit $e^x$ üben}
Bilde die erste Ableitung der folgenden Funktionen und vereinfache so weit wie möglich:
\begin{enumerate}
    \item $f_1(x) = (3x-1)e^x$
    \item $f_2(x) = (x^2+2x-5)e^x$
    \item $f_3(t) = t \cdot e^{2t}$ (Hier ist die Kettenregel für $e^{2t}$ zusätzlich nötig!)
\end{enumerate}
\end{aufgabenumgebung}

\begin{loesungsumgebung}[loes:produktregel-ex-ueben]{Produktregel mit $e^x$ üben}
Wir wenden die Produktregel $f'(x) = u'(x)v(x) + u(x)v'(x)$ an. Bei Bedarf wird zusätzlich die Kettenregel für die Ableitung der e-Funktion genutzt: $(e^{i(x)})' = e^{i(x)} \cdot i'(x)$.

\begin{enumerate}[label=(\alph*)]
    \item \textbf{Funktion $f_1(x) = (3x-1)e^x$} \\
    Sei $u(x) = 3x-1 \Rightarrow u'(x) = 3$. \\
    Sei $v(x) = e^x \Rightarrow v'(x) = e^x$.
    \begin{align*}
    f_1'(x) &= u'(x)v(x) + u(x)v'(x) \\
            &= 3 \cdot e^x + (3x-1) \cdot e^x \\
            &= e^x (3 + (3x-1)) \quad (\text{Ausklammern von } e^x) \\
            &= e^x (3 + 3x - 1) \\
            &= \mathbf{e^x (3x + 2)}
    \end{align*}

    \item \textbf{Funktion $f_2(x) = (x^2+2x-5)e^x$} \\
    Sei $u(x) = x^2+2x-5 \Rightarrow u'(x) = 2x+2$. \\
    Sei $v(x) = e^x \Rightarrow v'(x) = e^x$.
    \begin{align*}
    f_2'(x) &= u'(x)v(x) + u(x)v'(x) \\
            &= (2x+2)e^x + (x^2+2x-5)e^x \\
            &= e^x [(2x+2) + (x^2+2x-5)] \quad (\text{Ausklammern von } e^x) \\
            &= e^x (x^2 + 2x + 2x + 2 - 5) \\
            &= \mathbf{e^x (x^2 + 4x - 3)}
    \end{align*}

    \item \textbf{Funktion $f_3(t) = t \cdot e^{2t}$} \\
    Hier leiten wir nach $t$ ab.
    Sei $u(t) = t \Rightarrow u'(t) = 1$. \\
    Sei $v(t) = e^{2t}$. Für $v'(t)$ benötigen wir die Kettenregel:
    Die innere Funktion ist $i(t)=2t \Rightarrow i'(t)=2$. Die äußere Funktion ist $a_v(w)=e^w \Rightarrow a_v'(w)=e^w$.
    $v'(t) = e^{2t} \cdot 2 = 2e^{2t}$.
    \begin{align*}
    f_3'(t) &= u'(t)v(t) + u(t)v'(t) \\
             &= 1 \cdot e^{2t} + t \cdot (2e^{2t}) \\
             &= e^{2t} + 2te^{2t} \\
             &= \mathbf{e^{2t} (1 + 2t)} \quad (\text{Ausklammern von } e^{2t})
    \end{align*}
\end{enumerate}

\end{loesungsumgebung}

\begin{aufgabenumgebung}{Quotientenregel mit $e^x$ üben}
Bilde die erste Ableitung der folgenden Funktionen und vereinfache so weit wie möglich. Gib auch den Definitionsbereich an.
\begin{enumerate}
    \item $f_1(x) = \frac{x+2}{e^x}$
    \item $f_2(x) = \frac{e^{3x}}{2x-1}$ (Kettenregel für $e^{3x}$ nötig!)
\end{enumerate}
\end{aufgabenumgebung}

\begin{loesungsumgebung}[loes:quotientenregel-ex-ueben]{Quotientenregel mit $e^x$ üben}
Wir bestimmen die erste Ableitung der gegebenen Funktionen mithilfe der Quotientenregel $f'(x) = \frac{u'(x)v(x) - u(x)v'(x)}{[v(x)]^2}$ und geben den Definitionsbereich an.

\begin{enumerate}[label=(\alph*)]
    \item \textbf{Funktion $f_1(x) = \frac{x+2}{e^x}$}
    \begin{itemize}
        \item \textbf{Definitionsbereich:} Der Nenner $e^x$ ist für alle reellen Zahlen $x$ definiert und stets $e^x > 0$, also nie Null.
        Somit ist $D_{f_1} = \mathbb{R}$.
        \item \textbf{Ableitung mit Quotientenregel:} \\
        Sei $u(x) = x+2 \Rightarrow u'(x) = 1$. \\
        Sei $v(x) = e^x \Rightarrow v'(x) = e^x$.
        \begin{align*}
        f_1'(x) &= \frac{u'(x)v(x) - u(x)v'(x)}{[v(x)]^2} \\
                &= \frac{1 \cdot e^x - (x+2) \cdot e^x}{(e^x)^2} \\
                &= \frac{e^x (1 - (x+2))}{(e^x)^2} \quad (\text{Ausklammern von } e^x \text{ im Zähler}) \\
                &= \frac{1 - x - 2}{e^x} \quad (\text{Kürzen von } e^x) \\
                &= \mathbf{\frac{-x - 1}{e^x}} \quad \text{oder} \quad \mathbf{-\frac{x+1}{e^x}}
        \end{align*}
        \textit{Alternative Schreibweise:} $f_1'(x) = -(x+1)e^{-x}$.
    \end{itemize}

    \item \textbf{Funktion $f_2(x) = \frac{e^{3x}}{2x-1}$}
    \begin{itemize}
        \item \textbf{Definitionsbereich:} Der Nenner $2x-1$ darf nicht Null sein: $2x-1 \neq 0 \Rightarrow 2x \neq 1 \Rightarrow x \neq \frac{1}{2}$.
        Somit ist $D_{f_2} = \mathbb{R} \setminus \{\frac{1}{2}\}$.
        \item \textbf{Ableitung mit Quotientenregel:} \\
        Sei $u(x) = e^{3x}$. Für $u'(x)$ benötigen wir die Kettenregel:
        Innere Funktion $i(x)=3x \Rightarrow i'(x)=3$. Äußere Funktion $a_u(w)=e^w \Rightarrow a_u'(w)=e^w$.
        $u'(x) = e^{3x} \cdot 3 = 3e^{3x}$. \\
        Sei $v(x) = 2x-1 \Rightarrow v'(x) = 2$.
        \begin{align*}
        f_2'(x) &= \frac{u'(x)v(x) - u(x)v'(x)}{[v(x)]^2} \\
                &= \frac{(3e^{3x})(2x-1) - (e^{3x})(2)}{(2x-1)^2} \\
                &= \frac{e^{3x} [3(2x-1) - 2]}{(2x-1)^2} \quad (\text{Ausklammern von } e^{3x} \text{ im Zähler}) \\
                &= \frac{e^{3x} (6x - 3 - 2)}{(2x-1)^2} \\
                &= \mathbf{\frac{e^{3x}(6x-5)}{(2x-1)^2}}
        \end{align*}
    \end{itemize}
\end{enumerate}

\end{loesungsumgebung}

\begin{aufgabenumgebung}{Kettenregel mit $e^x$ weiter üben}
Bilde die erste Ableitung:
\begin{enumerate}
    \item $f_1(x) = e^{-x^2}$
    \item $f_2(x) = 5e^{2x^3-4x+1}$
    \item $f_3(x) = (e^x+1)^3$ (Hier ist $e^x+1$ die innere Funktion und $(\dots)^3$ die äußere.)
\end{enumerate}
\end{aufgabenumgebung}

\begin{loesungsumgebung}[loes:kettenregel-ex-weiter-ueben]{Kettenregel mit $e^x$ weiter üben}
Wir bilden die erste Ableitung der gegebenen Funktionen mithilfe der Kettenregel.

\begin{enumerate}[label=(\alph*)]
    \item \textbf{Funktion $f_1(x) = e^{-x^2}$}
    \begin{itemize}
        \item Äußere Funktion: $a(u) = e^u \Rightarrow a'(u) = e^u$.
        \item Innere Funktion: $i(x) = -x^2 \Rightarrow i'(x) = -2x$.
    \end{itemize}
    Anwendung der Kettenregel:
    \begin{align*}
    f_1'(x) &= a'(i(x)) \cdot i'(x) \\
            &= e^{-x^2} \cdot (-2x) \\
            &= \mathbf{-2xe^{-x^2}}
    \end{align*}

    \item \textbf{Funktion $f_2(x) = 5e^{2x^3-4x+1}$}
    \begin{itemize}
        \item Äußere Funktion: $a(u) = 5e^u \Rightarrow a'(u) = 5e^u$.
        \item Innere Funktion: $i(x) = 2x^3-4x+1 \Rightarrow i'(x) = 6x^2-4$.
    \end{itemize}
    Anwendung der Kettenregel:
    \begin{align*}
    f_2'(x) &= a'(i(x)) \cdot i'(x) \\
            &= 5e^{2x^3-4x+1} \cdot (6x^2-4) \\
            &= \mathbf{5(6x^2-4)e^{2x^3-4x+1}} \quad \text{oder} \quad \mathbf{(30x^2-20)e^{2x^3-4x+1}}
    \end{align*}

    \item \textbf{Funktion $f_3(x) = (e^x+1)^3$} \\
    Hier ist $e^x+1$ die innere Funktion und $(\dots)^3$ die äußere.
    \begin{itemize}
        \item Äußere Funktion: $a(u) = u^3 \Rightarrow a'(u) = 3u^2$.
        \item Innere Funktion: $i(x) = e^x+1 \Rightarrow i'(x) = e^x$.
    \end{itemize}
    Anwendung der Kettenregel:
    \begin{align*}
    f_3'(x) &= a'(i(x)) \cdot i'(x) \\
            &= 3(e^x+1)^2 \cdot e^x \\
            &= \mathbf{3e^x(e^x+1)^2}
    \end{align*}
\end{enumerate}

\end{loesungsumgebung}


\begin{aufgabenumgebung}{Kombinierte Anwendung: Produkt- und Kettenregel bei e-Funktionen}
Bilde die erste Ableitung der folgenden Funktionen. Vereinfache das Ergebnis so weit wie möglich, indem du z.B. gemeinsame Faktoren (insbesondere den $e$-Term) ausklammerst.
\begin{enumerate}
    \item $f(x) = (2x^2 - 3x + 1) \cdot e^{x^2 - 4}$
    \item $g(x) = (x+1)^2 \cdot e^{-2x}$
\end{enumerate}
\end{aufgabenumgebung}

\begin{loesungsumgebung}[loes:produkt-kette-e-funktionen]{Kombinierte Anwendung: Produkt- und Kettenregel bei e-Funktionen}
Wir bilden die erste Ableitung der gegebenen Funktionen und vereinfachen die Ergebnisse.

\begin{enumerate}[label=(\alph*)]
    \item \textbf{Funktion $f(x) = (2x^2 - 3x + 1) \cdot e^{x^2 - 4}$} \\
    Diese Funktion ist ein Produkt $f(x) = u(x) \cdot v(x)$. Wir wenden die Produktregel an: $f'(x) = u'(x)v(x) + u(x)v'(x)$.
    \begin{itemize}
        \item Sei $u(x) = 2x^2 - 3x + 1 \Rightarrow u'(x) = 4x - 3$.
        \item Sei $v(x) = e^{x^2 - 4}$. Für $v'(x)$ benötigen wir die Kettenregel:
        Die äußere Funktion ist $a_v(w) = e^w \Rightarrow a_v'(w) = e^w$.
        Die innere Funktion ist $i_v(x) = x^2 - 4 \Rightarrow i_v'(x) = 2x$.
        Somit ist $v'(x) = a_v'(i_v(x)) \cdot i_v'(x) = e^{x^2 - 4} \cdot 2x = 2xe^{x^2 - 4}$.
    \end{itemize}
    Anwendung der Produktregel:
    \begin{align*}
    f'(x) &= (4x - 3) \cdot e^{x^2 - 4} + (2x^2 - 3x + 1) \cdot (2xe^{x^2 - 4}) \\
          &= e^{x^2 - 4} \left[ (4x - 3) + (2x^2 - 3x + 1)(2x) \right] \quad (\text{Ausklammern von } e^{x^2-4}) \\
          &= e^{x^2 - 4} [4x - 3 + 4x^3 - 6x^2 + 2x] \\
          &= e^{x^2 - 4} (4x^3 - 6x^2 + 6x - 3) \\
          &= \mathbf{(4x^3 - 6x^2 + 6x - 3)e^{x^2 - 4}}
    \end{align*}

    \item \textbf{Funktion $g(x) = (x+1)^2 \cdot e^{-2x}$} \\
    Dies ist ein Produkt $g(x) = u(x) \cdot v(x)$.
    \begin{itemize}
        \item Sei $u(x) = (x+1)^2$. Für $u'(x)$ mit Kettenregel:
        Äußere Funktion $a_u(w) = w^2 \Rightarrow a_u'(w) = 2w$.
        Innere Funktion $i_u(x) = x+1 \Rightarrow i_u'(x) = 1$.
        $u'(x) = 2(x+1) \cdot 1 = 2(x+1)$.
        \item Sei $v(x) = e^{-2x}$. Für $v'(x)$ mit Kettenregel:
        Äußere Funktion $a_v(w) = e^w \Rightarrow a_v'(w) = e^w$.
        Innere Funktion $i_v(x) = -2x \Rightarrow i_v'(x) = -2$.
        $v'(x) = e^{-2x} \cdot (-2) = -2e^{-2x}$.
    \end{itemize}
    Anwendung der Produktregel:
    \begin{align*}
    g'(x) &= u'(x)v(x) + u(x)v'(x) \\
          &= 2(x+1)e^{-2x} + (x+1)^2(-2e^{-2x}) \\
          &= 2(x+1)e^{-2x} [1 - (x+1)] \quad (\text{Ausklammern von } 2(x+1)e^{-2x}) \\
          &= 2(x+1)e^{-2x} [1 - x - 1] \\
          &= 2(x+1)e^{-2x} [-x] \\
          &= \mathbf{-2x(x+1)e^{-2x}}
    \end{align*}

    \textbf{Alternative Lösung für $g(x)$ durch Ausmultiplizieren des Binoms $(x+1)^2$ zuerst:} \\
    $g(x) = (x^2+2x+1) \cdot e^{-2x}$.
    Sei nun $u(x) = x^2+2x+1 \Rightarrow u'(x) = 2x+2$.
    Sei $v(x) = e^{-2x} \Rightarrow v'(x) = -2e^{-2x}$ (wie oben).
    Anwendung der Produktregel:
    \begin{align*}
    g'(x) &= (2x+2)e^{-2x} + (x^2+2x+1)(-2e^{-2x}) \\
          &= e^{-2x} [(2x+2) - 2(x^2+2x+1)] \quad (\text{Ausklammern von } e^{-2x}) \\
          &= e^{-2x} [2x+2 - 2x^2-4x-2] \\
          &= e^{-2x} [-2x^2 - 2x] \\
          &= e^{-2x} [-2x(x+1)] \\
          &= \mathbf{-2x(x+1)e^{-2x}}
    \end{align*}
    Beide Lösungswege führen zum selben Ergebnis.
\end{enumerate}

\end{loesungsumgebung}


\begin{aufgabenumgebung}{Kurvendiskussionen mit Exponentialfunktionen}
Führe eine vollständige Kurvendiskussion für die folgenden Funktionen durch.
\begin{enumerate}
    \item $f(x) = x e^{-x}$
    \item $g(x) = (x^2-2)e^x$
    \item \textbf{Für Experten:} $h(x) = e^{x^2-2x}$. (Hier wird die Kettenregel komplexer, aber die Struktur der Kurvendiskussion bleibt gleich.)
\end{enumerate}
\end{aufgabenumgebung}



\begin{loesungsumgebung}[loes:kurvendiskussion-ex-funktionen]{Kurvendiskussionen mit Exponentialfunktionen}

\begin{enumerate}[label=(\alph*)]
    \item \textbf{Funktion $f(x) = x e^{-x}$}

    \subsubsection*{1. Definitionsbereich}
    Die Funktion $f(x)$ ist als Produkt einer linearen Funktion ($x$) und einer Exponentialfunktion ($e^{-x}$) für alle reellen Zahlen definiert.
    $D_f = \mathbb{R}$.

    \subsubsection*{2. Symmetrie}
    $f(-x) = (-x)e^{-(-x)} = -xe^x$.
    Da $f(-x) \neq f(x)$ und $f(-x) \neq -f(x)$ (z.B. $f(1)=e^{-1}$, $f(-1)=-e$, $-f(1)=-e^{-1}$), liegt keine einfache Achsen- oder Punktsymmetrie zum Ursprung vor.

    \subsubsection*{3. Verhalten im Unendlichen}
    \begin{itemize}
        \item Für $x \to \infty$: $\lim_{x \to \infty} x e^{-x} = \lim_{x \to \infty} \frac{x}{e^x}$. Da $e^x$ schneller wächst als $x$ (Regel von L'Hôpital anwendbar: $\lim \frac{1}{e^x} = 0$), gilt $\lim_{x \to \infty} f(x) = 0$.
        Somit ist $y=0$ eine horizontale Asymptote für $x \to \infty$.
        \item Für $x \to -\infty$: Sei $u = -x$. Dann $u \to \infty$.
        $\lim_{x \to -\infty} x e^{-x} = \lim_{u \to \infty} (-u)e^u = -\infty \cdot \infty = -\infty$.
    \end{itemize}

    \subsubsection*{4. Achsenschnittpunkte}
    \begin{itemize}
        \item y-Achsenabschnitt (für $x=0$): $f(0) = 0 \cdot e^0 = 0 \cdot 1 = 0$. Punkt $P_y(0|0)$.
        \item Nullstellen (für $f(x)=0$): $xe^{-x} = 0$. Da $e^{-x} > 0$ für alle $x$, muss $x=0$ sein.
        Die einzige Nullstelle ist $x_N=0$.
    \end{itemize}

    \subsubsection*{5. Erste Ableitung $f'(x)$}
    Mit der Produktregel ($u=x, u'=1; v=e^{-x}, v'=-e^{-x}$):
    $f'(x) = 1 \cdot e^{-x} + x \cdot (-e^{-x}) = e^{-x} - xe^{-x} = \mathbf{e^{-x}(1-x)}$.

    \subsubsection*{6. Extrempunkte}
    Notwendige Bedingung: $f'(x)=0$.
    $e^{-x}(1-x) = 0$. Da $e^{-x} \neq 0$, muss $1-x=0 \Rightarrow x=1$.
    Potenzielle Extremstelle $x_E=1$. $f(1) = 1 \cdot e^{-1} = \frac{1}{e}$.
    Hinreichende Bedingung (mit $f''(x)$, siehe unten): $f''(1) = e^{-1}(1-2) = -e^{-1} < 0$.
    Also liegt ein lokaler Hochpunkt bei $H(1 | \frac{1}{e} \approx 0.368)$ vor.

    \subsubsection*{7. Monotonieverhalten}
    Das Vorzeichen von $f'(x) = e^{-x}(1-x)$ hängt vom Faktor $(1-x)$ ab, da $e^{-x}>0$.
    \begin{itemize}
        \item Für $x < 1$: $1-x > 0 \Rightarrow f'(x) > 0 \Rightarrow f(x)$ ist streng monoton steigend.
        \item Für $x > 1$: $1-x < 0 \Rightarrow f'(x) < 0 \Rightarrow f(x)$ ist streng monoton fallend.
    \end{itemize}

    \subsubsection*{8. Zweite Ableitung $f''(x)$}
    Mit der Produktregel für $f'(x)=e^{-x}(1-x)$ ($u=e^{-x}, u'=-e^{-x}; v=1-x, v'=-1$):
    $f''(x) = (-e^{-x})(1-x) + e^{-x}(-1) = -e^{-x} + xe^{-x} - e^{-x} = xe^{-x} - 2e^{-x} = \mathbf{e^{-x}(x-2)}$.

    \subsubsection*{9. Wendepunkte}
    Notwendige Bedingung: $f''(x)=0$.
    $e^{-x}(x-2) = 0$. Da $e^{-x} \neq 0$, muss $x-2=0 \Rightarrow x=2$.
    Potenzielle Wendestelle $x_W=2$. $f(2) = 2e^{-2} = \frac{2}{e^2}$.
    Hinreichende Bedingung (mit $f'''(x)$):
    $f'''(x) = \frac{d}{dx}(e^{-x}(x-2)) = (-e^{-x})(x-2) + e^{-x}(1) = e^{-x}(-x+2+1) = e^{-x}(3-x)$.
    $f'''(2) = e^{-2}(3-2) = e^{-2} \neq 0$.
    Also liegt ein Wendepunkt bei $W(2 | \frac{2}{e^2} \approx 0.271)$ vor.

    \subsubsection*{10. Krümmungsverhalten}
    Das Vorzeichen von $f''(x) = e^{-x}(x-2)$ hängt von $(x-2)$ ab.
    \begin{itemize}
        \item Für $x < 2$: $x-2 < 0 \Rightarrow f''(x) < 0 \Rightarrow f(x)$ ist rechtsgekrümmt (konkav).
        \item Für $x > 2$: $x-2 > 0 \Rightarrow f''(x) > 0 \Rightarrow f(x)$ ist linksgekrümmt (konvex).
    \end{itemize}

    \subsubsection*{11. Wertebereich}
    $W_f = (-\infty, \frac{1}{e}]$.

    \item \textbf{Funktion $g(x) = (x^2-2)e^x$}

    \subsubsection*{1. Definitionsbereich}
    $D_g = \mathbb{R}$.

    \subsubsection*{2. Symmetrie}
    $g(-x) = ((-x)^2-2)e^{-x} = (x^2-2)e^{-x}$. Keine einfache Symmetrie.

    \subsubsection*{3. Verhalten im Unendlichen}
    \begin{itemize}
        \item Für $x \to \infty$: $\lim_{x \to \infty} (x^2-2)e^x = \infty \cdot \infty = \infty$.
        \item Für $x \to -\infty$: $\lim_{x \to -\infty} (x^2-2)e^x = \lim_{x \to -\infty} \frac{x^2-2}{e^{-x}}$. Mit L'Hôpital (zweimal):
        $\lim_{x \to -\infty} \frac{2x}{-e^{-x}} = \lim_{x \to -\infty} \frac{2}{e^{-x}} = 0$.
        Somit ist $y=0$ eine horizontale Asymptote für $x \to -\infty$.
    \end{itemize}

    \subsubsection*{4. Achsenschnittpunkte}
    \begin{itemize}
        \item y-Achsenabschnitt ($x=0$): $g(0) = (0^2-2)e^0 = -2 \cdot 1 = -2$. Punkt $P_y(0|-2)$.
        \item Nullstellen ($g(x)=0$): $(x^2-2)e^x = 0$. Da $e^x \neq 0$, muss $x^2-2=0 \Rightarrow x^2=2 \Rightarrow x_{N1,2} = \pm\sqrt{2}$.
    \end{itemize}

    \subsubsection*{5. Erste Ableitung $g'(x)$}
    Mit der Produktregel ($u=x^2-2, u'=2x; v=e^x, v'=e^x$):
    $g'(x) = 2xe^x + (x^2-2)e^x = e^x(2x+x^2-2) = \mathbf{e^x(x^2+2x-2)}$.

    \subsubsection*{6. Extrempunkte}
    $g'(x)=0 \Rightarrow e^x(x^2+2x-2)=0 \Rightarrow x^2+2x-2=0$.
    $x_{1,2} = \frac{-2 \pm \sqrt{4-4(1)(-2)}}{2} = \frac{-2 \pm \sqrt{12}}{2} = \frac{-2 \pm 2\sqrt{3}}{2} = -1 \pm \sqrt{3}$.
    $x_{E1} = -1-\sqrt{3} \approx -2.732$, $x_{E2} = -1+\sqrt{3} \approx 0.732$.
    Hinreichende Bedingung (mit $g''(x)$, siehe unten):
    $g''(x_{E1}) = e^{-1-\sqrt{3}}(-1-\sqrt{3})^2 + 4(-1-\sqrt{3})) = e^{-1-\sqrt{3}}(4+2\sqrt{3}-4-4\sqrt{3}) = -2\sqrt{3}e^{-1-\sqrt{3}} < 0 \Rightarrow$ Hochpunkt.
    $g(x_{E1}) = ((-1-\sqrt{3})^2-2)e^{-1-\sqrt{3}} = (1+2\sqrt{3}+3-2)e^{-1-\sqrt{3}} = (2+2\sqrt{3})e^{-1-\sqrt{3}} \approx 0.355$.
    $H(-1-\sqrt{3} | (2+2\sqrt{3})e^{-1-\sqrt{3}})$.
    $g''(x_{E2}) = e^{-1+\sqrt{3}}((-1+\sqrt{3})^2 + 4(-1+\sqrt{3})) = e^{-1+\sqrt{3}}(4-2\sqrt{3}-4+4\sqrt{3}) = 2\sqrt{3}e^{-1+\sqrt{3}} > 0 \Rightarrow$ Tiefpunkt.
    $g(x_{E2}) = ((-1+\sqrt{3})^2-2)e^{-1+\sqrt{3}} = (1-2\sqrt{3}+3-2)e^{-1+\sqrt{3}} = (2-2\sqrt{3})e^{-1+\sqrt{3}} \approx -3.044$.
    $T(-1+\sqrt{3} | (2-2\sqrt{3})e^{-1+\sqrt{3}})$.

    \subsubsection*{7. Monotonieverhalten}
    Vorzeichen von $g'(x)=e^x(x^2+2x-2)$ hängt von $x^2+2x-2$ ab. Dies ist eine nach oben geöffnete Parabel mit Nullstellen $x_{E1}, x_{E2}$.
    \begin{itemize}
        \item Für $x < -1-\sqrt{3}$: $g'(x) > 0 \Rightarrow g(x)$ steigend.
        \item Für $-1-\sqrt{3} < x < -1+\sqrt{3}$: $g'(x) < 0 \Rightarrow g(x)$ fallend.
        \item Für $x > -1+\sqrt{3}$: $g'(x) > 0 \Rightarrow g(x)$ steigend.
    \end{itemize}

    \subsubsection*{8. Zweite Ableitung $g''(x)$}
    Mit Produktregel für $g'(x)=e^x(x^2+2x-2)$ ($u=e^x, u'=e^x; v=x^2+2x-2, v'=2x+2$):
    $g''(x) = e^x(x^2+2x-2) + e^x(2x+2) = e^x(x^2+2x-2+2x+2) = \mathbf{e^x(x^2+4x)}$.

    \subsubsection*{9. Wendepunkte}
    $g''(x)=0 \Rightarrow e^x(x^2+4x)=0 \Rightarrow x(x+4)=0$.
    $x_{W1}=0$, $x_{W2}=-4$.
    Hinreichende Bedingung (mit $g'''(x)$):
    $g'''(x) = \frac{d}{dx}(e^x(x^2+4x)) = e^x(x^2+4x) + e^x(2x+4) = e^x(x^2+6x+4)$.
    $g'''(0) = e^0(4) = 4 \neq 0 \Rightarrow W_1(0|g(0)) = (0|-2)$.
    $g'''(-4) = e^{-4}(16-24+4) = -4e^{-4} \neq 0 \Rightarrow W_2(-4|g(-4))$.
    $g(-4) = ((-4)^2-2)e^{-4} = (16-2)e^{-4} = 14e^{-4} \approx 0.256$.
    $W_2(-4 | 14e^{-4})$.

    \subsubsection*{10. Krümmungsverhalten}
    Vorzeichen von $g''(x)=e^x x(x+4)$ hängt von $x(x+4)$ ab. Nach oben geöffnete Parabel mit Nullstellen $0, -4$.
    \begin{itemize}
        \item Für $x < -4$: $g''(x) > 0 \Rightarrow g(x)$ linksgekrümmt.
        \item Für $-4 < x < 0$: $g''(x) < 0 \Rightarrow g(x)$ rechtsgekrümmt.
        \item Für $x > 0$: $g''(x) > 0 \Rightarrow g(x)$ linksgekrümmt.
    \end{itemize}

    \subsubsection*{11. Wertebereich}
    $W_g = [g(x_{E2}), \infty) = [(2-2\sqrt{3})e^{-1+\sqrt{3}}, \infty) \approx [-3.044, \infty)$.

    \item \textbf{Für Experten: Funktion $h(x) = e^{x^2-2x}$}

    \subsubsection*{1. Definitionsbereich}
    $D_h = \mathbb{R}$.

    \subsubsection*{2. Symmetrie}
    $h(-x) = e^{(-x)^2-2(-x)} = e^{x^2+2x}$. Keine einfache Symmetrie.

    \subsubsection*{3. Verhalten im Unendlichen}
    Der Exponent ist $p(x)=x^2-2x$.
    \begin{itemize}
        \item Für $x \to \infty$: $p(x) \to \infty \Rightarrow h(x) = e^{p(x)} \to \infty$.
        \item Für $x \to -\infty$: $p(x) \to \infty \Rightarrow h(x) = e^{p(x)} \to \infty$.
    \end{itemize}

    \subsubsection*{4. Achsenschnittpunkte}
    \begin{itemize}
        \item y-Achsenabschnitt ($x=0$): $h(0) = e^{0-0} = e^0 = 1$. Punkt $P_y(0|1)$.
        \item Nullstellen ($h(x)=0$): $e^{x^2-2x}=0$. Da $e^u > 0$ für alle $u$, gibt es keine Nullstellen.
    \end{itemize}

    \subsubsection*{5. Erste Ableitung $h'(x)$}
    Mit der Kettenregel (innere Funktion $i(x)=x^2-2x \Rightarrow i'(x)=2x-2$; äußere $a(u)=e^u \Rightarrow a'(u)=e^u$):
    $h'(x) = e^{x^2-2x} \cdot (2x-2) = \mathbf{(2x-2)e^{x^2-2x}}$.

    \subsubsection*{6. Extrempunkte}
    $h'(x)=0 \Rightarrow (2x-2)e^{x^2-2x}=0$. Da $e^{x^2-2x} \neq 0$, muss $2x-2=0 \Rightarrow x=1$.
    Potenzielle Extremstelle $x_E=1$. $h(1) = e^{1^2-2(1)} = e^{-1} = \frac{1}{e}$.
    Hinreichende Bedingung (mit $h''(x)$, siehe unten): $h''(1) = (4(1)^2-8(1)+6)e^{1-2} = (4-8+6)e^{-1} = 2e^{-1} > 0$.
    Also liegt ein lokaler (und globaler) Tiefpunkt bei $T(1 | \frac{1}{e} \approx 0.368)$ vor.

    \subsubsection*{7. Monotonieverhalten}
    Das Vorzeichen von $h'(x)=(2x-2)e^{x^2-2x}$ hängt von $(2x-2)$ ab.
    \begin{itemize}
        \item Für $x < 1$: $2x-2 < 0 \Rightarrow h'(x) < 0 \Rightarrow h(x)$ streng monoton fallend.
        \item Für $x > 1$: $2x-2 > 0 \Rightarrow h'(x) > 0 \Rightarrow h(x)$ streng monoton steigend.
    \end{itemize}

    \subsubsection*{8. Zweite Ableitung $h''(x)$}
    Mit Produktregel für $h'(x)=(2x-2)e^{x^2-2x}$ ($u=2x-2, u'=2; v=e^{x^2-2x}, v'=(2x-2)e^{x^2-2x}$):
    $h''(x) = 2 \cdot e^{x^2-2x} + (2x-2) \cdot (2x-2)e^{x^2-2x}$
    $h''(x) = e^{x^2-2x} [2 + (2x-2)^2] = e^{x^2-2x} [2 + 4x^2-8x+4] = \mathbf{(4x^2-8x+6)e^{x^2-2x}}$.

    \subsubsection*{9. Wendepunkte}
    $h''(x)=0 \Rightarrow (4x^2-8x+6)e^{x^2-2x}=0$.
    Da $e^{x^2-2x} \neq 0$, muss $4x^2-8x+6=0 \Rightarrow 2x^2-4x+3=0$.
    Diskriminante $D = (-4)^2 - 4(2)(3) = 16-24 = -8$.
    Da $D<0$, gibt es keine reellen Lösungen für $2x^2-4x+3=0$.
    Somit hat $h(x)$ \textbf{keine Wendepunkte}.

    \subsubsection*{10. Krümmungsverhalten}
    Das Vorzeichen von $h''(x)=(4x^2-8x+6)e^{x^2-2x}$ hängt von $(4x^2-8x+6)$ ab.
    Die quadratische Funktion $q(x)=4x^2-8x+6$ ist eine nach oben geöffnete Parabel. Da sie keine Nullstellen hat und $q(0)=6>0$, ist $q(x)$ immer positiv.
    Somit ist $h''(x) > 0$ für alle $x \in \mathbb{R}$.
    Der Graph von $h(x)$ ist \textbf{immer linksgekrümmt (konvex)}.

    \subsubsection*{11. Wertebereich}
    $W_h = [\frac{1}{e}, \infty)$.
\end{enumerate}

\subsubsection*{Gemeinsame Skizze der Graphen}
Die Anfertigung einer aussagekräftigen gemeinsamen Skizze erfordert die Berücksichtigung der charakteristischen Punkte und Verhaltensweisen aller drei Funktionen.
\begin{itemize}
    \item $f(x)=xe^{-x}$: Startet bei $-\infty$ für $x \to -\infty$, geht durch $(0|0)$, hat ein Maximum bei $(1|1/e)$, einen Wendepunkt bei $(2|2/e^2)$ und nähert sich $y=0$ für $x \to \infty$.
    \item $g(x)=(x^2-2)e^x$: Nähert sich $y=0$ für $x \to -\infty$, y-Achsenabschnitt $(0|-2)$, Nullstellen bei $\pm\sqrt{2}$, Hochpunkt bei $x \approx -2.73$, Tiefpunkt bei $x \approx 0.73$, Wendepunkte bei $x=-4$ und $x=0$, geht nach $\infty$ für $x \to \infty$.
    \item $h(x)=e^{x^2-2x}$: Geht nach $\infty$ für $x \to \pm\infty$, y-Achsenabschnitt $(0|1)$, Tiefpunkt bei $(1|1/e)$, immer linksgekrümmt, keine Nullstellen.
\end{itemize}
Da die Wertebereiche und das Verhalten stark variieren (z.B. $f(x)$ und $g(x)$ haben Asymptote $y=0$, $h(x)$ nicht), ist eine einzelne Skizze, die alle Details gut darstellt, anspruchsvoll. Man würde verschiedene y-Skalierungen oder mehrere Detailansichten benötigen, um alle charakteristischen Punkte gut sichtbar zu machen.

\begin{center}
\includegraphics[width=0.8\textwidth]{grafiken/kurvendiskussion_ex_kombiniert.png}
% --- Beschreibung der Skizze ---
% Die Skizze sollte versuchen, die drei Graphen f(x), g(x), und h(x) darzustellen.
% f(x) = xe^{-x}: Charakteristischer Hügel im positiven Bereich, nähert sich der x-Achse für x -> unendlich.
% g(x) = (x^2-2)e^x: Nähert sich der x-Achse für x -> -unendlich, hat Nullstellen, ein lokales Maximum und Minimum.
% h(x) = e^{x^2-2x}: Parabelähnlicher Verlauf (obwohl es eine e-Funktion ist), mit Minimum bei (1, 1/e), steigt stark an für x -> +/- unendlich.
% Wegen der stark unterschiedlichen Wertebereiche und Verläufe ist eine einzelne übersichtliche Darstellung schwierig.
% Die x-Achse sollte zumindest den Bereich von ca. -4 bis 4 abdecken.
% Die y-Achse müsste einen Bereich von z.B. -4 bis 10 oder mehr abdecken, wobei Details von f(x) dann klein werden könnten.
\captionof{figure}{Skizze der Graphen von $f(x)$, $g(x)$ und $h(x)$ (konzeptionell).}
\label{fig:kurvendiskussion_ex_kombiniert}
\end{center}

\end{loesungsumgebung}




\begin{aufgabenumgebung}{Anwendung: Radioaktiver Zerfall}
Eine radioaktive Substanz zerfällt so, dass die noch vorhandene Menge $M(t)$ (in Gramm) nach $t$ Jahren durch die Funktion $M(t) = 100 \cdot e^{-0.05t}$ beschrieben wird.
\begin{enumerate}
    \item Wie groß ist die Anfangsmenge der Substanz?
    \item Wie viel Gramm sind nach 10 Jahren noch vorhanden?
    \item Mit welcher Rate zerfällt die Substanz zum Zeitpunkt $t=0$ und zum Zeitpunkt $t=10$ Jahre (in Gramm pro Jahr)? (Tipp: Ableitung $M'(t)$).
    \item \textbf{Halbwertszeit:} Nach welcher Zeit $T_H$ ist nur noch die Hälfte der ursprünglichen Menge vorhanden? (Setze $M(T_H) = \frac{1}{2}M(0)$ und löse nach $T_H$. Hierfür benötigst du den natürlichen Logarithmus $\ln$, die Umkehrfunktion von $e^x$.)
\end{enumerate}
\end{aufgabenumgebung}

\begin{loesungsumgebung}[loes:radioaktiver-zerfall-anwendung]{Anwendung: Radioaktiver Zerfall}
Die vorhandene Menge der Substanz $M(t)$ (in Gramm) nach $t$ Jahren ist gegeben durch $M(t) = 100 \cdot e^{-0.05t}$.

\begin{enumerate}[label=(\alph*)]
    \item \textbf{Wie groß ist die Anfangsmenge der Substanz?}
    Die Anfangsmenge ist die Menge zum Zeitpunkt $t=0$:
    $$ M(0) = 100 \cdot e^{-0.05 \cdot 0} = 100 \cdot e^0 = 100 \cdot 1 = 100 $$
    Die Anfangsmenge der Substanz beträgt \textbf{100 Gramm}.

    \item \textbf{Wie viel Gramm sind nach 10 Jahren noch vorhanden?}
    Wir setzen $t=10$ Jahre in die Funktion ein:
    $$ M(10) = 100 \cdot e^{-0.05 \cdot 10} = 100 \cdot e^{-0.5} $$
    Der exakte Wert ist $100e^{-0.5}$ Gramm.
    Näherungswert: $e^{-0.5} \approx 0.60653$.
    $$ M(10) \approx 100 \cdot 0.60653 = 60.653 $$
    Nach 10 Jahren sind noch ca. \textbf{60,65 Gramm} der Substanz vorhanden.

    \item \textbf{Mit welcher Rate zerfällt die Substanz zum Zeitpunkt $t=0$ und zum Zeitpunkt $t=10$ Jahre?}
    Die Zerfallsrate ist die erste Ableitung $M'(t)$ der Mengenfunktion.
    $M(t) = 100 e^{-0.05t}$.
    Mit der Kettenregel ($u(t)=-0.05t \Rightarrow u'(t)=-0.05$):
    $M'(t) = 100 \cdot e^{-0.05t} \cdot (-0.05) = -5e^{-0.05t}$.
    Die Einheit der Rate ist Gramm pro Jahr (g/Jahr). Das negative Vorzeichen zeigt an, dass die Menge abnimmt.

    \begin{itemize}
        \item \textbf{Zerfallsrate zum Zeitpunkt $t=0$ Jahre:}
        $M'(0) = -5e^{-0.05 \cdot 0} = -5e^0 = -5 \cdot 1 = -5$.
        Zum Zeitpunkt $t=0$ zerfällt die Substanz mit einer Rate von \textbf{5 Gramm pro Jahr}.
        \item \textbf{Zerfallsrate zum Zeitpunkt $t=10$ Jahre:}
        $M'(10) = -5e^{-0.05 \cdot 10} = -5e^{-0.5}$.
        $M'(10) \approx -5 \cdot 0.60653 \approx -3.03265$.
        Zum Zeitpunkt $t=10$ Jahre zerfällt die Substanz mit einer Rate von ca. \textbf{3,03 Gramm pro Jahr}.
    \end{itemize}

    \item \textbf{Halbwertszeit $T_H$:}
    Die Halbwertszeit ist die Zeit, nach der nur noch die Hälfte der ursprünglichen Menge vorhanden ist.
    Die Anfangsmenge ist $M(0) = 100\,$g. Die Hälfte davon ist $50\,$g.
    Wir setzen $M(T_H) = 50$:
    $$ 100 \cdot e^{-0.05 T_H} = 50 $$
    Wir lösen diese Gleichung nach $T_H$ auf:
    $$
    \begin{array}{r c l c l}
    \umformung{100 \cdot e^{-0.05 T_H}}{50}{\div}{100}
    \umformung{e^{-0.05 T_H}}{0.5}{\text{nimm } \ln(\dots)}{\text{auf beiden Seiten}}
    \umformung{\ln(e^{-0.05 T_H})}{\ln(0.5)}{}{ } % \ln(e^x)=x
    \umformung{-0.05 T_H}{\ln(0.5)}{\div}{(-0.05)}
    \umformungend{T_H}{\frac{\ln(0.5)}{-0.05}}
    \end{array}
    $$
    Da $\ln(0.5) = \ln(1/2) = \ln(1) - \ln(2) = 0 - \ln(2) = -\ln(2)$, können wir schreiben:
    $$ T_H = \frac{-\ln(2)}{-0.05} = \frac{\ln(2)}{0.05} $$
    Mit $\ln(2) \approx 0.693147$:
    $$ T_H \approx \frac{0.693147}{0.05} = 13.86294 $$
    Die Halbwertszeit beträgt $\mathbf{T_H = \frac{\ln 2}{0.05} \approx 13.86}$ Jahre.
\end{enumerate}

\end{loesungsumgebung}

\begin{aufgabenumgebung}{Optimierung im biologischen Kontext – Wachstum und Hemmung}
Eine Bakterienpopulation wächst zunächst, wird aber durch einen hemmenden Faktor (z.B. begrenzte Nährstoffe) beeinflusst. Die Anzahl der Bakterien $N$ (in Tausend) nach $t$ Stunden kann modelliert werden durch die Funktion:
\[ N(t) = 5t \cdot e^{-0.1t} \quad (\text{für } t \ge 0) \]
\begin{enumerate}
    \item \textbf{Anfangsbestand:} Wie viele Bakterien sind zum Zeitpunkt $t=0$ vorhanden? Interpretiere das Ergebnis im Kontext.
    \item \textbf{Wachstumsrate:} Bestimme die Funktion $N'(t)$, welche die Wachstumsrate der Bakterienpopulation zum Zeitpunkt $t$ angibt. (Produkt- und Kettenregel sind hier gefragt!)
    \item \textbf{Maximale Population:}
        \begin{itemize}
            \item Zu welchem Zeitpunkt $t_{max}$ erreicht die Bakterienpopulation ihr Maximum? (Tipp: Notwendige Bedingung für Extremstellen $N'(t)=0$. Da $e^{-0.1t}$ nie Null wird, musst du nur den anderen Faktor betrachten.)
            \item Überprüfe mit der zweiten Ableitung $N''(t)$ oder dem Vorzeichenwechselkriterium von $N'(t)$, ob es sich tatsächlich um ein Maximum handelt.
            \item Wie groß ist die maximale Bakterienpopulation $N(t_{max})$?
        \end{itemize}
    \item \textbf{Verhalten für $t \to \infty$:} Was passiert mit der Bakterienpopulation für sehr große Zeiten? (Untersuche $\lim_{t \to \infty} N(t)$). Ist das biologisch sinnvoll?
    \item \textbf{Stärkste Zunahme/Abnahme der Wachstumsrate (für Experten):}
        Die Änderungsrate der Wachstumsrate wird durch $N''(t)$ beschrieben. Wann ist die Zunahme der Wachstumsrate maximal (d.h. wann wächst die Population am schnellsten schneller)? Wann ist die Abnahme der Wachstumsrate maximal (d.h. wann verlangsamt sich das Wachstum am stärksten)? (Tipp: Untersuche $N''(t)$ auf Extremstellen, d.h. bilde $N'''(t)$).
    \item \textbf{Skizze:} Skizziere den Graphen von $N(t)$ für $t \ge 0$ und markiere den maximalen Bestand.
\end{enumerate}
\end{aufgabenumgebung}



\begin{loesungsumgebung}[loes:optimierung-bakterienwachstum]{Optimierung im biologischen Kontext – Wachstum und Hemmung}
Die Anzahl der Bakterien $N$ (in Tausend) nach $t$ Stunden wird modelliert durch $N(t) = 5t \cdot e^{-0.1t}$ für $t \ge 0$.

\begin{enumerate}[label=(\alph*)]
    \item \textbf{Anfangsbestand:}
    Der Anfangsbestand ist die Anzahl der Bakterien zum Zeitpunkt $t=0$:
    $$ N(0) = 5 \cdot 0 \cdot e^{-0.1 \cdot 0} = 0 \cdot e^0 = 0 \cdot 1 = 0 $$
    Zum Zeitpunkt $t=0$ sind \textbf{0 Tausend Bakterien} (also keine Bakterien laut Modell) vorhanden. Dies könnte bedeuten, dass die Beobachtung mit einer vernachlässigbar kleinen Startpopulation beginnt oder dass $t=0$ den Zeitpunkt unmittelbar vor Beginn des exponentiellen Wachstums darstellt, das dann durch die Hemmung beeinflusst wird.

    \item \textbf{Wachstumsrate $N'(t)$:}
    Die Wachstumsrate ist die erste Ableitung von $N(t)$. Wir verwenden die Produktregel: $u(t)=5t \Rightarrow u'(t)=5$; $v(t)=e^{-0.1t}$.
    Für $v'(t)$ verwenden wir die Kettenregel: $v'(t) = e^{-0.1t} \cdot (-0.1) = -0.1e^{-0.1t}$.
    \begin{align*}
    N'(t) &= u'(t)v(t) + u(t)v'(t) \\
           &= 5 \cdot e^{-0.1t} + 5t \cdot (-0.1e^{-0.1t}) \\
           &= 5e^{-0.1t} - 0.5te^{-0.1t} \\
           &= \mathbf{e^{-0.1t}(5 - 0.5t)}
    \end{align*}
    Die Einheit der Wachstumsrate ist Tausend Bakterien pro Stunde.

    \item \textbf{Maximale Population:}
    \begin{itemize}
        \item \textbf{Zeitpunkt $t_{max}$ der maximalen Population:}
        Wir setzen $N'(t)=0$: $e^{-0.1t}(5 - 0.5t) = 0$.
        Da $e^{-0.1t}$ stets positiv ist, muss der zweite Faktor Null sein:
        $5 - 0.5t = 0 \Rightarrow 0.5t = 5 \Rightarrow t = \frac{5}{0.5} = 10$.
        Der Zeitpunkt der potentiell maximalen Population ist $\mathbf{t_{max} = 10}$ Stunden.
        \item \textbf{Überprüfung mit der zweiten Ableitung $N''(t)$:}
        Wir leiten $N'(t) = 5e^{-0.1t} - 0.5te^{-0.1t}$ ab (Produktregel für den zweiten Term oder für die faktorisierte Form).
        Nehmen wir $N'(t) = e^{-0.1t}(5 - 0.5t)$.
        $u_2(t)=e^{-0.1t} \Rightarrow u_2'(t)=-0.1e^{-0.1t}$.
        $v_2(t)=5-0.5t \Rightarrow v_2'(t)=-0.5$.
        $N''(t) = u_2'(t)v_2(t) + u_2(t)v_2'(t)$
        $N''(t) = -0.1e^{-0.1t}(5 - 0.5t) + e^{-0.1t}(-0.5)$
        $N''(t) = e^{-0.1t}[-0.1(5 - 0.5t) - 0.5]$
        $N''(t) = e^{-0.1t}[-0.5 + 0.05t - 0.5] = e^{-0.1t}(0.05t - 1)$.
        Für $t=10$:
        $N''(10) = e^{-0.1 \cdot 10}(0.05 \cdot 10 - 1) = e^{-1}(0.5 - 1) = e^{-1}(-0.5) = -\frac{0.5}{e}$.
        Da $N''(10) < 0$, handelt es sich bei $t=10$ Stunden tatsächlich um ein lokales Maximum.
        \item \textbf{Maximale Bakterienpopulation $N(t_{max})$:}
        $N(10) = 5 \cdot 10 \cdot e^{-0.1 \cdot 10} = 50e^{-1} = \frac{50}{e}$.
        $N(10) \approx \frac{50}{2.71828} \approx 18.394$.
        Die maximale Population beträgt $\mathbf{\frac{50}{e} \approx 18.394}$ Tausend Bakterien (also ca. 18394 Bakterien).
    \end{itemize}

    \item \textbf{Verhalten für $t \to \infty$:}
    Wir untersuchen den Grenzwert der Populationsfunktion $N(t) = 5t e^{-0.1t} = \frac{5t}{e^{0.1t}}$ für $t \to \infty$.
    Da die Exponentialfunktion im Nenner schneller wächst als die lineare Funktion im Zähler, gilt (Standardgrenzwert):
    $$ \lim_{t \to \infty} \frac{5t}{e^{0.1t}} \stackrel{L'H}{=} \lim_{t \to \infty} \frac{5}{0.1e^{0.1t}} = \frac{5}{\infty} = 0 $$
    Für sehr große Zeiten nähert sich die Bakterienpopulation \textbf{Null} an.
    \textbf{Biologische Sinnhaftigkeit:} Dies ist biologisch sinnvoll. Auch wenn die Population anfangs wächst, führen begrenzte Nährstoffe, Anhäufung von Abfallprodukten oder andere limitierende Faktoren dazu, dass die Population nicht unbegrenzt wachsen kann und schließlich wieder abnimmt und ausstirbt oder ein sehr niedriges Niveau erreicht.

    \item \textbf{Stärkste Zunahme/Abnahme der Wachstumsrate (für Experten):}
    Die Wachstumsrate ist $N'(t) = e^{-0.1t}(5 - 0.5t)$. Die Änderungsrate der Wachstumsrate ist $N''(t) = e^{-0.1t}(0.05t - 1)$. Wir suchen die Extrema von $N''(t)$. Dazu bilden wir die Ableitung von $N''(t)$, also $N'''(t)$.
    $N'''(t) = \frac{d}{dt} [e^{-0.1t}(0.05t - 1)]$.
    Mit $u(t)=e^{-0.1t} \Rightarrow u'(t)=-0.1e^{-0.1t}$ und $v(t)=0.05t-1 \Rightarrow v'(t)=0.05$.
    $N'''(t) = -0.1e^{-0.1t}(0.05t - 1) + e^{-0.1t}(0.05)$
    $N'''(t) = e^{-0.1t}[-0.1(0.05t - 1) + 0.05]$
    $N'''(t) = e^{-0.1t}[-0.005t + 0.1 + 0.05] = e^{-0.1t}(-0.005t + 0.15)$.
    Setze $N'''(t)=0$ für kritische Stellen von $N''(t)$:
    $e^{-0.1t}(-0.005t + 0.15) = 0$.
    Da $e^{-0.1t} \neq 0$, muss $-0.005t + 0.15 = 0 \Rightarrow 0.005t = 0.15 \Rightarrow t = \frac{0.15}{0.005} = \frac{150}{5} = 30$.
    Um die Art des Extremums von $N''(t)$ bei $t=30$ zu bestimmen, bilden wir $N^{(4)}(t)$:
    $N^{(4)}(t) = \frac{d}{dt} [e^{-0.1t}(-0.005t + 0.15)]$.
    Mit $u(t)=e^{-0.1t} \Rightarrow u'(t)=-0.1e^{-0.1t}$ und $v(t)=-0.005t+0.15 \Rightarrow v'(t)=-0.005$.
    $N^{(4)}(t) = -0.1e^{-0.1t}(-0.005t + 0.15) + e^{-0.1t}(-0.005)$
    $N^{(4)}(t) = e^{-0.1t}[-0.1(-0.005t + 0.15) - 0.005] = e^{-0.1t}[0.0005t - 0.015 - 0.005] = e^{-0.1t}(0.0005t - 0.02)$.
    $N^{(4)}(30) = e^{-3}(0.0005 \cdot 30 - 0.02) = e^{-3}(0.015 - 0.02) = e^{-3}(-0.005)$.
    Da $N^{(4)}(30) < 0$, hat $N''(t)$ bei $t=30$ ein \textbf{lokales Maximum}.
    $N''(30) = e^{-0.1 \cdot 30}(0.05 \cdot 30 - 1) = e^{-3}(1.5 - 1) = 0.5e^{-3} \approx 0.0249 > 0$.
    Die Zunahme der Wachstumsrate ist also bei $\mathbf{t=30}$ Stunden maximal (d.h. das Wachstum beschleunigt sich dort am stärksten positiv).
    Um die stärkste Abnahme der Wachstumsrate zu finden (wo $N''(t)$ am negativsten ist), betrachten wir die Ränder des sinnvollen Definitionsbereichs und andere kritische Punkte, falls vorhanden. $N''(t) = e^{-0.1t}(0.05t - 1)$.
    $N''(0) = e^0(0-1) = -1$.
    Die Nullstelle von $N''(t)$ ist bei $t=20$.
    Für $0 \le t < 20$: $0.05t-1 < 0 \Rightarrow N''(t) < 0$ (Wachstumsrate nimmt ab).
    Für $t > 20$: $0.05t-1 > 0 \Rightarrow N''(t) > 0$ (Wachstumsrate nimmt zu, bis $t=30$, dann wird Zunahme der Zunahme geringer).
    Das Minimum von $N''(t)$ im Intervall $t \ge 0$ tritt also bei $t=0$ auf.
    Die Abnahme der Wachstumsrate ist maximal (d.h. $N''(t)$ ist am negativsten) bei $\mathbf{t=0}$ Stunden, mit $N''(0) = -1$. (Hier verlangsamt sich das Wachstum am stärksten, bzw. die anfängliche 'Bremsung' des Wachstums ist am stärksten, da $N'(0)=5$ positiv ist, aber $N''(0)=-1$ negativ).
\end{enumerate}

\end{loesungsumgebung}



\begin{aufgabenumgebung}{Tangenten an Exponentialfunktionen}
Gegeben ist die Funktion $f(x) = (x-2)e^x$.
\begin{enumerate}
    \item Bestimme die Gleichung der Tangente an den Graphen von $f$ im Punkt $P(2|f(2))$.
    \item In welchem Punkt schneidet diese Tangente die y-Achse?
    \item Gibt es eine Stelle $x_0$, an der die Tangente an den Graphen von $f$ parallel zur x-Achse verläuft? Wenn ja, bestimme $x_0$ und die Art des Extrempunktes an dieser Stelle.
    \item (Für Experten): Gibt es eine Tangente an den Graphen von $f$, die durch den Ursprung $(0|0)$ verläuft, aber nicht im Ursprung berührt?

\end{enumerate}
\end{aufgabenumgebung}


\begin{loesungsumgebung}[loes:tangenten-ex-funktionen]{Tangenten an Exponentialfunktionen}
Gegeben ist die Funktion $f(x) = (x-2)e^x$.
Die erste Ableitung $f'(x)$ bestimmen wir mit der Produktregel:
Sei $u(x) = x-2 \Rightarrow u'(x) = 1$.
Sei $v(x) = e^x \Rightarrow v'(x) = e^x$.
$f'(x) = u'(x)v(x) + u(x)v'(x) = 1 \cdot e^x + (x-2)e^x = e^x(1 + x - 2) = (x-1)e^x$.

\begin{enumerate}[label=(\alph*)]
    \item \textbf{Gleichung der Tangente und der Normalen an den Graphen von $f$ im Punkt $P(2|f(2))$.}
    \begin{itemize}
        \item Zuerst bestimmen wir die y-Koordinate des Punktes $P$:
        $f(2) = (2-2)e^2 = 0 \cdot e^2 = 0$. Der Berührpunkt ist also $P(2|0)$.
        \item Die Steigung der Tangente $m_t$ im Punkt $P(2|0)$ ist $f'(2)$:
        $m_t = f'(2) = (2-1)e^2 = 1 \cdot e^2 = e^2$.
        \item Die Gleichung der Tangente $y_t = m_t(x-x_0) + y_0$ ist:
        $y_t = e^2(x-2) + 0 \Rightarrow \mathbf{y_t = e^2x - 2e^2}$.
        % Für die Normale (nicht explizit gefragt, aber oft Teil solcher Aufgaben):
        % Die Steigung der Normalen $m_n = -1/m_t = -1/e^2$.
        % Normalengleichung: $y_n - 0 = -\frac{1}{e^2}(x-2) \Rightarrow y_n = -\frac{1}{e^2}x + \frac{2}{e^2}$.
    \end{itemize}

    \item \textbf{In welchem Punkt schneidet diese Tangente die y-Achse?}
    Die Tangentengleichung ist $y_t = e^2x - 2e^2$.
    Um den Schnittpunkt mit der y-Achse zu finden, setzen wir $x=0$:
    $y_t(0) = e^2(0) - 2e^2 = -2e^2$.
    Der Schnittpunkt mit der y-Achse ist $\mathbf{P_y(0|-2e^2)}$.

    \item \textbf{Gibt es eine Stelle $x_0$, an der die Tangente an den Graphen von $f$ parallel zur x-Achse verläuft? Wenn ja, bestimme $x_0$ und die Art des Extrempunktes an dieser Stelle.}
    Eine Tangente ist parallel zur x-Achse, wenn ihre Steigung Null ist, d.h. $f'(x_0)=0$.
    $f'(x_0) = (x_0-1)e^{x_0} = 0$.
    Da $e^{x_0}$ immer positiv ist ($e^{x_0} > 0$), muss gelten:
    $x_0-1 = 0 \Rightarrow \mathbf{x_0 = 1}$.
    Um die Art des Extrempunktes zu bestimmen, bilden wir die zweite Ableitung $f''(x)$:
    $f'(x) = (x-1)e^x$. Mit Produktregel ($u=x-1, u'=1; v=e^x, v'=e^x$):
    $f''(x) = 1 \cdot e^x + (x-1)e^x = e^x(1+x-1) = xe^x$.
    An der Stelle $x_0=1$:
    $f''(1) = 1 \cdot e^1 = e$.
    Da $f''(1) = e > 0$, liegt bei $x_0=1$ ein \textbf{lokaler Tiefpunkt} vor.
    Die y-Koordinate ist $f(1) = (1-2)e^1 = -e$.
    Der Tiefpunkt ist $TP(1|-e)$.

    \item \textbf{(Für Experten): Gibt es eine Tangente an den Graphen von $f$, die durch den Ursprung $(0|0)$ verläuft, aber nicht im Ursprung berührt?}
    Der Punkt $(0|0)$ liegt nicht auf dem Graphen von $f$, da $f(0) = (0-2)e^0 = -2 \neq 0$. Also suchen wir eine Tangente von einem externen Punkt $(0|0)$ an den Graphen.
    Sei $B(x_B | f(x_B))$ der Berührpunkt der Tangente am Graphen. Die Steigung der Tangente in $B$ ist $m_T = f'(x_B) = (x_B-1)e^{x_B}$.
    Die Steigung der Geraden durch den Ursprung $O(0|0)$ und den Berührpunkt $B(x_B | f(x_B))$ ist (für $x_B \neq 0$):
    $m_{OB} = \frac{f(x_B) - 0}{x_B - 0} = \frac{(x_B-2)e^{x_B}}{x_B}$.
    Für eine Tangente müssen diese Steigungen gleich sein: $m_T = m_{OB}$.
    $$ (x_B-1)e^{x_B} = \frac{(x_B-2)e^{x_B}}{x_B} $$
    Da $e^{x_B} \neq 0$, können wir dadurch teilen:
    $$ x_B-1 = \frac{x_B-2}{x_B} $$
    Multiplikation mit $x_B$ (wir suchen einen Berührpunkt $x_B \neq 0$, da sonst der Nenner Null wäre und der Ursprung kein Berührpunkt ist, da $f(0) \neq 0$):
    $$ x_B(x_B-1) = x_B-2 $$
    $$ x_B^2 - x_B = x_B - 2 $$
    $$ x_B^2 - 2x_B + 2 = 0 $$
    Wir untersuchen die Diskriminante dieser quadratischen Gleichung ($a=1, b=-2, c=2$):
    $D = b^2 - 4ac = (-2)^2 - 4(1)(2) = 4 - 8 = -4$.
    Da die Diskriminante $D < 0$ ist, hat die quadratische Gleichung keine reellen Lösungen für $x_B$.
    \textbf{Schlussfolgerung:} Es gibt keine solche Stelle $x_B$ und somit keine Tangente an den Graphen von $f(x)$, die durch den Ursprung $(0|0)$ verläuft (und nicht im Ursprung berührt, was ohnehin nicht der Fall ist, da $f(0)\neq0$).
\end{enumerate}

\end{loesungsumgebung}

\begin{aufgabenumgebung}{Kurvendiskussion einer komplexeren e-Funktion}
Führe eine vollständige Kurvendiskussion für die Funktion $f(x) = (x^2 - 3)e^{-x}$ durch. Untersuche dabei insbesondere:
\begin{itemize}
    \item Definitionsbereich, Symmetrie
    \item Verhalten im Unendlichen (Grenzwerte, Asymptoten)
    \item Nullstellen
    \item Extrempunkte (Lage und Art)
    \item Wendepunkte (Lage)
    \item Skizziere den Graphen von $f(x)$.
\end{itemize}
\end{aufgabenumgebung}

\begin{loesungsumgebung}[loes:kurvendiskussion-komplex-ex]{Kurvendiskussion einer komplexeren e-Funktion}
Wir führen eine vollständige Kurvendiskussion für die Funktion $f(x) = (x^2 - 3)e^{-x}$ durch.

\subsubsection*{1. Definitionsbereich ($D_f$)}
Die Terme $x^2-3$ und $e^{-x}$ sind für alle reellen Zahlen definiert.
Somit ist der Definitionsbereich $\mathbf{D_f = \mathbb{R}}$.

\subsubsection*{2. Symmetrie}
Wir untersuchen $f(-x)$:
$f(-x) = ((-x)^2 - 3)e^{-(-x)} = (x^2 - 3)e^x$.
Da $f(-x) \neq f(x)$ und $f(-x) \neq -f(x)$ (z.B. $f(1) = (1-3)e^{-1} = -2e^{-1}$, während $f(-1) = (1-3)e^1 = -2e$, und $-f(1) = 2e^{-1}$), liegt \textbf{keine einfache Achsen- oder Punktsymmetrie} zum Ursprung vor.

\subsubsection*{3. Verhalten im Unendlichen (Grenzwerte, Asymptoten)}
\begin{itemize}
    \item Für $x \to \infty$:
    $f(x) = \frac{x^2 - 3}{e^x}$. Da die Exponentialfunktion $e^x$ schneller wächst als jede Potenz von $x$, gilt:
    $$ \lim_{x \to \infty} \frac{x^2 - 3}{e^x} = 0 $$
    (Dies kann auch zweimal mit der Regel von L'Hôpital gezeigt werden: $\lim_{x \to \infty} \frac{2x}{e^x} = \lim_{x \to \infty} \frac{2}{e^x} = 0$).
    Somit hat die Funktion eine \textbf{horizontale Asymptote $y=0$} für $x \to \infty$.
    \item Für $x \to -\infty$:
    Sei $u = -x$. Wenn $x \to -\infty$, dann $u \to \infty$.
    $f(x) = ((-u)^2 - 3)e^{-(-u)} = (u^2 - 3)e^u$.
    $$ \lim_{u \to \infty} (u^2 - 3)e^u = (\infty - 3) \cdot \infty = \infty \cdot \infty = \infty $$
    Also $\lim_{x \to -\infty} f(x) = \mathbf{\infty}$.
\end{itemize}

\subsubsection*{4. Achsenschnittpunkte}
\begin{itemize}
    \item \textbf{Schnittpunkt mit der y-Achse} (setze $x=0$):
    $f(0) = (0^2 - 3)e^0 = (-3) \cdot 1 = -3$.
    Der y-Achsenabschnitt ist $P_y(0|-3)$.
    \item \textbf{Nullstellen} (setze $f(x)=0$):
    $(x^2 - 3)e^{-x} = 0$.
    Da $e^{-x} > 0$ für alle $x \in \mathbb{R}$, muss $x^2 - 3 = 0$ sein.
    $x^2 = 3 \Rightarrow x = \pm\sqrt{3}$.
    Die Nullstellen sind $\mathbf{x_{N1} = -\sqrt{3} \approx -1.732}$ und $\mathbf{x_{N2} = \sqrt{3} \approx 1.732}$.
\end{itemize}

\subsubsection*{5. Erste Ableitung $f'(x)$}
Wir verwenden die Produktregel ($u=x^2-3, v=e^{-x}$):
$u'(x) = 2x$.
$v'(x) = -e^{-x}$.
\begin{align*}
f'(x) &= u'(x)v(x) + u(x)v'(x) \\
      &= 2x \cdot e^{-x} + (x^2-3) \cdot (-e^{-x}) \\
      &= e^{-x} [2x - (x^2-3)] \\
      &= e^{-x} (2x - x^2 + 3) \\
      &= \mathbf{(-x^2 + 2x + 3)e^{-x}}
\end{align*}

\subsubsection*{6. Extrempunkte (Lage und Art)}
Notwendige Bedingung: $f'(x_E)=0$.
$(-x^2 + 2x + 3)e^{-x} = 0$.
Da $e^{-x} \neq 0$, muss $-x^2 + 2x + 3 = 0 \Rightarrow x^2 - 2x - 3 = 0$.
Diese quadratische Gleichung lässt sich faktorisieren: $(x-3)(x+1)=0$.
Die kritischen Stellen sind $x_{E1} = -1$ und $x_{E2} = 3$.
Hinreichende Bedingung: Wir untersuchen das Vorzeichen von $f'(x)$. Der Faktor $e^{-x}$ ist immer positiv. Das Vorzeichen von $f'(x)$ wird also durch $P(x) = -x^2+2x+3$ bestimmt. Dies ist eine nach unten geöffnete Parabel mit Nullstellen bei $-1$ und $3$.
\begin{itemize}
    \item Für $x < -1$ (z.B. $x=-2$): $P(-2) = -4-4+3 = -5 < 0 \Rightarrow f'(x) < 0$.
    \item Für $-1 < x < 3$ (z.B. $x=0$): $P(0) = 3 > 0 \Rightarrow f'(x) > 0$.
    \item Für $x > 3$ (z.B. $x=4$): $P(4) = -16+8+3 = -5 < 0 \Rightarrow f'(x) < 0$.
\end{itemize}
Daraus folgt:
\begin{itemize}
    \item Bei $x_{E1} = -1$: Vorzeichenwechsel von $f'(x)$ von $-$ nach $+ \Rightarrow$ \textbf{Lokaler Tiefpunkt}.
    $y_{E1} = f(-1) = ((-1)^2-3)e^{-(-1)} = (1-3)e^1 = -2e \approx -5.437$.
    $\mathbf{TP(-1|-2e)}$.
    \item Bei $x_{E2} = 3$: Vorzeichenwechsel von $f'(x)$ von $+$ nach $- \Rightarrow$ \textbf{Lokaler Hochpunkt}.
    $y_{E2} = f(3) = ((3)^2-3)e^{-3} = (9-3)e^{-3} = 6e^{-3} = \frac{6}{e^3} \approx 0.299$.
    $\mathbf{HP(3|6e^{-3})}$.
\end{itemize}

\subsubsection*{7. Monotonieverhalten}
Aus der Untersuchung von $f'(x)$:
\begin{itemize}
    \item Streng monoton fallend in $(-\infty, -1]$.
    \item Streng monoton steigend in $[-1, 3]$.
    \item Streng monoton fallend in $[3, \infty)$.
\end{itemize}

\subsubsection*{8. Zweite Ableitung $f''(x)$}
Wir leiten $f'(x) = (-x^2 + 2x + 3)e^{-x}$ mit der Produktregel ab:
$u(x) = -x^2+2x+3 \Rightarrow u'(x) = -2x+2$.
$v(x) = e^{-x} \Rightarrow v'(x) = -e^{-x}$.
\begin{align*}
f''(x) &= (-2x+2)e^{-x} + (-x^2+2x+3)(-e^{-x}) \\
       &= e^{-x} [(-2x+2) - (-x^2+2x+3)] \\
       &= e^{-x} [-2x+2 + x^2-2x-3] \\
       &= \mathbf{(x^2 - 4x - 1)e^{-x}}
\end{align*}

\subsubsection*{9. Wendepunkte (Lage)}
Notwendige Bedingung: $f''(x_W)=0$.
$(x^2 - 4x - 1)e^{-x} = 0$.
Da $e^{-x} \neq 0$, muss $x^2 - 4x - 1 = 0$.
Mit der Mitternachtsformel: $x_W = \frac{-(-4) \pm \sqrt{(-4)^2 - 4(1)(-1)}}{2(1)} = \frac{4 \pm \sqrt{16+4}}{2} = \frac{4 \pm \sqrt{20}}{2}$.
$\sqrt{20} = \sqrt{4 \cdot 5} = 2\sqrt{5}$.
$x_W = \frac{4 \pm 2\sqrt{5}}{2} = 2 \pm \sqrt{5}$.
Die potentiellen Wendestellen sind $x_{W1} = 2 - \sqrt{5} \approx -0.236$ und $x_{W2} = 2 + \sqrt{5} \approx 4.236$.
Hinreichende Bedingung: Wir prüfen, ob $f'''(x_W) \neq 0$ oder ob $f''(x)$ das Vorzeichen wechselt. Da $x^2-4x-1$ eine Parabel ist, die ihre Nullstellen schneidet, wechselt $f''(x)$ das Vorzeichen an diesen Stellen. Somit liegen Wendepunkte vor.
y-Koordinaten:
$y_{W1} = f(2-\sqrt{5}) = ((2-\sqrt{5})^2-3)e^{-(2-\sqrt{5})} = (4-4\sqrt{5}+5-3)e^{\sqrt{5}-2} = (6-4\sqrt{5})e^{\sqrt{5}-2} \approx -3.727$.
$y_{W2} = f(2+\sqrt{5}) = ((2+\sqrt{5})^2-3)e^{-(2+\sqrt{5})} = (4+4\sqrt{5}+5-3)e^{-2-\sqrt{5}} = (6+4\sqrt{5})e^{-2-\sqrt{5}} \approx 0.216$.
Wendepunkte: $\mathbf{WP_1(2-\sqrt{5} | (6-4\sqrt{5})e^{\sqrt{5}-2})}$ und $\mathbf{WP_2(2+\sqrt{5} | (6+4\sqrt{5})e^{-2-\sqrt{5}})}$.

\subsubsection*{10. Krümmungsverhalten}
Das Vorzeichen von $f''(x) = (x^2 - 4x - 1)e^{-x}$ wird durch den quadratischen Term $P(x)=x^2-4x-1$ bestimmt (da $e^{-x}>0$). $P(x)$ ist eine nach oben geöffnete Parabel mit Nullstellen $x_{W1}$ und $x_{W2}$.
\begin{itemize}
    \item Für $x < 2-\sqrt{5}$: $P(x) > 0 \Rightarrow f''(x) > 0 \Rightarrow f(x)$ ist linksgekrümmt (konvex).
    \item Für $2-\sqrt{5} < x < 2+\sqrt{5}$: $P(x) < 0 \Rightarrow f''(x) < 0 \Rightarrow f(x)$ ist rechtsgekrümmt (konkav).
    \item Für $x > 2+\sqrt{5}$: $P(x) > 0 \Rightarrow f''(x) > 0 \Rightarrow f(x)$ ist linksgekrümmt (konvex).
\end{itemize}

\subsubsection*{11. Wertebereich ($W_f$)}
Der globale Tiefpunkt ist $TP(-1|-2e \approx -5.437)$. Der Graph geht für $x \to -\infty$ nach $\infty$. Der Hochpunkt ist $HP(3|6e^{-3} \approx 0.299)$ und für $x \to \infty$ geht $f(x) \to 0$.
Somit ist der Wertebereich $\mathbf{W_f = [-2e, \infty)}$.

\subsubsection*{12. Skizze des Graphen}
Der Graph startet im Unendlichen für $x \to -\infty$, fällt zu einem Tiefpunkt bei $TP(-1|-2e)$, steigt dann an, durchläuft einen Wendepunkt bei $WP_1(2-\sqrt{5}| \dots)$, schneidet die y-Achse bei $(0|-3)$, steigt weiter zu einem Hochpunkt bei $HP(3|6e^{-3})$, fällt dann, durchläuft einen weiteren Wendepunkt bei $WP_2(2+\sqrt{5}| \dots)$ und nähert sich der x-Achse ($y=0$) als Asymptote für $x \to \infty$. Die Nullstellen sind bei $x=\pm\sqrt{3}$.
\begin{center}
\includegraphics[width=0.8\textwidth]{grafiken/kurvendisk_komplex_ex.png}
% --- Beschreibung der Skizze ---
% Die Skizze zeigt einen Graphen, der von links oben aus dem Unendlichen kommt.
% Er schneidet die x-Achse bei x = -sqrt(3) (ca. -1.73).
% Erreicht einen Tiefpunkt bei TP(-1 | -2e) (ca. -1 | -5.44).
% Steigt an, durchläuft einen Wendepunkt bei WP1(2-sqrt(5) | ...) (ca. -0.24 | -3.73).
% Schneidet die y-Achse bei (0 | -3).
% Steigt weiter zu einem Hochpunkt bei HP(3 | 6e^-3) (ca. 3 | 0.30).
% Fällt dann ab, schneidet die x-Achse bei x = sqrt(3) (ca. 1.73).
% Durchläuft einen Wendepunkt bei WP2(2+sqrt(5) | ...) (ca. 4.24 | 0.22).
% Nähert sich der x-Achse (y=0) asymptotisch für x -> +unendlich.
\captionof{figure}{Graph der Funktion $f(x) = (x^2 - 3)e^{-x}$.}
\label{fig:kurvendisk_komplex_ex}
\end{center}

\end{loesungsumgebung}

\begin{aufgabenumgebung}{Partielle Integration üben – Mehr Vielfalt}
Berechne die folgenden unbestimmten Integrale mit partieller Integration:
\begin{enumerate}
    \item $\int (2x+1)e^x \,dx$
    \item $\int x^2 e^x \,dx$ (Tipp: Hier musst du die partielle Integration zweimal anwenden!)
    \item $\int \ln(x) \,dx$
    \item $\int x \cos(x) \,dx$
    \item $\int (x^2+1)\ln(x) \,dx$
    \item \textbf{Herausforderung (das 'Trick-Integral'):} $\int e^x \sin(x) \,dx$

\end{enumerate}
\end{aufgabenumgebung}

\begin{loesungsumgebung}[loes:partielle-integration-ueben-vielfalt]{Partielle Integration üben – Mehr Vielfalt}
Wir berechnen die unbestimmten Integrale mithilfe der partiellen Integration.

\begin{enumerate}[label=(\alph*)]
    \item $\mathbf{\int (2x+1)e^x \,dx}$ \\
    Wir wählen:
    \begin{itemize}
        \item $f(x) = 2x+1 \Rightarrow f'(x) = 2$ (wird einfacher beim Ableiten)
        \item $g'(x) = e^x \Rightarrow g(x) = e^x$ (leicht zu integrieren)
    \end{itemize}
    Anwendung der Formel $\int f(x)g'(x) \,dx = f(x)g(x) - \int f'(x)g(x) \,dx$:
    \begin{align*} \int (2x+1)e^x \,dx &= (2x+1)e^x - \int 2e^x \,dx \\ &= (2x+1)e^x - 2\int e^x \,dx \\ &= (2x+1)e^x - 2e^x + C \\ &= e^x(2x+1-2) + C \\ &= \mathbf{e^x(2x-1) + C} \end{align*}

    \item $\mathbf{\int x^2 e^x \,dx}$ (Tipp: Zweimal anwenden) \\
    \textbf{Erste partielle Integration:}
    Wir wählen:
    \begin{itemize}
        \item $f(x) = x^2 \Rightarrow f'(x) = 2x$
        \item $g'(x) = e^x \Rightarrow g(x) = e^x$
    \end{itemize}
    $$ \int x^2 e^x \,dx = x^2e^x - \int 2x e^x \,dx = x^2e^x - 2 \int x e^x \,dx \quad (*) $$
    \textbf{Zweite partielle Integration für $\int x e^x \,dx$:}
    Wir wählen für dieses Integral:
    \begin{itemize}
        \item $f_2(x) = x \Rightarrow f_2'(x) = 1$
        \item $g_2'(x) = e^x \Rightarrow g_2(x) = e^x$
    \end{itemize}
    $$ \int x e^x \,dx = xe^x - \int 1 \cdot e^x \,dx = xe^x - e^x $$
    Setzen wir dies zurück in $(*)$:
    \begin{align*} \int x^2 e^x \,dx &= x^2e^x - 2(xe^x - e^x) + C \\ &= x^2e^x - 2xe^x + 2e^x + C \\ &= \mathbf{e^x(x^2 - 2x + 2) + C} \end{align*}

    \item $\mathbf{\int \ln(x) \,dx}$ \\
    Tipp: Wähle $f(x)=\ln(x)$ und $g'(x)=1$. Die Ableitung von $\ln(x)$ ist $(\ln x)' = \frac{1}{x}$.
    Wir wählen:
    \begin{itemize}
        \item $f(x) = \ln(x) \Rightarrow f'(x) = \frac{1}{x}$
        \item $g'(x) = 1 \Rightarrow g(x) = x$
    \end{itemize}
    $$ \int \ln(x) \cdot 1 \,dx = \ln(x) \cdot x - \int \frac{1}{x} \cdot x \,dx $$
    $$ = x\ln(x) - \int 1 \,dx $$
    $$ = \mathbf{x\ln(x) - x + C} $$

    \item $\mathbf{\int x \cos(x) \,dx}$ \\
    Tipp: $(\sin x)' = \cos x$, $(\cos x)' = -\sin x$, $\int \cos x \,dx = \sin x + C$. Wähle $f(x)=x$ und $g'(x)=\cos x$.
    Wir wählen:
    \begin{itemize}
        \item $f(x) = x \Rightarrow f'(x) = 1$
        \item $g'(x) = \cos(x) \Rightarrow g(x) = \sin(x)$
    \end{itemize}
    \begin{align*} \int x \cos(x) \,dx &= x\sin(x) - \int 1 \cdot \sin(x) \,dx \\ &= x\sin(x) - \int \sin(x) \,dx \\ &= x\sin(x) - (-\cos(x)) + C \\ &= \mathbf{x\sin(x) + \cos(x) + C} \end{align*}

    \item $\mathbf{\int (x^2+1)\ln(x) \,dx}$ \\
    Tipp: Wähle $f(x)=\ln(x)$ und $g'(x)=x^2+1$.
    Wir wählen:
    \begin{itemize}
        \item $f(x) = \ln(x) \Rightarrow f'(x) = \frac{1}{x}$
        \item $g'(x) = x^2+1 \Rightarrow g(x) = \int (x^2+1) \,dx = \frac{x^3}{3} + x$
    \end{itemize}
    \begin{align*} \int (x^2+1)\ln(x) \,dx &= \ln(x) \left(\frac{x^3}{3} + x\right) - \int \frac{1}{x} \left(\frac{x^3}{3} + x\right) \,dx \\ &= \left(\frac{x^3}{3} + x\right)\ln(x) - \int \left(\frac{x^2}{3} + 1\right) \,dx \\ &= \left(\frac{x^3}{3} + x\right)\ln(x) - \left(\frac{1}{3}\frac{x^3}{3} + x\right) + C \\ &= \mathbf{\left(\frac{x^3}{3} + x\right)\ln(x) - \frac{x^3}{9} - x + C} \end{align*}

    \item \textbf{Herausforderung: $\int e^x \sin(x) \,dx$} \\
    Sei $I = \int e^x \sin(x) \,dx$.
    \textbf{Erste partielle Integration:}
    Wähle $f(x) = \sin(x) \Rightarrow f'(x) = \cos(x)$.
    Wähle $g'(x) = e^x \Rightarrow g(x) = e^x$.
    $$ I = e^x \sin(x) - \int e^x \cos(x) \,dx \quad (1) $$
    \textbf{Zweite partielle Integration für $\int e^x \cos(x) \,dx$:}
    Sei $J = \int e^x \cos(x) \,dx$.
    Wähle (konsistent zum ersten Mal) $u_2(x) = \cos(x) \Rightarrow u_2'(x) = -\sin(x)$.
    Wähle $v_2'(x) = e^x \Rightarrow v_2(x) = e^x$.
    $$ J = e^x \cos(x) - \int e^x (-\sin(x)) \,dx = e^x \cos(x) + \int e^x \sin(x) \,dx = e^x \cos(x) + I $$
    Setze dies für $J$ in Gleichung (1) ein:
    $$ I = e^x \sin(x) - (e^x \cos(x) + I) $$
    $$ I = e^x \sin(x) - e^x \cos(x) - I $$
    Jetzt stellen wir nach $I$ um:
    $$ 2I = e^x \sin(x) - e^x \cos(x) $$
    $$ 2I = e^x (\sin(x) - \cos(x)) $$
    $$ I = \frac{1}{2}e^x (\sin(x) - \cos(x)) $$
    Vergiss die Integrationskonstante nicht:
    $$ \mathbf{\int e^x \sin(x) \,dx = \frac{1}{2}e^x (\sin(x) - \cos(x)) + C} $$
\end{enumerate}

\end{loesungsumgebung}


\begin{aufgabenumgebung}{Integration durch Substitution – Vielfältige Übungen}
Berechne die folgenden unbestimmten Integrale mit der Substitutionsmethode:
\begin{enumerate}
    \item $\int (2x+1)^4 \,dx$ (Tipp: $u=2x+1$)
    \item $\int x \cdot e^{x^2} \,dx$ (Tipp: $u=x^2$. Was ist $du$? Du musst eventuell einen Faktor anpassen.)
    \item $\int \frac{1}{(3x-5)^2} \,dx$ (Tipp: Schreibe als $(3x-5)^{-2}$ und substituiere $u=3x-5$)
    \item $\int \cos(2x) \,dx$ (Stammfunktion von $\cos(u)$ ist $\sin(u)$. Substituiere $u=2x$.)
    \item $\int 3x^2 \cdot (x^3+7)^5 \,dx$ 

    \item $\int \sqrt{4x-3} \,dx$ 

    \item $\int \frac{5x}{(x^2-1)^3} \,dx$

    \item $\int (x+2) \cdot e^{x^2+4x-1} \,dx$

    \item \textbf{Herausforderung:} $\int \frac{x^3}{\sqrt{1+x^4}} \,dx$

\end{enumerate}
\end{aufgabenumgebung}

\begin{loesungsumgebung}[loes:integration-substitution-uebungen]{Integration durch Substitution – Vielfältige Übungen}
Wir berechnen die unbestimmten Integrale mithilfe der Substitutionsmethode.

\begin{enumerate}[label=(\alph*)]
    \item $\mathbf{\int (2x+1)^4 \,dx}$ \\
    Substitution: $u = 2x+1$. \\
    Ableitung der inneren Funktion: $\frac{du}{dx} = 2 \Rightarrow du = 2 \,dx \Rightarrow dx = \frac{1}{2} \,du$.
    \begin{align*} \int (2x+1)^4 \,dx &= \int u^4 \cdot \frac{1}{2} \,du \\ &= \frac{1}{2} \int u^4 \,du \\ &= \frac{1}{2} \cdot \frac{u^5}{5} + C \\ &= \frac{1}{10}u^5 + C \end{align*}
    Rücksubstitution $u=2x+1$:
    $$ \mathbf{\frac{1}{10}(2x+1)^5 + C} $$

    \item $\mathbf{\int x \cdot e^{x^2} \,dx}$ \\
    Substitution: $u = x^2$. \\
    Ableitung der inneren Funktion: $\frac{du}{dx} = 2x \Rightarrow du = 2x \,dx \Rightarrow x \,dx = \frac{1}{2} \,du$.
    \begin{align*} \int x e^{x^2} \,dx &= \int e^{x^2} (x \,dx) \\ &= \int e^u \cdot \frac{1}{2} \,du \\ &= \frac{1}{2} \int e^u \,du \\ &= \frac{1}{2} e^u + C \end{align*}
    Rücksubstitution $u=x^2$:
    $$ \mathbf{\frac{1}{2}e^{x^2} + C} $$

    \item $\mathbf{\int \frac{1}{(3x-5)^2} \,dx}$ \\
    Schreibe als $\int (3x-5)^{-2} \,dx$.
    Substitution: $u = 3x-5$. \\
    Ableitung der inneren Funktion: $\frac{du}{dx} = 3 \Rightarrow du = 3 \,dx \Rightarrow dx = \frac{1}{3} \,du$.
    \begin{align*} \int (3x-5)^{-2} \,dx &= \int u^{-2} \cdot \frac{1}{3} \,du \\ &= \frac{1}{3} \int u^{-2} \,du \\ &= \frac{1}{3} \cdot \frac{u^{-1}}{-1} + C \\ &= -\frac{1}{3}u^{-1} + C = -\frac{1}{3u} + C \end{align*}
    Rücksubstitution $u=3x-5$:
    $$ \mathbf{-\frac{1}{3(3x-5)} + C} $$

    \item $\mathbf{\int \cos(2x) \,dx}$ \\
    Substitution: $u = 2x$. \\
    Ableitung der inneren Funktion: $\frac{du}{dx} = 2 \Rightarrow du = 2 \,dx \Rightarrow dx = \frac{1}{2} \,du$.
    (Stammfunktion von $\cos(u)$ ist $\sin(u)$).
    \begin{align*} \int \cos(2x) \,dx &= \int \cos(u) \cdot \frac{1}{2} \,du \\ &= \frac{1}{2} \int \cos(u) \,du \\ &= \frac{1}{2} \sin(u) + C \end{align*}
    Rücksubstitution $u=2x$:
    $$ \mathbf{\frac{1}{2}\sin(2x) + C} $$

    \item $\mathbf{\int 3x^2 \cdot (x^3+7)^5 \,dx}$ \\
    Substitution: $u = x^3+7$. \\
    Ableitung der inneren Funktion: $\frac{du}{dx} = 3x^2 \Rightarrow du = 3x^2 \,dx$.
    Der Term $3x^2 \,dx$ ist im Integranden vorhanden.
    \begin{align*} \int (x^3+7)^5 \cdot (3x^2 \,dx) &= \int u^5 \,du \\ &= \frac{u^6}{6} + C \end{align*}
    Rücksubstitution $u=x^3+7$:
    $$ \mathbf{\frac{1}{6}(x^3+7)^6 + C} $$

    \item $\mathbf{\int \sqrt{4x-3} \,dx}$ \\
    Schreibe als $\int (4x-3)^{1/2} \,dx$.
    Substitution: $u = 4x-3$. \\
    Ableitung der inneren Funktion: $\frac{du}{dx} = 4 \Rightarrow du = 4 \,dx \Rightarrow dx = \frac{1}{4} \,du$.
    \begin{align*} \int (4x-3)^{1/2} \,dx &= \int u^{1/2} \cdot \frac{1}{4} \,du \\ &= \frac{1}{4} \int u^{1/2} \,du \\ &= \frac{1}{4} \cdot \frac{u^{\frac{1}{2}+1}}{\frac{1}{2}+1} + C \\ &= \frac{1}{4} \cdot \frac{u^{3/2}}{3/2} + C = \frac{1}{4} \cdot \frac{2}{3} u^{3/2} + C = \frac{1}{6}u^{3/2} + C \end{align*}
    Rücksubstitution $u=4x-3$:
    $$ \mathbf{\frac{1}{6}(4x-3)^{3/2} + C} \quad \text{oder} \quad \mathbf{\frac{1}{6}\sqrt{(4x-3)^3} + C} $$

    \item $\mathbf{\int \frac{5x}{(x^2-1)^3} \,dx}$ \\
    Schreibe als $\int 5x(x^2-1)^{-3} \,dx = 5 \int x(x^2-1)^{-3} \,dx$.
    Substitution: $u = x^2-1$. \\
    Ableitung der inneren Funktion: $\frac{du}{dx} = 2x \Rightarrow du = 2x \,dx \Rightarrow x \,dx = \frac{1}{2} \,du$.
    \begin{align*} 5 \int (x^2-1)^{-3} (x \,dx) &= 5 \int u^{-3} \cdot \frac{1}{2} \,du \\ &= \frac{5}{2} \int u^{-3} \,du \\ &= \frac{5}{2} \cdot \frac{u^{-3+1}}{-3+1} + C \\ &= \frac{5}{2} \cdot \frac{u^{-2}}{-2} + C = -\frac{5}{4}u^{-2} + C \end{align*}
    Rücksubstitution $u=x^2-1$:
    $$ \mathbf{-\frac{5}{4}(x^2-1)^{-2} + C} \quad \text{oder} \quad \mathbf{-\frac{5}{4(x^2-1)^2} + C} $$

    \item $\mathbf{\int (x+2) \cdot e^{x^2+4x-1} \,dx}$ \\
    Substitution: $u = x^2+4x-1$. \\
    Ableitung der inneren Funktion: $\frac{du}{dx} = 2x+4 = 2(x+2) \Rightarrow du = 2(x+2) \,dx \Rightarrow (x+2) \,dx = \frac{1}{2} \,du$.
    \begin{align*} \int e^{x^2+4x-1} \cdot ((x+2) \,dx) &= \int e^u \cdot \frac{1}{2} \,du \\ &= \frac{1}{2} \int e^u \,du \\ &= \frac{1}{2}e^u + C \end{align*}
    Rücksubstitution $u=x^2+4x-1$:
    $$ \mathbf{\frac{1}{2}e^{x^2+4x-1} + C} $$

    \item \textbf{Herausforderung: $\int \frac{x^3}{\sqrt{1+x^4}} \,dx$} \\
    Schreibe als $\int x^3 (1+x^4)^{-1/2} \,dx$.
    Substitution: $u = 1+x^4$. \\
    Ableitung der inneren Funktion: $\frac{du}{dx} = 4x^3 \Rightarrow du = 4x^3 \,dx \Rightarrow x^3 \,dx = \frac{1}{4} \,du$.
    \begin{align*} \int (1+x^4)^{-1/2} (x^3 \,dx) &= \int u^{-1/2} \cdot \frac{1}{4} \,du \\ &= \frac{1}{4} \int u^{-1/2} \,du \\ &= \frac{1}{4} \cdot \frac{u^{-\frac{1}{2}+1}}{-\frac{1}{2}+1} + C \\ &= \frac{1}{4} \cdot \frac{u^{1/2}}{1/2} + C = \frac{1}{4} \cdot 2u^{1/2} + C = \frac{1}{2}u^{1/2} + C \end{align*}
    Rücksubstitution $u=1+x^4$:
    $$ \mathbf{\frac{1}{2}\sqrt{1+x^4} + C} $$
\end{enumerate}

\end{loesungsumgebung}

\begin{aufgabenumgebung}{Anwendungsaufgabe: Fläche unter $xe^{-x}$}
Die Funktion $f(x) = xe^{-x}$ spielt in einigen Anwendungsbereichen eine Rolle.
\begin{enumerate}
    \item Bestimme die Stammfunktion $F(x)$ von $f(x)$ mithilfe partieller Integration.
    \item Berechne den Inhalt der Fläche, die der Graph von $f(x)$ mit der x-Achse im Intervall $[0, 2]$ einschließt. (Hinweis: Untersuche, ob $f(x)$ im Intervall positiv ist. $e^{-x}$ ist immer positiv).
    \item (Für Experten): Untersuche das Verhalten von $f(x)$ für $x \to \infty$. (Tipp: $\lim_{x \to \infty} xe^{-x} = \lim_{x \to \infty} \frac{x}{e^x} = 0$). Was bedeutet das für die Fläche unter dem Graphen von $x=0$ bis 'unendlich'? Solche Integrale nennt man \textit{uneigentliche Integrale}.
\end{enumerate}
\end{aufgabenumgebung}

\begin{loesungsumgebung}[loes:anwendung-flaeche-xe-hoch-minus-x]{Anwendungsaufgabe: Fläche unter $xe^{-x}$}
Gegeben ist die Funktion $f(x) = xe^{-x}$.

\begin{enumerate}[label=(\alph*)]
    \item \textbf{Bestimme die Stammfunktion $F(x)$ von $f(x)$ mithilfe partieller Integration.} \\
    Wir verwenden die Formel für partielle Integration: $\int u(x)v'(x) \,dx = u(x)v(x) - \int u'(x)v(x) \,dx$.
    Wähle:
    \begin{itemize}
        \item $u(x) = x \Rightarrow u'(x) = 1$.
        \item $v'(x) = e^{-x} \Rightarrow v(x) = \int e^{-x} \,dx = -e^{-x}$.
    \end{itemize}
    Dann ist die Stammfunktion $F(x)$:
    \begin{align*} F(x) &= \int xe^{-x} \,dx \\ &= x(-e^{-x}) - \int 1 \cdot (-e^{-x}) \,dx \\ &= -xe^{-x} - \int -e^{-x} \,dx \\ &= -xe^{-x} + \int e^{-x} \,dx \\ &= -xe^{-x} + (-e^{-x}) + C \\ &= -xe^{-x} - e^{-x} + C \\ &= \mathbf{-(x+1)e^{-x} + C} \end{align*}

    \item \textbf{Berechne den Inhalt der Fläche, die der Graph von $f(x)$ mit der x-Achse im Intervall $[0, 2]$ einschließt.} \\
    Zuerst prüfen wir das Vorzeichen von $f(x) = xe^{-x}$ im Intervall $[0,2]$.
    Der Faktor $e^{-x}$ ist immer positiv ($e^{-x} > 0$).
    Für $x \in [0,2]$ gilt:
    \begin{itemize}
        \item Wenn $x=0$, ist $f(0)=0 \cdot e^0 = 0$.
        \item Wenn $x \in (0,2]$, ist $x > 0$, also ist $f(x) = x e^{-x} > 0$.
    \end{itemize}
    Da $f(x) \ge 0$ im Intervall $[0,2]$, entspricht der Flächeninhalt $A$ dem bestimmten Integral:
    $$ A = \int_0^2 xe^{-x} \,dx $$
    Mit der Stammfunktion $F(x) = -(x+1)e^{-x}$ (wir können $C=0$ wählen):
    \begin{align*} A &= [-(x+1)e^{-x}]_0^2 \\ &= (-(2+1)e^{-2}) - (-(0+1)e^0) \\ &= (-3e^{-2}) - (-1 \cdot 1) \\ &= -3e^{-2} - (-1) \\ &= 1 - 3e^{-2} = \mathbf{1 - \frac{3}{e^2}} \end{align*}
    Numerischer Wert: $A \approx 1 - \frac{3}{(2.71828)^2} \approx 1 - \frac{3}{7.38906} \approx 1 - 0.40601 \approx 0.59399$.
    Der Flächeninhalt beträgt $1 - \frac{3}{e^2} \approx 0.594$ Flächeneinheiten.

    \item \textbf{(Für Experten): Verhalten von $f(x)$ für $x \to \infty$ und Bedeutung für die Fläche von $x=0$ bis 'unendlich'.}
    \begin{itemize}
        \item \textbf{Verhalten für $x \to \infty$:}
        Wir untersuchen $\lim_{x \to \infty} f(x) = \lim_{x \to \infty} xe^{-x} = \lim_{x \to \infty} \frac{x}{e^x}$.
        Da die Exponentialfunktion $e^x$ schneller wächst als die lineare Funktion $x$, ist dieser Grenzwert (wie auch im Tipp angedeutet und z.B. mit der Regel von L'Hôpital zeigbar):
        $$ \lim_{x \to \infty} \frac{x}{e^x} = 0 $$
        Der Graph von $f(x)$ nähert sich also für $x \to \infty$ der x-Achse asymptotisch an ($y=0$ ist eine horizontale Asymptote).
        \item \textbf{Bedeutung für die Fläche von $x=0$ bis 'unendlich':}
        Die Fläche unter dem Graphen von $x=0$ bis 'unendlich' wird durch das uneigentliche Integral $\int_0^\infty xe^{-x} \,dx$ beschrieben. Dieses ist definiert als:
        $$ \int_0^\infty xe^{-x} \,dx = \lim_{b \to \infty} \int_0^b xe^{-x} \,dx $$
        Mit der Stammfunktion $F(x) = -(x+1)e^{-x}$:
        \begin{align*} \lim_{b \to \infty} [-(x+1)e^{-x}]_0^b &= \lim_{b \to \infty} \left( (-(b+1)e^{-b}) - (-(0+1)e^0) \right) \\ &= \lim_{b \to \infty} \left( -\frac{b+1}{e^b} - (-1) \right) \\ &= \lim_{b \to \infty} \left( 1 - \frac{b+1}{e^b} \right) \end{align*}
        Den Grenzwert $\lim_{b \to \infty} \frac{b+1}{e^b}$ bestimmen wir mit L'Hôpital:
        $$ \lim_{b \to \infty} \frac{b+1}{e^b} \stackrel{L'H}{=} \lim_{b \to \infty} \frac{1}{e^b} = 0 $$
        Somit ist der Wert des uneigentlichen Integrals:
        $$ \int_0^\infty xe^{-x} \,dx = 1 - 0 = \mathbf{1} $$
        \textbf{Bedeutung:} Obwohl sich die Fläche unter dem Graphen von $f(x)=xe^{-x}$ unendlich weit entlang der positiven x-Achse erstreckt, hat sie einen endlichen Gesamtinhalt von 1 Flächeneinheit. 
    \end{itemize}
\end{enumerate}

\end{loesungsumgebung}


\begin{aufgabenumgebung}{Optimierung im biologischen Kontext – Wachstum und Hemmung}
Eine Bakterienpopulation wächst zunächst, wird aber durch einen hemmenden Faktor (z.B. begrenzte Nährstoffe) beeinflusst. Die Anzahl der Bakterien $N$ (in Tausend) nach $t$ Stunden kann modelliert werden durch die Funktion:
\[ N(t) = 5t \cdot e^{-0.1t} \quad (\text{für } t \ge 0) \]
\begin{enumerate}
    \item \textbf{Anfangsbestand:} Wie viele Bakterien sind zum Zeitpunkt $t=0$ vorhanden? Interpretiere das Ergebnis im Kontext.
    \item \textbf{Wachstumsrate:} Bestimme die Funktion $N'(t)$, welche die Wachstumsrate der Bakterienpopulation zum Zeitpunkt $t$ angibt. (Produkt- und Kettenregel sind hier gefragt!)
    \item \textbf{Maximale Population:}
        \begin{itemize}
            \item Zu welchem Zeitpunkt $t_{max}$ erreicht die Bakterienpopulation ihr Maximum? (Tipp: Notwendige Bedingung für Extremstellen $N'(t)=0$. Da $e^{-0.1t}$ nie Null wird, musst du nur den anderen Faktor betrachten.)
            \item Überprüfe mit der zweiten Ableitung $N''(t)$ oder dem Vorzeichenwechselkriterium von $N'(t)$, ob es sich tatsächlich um ein Maximum handelt.
            \item Wie groß ist die maximale Bakterienpopulation $N(t_{max})$?
        \end{itemize}
    \item \textbf{Verhalten für $t \to \infty$:} Was passiert mit der Bakterienpopulation für sehr große Zeiten? (Untersuche $\lim_{t \to \infty} N(t)$). Ist das biologisch sinnvoll?
    \item \textbf{Stärkste Zunahme/Abnahme der Wachstumsrate (für Experten):}
        Die Änderungsrate der Wachstumsrate wird durch $N''(t)$ beschrieben. Wann ist die Zunahme der Wachstumsrate maximal (d.h. wann wächst die Population am schnellsten schneller)? Wann ist die Abnahme der Wachstumsrate maximal (d.h. wann verlangsamt sich das Wachstum am stärksten)? (Tipp: Untersuche $N''(t)$ auf Extremstellen, d.h. bilde $N'''(t)$).
    \item \textbf{Skizze:} Skizziere den Graphen von $N(t)$ für $t \ge 0$ und markiere den maximalen Bestand.
\end{enumerate}
\end{aufgabenumgebung}

\begin{loesungsumgebung}[loes:optimierung-bakterienwachstum-wiederholung]{Optimierung im biologischen Kontext – Wachstum und Hemmung}
Die Anzahl der Bakterien $N$ (in Tausend) nach $t$ Stunden wird modelliert durch $N(t) = 5t \cdot e^{-0.1t}$ für $t \ge 0$.

\begin{enumerate}[label=(\alph*)]
    \item \textbf{Anfangsbestand:}
    Der Anfangsbestand ist die Anzahl der Bakterien zum Zeitpunkt $t=0$:
    $$ N(0) = 5 \cdot 0 \cdot e^{-0.1 \cdot 0} = 0 \cdot e^0 = 0 \cdot 1 = 0 $$
    Zum Zeitpunkt $t=0$ sind \textbf{0 Tausend Bakterien} (also keine Bakterien laut Modell) vorhanden. Dies könnte bedeuten, dass die Beobachtung mit einer vernachlässigbar kleinen Startpopulation beginnt oder dass $t=0$ den Zeitpunkt unmittelbar vor Beginn des exponentiellen Wachstums darstellt, das dann durch die Hemmung beeinflusst wird.

    \item \textbf{Wachstumsrate $N'(t)$:}
    Die Wachstumsrate ist die erste Ableitung von $N(t)$. Wir verwenden die Produktregel: $u(t)=5t \Rightarrow u'(t)=5$; $v(t)=e^{-0.1t}$.
    Für $v'(t)$ verwenden wir die Kettenregel: $v'(t) = e^{-0.1t} \cdot (-0.1) = -0.1e^{-0.1t}$.
    \begin{align*}
    N'(t) &= u'(t)v(t) + u(t)v'(t) \\
           &= 5 \cdot e^{-0.1t} + 5t \cdot (-0.1e^{-0.1t}) \\
           &= 5e^{-0.1t} - 0.5te^{-0.1t} \\
           &= \mathbf{e^{-0.1t}(5 - 0.5t)}
    \end{align*}
    Die Einheit der Wachstumsrate ist Tausend Bakterien pro Stunde.

    \item \textbf{Maximale Population:}
    \begin{itemize}
        \item \textbf{Zeitpunkt $t_{max}$ der maximalen Population:}
        Wir setzen $N'(t)=0$: $e^{-0.1t}(5 - 0.5t) = 0$.
        Da $e^{-0.1t}$ stets positiv ist, muss der zweite Faktor Null sein:
        $5 - 0.5t = 0 \Rightarrow 0.5t = 5 \Rightarrow t = \frac{5}{0.5} = 10$.
        Der Zeitpunkt der potentiell maximalen Population ist $\mathbf{t_{max} = 10}$ Stunden.
        \item \textbf{Überprüfung mit der zweiten Ableitung $N''(t)$:}
        Wir leiten $N'(t) = 5e^{-0.1t} - 0.5te^{-0.1t}$ ab.
        $N''(t) = \frac{d}{dt}(5e^{-0.1t}) - \frac{d}{dt}(0.5te^{-0.1t})$
        $N''(t) = 5(-0.1)e^{-0.1t} - [0.5e^{-0.1t} + 0.5t(-0.1)e^{-0.1t}]$ (Produktregel für den zweiten Term)
        $N''(t) = -0.5e^{-0.1t} - 0.5e^{-0.1t} + 0.05te^{-0.1t}$
        $N''(t) = e^{-0.1t}(-0.5 - 0.5 + 0.05t) = e^{-0.1t}(0.05t - 1)$.
        Für $t=10$:
        $N''(10) = e^{-0.1 \cdot 10}(0.05 \cdot 10 - 1) = e^{-1}(0.5 - 1) = e^{-1}(-0.5) = -\frac{0.5}{e}$.
        Da $N''(10) < 0$, handelt es sich bei $t=10$ Stunden tatsächlich um ein lokales Maximum.
        \item \textbf{Maximale Bakterienpopulation $N(t_{max})$:}
        $N(10) = 5 \cdot 10 \cdot e^{-0.1 \cdot 10} = 50e^{-1} = \frac{50}{e}$.
        $N(10) \approx \frac{50}{2.71828} \approx 18.394$.
        Die maximale Population beträgt $\mathbf{\frac{50}{e} \approx 18.394}$ Tausend Bakterien (also ca. 18394 Bakterien).
    \end{itemize}

    \item \textbf{Verhalten für $t \to \infty$:}
    Wir untersuchen den Grenzwert der Populationsfunktion $N(t) = 5t e^{-0.1t} = \frac{5t}{e^{0.1t}}$ für $t \to \infty$.
    Da die Exponentialfunktion im Nenner schneller wächst als die lineare Funktion im Zähler, gilt (mit der Regel von L'Hôpital):
    $$ \lim_{t \to \infty} \frac{5t}{e^{0.1t}} \stackrel{L'H}{=} \lim_{t \to \infty} \frac{5}{0.1e^{0.1t}} = \frac{5}{\infty} = 0 $$
    Für sehr große Zeiten nähert sich die Bakterienpopulation \textbf{Null} an.
    \textbf{Biologische Sinnhaftigkeit:} Dies ist biologisch sinnvoll. Auch wenn die Population anfangs wächst, führen begrenzte Nährstoffe, Anhäufung von Abfallprodukten oder andere limitierende Faktoren dazu, dass die Population nicht unbegrenzt wachsen kann und schließlich wieder abnimmt und ausstirbt oder ein sehr niedriges Niveau erreicht.

    \item \textbf{Stärkste Zunahme/Abnahme der Wachstumsrate (für Experten):}
    Die Wachstumsrate ist $N'(t) = e^{-0.1t}(5 - 0.5t)$. Die Änderungsrate der Wachstumsrate ist $N''(t) = e^{-0.1t}(0.05t - 1)$. Wir suchen die Extrema von $N''(t)$. Dazu bilden wir die Ableitung von $N''(t)$, also $N'''(t)$.
    $N'''(t) = \frac{d}{dt} [e^{-0.1t}(0.05t - 1)]$.
    Mit $u(t)=e^{-0.1t} \Rightarrow u'(t)=-0.1e^{-0.1t}$ und $v(t)=0.05t-1 \Rightarrow v'(t)=0.05$.
    $N'''(t) = -0.1e^{-0.1t}(0.05t - 1) + e^{-0.1t}(0.05)$
    $N'''(t) = e^{-0.1t}[-0.1(0.05t - 1) + 0.05]$
    $N'''(t) = e^{-0.1t}[-0.005t + 0.1 + 0.05] = e^{-0.1t}(-0.005t + 0.15)$.
    Setze $N'''(t)=0$ für kritische Stellen von $N''(t)$:
    $e^{-0.1t}(-0.005t + 0.15) = 0$.
    Da $e^{-0.1t} \neq 0$, muss $-0.005t + 0.15 = 0 \Rightarrow 0.005t = 0.15 \Rightarrow t = \frac{0.15}{0.005} = 30$.
    Um die Art des Extremums von $N''(t)$ bei $t=30$ zu bestimmen, bilden wir $N^{(4)}(t)$:
    $N^{(4)}(t) = \frac{d}{dt} [e^{-0.1t}(-0.005t + 0.15)]$.
    Mit $u(t)=e^{-0.1t} \Rightarrow u'(t)=-0.1e^{-0.1t}$ und $v(t)=-0.005t+0.15 \Rightarrow v'(t)=-0.005$.
    $N^{(4)}(t) = -0.1e^{-0.1t}(-0.005t + 0.15) + e^{-0.1t}(-0.005)$
    $N^{(4)}(t) = e^{-0.1t}[-0.1(-0.005t + 0.15) - 0.005] = e^{-0.1t}(0.0005t - 0.02)$.
    $N^{(4)}(30) = e^{-3}(0.0005 \cdot 30 - 0.02) = e^{-3}(0.015 - 0.02) = e^{-3}(-0.005)$.
    Da $N^{(4)}(30) < 0$, hat $N''(t)$ bei $t=30$ ein \textbf{lokales Maximum}.
    Der Wert ist $N''(30) = e^{-3}(0.05 \cdot 30 - 1) = e^{-3}(1.5 - 1) = 0.5e^{-3} \approx 0.0249$.
    Die \textbf{Zunahme der Wachstumsrate ist also bei $t=30$ Stunden maximal}.
    Um die stärkste Abnahme der Wachstumsrate zu finden (wo $N''(t)$ am negativsten ist), betrachten wir die Ränder und andere kritische Punkte von $N''(t)$. Die einzige kritische Stelle von $N''(t)$ (Nullstelle von $N'''(t)$) ist $t=30$, wo ein Maximum vorliegt.
    Wir untersuchen die Ränder des sinnvollen Definitionsbereichs (hier $t \ge 0$) und den Verlauf.
    $N''(0) = e^0(0-1) = -1$.
    $\lim_{t \to \infty} N''(t) = \lim_{t \to \infty} e^{-0.1t}(0.05t - 1) = \lim_{t \to \infty} \frac{0.05t-1}{e^{0.1t}} \stackrel{L'H}{=} \lim_{t \to \infty} \frac{0.05}{0.1e^{0.1t}} = 0$.
    Da $N''(t)$ bei $t=0$ den Wert $-1$ hat, bei $t=20$ eine Nullstelle ($N''(20)=0$), bei $t=30$ ein positives Maximum ($0.5e^{-3}$) und dann gegen $0$ strebt, ist der minimalste Wert von $N''(t)$ (stärkste Abnahme der Wachstumsrate) bei $\mathbf{t=0}$ Stunden, mit $N''(0) = -1$.
\end{enumerate}

\end{loesungsumgebung}

\begin{aufgabenumgebung}{Flächenberechnung mit Exponentialfunktionen}
\begin{enumerate}
    \item \textbf{Fläche unter $e^x$:}
        Berechne den Inhalt der Fläche, die vom Graphen der Funktion $f(x) = e^x$, der x-Achse und den Geraden $x=0$ und $x=2$ eingeschlossen wird. Fertige eine Skizze an und markiere die Fläche.
    \item \textbf{Fläche unter $e^{-x}$:}
        Berechne den Inhalt der Fläche, die vom Graphen der Funktion $g(x) = 2e^{-0.5x}$, der x-Achse und den Geraden $x=0$ und $x=4$ eingeschlossen wird. Skizziere auch hier den Graphen und die Fläche.
    \item \textbf{Fläche zwischen $e^x$ und einer Geraden (für Experten):}
        Die Graphen der Funktionen $f(x) = e^x$ und $h(x) = e\cdot x + 1$ schließen eine Fläche ein.
        \begin{itemize}
            \item Zeige, dass sich die Graphen an der Stelle $x=0$ und $x=1$ schneiden.
            \item Bestimme, welche Funktion im Intervall $[0,1]$ oben bzw. unten liegt.
            \item Berechne den Inhalt der eingeschlossenen Fläche.
        \end{itemize}
\end{enumerate}
\end{aufgabenumgebung}


\begin{loesungsumgebung}[loes:flaechen-ex-funktionen]{Flächenberechnung mit Exponentialfunktionen}

\begin{enumerate}[label=(\alph*)]
    \item \textbf{Fläche unter $f(x) = e^x$ im Intervall $[0, 2]$:}
    Die Funktion $f(x)=e^x$ ist im gesamten Intervall $[0,2]$ positiv ($e^x > 0$ für alle $x$). Der gesuchte Flächeninhalt $A_1$ ist daher gleich dem bestimmten Integral:
    $$ A_1 = \int_0^2 e^x \,dx $$
    Eine Stammfunktion von $e^x$ ist $F(x) = e^x$.
    \begin{align*}
    A_1 &= [e^x]_0^2 \\
        &= e^2 - e^0 \\
        &= e^2 - 1
    \end{align*}
    Numerisch ist $A_1 \approx (2.71828)^2 - 1 \approx 7.38906 - 1 = 6.38906$.
    Der Flächeninhalt beträgt $\mathbf{e^2 - 1 \approx 6.389}$ Flächeneinheiten.
    \begin{center}
    \includegraphics[width=0.8\textwidth]{grafiken/flaeche_ex_0_2.png}
    % --- Beschreibung der Skizze ---
    % Die Skizze zeigt den Graphen der Funktion f(x)=e^x.
    % Die x-Achse ist von etwa -1 bis 3 skaliert, die y-Achse von 0 bis etwa 8.
    % Der Graph steigt exponentiell an und geht durch (0|1) und (1|e) und (2|e^2).
    % Die Fläche unter dem Graphen zwischen x=0 (y-Achse) und x=2 ist markiert.
    \captionof{figure}{Fläche unter $f(x)=e^x$ im Intervall $[0,2]$.}
    \label{fig:flaeche_ex_0_2}
    \end{center}

    \item \textbf{Fläche unter $g(x) = 2e^{-0.5x}$ im Intervall $[0, 4]$:}
    Die Funktion $g(x)=2e^{-0.5x}$ ist im gesamten Intervall $[0,4]$ positiv. Der gesuchte Flächeninhalt $A_2$ ist daher gleich dem bestimmten Integral:
    $$ A_2 = \int_0^4 2e^{-0.5x} \,dx $$
    Eine Stammfunktion von $2e^{-0.5x}$ ist $G(x) = 2 \cdot \frac{1}{-0.5}e^{-0.5x} = 2 \cdot (-2)e^{-0.5x} = -4e^{-0.5x}$.
    \begin{align*}
    A_2 &= [-4e^{-0.5x}]_0^4 \\
        &= (-4e^{-0.5 \cdot 4}) - (-4e^{-0.5 \cdot 0}) \\
        &= (-4e^{-2}) - (-4e^0) \\
        &= -4e^{-2} - (-4 \cdot 1) \\
        &= 4 - 4e^{-2} = \mathbf{4 - \frac{4}{e^2}}
    \end{align*}
    Numerisch ist $A_2 \approx 4 - \frac{4}{(2.71828)^2} \approx 4 - \frac{4}{7.38906} \approx 4 - 0.54134 \approx 3.45866$.
    Der Flächeninhalt beträgt $4 - \frac{4}{e^2} \approx 3.459$ Flächeneinheiten.
    \begin{center}
    \includegraphics[width=0.8\textwidth]{grafiken/flaeche_2e_neg05x_0_4.png}
    % --- Beschreibung der Skizze ---
    % Die Skizze zeigt den Graphen der Funktion g(x)=2e^(-0.5x).
    % Die x-Achse ist von etwa -1 bis 5 skaliert, die y-Achse von 0 bis etwa 2.5.
    % Der Graph ist eine fallende Exponentialfunktion, die bei (0|2) beginnt und sich für x->unendlich der x-Achse nähert.
    % Die Fläche unter dem Graphen zwischen x=0 und x=4 ist markiert. Bei x=4 ist g(4) = 2e^-2 approx 0.27.
    \captionof{figure}{Fläche unter $g(x)=2e^{-0.5x}$ im Intervall $[0,4]$.}
    \label{fig:flaeche_2e_neg05x_0_4}
    \end{center}

    \item \textbf{Fläche zwischen $f(x) = e^x$ und $h(x) = ex + 1$:}
    \begin{itemize}
        \item \textbf{Zeige, dass sich die Graphen an der Stelle $x=0$ und $x=1$ schneiden.} \\
        Für $x=0$:
        $f(0) = e^0 = 1$.
        $h(0) = e \cdot 0 + 1 = 1$.
        Da $f(0)=h(0)=1$, schneiden sich die Graphen bei $x=0$.
        Für $x=1$:
        $f(1) = e^1 = e$.
        $h(1) = e \cdot 1 + 1 = e+1$.
        Da $f(1) = e$ und $h(1) = e+1$ und $e \neq e+1$, ist die Aussage, dass sich die Graphen auch bei $x=1$ schneiden, für die gegebene Funktion $h(x)=ex+1$ \textbf{nicht korrekt}.
        Wir fahren mit der Aufgabenstellung fort unter der Annahme, dass das Intervall $[0,1]$ für die Flächenberechnung relevant ist, auch wenn $x=1$ kein Schnittpunkt ist, oder dass eine andere Funktion $h(x)$ gemeint war, die $f(x)$ bei $x=1$ schneidet. Die Abbildung \ref{fig:flaeche_ex_gerade} im Aufgabentext suggeriert ein abgeschlossenes Flächenstück. Wenn wir die Funktionen wie gegeben verwenden, ist $x=0$ der einzige Schnittpunkt.

        \item \textbf{Bestimme, welche Funktion im Intervall $[0,1]$ oben bzw. unten liegt.}
        Wir vergleichen $f(x)=e^x$ und $h(x)=ex+1$ im Intervall $[0,1]$.
        Betrachten wir die Differenzfunktion $d(x) = h(x) - f(x) = ex+1 - e^x$.
        $d(0) = e(0)+1 - e^0 = 1-1 = 0$.
        $d(1) = e(1)+1 - e^1 = e+1-e = 1$.
        Um das Verhalten dazwischen zu prüfen, betrachten wir $d'(x) = e - e^x$.
        $d'(x)=0 \Rightarrow e^x=e \Rightarrow x=1$.
        $d''(x) = -e^x$. $d''(1) = -e < 0$. Also hat $d(x)$ ein Maximum bei $x=1$.
        Da $d(0)=0$ und $d(1)=1$ (Maximum), und $d(x)$ stetig ist, muss $d(x) \ge 0$ für $x \in [0,1]$ gelten.
        Somit liegt $\mathbf{h(x) = ex+1}$ im Intervall $[0,1]$ \textbf{oberhalb oder gleich} $\mathbf{f(x) = e^x}$.

        \item \textbf{Berechne den Inhalt der eingeschlossenen Fläche.}
        Da $h(x) \ge f(x)$ im Intervall $[0,1]$ und $x=0$ der einzige Schnittpunkt in diesem Bereich ist (und $x=1$ die rechte Grenze des Intervalls, aber kein Schnittpunkt mit $f(x)$ ist, wie oben gezeigt), berechnen wir die Fläche $A_3 = \int_0^1 (h(x)-f(x)) \,dx$.
        \begin{align*}
        A_3 &= \int_0^1 (ex+1 - e^x) \,dx \\
            &= \left[ e\frac{x^2}{2} + x - e^x \right]_0^1 \\
            &= \left( e\frac{1^2}{2} + 1 - e^1 \right) - \left( e\frac{0^2}{2} + 0 - e^0 \right) \\
            &= \left( \frac{e}{2} + 1 - e \right) - (0 + 0 - 1) \\
            &= \left( 1 - \frac{e}{2} \right) - (-1) \\
            &= 1 - \frac{e}{2} + 1 \\
            &= \mathbf{2 - \frac{e}{2}}
        \end{align*}
        Numerisch ist $A_3 \approx 2 - \frac{2.71828}{2} \approx 2 - 1.35914 = 0.64086$.
        Der Flächeninhalt beträgt $2 - \frac{e}{2} \approx 0.641$ Flächeneinheiten.
    \end{itemize}
\end{enumerate}

\end{loesungsumgebung}



\begin{aufgabenumgebung}{Anwendung: Medikamentenabbau im Körper}
Die Konzentration eines Medikaments im Blut eines Patienten (in mg/Liter) kann nach der Einnahme durch die Funktion $K(t) = 10 \cdot t \cdot e^{-0.2t}$ beschrieben werden, wobei $t$ die Zeit in Stunden nach der Einnahme ist ($t \ge 0$).
\begin{enumerate}
    \item Zu welchem Zeitpunkt ist die Konzentration des Medikaments maximal? Wie hoch ist diese maximale Konzentration? (Nutze die Differentialrechnung).
    \item Die 'Gesamtwirkung' eines Medikaments über einen bestimmten Zeitraum kann manchmal durch das Integral der Konzentrationsfunktion über diesen Zeitraum angenähert werden (dies ist eine Vereinfachung, aber das Integral gibt eine Art 'kumulierte Dosis' an).
        Berechne $\int_0^{10} K(t) \,dt$. (Tipp: Partielle Integration ist hier notwendig! Wähle $u'(t)=e^{-0.2t}$ und $v(t)=10t$).
        Was könnte dieser Wert im Kontext bedeuten?
    \item (Für Experten): Was ist $\lim_{t \to \infty} K(t)$? Was bedeutet das für die Konzentration des Medikaments nach sehr langer Zeit?
\end{enumerate}
\end{aufgabenumgebung}


\begin{loesungsumgebung}[loes:anwendung-medikamentenabbau]{Anwendung: Medikamentenabbau im Körper}
Die Konzentration des Medikaments wird durch $K(t) = 10t \cdot e^{-0.2t}$ beschrieben ($t \ge 0$).

\begin{enumerate}[label=(\alph*)]
    \item \textbf{Zu welchem Zeitpunkt ist die Konzentration des Medikaments maximal? Wie hoch ist diese maximale Konzentration?}
    Um das Maximum zu finden, bilden wir die erste Ableitung $K'(t)$ und setzen sie gleich Null.
    $K(t) = 10t \cdot e^{-0.2t}$. Wir verwenden die Produktregel ($u=10t, v=e^{-0.2t}$):
    $u'(t) = 10$.
    $v'(t) = e^{-0.2t} \cdot (-0.2) = -0.2e^{-0.2t}$ (Kettenregel).
    \begin{align*} K'(t) &= 10 \cdot e^{-0.2t} + 10t \cdot (-0.2e^{-0.2t}) \\ &= 10e^{-0.2t} - 2te^{-0.2t} \\ &= e^{-0.2t}(10 - 2t) \end{align*}
    Setze $K'(t)=0$:
    $e^{-0.2t}(10 - 2t) = 0$.
    Da $e^{-0.2t} > 0$ für alle $t$, muss $10 - 2t = 0$ sein.
    $10 = 2t \Rightarrow t = 5$.
    Die kritische Stelle ist $t_{max} = 5$ Stunden.
    Um zu überprüfen, ob es sich um ein Maximum handelt, bilden wir die zweite Ableitung $K''(t)$:
    $K'(t) = (10 - 2t)e^{-0.2t}$. Mit Produktregel ($u_2=10-2t, v_2=e^{-0.2t}$):
    $u_2'(t) = -2$.
    $v_2'(t) = -0.2e^{-0.2t}$.
    \begin{align*} K''(t) &= -2 \cdot e^{-0.2t} + (10 - 2t) \cdot (-0.2e^{-0.2t}) \\ &= e^{-0.2t}[-2 - 0.2(10 - 2t)] \\ &= e^{-0.2t}[-2 - 2 + 0.4t] \\ &= e^{-0.2t}(0.4t - 4) \end{align*}
    Setze $t=5$ in $K''(t)$ ein:
    $K''(5) = e^{-0.2 \cdot 5}(0.4 \cdot 5 - 4) = e^{-1}(2 - 4) = e^{-1}(-2) = -\frac{2}{e}$.
    Da $K''(5) < 0$, liegt bei $t=5$ Stunden ein lokales Maximum vor.
    Die maximale Konzentration ist:
    $K(5) = 10 \cdot 5 \cdot e^{-0.2 \cdot 5} = 50e^{-1} = \frac{50}{e}$.
    $K(5) \approx \frac{50}{2.71828} \approx 18.394$.
    Die Konzentration ist \textbf{nach 5 Stunden maximal} und beträgt dann $\mathbf{\frac{50}{e} \approx 18.394}$ mg/Liter.

    \item \textbf{Berechne $\int_0^{10} K(t) \,dt$.}
    Wir verwenden partielle Integration für $\int 10t e^{-0.2t} \,dt$.
    Sei $u(t) = 10t \Rightarrow u'(t) = 10$.
    Sei $v'(t) = e^{-0.2t} \Rightarrow v(t) = \int e^{-0.2t} \,dt = \frac{1}{-0.2}e^{-0.2t} = -5e^{-0.2t}$.
    \begin{align*} \int 10t e^{-0.2t} \,dt &= 10t(-5e^{-0.2t}) - \int 10(-5e^{-0.2t}) \,dt \\ &= -50te^{-0.2t} - \int -50e^{-0.2t} \,dt \\ &= -50te^{-0.2t} + 50 \int e^{-0.2t} \,dt \\ &= -50te^{-0.2t} + 50 \left(\frac{1}{-0.2}e^{-0.2t}\right) + C \\ &= -50te^{-0.2t} - 250e^{-0.2t} + C \\ &= -50e^{-0.2t}(t+5) + C \end{align*}
    Nun das bestimmte Integral von $0$ bis $10$:
    \begin{align*} \int_0^{10} 10t e^{-0.2t} \,dt &= [-50e^{-0.2t}(t+5)]_0^{10} \\ &= \left(-50e^{-0.2 \cdot 10}(10+5)\right) - \left(-50e^{-0.2 \cdot 0}(0+5)\right) \\ &= \left(-50e^{-2}(15)\right) - \left(-50e^0(5)\right) \\ &= -750e^{-2} - (-250 \cdot 1) \\ &= 250 - 750e^{-2} = \mathbf{250 - \frac{750}{e^2}} \end{align*}
    Numerischer Wert: $250 - \frac{750}{(2.71828)^2} \approx 250 - \frac{750}{7.38906} \approx 250 - 101.501 \approx 148.499$.
    \textit{Bedeutung:} Der Wert $\mathbf{250 - \frac{750}{e^2} \approx 148.5}$ (mg/Liter)$\cdot$Stunden stellt die kumulierte Medikamentenexposition des Patienten über die ersten 10 Stunden dar. Es ist ein Maß für die Gesamtmenge des Medikaments, der der Körper in diesem Zeitraum ausgesetzt war (Fläche unter der Konzentrationskurve).

    \item \textbf{(Für Experten): Was ist $\lim_{t \to \infty} K(t)$?}
    $$ K(t) = 10t e^{-0.2t} = \frac{10t}{e^{0.2t}} $$
    Für $t \to \infty$ wächst der Nenner $e^{0.2t}$ schneller als der Zähler $10t$. Mit der Regel von L'Hôpital:
    $$ \lim_{t \to \infty} \frac{10t}{e^{0.2t}} \stackrel{L'H}{=} \lim_{t \to \infty} \frac{\frac{d}{dt}(10t)}{\frac{d}{dt}(e^{0.2t})} = \lim_{t \to \infty} \frac{10}{0.2e^{0.2t}} $$
    Da $e^{0.2t} \to \infty$ für $t \to \infty$, ist der Grenzwert:
    $$ \lim_{t \to \infty} \frac{10}{0.2e^{0.2t}} = \frac{10}{\infty} = \mathbf{0} $$
    \textit{Bedeutung:} Für sehr große Zeiten ($t \to \infty$) nähert sich die Konzentration des Medikaments im Blut Null an. Das bedeutet, dass das Medikament im Laufe der Zeit vollständig aus dem Körper eliminiert oder abgebaut wird, was biologisch erwartet wird.
\end{enumerate}

\end{loesungsumgebung}



\begin{aufgabenumgebung}{Uneigentliches Integral – Fläche bis ins Unendliche?}
Wir betrachten die Funktion $f(x) = e^{-x}$ für $x \ge 0$.
\begin{enumerate}
    \item Skizziere den Graphen von $f(x)$.
    \item Berechne den Flächeninhalt $A_b = \int_0^b e^{-x} \,dx$ für eine beliebige obere Grenze $b > 0$.
    \item Was passiert mit diesem Flächeninhalt $A_b$, wenn $b$ unendlich groß wird? Berechne also den Grenzwert $\lim_{b \to \infty} A_b$.
    \item Kann eine Fläche, die sich ins Unendliche erstreckt, einen endlichen Inhalt haben? Diskutiere dein Ergebnis aus c).
\end{enumerate}
\end{aufgabenumgebung}

\begin{loesungsumgebung}[loes:uneigentliches-integral-flaeche]{Uneigentliches Integral – Fläche bis ins Unendliche?}
Wir betrachten die Funktion $f(x) = e^{-x}$ für $x \ge 0$.

\begin{enumerate}[label=(\alph*)]
    \item \textbf{Skizziere den Graphen von $f(x)$.}
    Die Funktion $f(x)=e^{-x}$ ist eine monoton fallende Exponentialfunktion.
    \begin{itemize}
        \item Bei $x=0$ ist $f(0)=e^0=1$. Der Graph startet im Punkt $(0|1)$.
        \item Für $x > 0$ ist $f(x) > 0$. Der Graph verläuft vollständig oberhalb der x-Achse.
        \item Für $x \to \infty$ nähert sich $f(x)=e^{-x} = \frac{1}{e^x}$ dem Wert $0$. Die x-Achse ($y=0$) ist eine horizontale Asymptote.
    \end{itemize}
    \begin{center}
    % \includegraphics[width=0.8\textwidth]{grafiken/uneigentliches_integral_ex.png}
    % --- Beschreibung der Skizze ---
    % Die Skizze zeigt ein Koordinatensystem mit x- und y-Achse.
    % Der Graph der Funktion f(x) = e^(-x) ist für x >= 0 eingezeichnet.
    % Er startet bei (0|1) auf der y-Achse.
    % Er fällt exponentiell ab und nähert sich für x -> unendlich der x-Achse (horizontale Asymptote y=0).
    % Die Fläche zwischen dem Graphen, der x-Achse, der y-Achse (x=0) und einer vertikalen Linie bei x=b ist schattiert.
    % Eine Andeutung, dass b gegen unendlich geht, kann durch einen Pfeil oder die Beschriftung der schattierten Fläche als A_unendlich erfolgen.
    \captionof{figure}{Graph der Funktion $f(x)=e^{-x}$ für $x \ge 0$.}
    \label{fig:uneigentliches_integral_ex}
    \end{center}

    \item \textbf{Berechne den Flächeninhalt $A_b = \int_0^b e^{-x} \,dx$ für eine beliebige obere Grenze $b > 0$.}
    Eine Stammfunktion von $f(x)=e^{-x}$ ist $F(x) = -e^{-x}$.
    \begin{align*} A_b &= \int_0^b e^{-x} \,dx \\ &= [-e^{-x}]_0^b \\ &= (-e^{-b}) - (-e^{-0}) \\ &= -e^{-b} - (-1) \\ &= \mathbf{1 - e^{-b}} \quad \text{oder} \quad \mathbf{1 - \frac{1}{e^b}} \end{align*}

    \item \textbf{Was passiert mit diesem Flächeninhalt $A_b$, wenn $b$ unendlich groß wird? Berechne also den Grenzwert $\lim_{b \to \infty} A_b$.}
    Wir berechnen den Grenzwert des Ergebnisses aus Teil (b) für $b \to \infty$:
    $$ \lim_{b \to \infty} A_b = \lim_{b \to \infty} \left(1 - e^{-b}\right) = \lim_{b \to \infty} \left(1 - \frac{1}{e^b}\right) $$
    Wenn $b \to \infty$, dann $e^b \to \infty$, und somit $\frac{1}{e^b} \to 0$.
    Also:
    $$ \lim_{b \to \infty} A_b = 1 - 0 = \mathbf{1} $$
    Dieser Grenzwert ist der Wert des uneigentlichen Integrals $\int_0^\infty e^{-x} \,dx$.

    \item \textbf{Kann eine Fläche, die sich ins Unendliche erstreckt, einen endlichen Inhalt haben? Diskutiere dein Ergebnis aus c).}
    \textbf{Ja}, eine Fläche, die sich ins Unendliche erstreckt, kann einen endlichen Inhalt haben. Das Ergebnis aus Teil (c), $\int_0^\infty e^{-x} \,dx = 1$, ist ein Beispiel dafür.
    \textit{Diskussion:}
    Obwohl sich der Integrationsbereich $[0, \infty)$ unendlich weit entlang der x-Achse ausdehnt, nähert sich die Funktion $f(x)=e^{-x}$ für $x \to \infty$ sehr schnell der x-Achse an (d.h., die 'Höhe' der Fläche wird sehr klein). Der Beitrag der Fläche für immer größere $x$-Werte wird so gering, dass die Summe aller Flächenbeiträge (das Integral) gegen einen endlichen Wert konvergiert.
    Man kann sich das so vorstellen, dass der 'Schwanz' der Fläche so schnell dünn wird, dass sein Gesamtinhalt begrenzt ist. Nicht jede Funktion, die sich ins Unendliche erstreckt, hat eine endliche Fläche (z.B. $\int_1^\infty \frac{1}{x} \,dx$ divergiert), aber für Funktionen, die schnell genug gegen Null gehen, wie $e^{-x}$, ist dies möglich. Das Konzept der Konvergenz von Reihen und uneigentlichen Integralen beschäftigt sich genau mit solchen Phänomenen.
\end{enumerate}

\end{loesungsumgebung}


\begin{aufgabenumgebung}{Exponentialfunktionen – Übergreifende und anspruchsvolle Aufgaben}
Die folgenden Aufgaben sollen dein Verständnis für Exponentialfunktionen, ihre Ableitungen, Stammfunktionen und Anwendungen umfassend prüfen. Versuche, die Aufgaben sorgfältig und schrittweise zu lösen.
\begin{enumerate}
    \item \textbf{Kurvendiskussion einer e-Funktion mit Polynomfaktor:}
        Führe eine vollständige Kurvendiskussion für die Funktion $f(x) = (4-x^2)e^{-0.5x}$ durch. Untersuche dabei insbesondere:
        \begin{itemize}
            \item Definitionsbereich und Symmetrie.
            \item Verhalten im Unendlichen und Asymptoten. (Tipp: $\lim_{x\to\infty} x^n e^{-kx} = 0$ für $k>0$).
            \item Nullstellen.
            \item Extrempunkte (Lage und Art).
            \item Wendepunkte (Lage und Krümmungsverhalten).
            \item Skizziere den Graphen von $f(x)$.
        \end{itemize}

    \item \textbf{Optimierung: Maximale Konzentration eines Medikaments}
        Die Konzentration $K(t)$ eines Medikaments im Blut eines Patienten (in mg/l) zum Zeitpunkt $t$ (in Stunden nach Einnahme) wird durch die Funktion $K(t) = 20t \cdot e^{-0.25t}$ für $t \ge 0$ beschrieben.
        \begin{itemize}
            \item Zu welchem Zeitpunkt $t_{max}$ erreicht die Konzentration des Medikaments ihr Maximum?
            \item Wie hoch ist diese maximale Konzentration?
            \item Bestimme die Funktion $K'(t)$, die die Änderungsrate der Konzentration angibt. Wann nimmt die Konzentration am stärksten zu? (Suche nach dem Maximum von $K'(t)$, also einem Wendepunkt von $K(t)$ mit bestimmten Eigenschaften).
        \end{itemize}

    \item \textbf{Flächenberechnung und uneigentliches Integral:}
        Gegeben ist die Funktion $f(x) = (x+1)e^{-x}$.
        \begin{itemize}
            \item Bestimme die Stammfunktion $F(x)$ von $f(x)$ mithilfe partieller Integration.
            \item Berechne den Inhalt der Fläche, die der Graph von $f(x)$ mit der x-Achse und der y-Achse im ersten Quadranten einschließt. (Finde dazu die Nullstelle von $f(x)$ für $x \ge 0$).
            \item (Für Experten): Untersuche, ob die Fläche, die der Graph von $f(x)$ mit der x-Achse für $x \ge 0$ einschließt, einen endlichen Inhalt hat. Berechne dazu das uneigentliche Integral $\int_0^\infty (x+1)e^{-x} \,dx = \lim_{b \to \infty} \int_0^b (x+1)e^{-x} \,dx$.
        \end{itemize}

    \item \textbf{Rekonstruktion einer Exponentialfunktion und Tangente:}
        Eine Funktion $f$ ist gegeben durch $f(x) = (ax+b)e^{-x}$. Der Graph von $f$ hat im Punkt $P(0|2)$ eine Tangente, die parallel zur Geraden $y=3x-5$ ist.
        \begin{itemize}
            \item Bestimme die Werte der Parameter $a$ und $b$.
            \item Bestimme die Gleichung der Tangente an den Graphen von $f$ im Punkt $P(0|2)$.
            \item Untersuche die Funktion $f(x)$ mit den gefundenen Parametern auf Nullstellen und Extrempunkte.
        \end{itemize}

    \item \textbf{Schar von Exponentialfunktionen (Schwer):}
        Gegeben ist die Funktionenschar $f_k(x) = x^2 e^{kx}$ mit dem Scharparameter $k \in \mathbb{R}$, $k \neq 0$.
        \begin{itemize}
            \item Bestimme die Nullstellen von $f_k(x)$.
            \item Bestimme die erste Ableitung $f_k'(x)$. Für welche Werte von $x$ (in Abhängigkeit von $k$) ist $f_k'(x)=0$?
            \item Untersuche die Art der Extremstellen in Abhängigkeit von $k$. (Tipp: Betrachte die Fälle $k>0$ und $k<0$ getrennt für das Verhalten der zweiten Ableitung oder den Vorzeichenwechsel der ersten Ableitung).
            \item Gibt es Werte für $k$, sodass die Funktion keine Extrempunkte besitzt?
        \end{itemize}
\end{enumerate}
\end{aufgabenumgebung}

\begin{loesungsumgebung}[loes:A:DiffUebergreifend]{Übergreifende Übungsaufgaben}

\begin{enumerate}
    \item \textbf{Kurvendiskussion einer e-Funktion mit Polynomfaktor: $f(x) = (4-x^2)e^{-0.5x}$}
    \begin{itemize}
        \item \textbf{Definitionsbereich und Symmetrie:}
        $D_f = \mathbb{R}$, da sowohl der Polynomfaktor als auch die e-Funktion für alle $x \in \mathbb{R}$ definiert sind.
        $f(-x) = (4-(-x)^2)e^{-0.5(-x)} = (4-x^2)e^{0.5x}$.
        Da $f(-x) \neq f(x)$ und $f(-x) \neq -f(x)$, liegt keine einfache Achsen- oder Punktsymmetrie zum Ursprung vor.

        \item \textbf{Verhalten im Unendlichen und Asymptoten:}
        Für $x \to \infty$: $f(x) = \frac{4-x^2}{e^{0.5x}}$. Da die e-Funktion im Nenner schneller wächst als jedes Polynom im Zähler (vgl. Regel von L'Hôpital, zweimal angewendet $\lim \frac{-2x}{0.5e^{0.5x}} = \lim \frac{-2}{0.25e^{0.5x}} = 0$), gilt $\lim_{x\to\infty} f(x) = 0$.
        Somit ist $\mathbf{y=0}$ eine horizontale Asymptote für $x \to \infty$.
        Für $x \to -\infty$: Sei $u=-x$, dann $u \to \infty$. $f(x) = (4-u^2)e^{0.5u}$.
        $\lim_{u\to\infty} (4-u^2)e^{0.5u} = (-\infty) \cdot (\infty) = -\infty$. Also $\lim_{x\to-\infty} f(x) = -\infty$.

        \item \textbf{Nullstellen:}
        $f(x)=0 \Rightarrow (4-x^2)e^{-0.5x}=0$. Da $e^{-0.5x} > 0$ immer, muss $4-x^2=0 \Rightarrow x^2=4 \Rightarrow x=\pm 2$.
        Nullstellen: $\mathbf{N_1(-2|0)}$ und $\mathbf{N_2(2|0)}$.
        Y-Achsenabschnitt: $f(0)=(4-0)e^0 = 4 \cdot 1 = 4$. $\mathbf{P_y(0|4)}$.

        \item \textbf{Extrempunkte (Lage und Art):}
        $f'(x) = (-2x)e^{-0.5x} + (4-x^2)(-0.5e^{-0.5x}) = e^{-0.5x}[-2x - 0.5(4-x^2)]$
        $f'(x) = e^{-0.5x}[-2x - 2 + 0.5x^2] = \frac{1}{2}e^{-0.5x}(x^2 - 4x - 4)$.
        $f'(x)=0 \Rightarrow x^2 - 4x - 4 = 0$.
        $x_{1,2} = \frac{4 \pm \sqrt{16-4(1)(-4)}}{2} = \frac{4 \pm \sqrt{32}}{2} = \frac{4 \pm 4\sqrt{2}}{2} = 2 \pm 2\sqrt{2}$.
        $x_{E1} = 2 - 2\sqrt{2} \approx -0.828$, $x_{E2} = 2 + 2\sqrt{2} \approx 4.828$.
        $f''(x) = \frac{d}{dx}\left(\frac{1}{2}e^{-0.5x}(x^2 - 4x - 4)\right)$
        $f''(x) = \frac{1}{2}[-0.5e^{-0.5x}(x^2-4x-4) + e^{-0.5x}(2x-4)]$
        $f''(x) = \frac{1}{2}e^{-0.5x}[-0.5x^2+2x+2 + 2x-4] = \frac{1}{2}e^{-0.5x}(-0.5x^2+4x-2) = \frac{1}{4}e^{-0.5x}(-x^2+8x-4)$.
        $f''(2-2\sqrt{2}) = \frac{1}{4}e^{-0.5(2-2\sqrt{2})} \underbrace{(-(2-2\sqrt{2})^2+8(2-2\sqrt{2})-4)}_{ \approx - (0.686) + 8(-0.828) - 4 = -0.686 -6.624 -4 < 0 \text{ (genauer Wert über } x^2-4x-4=0 \text{ für } f' \text{ nutzen)}}$.
        Für $x_E = 2 \pm 2\sqrt{2}$ ist $x_E^2-4x_E-4=0$. Die Klammer in $f''(x_E)$ ist $\frac{1}{2}e^{-0.5x_E}[ (2x_E-4)]$.
        Für $x_{E1}=2-2\sqrt{2} \approx -0.828$: $2x_{E1}-4 = 4-4\sqrt{2}-4 = -4\sqrt{2} < 0$. Also $f''(x_{E1}) = \frac{1}{2}e^{-0.5x_{E1}}(-4\sqrt{2}) < 0 \implies \mathbf{HP}$.
        $f(2-2\sqrt{2}) = (4-(2-2\sqrt{2})^2)e^{-0.5(2-2\sqrt{2})} = (4-(4-8\sqrt{2}+8))e^{\sqrt{2}-1} = (4-12+8\sqrt{2})e^{\sqrt{2}-1} = (8\sqrt{2}-8)e^{\sqrt{2}-1} \approx 4.54$.
        $HP(2-2\sqrt{2} | (8\sqrt{2}-8)e^{\sqrt{2}-1})$.
        Für $x_{E2}=2+2\sqrt{2} \approx 4.828$: $2x_{E2}-4 = 4+4\sqrt{2}-4 = 4\sqrt{2} > 0$. Also $f''(x_{E2}) = \frac{1}{2}e^{-0.5x_{E2}}(4\sqrt{2}) > 0 \implies \mathbf{TP}$.
        $f(2+2\sqrt{2}) = (4-(2+2\sqrt{2})^2)e^{-0.5(2+2\sqrt{2})} = (4-(4+8\sqrt{2}+8))e^{-1-2\sqrt{2}} = (-8-8\sqrt{2})e^{-1-2\sqrt{2}} \approx -0.21$.
        $TP(2+2\sqrt{2} | (-8-8\sqrt{2})e^{-1-2\sqrt{2}})$.

        \item \textbf{Wendepunkte und Krümmungsverhalten:}
        $f''(x) = \frac{1}{4}e^{-0.5x}(-x^2+8x-4)$. $f''(x)=0 \Rightarrow -x^2+8x-4=0 \Rightarrow x^2-8x+4=0$.
        $x_{W1,2} = \frac{8 \pm \sqrt{64-16}}{2} = \frac{8 \pm \sqrt{48}}{2} = \frac{8 \pm 4\sqrt{3}}{2} = 4 \pm 2\sqrt{3}$.
        $x_{W1} = 4-2\sqrt{3} \approx 0.536$, $x_{W2} = 4+2\sqrt{3} \approx 7.464$.
        Da $-x^2+8x-4$ eine nach unten geöffnete Parabel ist:
        Linksgekrümmt ($f''(x)>0$) für $4-2\sqrt{3} < x < 4+2\sqrt{3}$.
        Rechtsgekrümmt ($f''(x)<0$) für $x < 4-2\sqrt{3}$ und $x > 4+2\sqrt{3}$.
        Wendepunkte bei $x_{W1}$ und $x_{W2}$.
        $WP_1(4-2\sqrt{3} | f(4-2\sqrt{3}))$, $WP_2(4+2\sqrt{3} | f(4+2\sqrt{3}))$.

        \item \textbf{Skizze des Graphen:}
        Der Graph von $f(x)$ startet für $x \to -\infty$ bei $f(x) \to -\infty$. Er hat Nullstellen bei $x=\pm 2$. Der y-Achsenabschnitt ist $P_y(0|4)$. Es gibt einen Hochpunkt bei $x \approx -0.828$ ($y \approx 4.54$) und einen Tiefpunkt bei $x \approx 4.828$ ($y \approx -0.21$). Für $x \to \infty$ nähert sich der Graph der Asymptote $y=0$. Wendepunkte liegen bei $x \approx 0.536$ und $x \approx 7.464$.
    \end{itemize}

    \item \textbf{Optimierung: Maximale Konzentration eines Medikaments} $K(t) = 20t e^{-0.25t}$ für $t \ge 0$.
    \begin{itemize}
        \item \textbf{Zeitpunkt $t_{max}$ der maximalen Konzentration:}
        $K'(t) = 20e^{-0.25t} + 20t(-0.25e^{-0.25t}) = 20e^{-0.25t}(1-0.25t)$.
        $K'(t)=0 \Rightarrow 1-0.25t=0 \Rightarrow 0.25t=1 \Rightarrow t_{max}=4$ Stunden.
        \item \textbf{Maximale Konzentration:}
        $K(4) = 20(4)e^{-0.25(4)} = 80e^{-1} = \frac{80}{e} \approx 29.43\,$mg/l.
        Zur Überprüfung: $K''(t) = 20(-0.25)e^{-0.25t}(1-0.25t) + 20e^{-0.25t}(-0.25) = -5e^{-0.25t}(1-0.25t+1) = -5e^{-0.25t}(2-0.25t)$.
        $K''(4) = -5e^{-1}(2-1) = -5e^{-1} < 0 \implies$ Maximum.
        \item \textbf{Änderungsrate $K'(t)$. Wann Zunahme am stärksten?}
        Die Zunahme ist am stärksten, wenn $K'(t)$ maximal ist, d.h. $(K'(t))'=K''(t)=0$.
        $K''(t) = -5e^{-0.25t}(2-0.25t)=0 \Rightarrow 2-0.25t=0 \Rightarrow 0.25t=2 \Rightarrow t_W=8$ Stunden.
        $(K'(t))''' = K'''(t) = -5(-0.25)e^{-0.25t}(2-0.25t) + (-5)e^{-0.25t}(-0.25) = 1.25e^{-0.25t}(2-0.25t+1) = 1.25e^{-0.25t}(3-0.25t)$.
        $K'''(8) = 1.25e^{-2}(3-2) = 1.25e^{-2} > 0$.
        Also hat $K'(t)$ bei $t=8$ ein lokales Minimum. Die stärkste Zunahme von $K(t)$ (Maximum von $K'(t)$) ist also am Rand des Intervalls $t \ge 0$ zu suchen, bevor $K'(t)$ zu fallen beginnt. Da $K'(t)$ von $K'(0)=20$ bis $K'(8)=-5e^{-2}(0)=-20e^{-2}$ fällt (Minimum von $K'(t)$ bei $t=8$, wo $K'(8) \approx -2.7$), ist die stärkste Zunahme bei $\mathbf{t=0}$ mit $K'(0)=20\,$mg/l/h.
        Der Wendepunkt von $K(t)$ bei $t_W=8$ markiert den Punkt, an dem die Konzentration am stärksten abnimmt (die Wachstumsrate $K'(t)$ ist dort minimal, d.h. am stärksten negativ oder am wenigsten positiv).
    \end{itemize}

    \item \textbf{Flächenberechnung und uneigentliches Integral: $f(x) = (x+1)e^{-x}$}
    \begin{itemize}
        \item \textbf{Stammfunktion $F(x)$:} Mit partieller Integration ($u=x+1, v'=e^{-x}$):
        $u'=1, v=-e^{-x}$.
        $F(x) = -(x+1)e^{-x} - \int 1(-e^{-x})dx = -(x+1)e^{-x} + \int e^{-x}dx = -(x+1)e^{-x} - e^{-x} + C = \mathbf{-(x+2)e^{-x} + C}$.
        \item \textbf{Fläche ... im ersten Quadranten:} Nullstelle von $f(x)=(x+1)e^{-x}$ ist $x=-1$. Für $x \ge 0$ ist $x+1 > 0$ und $e^{-x} > 0$, also $f(x)>0$.
        Die Fläche, die der Graph von $f(x)$ mit der x-Achse und der y-Achse im ersten Quadranten (also für $x \ge 0$) einschließt, ist durch das uneigentliche Integral $\int_0^\infty (x+1)e^{-x} \,dx$ gegeben.
        \item \textbf{Uneigentliches Integral $\int_0^\infty (x+1)e^{-x} \,dx$:}
        $= \lim_{b \to \infty} \int_0^b (x+1)e^{-x} \,dx = \lim_{b \to \infty} [-(x+2)e^{-x}]_0^b$
        $= \lim_{b \to \infty} (-(b+2)e^{-b} - (-(0+2)e^0))$
        $= \lim_{b \to \infty} (-\frac{b+2}{e^b} - (-2)) = \lim_{b \to \infty} (2 - \frac{b+2}{e^b})$.
        Mit L'Hôpital: $\lim_{b \to \infty} \frac{b+2}{e^b} = \lim_{b \to \infty} \frac{1}{e^b} = 0$.
        Das Integral ist $2-0 = \mathbf{2}$. Die Fläche hat den endlichen Inhalt 2.
    \end{itemize}

    \item \textbf{Rekonstruktion einer Exponentialfunktion und Tangente: $f(x) = (ax+b)e^{-x}$}
    \begin{itemize}
        \item \textbf{Bestimme $a$ und $b$:}
        $P(0|2)$ auf Graph $\implies f(0)=2 \Rightarrow (a \cdot 0 + b)e^0 = 2 \Rightarrow b \cdot 1 = 2 \Rightarrow \mathbf{b=2}$.
        $f'(x)$: Produktregel. $u=ax+b \Rightarrow u'=a$. $v=e^{-x} \Rightarrow v'=-e^{-x}$.
        $f'(x) = a e^{-x} + (ax+b)(-e^{-x}) = e^{-x}(a - (ax+b)) = e^{-x}(a-ax-b)$.
        Tangente in $P(0|2)$ parallel zu $y=3x-5 \implies f'(0)=3$.
        $f'(0) = e^0(a-a(0)-b) = 1(a-b) = 3 \Rightarrow a-b=3$.
        Mit $b=2$: $a-2=3 \Rightarrow \mathbf{a=5}$.
        Also $f(x) = (5x+2)e^{-x}$.
        \item \textbf{Gleichung der Tangente in $P(0|2)$:} Punkt $(0|2)$, Steigung $m=3$.
        $y-2 = 3(x-0) \Rightarrow \mathbf{y=3x+2}$.
        \item \textbf{Untersuche $f(x)=(5x+2)e^{-x}$ auf Nullstellen und Extrempunkte:}
        Nullstellen: $(5x+2)e^{-x}=0 \Rightarrow 5x+2=0 \Rightarrow \mathbf{x = -2/5 = -0.4}$.
        Extrempunkte: $f'(x) = e^{-x}(a-ax-b) = e^{-x}(5-5x-2) = e^{-x}(3-5x)$.
        $f'(x)=0 \Rightarrow 3-5x=0 \Rightarrow 5x=3 \Rightarrow x_E=3/5 = 0.6$.
        $f''(x) = \frac{d}{dx}(e^{-x}(3-5x)) = -e^{-x}(3-5x) + e^{-x}(-5) = e^{-x}(-3+5x-5) = e^{-x}(5x-8)$.
        $f''(3/5) = e^{-3/5}(5(3/5)-8) = e^{-3/5}(3-8) = -5e^{-3/5} < 0 \implies$ Lokaler Hochpunkt.
        $y_E = f(3/5) = (5(3/5)+2)e^{-3/5} = (3+2)e^{-3/5} = 5e^{-3/5}$.
        $\mathbf{HP(3/5 | 5e^{-3/5} \approx 2.74)}$.
    \end{itemize}

    \item \textbf{Schar von Exponentialfunktionen (Schwer): $f_k(x) = x^2 e^{kx}$ ($k \neq 0$)}
    \begin{itemize}
        \item \textbf{Nullstellen:} $x^2 e^{kx} = 0 \Rightarrow x^2=0 \Rightarrow \mathbf{x=0}$ (doppelte Nullstelle).
        \item \textbf{Erste Ableitung $f_k'(x)$ und kritische Stellen $f_k'(x)=0$:}
        $u=x^2 \Rightarrow u'=2x$. $v=e^{kx} \Rightarrow v'=ke^{kx}$.
        $f_k'(x) = 2xe^{kx} + x^2(ke^{kx}) = xe^{kx}(2+kx)$.
        $f_k'(x)=0 \Rightarrow x=0$ oder $2+kx=0 \Rightarrow kx=-2 \Rightarrow x = -2/k$.
        Kritische Stellen: $\mathbf{x_1=0}$ und $\mathbf{x_2=-2/k}$.
        \item \textbf{Art der Extremstellen in Abhängigkeit von $k$:}
        $f_k''(x) = \frac{d}{dx}((2x+kx^2)e^{kx})$.
        $u_2=2x+kx^2 \Rightarrow u_2'=2+2kx$. $v_2=e^{kx} \Rightarrow v_2'=ke^{kx}$.
        $f_k''(x) = (2+2kx)e^{kx} + (2x+kx^2)(ke^{kx}) = e^{kx}(2+2kx+2kx+k^2x^2) = e^{kx}(k^2x^2+4kx+2)$.
        \begin{itemize}
            \item Für $x_1=0$: $f_k''(0) = e^0(0+0+2) = 2$. Da $f_k''(0)=2>0 \implies$ Lokaler Tiefpunkt bei $(0|f_k(0)=0)$. $\mathbf{TP(0|0)}$.
            \item Für $x_2=-2/k$:
            $f_k''(-2/k) = e^{k(-2/k)}[k^2(-2/k)^2 + 4k(-2/k) + 2] = e^{-2}[k^2(4/k^2) - 8 + 2]$
            $= e^{-2}[4 - 8 + 2] = e^{-2}(-2) = -2e^{-2}$.
            Da $f_k''(-2/k) = -2e^{-2} < 0 \implies$ Lokaler Hochpunkt bei $x=-2/k$.
            $f_k(-2/k) = (-2/k)^2 e^{k(-2/k)} = \frac{4}{k^2}e^{-2}$. $\mathbf{HP(-2/k | \frac{4}{k^2e^2})}$.
        \end{itemize}
        Diese Unterscheidung gilt für alle $k \neq 0$.
        \item \textbf{Werte für $k$, sodass keine Extrempunkte?}
        Da für alle $k \neq 0$ die Stellen $x=0$ und $x=-2/k$ verschieden sind und zu einem Tiefpunkt bzw. einem Hochpunkt führen, besitzt die Funktion für alle $k \neq 0$ \textbf{immer genau zwei Extrempunkte}. Es gibt also keine Werte $k \neq 0$, für die die Funktion keine Extrempunkte besitzt.
    \end{itemize}
\end{enumerate}


\end{loesungsumgebung}






\begin{aufgabenumgebung}{Checkliste: Die Exponentialfunktion im Kern verstehen}
Die Exponentialfunktion ist in vielerlei Hinsicht besonders. Diese Fragen helfen dir, dein Verständnis zu vertiefen:

\begin{enumerate}[label=(\alph*)]
    \item \textbf{Die Eulersche Zahl $e$ und die natürliche Exponentialfunktion:}
    \begin{itemize}
        \item Warum wird $f(x)=e^x$ als die 'natürliche' Exponentialfunktion bezeichnet? Welche einzigartige Eigenschaft hat ihre Ableitung, und was bedeutet das für die Steigung ihres Graphen an jeder Stelle $x$?
        \item Vergleiche die Graphen von $y=2^x$, $y=e^x$ und $y=3^x$ in einer Skizze. Wo schneiden sie die y-Achse? Welche Funktion wächst für $x>0$ am schnellsten, welche am langsamsten? Wie verhält es sich mit ihren Steigungen an der Stelle $x=0$?
    \end{itemize}
    \item \textbf{Transformationen und Asymptoten:}
    Betrachte die Funktion $g(x) = A \cdot e^{k(x-B)} + C$.
    \begin{itemize}
        \item Erkläre, wie sich die Parameter $A, k, B, C$ jeweils auf den Graphen der Grundfunktion $y=e^x$ auswirken (Streckung, Stauchung, Spiegelung, Verschiebung).
        \item Welche Gleichung hat die waagerechte Asymptote von $g(x)$? Wie hängt diese vom Parameter $C$ ab? Warum kann der Graph diese Asymptote nie schneiden, wenn $A \neq 0$?
        \item Wenn $A>0$ und $k>0$: Ist die Funktion streng monoton steigend oder fallend? Was passiert, wenn $k<0$ ist?
    \end{itemize}
    \item \textbf{Von $b^x$ zu $e^{kx}$:}
    \begin{itemize}
        \item Erkläre mit eigenen Worten, warum jede Exponentialfunktion $f(x)=b^x$ (mit $b>0, b\neq1$) auch in der Form $f(x)=e^{kx}$ geschrieben werden kann. Welchen Wert hat $k$ in diesem Fall?
        \item Nutze diese Umschreibung, um die Ableitungsregel $(b^x)' = b^x \ln(b)$ aus der Regel $(e^{kx})' = k e^{kx}$ herzuleiten.
    \end{itemize}
    \item \textbf{Grenzwertverhalten und Dominanz:}
    \begin{itemize}
        \item Warum ist $\lim_{x \to -\infty} e^x = 0$, aber $\lim_{x \to \infty} e^x = \infty$?
        \item Betrachte eine Funktion $h(x) = x^n \cdot e^{-x}$ (mit $n \in \mathbb{N}$). Warum ist $\lim_{x \to \infty} h(x) = 0$, obwohl $x^n$ gegen $\infty$ geht? Welche Funktion 'dominiert' hier das Verhalten für große $x$?
    \end{itemize}
\end{enumerate}
\end{aufgabenumgebung}

\begin{loesungsumgebung}[loes:checkliste-ex-funktion-verstehen]{Checkliste: Die Exponentialfunktion im Kern verstehen}

\begin{enumerate}[label=(\alph*)]
    \item \textbf{Die Eulersche Zahl $e$ und die natürliche Exponentialfunktion:}
    \begin{itemize}
        \item \textbf{Warum 'natürliche' Exponentialfunktion? Einzigartige Eigenschaft der Ableitung und Bedeutung für Steigung.} \\
        Die Funktion $f(x)=e^x$ wird als die 'natürliche' Exponentialfunktion bezeichnet, weil die Basis $e$ (die Eulersche Zahl, $e \approx 2.71828$) diejenige eindeutig bestimmte Zahl ist, für die die Ableitung der Funktion $f(x)=e^x$ wieder die Funktion $f(x)=e^x$ selbst ist.
        Die einzigartige Eigenschaft ihrer Ableitung ist also: $(e^x)' = e^x$.
        Das bedeutet für die Steigung ihres Graphen an jeder Stelle $x_0$: Die Steigung der Tangente an den Graphen von $f(x)=e^x$ im Punkt $(x_0|e^{x_0})$ ist genau $e^{x_0}$, also gleich dem Funktionswert an dieser Stelle. Da $e^x$ immer positiv ist, ist die Steigung immer positiv, und die Funktion ist streng monoton steigend. Je größer der Funktionswert, desto steiler ist der Anstieg des Graphen.

        \item \textbf{Vergleich der Graphen von $y=2^x$, $y=e^x$ und $y=3^x$ (textuelle Beschreibung einer Skizze):}
        \begin{itemize}
            \item \textbf{Schnittpunkt mit der y-Achse:} Alle drei Funktionen $y=2^x$, $y=e^x$ und $y=3^x$ schneiden die y-Achse im Punkt $(0|1)$, da $2^0=1$, $e^0=1$ und $3^0=1$.
            \item \textbf{Wachstum für $x>0$:} Da $3 > e \approx 2.718 > 2$, wächst für $x>0$ die Funktion $y=3^x$ am schnellsten, gefolgt von $y=e^x$, und $y=2^x$ wächst am langsamsten. Das bedeutet, rechts von der y-Achse liegt der Graph von $3^x$ über dem von $e^x$, welcher wiederum über dem von $2^x$ liegt.
            \item \textbf{Verhalten für $x<0$:} Für $x<0$ kehrt sich die Reihenfolge um (bezogen auf die Nähe zur x-Achse). Zum Beispiel für $x=-1$: $2^{-1}=0.5$, $e^{-1} \approx 0.368$, $3^{-1} \approx 0.333$. Der Graph von $3^x$ ist am nächsten an der x-Achse, gefolgt von $e^x$, dann $2^x$. Alle nähern sich der x-Achse ($y=0$) als Asymptote für $x \to -\infty$.
            \item \textbf{Steigungen an der Stelle $x=0$:} Die Ableitung einer Funktion $f(x)=b^x$ ist $f'(x)=b^x \ln b$.
            \begin{itemize}
                \item Für $y=2^x$: $y'(0) = 2^0 \ln 2 = \ln 2 \approx 0.693$.
                \item Für $y=e^x$: $y'(0) = e^0 \ln e = 1 \cdot 1 = 1$.
                \item Für $y=3^x$: $y'(0) = 3^0 \ln 3 = \ln 3 \approx 1.0986$.
            \end{itemize}
            Die Funktion $e^x$ hat an der Stelle $x=0$ die Steigung $1$. $3^x$ wächst dort schneller (Steigung $>1$), $2^x$ wächst langsamer (Steigung $<1$).
        \end{itemize}
    \end{itemize}

    \item \textbf{Transformationen und Asymptoten von $g(x) = A \cdot e^{k(x-B)} + C$:}
    \begin{itemize}
        \item \textbf{Wirkung der Parameter $A, k, B, C$ auf $y=e^x$:}
        \begin{itemize}
            \item $\mathbf{C}$: Verschiebung des Graphen um $C$ Einheiten in y-Richtung (nach oben für $C>0$, nach unten für $C<0$).
            \item $\mathbf{B}$: Verschiebung des Graphen um $B$ Einheiten in x-Richtung (nach rechts für $B>0$ im Term $(x-B)$, nach links für $B<0$).
            \item $\mathbf{k}$:
            Wenn $|k|>1$, erfolgt eine horizontale Stauchung um den Faktor $1/|k|$ bezüglich der Achse $x=B$.
            Wenn $0<|k|<1$, erfolgt eine horizontale Streckung um den Faktor $1/|k|$ bezüglich der Achse $x=B$.
            Wenn $k<0$, erfolgt zusätzlich eine Spiegelung an der vertikalen Achse $x=B$.
            \item $\mathbf{A}$:
            Wenn $|A|>1$, erfolgt eine vertikale Streckung um den Faktor $|A|$.
            Wenn $0<|A|<1$, erfolgt eine vertikale Stauchung um den Faktor $|A|$.
            Wenn $A<0$, erfolgt zusätzlich eine Spiegelung an der (neuen) horizontalen Asymptote $y=C$.
        \end{itemize}
        \item \textbf{Waagerechte Asymptote von $g(x)$:}
        Der Term $e^{k(x-B)}$ geht entweder für $x \to \infty$ oder für $x \to -\infty$ gegen $0$ (je nach Vorzeichen von $k$).
        Wenn $k>0$, $\lim_{x \to -\infty} e^{k(x-B)} = 0$. Dann $\lim_{x \to -\infty} g(x) = A \cdot 0 + C = C$.
        Wenn $k<0$, $\lim_{x \to \infty} e^{k(x-B)} = 0$. Dann $\lim_{x \to \infty} g(x) = A \cdot 0 + C = C$.
        Die Gleichung der waagerechten Asymptote ist also $\mathbf{y=C}$.
        Der Graph kann diese Asymptote nie schneiden, wenn $A \neq 0$, weil $A \cdot e^{k(x-B)}$ niemals Null werden kann (da $e^{\text{Exponent}} > 0$ und $A \neq 0$). Somit ist $g(x) = A \cdot (\text{ein positiver Wert}) + C \neq C$.
        \item \textbf{Monotonie für $A>0$ und $k>0$ bzw. $k<0$:}
        Die Ableitung ist $g'(x) = A \cdot e^{k(x-B)} \cdot k = Ak \cdot e^{k(x-B)}$.
        Da $e^{k(x-B)} > 0$ immer gilt:
        \begin{itemize}
            \item Wenn $A>0$ und $k>0$: Dann ist $Ak > 0$, also $g'(x) > 0$. Die Funktion ist \textbf{streng monoton steigend}.
            \item Wenn $A>0$ und $k<0$: Dann ist $Ak < 0$, also $g'(x) < 0$. Die Funktion ist \textbf{streng monoton fallend}.
        \end{itemize}
    \end{itemize}

    \item \textbf{Von $b^x$ zu $e^{kx}$:}
    \begin{itemize}
        \item \textbf{Erklärung der Umschreibung:}
        Jede positive Basis $b$ ($b \neq 1$) kann als Potenz der Eulerschen Zahl $e$ geschrieben werden: $b = e^{\ln b}$. Dies folgt direkt aus der Definition des natürlichen Logarithmus als Umkehrfunktion zur e-Funktion ($e^{\ln(y)} = y$).
        Setzt man dies in $b^x$ ein, erhält man: $b^x = (e^{\ln b})^x$.
        Nach den Potenzgesetzen ($(a^m)^n = a^{m \cdot n}$) gilt: $(e^{\ln b})^x = e^{(\ln b) \cdot x} = e^{x \ln b}$.
        Dies ist von der Form $e^{kx}$ mit der Konstanten $\mathbf{k = \ln b}$.
        \item \textbf{Herleitung der Ableitungsregel $(b^x)' = b^x \ln(b)$:}
        Wir verwenden die Umschreibung $b^x = e^{x \ln b}$ und leiten diese Form mit der Kettenregel ab.
        Die äußere Funktion ist $a(u) = e^u$ mit $a'(u) = e^u$.
        Die innere Funktion ist $i(x) = x \ln b$. Da $\ln b$ eine Konstante ist, ist $i'(x) = \ln b$.
        Nach der Kettenregel:
        $(b^x)' = (e^{x \ln b})' = a'(i(x)) \cdot i'(x) = e^{x \ln b} \cdot (\ln b)$.
        Da $e^{x \ln b} = b^x$, folgt:
        $(b^x)' = \mathbf{b^x \ln b}$.
    \end{itemize}

    \item \textbf{Grenzwertverhalten und Dominanz:}
    \begin{itemize}
        \item \textbf{Warum $\lim_{x \to -\infty} e^x = 0$, aber $\lim_{x \to \infty} e^x = \infty$?}
        \begin{itemize}
            \item $\mathbf{\lim_{x \to \infty} e^x = \infty}$: Die natürliche Exponentialfunktion $y=e^x$ mit der Basis $e \approx 2.718 > 1$ ist streng monoton steigend. Für $x \to \infty$ (d.h. $x$ wird beliebig groß positiv) werden auch die Funktionswerte $e^x$ beliebig groß und wachsen über jede Schranke.
            \item $\mathbf{\lim_{x \to -\infty} e^x = 0}$: Wenn $x$ beliebig groß negativ wird (z.B. $x=-100, -1000, \dots$), schreiben wir $x=-u$ wobei $u$ beliebig groß positiv wird ($u \to \infty$). Dann ist $e^x = e^{-u} = \frac{1}{e^u}$. Da für $u \to \infty$ der Nenner $e^u \to \infty$ geht, geht der Bruch $\frac{1}{e^u} \to 0$. Die x-Achse ist also eine horizontale Asymptote für $x \to -\infty$.
        \end{itemize}
        \item \textbf{Betrachte $h(x) = x^n \cdot e^{-x}$ (mit $n \in \mathbb{N}$). Warum ist $\lim_{x \to \infty} h(x) = 0$, obwohl $x^n \to \infty$ geht? Welche Funktion 'dominiert'?}
        Wir betrachten $\lim_{x \to \infty} x^n e^{-x} = \lim_{x \to \infty} \frac{x^n}{e^x}$.
        Dies ist ein Grenzwert vom Typ '$\frac{\infty}{\infty}$'. Die Exponentialfunktion $e^x$ im Nenner wächst für $x \to \infty$ schneller als jede Potenzfunktion $x^n$ im Zähler, unabhängig vom Grad $n$ des Polynoms (solange $n$ endlich ist).
        Man kann dies formal zeigen, indem man die Regel von L'Hôpital $n$-mal anwendet:
        $$ \lim_{x \to \infty} \frac{x^n}{e^x} \stackrel{L'H}{=} \lim_{x \to \infty} \frac{nx^{n-1}}{e^x} \stackrel{L'H}{=} \lim_{x \to \infty} \frac{n(n-1)x^{n-2}}{e^x} \stackrel{L'H}{=} \dots \stackrel{L'H}{=} \lim_{x \to \infty} \frac{n!}{e^x} $$
        Da $n!$ eine Konstante ist und $e^x \to \infty$, ist der Grenzwert $\frac{n!}{\infty} = 0$.
        Die Funktion, die hier das Verhalten für große $x$ 'dominiert', ist die \textbf{Exponentialfunktion $e^x$} im Nenner. Ihre Zunahme ist so stark, dass sie das Wachstum der Potenzfunktion $x^n$ 'überwältigt' und den gesamten Bruch gegen Null treibt.
    \end{itemize}
\end{enumerate}

\end{loesungsumgebung}





\begin{aufgabenumgebung}{Checkliste: Exponentialfunktionen in der Analysis anwenden}
Die Analyse von Funktionen mit Exponentialtermen erfordert oft eine Kombination verschiedener Werkzeuge.

\begin{enumerate}[label=(\alph*)]
    \item \textbf{Nullstellen von $f(x) = P(x) \cdot e^{g(x)}$:}
    Wenn du die Nullstellen einer Funktion suchen sollest, die ein Produkt aus einem Polynom $P(x)$ und einem Exponentialterm $e^{g(x)}$ ist:
    \begin{itemize}
        \item Warum kannst du dich bei der Nullstellensuche ausschließlich auf den Faktor $P(x)$ konzentrieren?
        \item Gilt eine ähnliche Regel auch für die Nullstellensuche der Ableitung $f'(x)$ solcher Funktionen? (Tipp: Leite $P(x)e^{g(x)}$ allgemein mit Produkt- und Kettenregel ab und schaue, ob du $e^{g(x)}$ ausklammern kannst).
    \end{itemize}
    \item \textbf{Extremwertbestimmung bei $e$-Funktionen:}
    Betrachte die Funktion $f(x) = x^2 e^{-x}$.
    \begin{itemize}
        \item Welche Ableitungsregeln benötigst du, um $f'(x)$ und $f''(x)$ zu bilden?
        \item Wie gehst du vor, um die lokalen Extrempunkte zu finden? Erkläre die notwendige und die hinreichende Bedingung.
        \item Was erwartest du für das Verhalten von $f(x)$ für $x \to \infty$ und $x \to -\infty$? Wie hilft dir das, deine gefundenen Extrema einzuordnen (lokal vs. global)?
    \end{itemize}
    \item \textbf{Integrationstechniken im Kontext von $e$-Funktionen:}
    \begin{itemize}
        \item Für das Integral $\int (2x) e^{x^2} dx$: Welche Integrationstechnik bietet sich hier an und warum? Identifiziere die 'innere Funktion' und ihre Ableitung.
        \item Für das Integral $\int (x+1) e^x dx$: Welche Integrationstechnik ist hier typischerweise erfolgreich und warum? Wie würdest du die Faktoren für diese Technik wählen?
        \item Kannst du $\int e^{x^2} dx$ mit den in diesem Kapitel gelernten Methoden als elementare Funktion darstellen? (Tipp: Nicht jede Funktion hat eine einfach darstellbare Stammfunktion.)
    \end{itemize}
    \item \textbf{Interpretation im Anwendungskontext (Wachstum/Zerfall):}
    Ein Medikament wird im Körper exponentiell abgebaut, z.B. nach $M(t) = M_0 e^{-kt}$.
    \begin{itemize}
        \item Was bedeutet $M_0$ und was bedeutet ein größeres $k$ für den Abbauprozess?
        \item Die Ableitung $M'(t)$ gibt die momentane Abbaurate an. Ist $M'(t)$ positiv oder negativ? Warum ist das sinnvoll? Was passiert mit der Abbaurate für große $t$?
        \item Das bestimmte Integral $\int_{t_1}^{t_2} M(t) dt$ könnte als eine Art 'kumulierte Medikamentenbelastung' im Zeitintervall $[t_1, t_2]$ interpretiert werden. Welche Einheit hätte dieses Integral, wenn $M(t)$ in mg und $t$ in Stunden gemessen wird?
    \end{itemize}
\end{enumerate}
\end{aufgabenumgebung}



\begin{loesungsumgebung}[loes:checkliste-ex-anwendung]{Checkliste: Exponentialfunktionen in der Analysis anwenden}

\begin{enumerate}[label=(\alph*)]
    \item \textbf{Nullstellen von $f(x) = P(x) \cdot e^{g(x)}$:}
    \begin{itemize}
        \item \textbf{Konzentration auf $P(x)$ bei Nullstellensuche:}
        Der Exponentialterm $e^{g(x)}$ ist für alle reellen Werte von $g(x)$ immer positiv ($e^{g(x)} > 0$). Ein Produkt ist genau dann Null, wenn mindestens einer seiner Faktoren Null ist. Da $e^{g(x)}$ niemals Null werden kann, müssen die Nullstellen der Gesamtfunktion $f(x)$ ausschließlich aus den Nullstellen des Faktors $P(x)$ stammen. Man löst also die Gleichung $P(x)=0$.

        \item \textbf{Ähnliche Regel für Nullstellensuche der Ableitung $f'(x)$:}
        Ja, eine ähnliche Regel gilt. Leiten wir $f(x) = P(x)e^{g(x)}$ mit der Produkt- und Kettenregel ab:
        $u(x) = P(x) \Rightarrow u'(x) = P'(x)$.
        $v(x) = e^{g(x)} \Rightarrow v'(x) = e^{g(x)} \cdot g'(x)$ (Kettenregel).
        $f'(x) = u'(x)v(x) + u(x)v'(x) = P'(x)e^{g(x)} + P(x)e^{g(x)}g'(x)$.
        Man kann den Faktor $e^{g(x)}$ ausklammern:
        $f'(x) = e^{g(x)} [P'(x) + P(x)g'(x)]$.
        Um die Nullstellen von $f'(x)$ zu finden ($f'(x)=0$), kann man sich wieder darauf konzentrieren, wann der Klammerausdruck Null wird, da $e^{g(x)}$ immer positiv ist:
        $P'(x) + P(x)g'(x) = 0$.
    \end{itemize}

    \item \textbf{Extremwertbestimmung bei $f(x) = x^2 e^{-x}$:}
    \begin{itemize}
        \item \textbf{Benötigte Ableitungsregeln:}
        Für $f'(x)$: Produktregel und Kettenregel (für $e^{-x}$).
        Für $f''(x)$: Erneute Anwendung der Produktregel und Kettenregel auf die Terme von $f'(x)$.
        \item \textbf{Vorgehen zur Extrempunktfindung:}
        \begin{enumerate}
            \item \textbf{Notwendige Bedingung:} Erste Ableitung $f'(x)$ bilden und Null setzen ($f'(x)=0$). Die Lösungen $x_E$ sind die Kandidaten für Extremstellen.
            $f'(x) = 2xe^{-x} + x^2(-e^{-x}) = e^{-x}(2x-x^2) = xe^{-x}(2-x)$.
            $f'(x)=0 \Rightarrow xe^{-x}(2-x)=0 \Rightarrow x_1=0, x_2=2$.
            \item \textbf{Hinreichende Bedingung:}
            Entweder das Vorzeichenwechselkriterium von $f'(x)$ an den Stellen $x_E$ untersuchen oder die zweite Ableitung $f''(x_E)$ verwenden.
            $f''(x) = (2-2x)e^{-x} + (2x-x^2)(-e^{-x}) = e^{-x}(2-2x - (2x-x^2)) = e^{-x}(x^2-4x+2)$.
            Für $x_1=0$: $f''(0) = e^0(0-0+2) = 2 > 0 \Rightarrow$ Lokaler Tiefpunkt.
            Für $x_2=2$: $f''(2) = e^{-2}(4-8+2) = -2e^{-2} < 0 \Rightarrow$ Lokaler Hochpunkt.
            \item Die y-Koordinaten sind $f(0)=0$ und $f(2)=2^2e^{-2}=4e^{-2}$.
            Extrempunkte: $\mathbf{TP(0|0)}$ und $\mathbf{HP(2|4e^{-2})}$.
        \end{itemize}
        \item \textbf{Verhalten von $f(x)$ für $x \to \infty$ und $x \to -\infty$ und Einordnung der Extrema:}
        $\lim_{x \to \infty} x^2e^{-x} = \lim_{x \to \infty} \frac{x^2}{e^x} = 0$ (e-Funktion dominiert).
        $\lim_{x \to -\infty} x^2e^{-x}$: Sei $u=-x \Rightarrow u \to \infty$. $\lim_{u \to \infty} (-u)^2e^u = \lim_{u \to \infty} u^2e^u = \infty$.
        \textit{Einordnung:}
        Da $f(x) = x^2e^{-x} \ge 0$ für alle $x$ (da $x^2 \ge 0$ und $e^{-x} > 0$), ist der Tiefpunkt $TP(0|0)$ ein \textbf{globales Minimum}.
        Der Hochpunkt $HP(2|4e^{-2} \approx 0.54)$ ist ein \textbf{lokales Maximum}. Da die Funktion für $x \to -\infty$ gegen $\infty$ strebt, gibt es kein globales Maximum.
    \end{itemize}

    \item \textbf{Integrationstechniken im Kontext von $e$-Funktionen:}
    \begin{itemize}
        \item \textbf{Für $\int (2x) e^{x^2} dx$:}
        Die geeignete Technik ist die \textbf{Integration durch Substitution}.
        \textit{Grund:} Der Integrand hat die Form $k \cdot i'(x) \cdot e^{i(x)}$.
        Die 'innere Funktion' im Exponenten ist $i(x) = x^2$. Ihre Ableitung ist $i'(x) = 2x$. Dieser Faktor (oder ein Vielfaches) $2x$ ist als Faktor im Integranden vorhanden.
        Mit $u=x^2 \Rightarrow du = 2x \,dx$. Das Integral wird zu $\int e^u \,du = e^u + C = e^{x^2} + C$.

        \item \textbf{Für $\int (x+1) e^x dx$:}
        Die typischerweise erfolgreiche Technik ist die \textbf{partielle Integration} ($\int f g' = fg - \int f' g$).
        \textit{Grund:} Es handelt sich um ein Produkt aus einem Polynom $(x+1)$ und einer Exponentialfunktion $e^x$.
        \textit{Faktorwahl:} Man wählt den Polynomterm zum Ableiten, da er sich dadurch vereinfacht, und den Exponentialterm zum Integrieren, da er dabei einfach bleibt.
        $f(x) = x+1 \Rightarrow f'(x)=1$.
        $g'(x) = e^x \Rightarrow g(x)=e^x$.
        $\int (x+1)e^x dx = (x+1)e^x - \int 1 \cdot e^x dx = (x+1)e^x - e^x + C = xe^x + C$.

        \item \textbf{Kannst du $\int e^{x^2} dx$ mit den in diesem Kapitel gelernten Methoden als elementare Funktion darstellen?}
        \textbf{Nein}, die Stammfunktion von $e^{x^2}$ ist \textbf{nicht durch elementare Funktionen darstellbar}. Elementare Funktionen sind Polynome, rationale, trigonometrische, Exponential- und Logarithmusfunktionen sowie deren Verknüpfungen. Das Integral $\int e^{x^2} dx$ ist eng mit der (nicht-elementaren) Fehlerfunktion (Error Function) verbunden, die in der Wahrscheinlichkeitsrechnung und Statistik eine Rolle spielt.
    \end{itemize}

    \item \textbf{Interpretation im Anwendungskontext (Wachstum/Zerfall): $M(t) = M_0 e^{-kt}$}
    \begin{itemize}
        \item \textbf{Bedeutung $M_0$ und größeres $k$:}
        \begin{itemize}
            \item $\mathbf{M_0}$: Ist die Anfangsmenge der Substanz zum Zeitpunkt $t=0$, da $M(0) = M_0 e^0 = M_0$.
            \item \textbf{Größeres $\mathbf{k}$ ($k>0$):} Die Konstante $k$ ist die Zerfallskonstante. Ein größeres $k$ bedeutet einen \textbf{schnelleren Abbauprozess}. Die Substanz zerfällt rascher, und die Halbwertszeit $T_H = \frac{\ln 2}{k}$ wird kleiner.
        \end{itemize}
        \item \textbf{Ableitung $M'(t)$: positiv/negativ? Sinnvoll? Verhalten für große $t$?}
        $M'(t) = \frac{d}{dt}(M_0 e^{-kt}) = M_0 \cdot e^{-kt} \cdot (-k) = -kM_0e^{-kt}$.
        \begin{itemize}
            \item Da $M_0 > 0$ (Anfangsmenge), $k > 0$ (Zerfallskonstante) und $e^{-kt} > 0$, ist $M'(t)$ \textbf{immer negativ} für $t \ge 0$.
            \item \textit{Sinnvoll?:} Ja, das ist sinnvoll. Es beschreibt einen Zerfallsprozess, bei dem die Menge der Substanz mit der Zeit abnimmt. Eine negative Ableitung bedeutet eine negative Änderungsrate, also eine Abnahme.
            \item \textit{Verhalten für große $t$ ($\lim_{t \to \infty} M'(t)$):}
            $\lim_{t \to \infty} -kM_0e^{-kt} = -kM_0 \lim_{t \to \infty} \frac{1}{e^{kt}} = -kM_0 \cdot 0 = 0$.
            Die Abbaurate geht für große Zeiten gegen Null. Das bedeutet, der Zerfall verlangsamt sich, je weniger Substanz noch vorhanden ist.
        \end{itemize}
        \item \textbf{Einheit von $\int_{t_1}^{t_2} M(t) dt$, wenn $M(t)$ in mg und $t$ in Stunden?}
        Die Einheit des bestimmten Integrals ist das Produkt der Einheit der Funktion $M(t)$ und der Einheit der Integrationsvariablen $t$.
        Einheit von $M(t)$: mg.
        Einheit von $t$ (und somit $dt$): Stunden.
        Einheit von $\int M(t) dt$: $\mathbf{mg \cdot Stunden}$ (Milligramm-Stunden).
        Dieser Wert kann als eine Art 'kumulierte Medikamentenbelastung' oder die 'Gesamtmenge an Medikamentenwirkung' über das Zeitintervall $[t_1, t_2]$ interpretiert werden.
    \end{itemize}
\end{enumerate}

\end{loesungsumgebung}
\section{Logarithmusfunktionen – Die Welt des 'Zurückrechnens'}
\label{sec:logarithmusfunktionen_intro}

\begin{aufgabenumgebung}{Rechnen mit Logarithmen}
\begin{enumerate}
    \item Vereinfache die folgenden Terme so weit wie möglich (alle Variablen seien positiv):
        \begin{itemize}
            \item $\ln(x^3) + \ln(x^2)$
            \item $\ln(a^5) - \ln(a^2)$
            \item $2\ln(u) - 3\ln(v)$
            \item $\frac{1}{2}\ln(16x^4)$
        \end{itemize}
    \item Löse die folgenden Exponentialgleichungen nach $x$. Gib das Ergebnis sowohl exakt als auch als Dezimalzahl (auf 3 Nachkommastellen gerundet) an.
        \begin{itemize}
            \item $e^x = 20$
            \item $2e^{3x} = 18$
            \item $e^{-0.5x+1} = 5$
            \item $100 \cdot (0.7)^x = 10$ (Tipp: Erst isolieren, dann logarithmieren. Du kannst hier $\ln$ verwenden, obwohl die Basis $0.7$ ist.)
        \end{itemize}
\end{enumerate}
\end{aufgabenumgebung}




\begin{loesungsumgebung}[loes:rechnen-mit-logarithmen]{Rechnen mit Logarithmen}

\begin{enumerate}[label=(\alph*)]
    \item \textbf{Vereinfache die folgenden Terme so weit wie möglich (alle Variablen seien positiv):}
    \begin{itemize}
        \item $\mathbf{\ln(x^3) + \ln(x^2)}$
        $$ \ln(x^3) + \ln(x^2) = \ln(x^3 \cdot x^2) = \ln(x^{3+2}) = \ln(x^5) = \mathbf{5\ln(x)} $$
        Oder alternativ: $3\ln(x) + 2\ln(x) = (3+2)\ln(x) = 5\ln(x)$.

        \item $\mathbf{\ln(a^5) - \ln(a^2)}$
        $$ \ln(a^5) - \ln(a^2) = \ln\left(\frac{a^5}{a^2}\right) = \ln(a^{5-2}) = \ln(a^3) = \mathbf{3\ln(a)} $$
        Oder alternativ: $5\ln(a) - 2\ln(a) = (5-2)\ln(a) = 3\ln(a)$.

        \item $\mathbf{2\ln(u) - 3\ln(v)}$
        $$ 2\ln(u) - 3\ln(v) = \ln(u^2) - \ln(v^3) = \mathbf{\ln\left(\frac{u^2}{v^3}\right)} $$

        \item $\mathbf{\frac{1}{2}\ln(16x^4)}$
        \begin{align*} \frac{1}{2}\ln(16x^4) &= \frac{1}{2}(\ln(16) + \ln(x^4)) \\ &= \frac{1}{2}(\ln(2^4) + 4\ln(x)) \\ &= \frac{1}{2}(4\ln(2) + 4\ln(x)) \\ &= \mathbf{2\ln(2) + 2\ln(x)} \quad (\text{oder } 2(\ln(2) + \ln(x)) \text{ oder } \ln(4x^2)) \end{align*}
        Da $x>0$, ist $\ln(4x^2) = \ln(4) + \ln(x^2) = 2\ln 2 + 2\ln x$.
    \end{itemize}

    \item \textbf{Löse die folgenden Exponentialgleichungen nach $x$. Gib das Ergebnis sowohl exakt als auch als Dezimalzahl (auf 3 Nachkommastellen gerundet) an.}
    \begin{itemize}
        \item $\mathbf{e^x = 20}$
        $$
        \begin{array}{r c l c l}
        \umformung{e^x}{20}{\ln(\ldots)}{}
        \umformung{x}{\ln(20)}{}{}
        \end{array}
        $$
        Exakte Lösung: $x = \ln(20)$.
        Dezimalzahl: $x \approx \mathbf{2.996}$.

        \item $\mathbf{2e^{3x} = 18}$
        $$
        \begin{array}{r c l c l}
        \umformung{2e^{3x}}{18}{\div}{2}
        \umformung{e^{3x}}{9}{\ln(\ldots)}{}
        \umformung{3x}{\ln(9)}{\div}{3}
        \umformungend{x}{\frac{\ln(9)}{3}}
        \end{array}
        $$
        Exakte Lösung: $x = \frac{\ln(9)}{3} = \frac{\ln(3^2)}{3} = \frac{2\ln(3)}{3}$.
        Dezimalzahl: $x \approx \frac{2 \cdot 1.098612}{3} \approx \frac{2.197224}{3} \approx \mathbf{0.732}$.

        \item $\mathbf{e^{-0.5x+1} = 5}$
        $$
        \begin{array}{r c l c l}
        \umformung{e^{-0.5x+1}}{5}{\ln(\ldots)}{}
        \umformung{-0.5x+1}{\ln(5)}{-}{1}
        \umformung{-0.5x}{\ln(5)-1}{\div}{(-0.5)}
        \umformungend{x}{\frac{\ln(5)-1}{-0.5}}
        \end{array}
        $$
        Exakte Lösung: $x = \frac{\ln(5)-1}{-0.5} = -2(\ln(5)-1) = 2(1-\ln(5))$.
        Dezimalzahl: $x \approx 2(1-1.609438) = 2(-0.609438) \approx \mathbf{-1.219}$.

        \item $\mathbf{100 \cdot (0.7)^x = 10}$
        $$
        \begin{array}{r c l c l}
        \umformung{100 \cdot (0.7)^x}{10}{\div}{100}
        \umformung{(0.7)^x}{0.1}{\ln(\ldots)}{}
        \umformung{x \ln(0.7)}{\ln(0.1)}{\div}{\ln(0.7)}
        \umformungend{x}{\frac{\ln(0.1)}{\ln(0.7)}}
        \end{array}
        $$
        Exakte Lösung: $x = \frac{\ln(0.1)}{\ln(0.7)}$.
        Dezimalzahl: $x \approx \frac{-2.302585}{-0.356675} \approx \mathbf{6.456}$.
    \end{itemize}
\end{enumerate}

\end{loesungsumgebung}



\begin{aufgabenumgebung}{Ableiten und Integrieren mit \texorpdfstring{$\ln(x)$}{ln(x)}}
\begin{enumerate}
    \item Bilde die erste Ableitung der folgenden Funktionen:
        \begin{itemize}
            \item $f_1(x) = -2\ln(x) + e^x$
            \item $f_2(x) = x^3 \ln(x)$ (Produktregel!)
            \item $f_3(x) = \frac{\ln(x)}{x}$ (Quotientenregel!)
        \end{itemize}
    \item Bestimme die Menge aller Stammfunktionen:
        \begin{itemize}
            \item $g_1(x) = \frac{3}{x} - 2x + 1$
            \item $g_2(x) = \frac{x^2+x-1}{x}$ (Tipp: Den Bruch zuerst aufteilen!)
        \end{itemize}
\end{enumerate}
\end{aufgabenumgebung}



\begin{loesungsumgebung}[loes:ableiten-integrieren-lnx]{Ableiten und Integrieren mit \texorpdfstring{$\ln(x)$}{ln(x)}}

\begin{enumerate}[label=(\alph*)]
    \item \textbf{Bilde die erste Ableitung der folgenden Funktionen:}
    Der Definitionsbereich für Funktionen mit $\ln(x)$ ist $x>0$.
    \begin{itemize}
        \item \textbf{$f_1(x) = -2\ln(x) + e^x$} \\
        Definitionsbereich: $x>0$.
        $$ f_1'(x) = \frac{d}{dx}(-2\ln(x) + e^x) = -2 \cdot \frac{1}{x} + e^x = \mathbf{-\frac{2}{x} + e^x} $$

        \item \textbf{$f_2(x) = x^3 \ln(x)$} (Produktregel!) \\
        Definitionsbereich: $x>0$.
        Sei $u(x) = x^3 \Rightarrow u'(x) = 3x^2$. \\
        Sei $v(x) = \ln(x) \Rightarrow v'(x) = \frac{1}{x}$.
        \begin{align*}
        f_2'(x) &= u'(x)v(x) + u(x)v'(x) \\
                &= 3x^2 \cdot \ln(x) + x^3 \cdot \frac{1}{x} \\
                &= 3x^2 \ln(x) + x^2 \\
                &= \mathbf{x^2(3\ln(x) + 1)}
        \end{align*}

        \item \textbf{$f_3(x) = \frac{\ln(x)}{x}$} (Quotientenregel!) \\
        Definitionsbereich: $x>0$.
        Sei $u(x) = \ln(x) \Rightarrow u'(x) = \frac{1}{x}$. \\
        Sei $v(x) = x \Rightarrow v'(x) = 1$.
        \begin{align*}
        f_3'(x) &= \frac{u'(x)v(x) - u(x)v'(x)}{[v(x)]^2} \\
                &= \frac{\frac{1}{x} \cdot x - \ln(x) \cdot 1}{x^2} \\
                &= \mathbf{\frac{1 - \ln(x)}{x^2}}
        \end{align*}
    \end{itemize}

    \item \textbf{Bestimme die Menge aller Stammfunktionen:}
    Wir nehmen für die folgenden Aufgaben an, dass der Definitionsbereich $x>0$ ist, um $\ln(x)$ anstelle von $\ln|x|$ zu verwenden, passend zum Thema $\ln(x)$.
    \begin{itemize}
        \item \textbf{$g_1(x) = \frac{3}{x} - 2x + 1$} \\
        \begin{align*} G_1(x) &= \int \left(\frac{3}{x} - 2x + 1\right) \,dx \\ &= 3\int \frac{1}{x} \,dx - 2\int x \,dx + \int 1 \,dx \\ &= 3\ln(x) - 2\frac{x^2}{2} + x + C \\ &= \mathbf{3\ln(x) - x^2 + x + C} \quad (\text{für } x>0) \end{align*}

        \item \textbf{$g_2(x) = \frac{x^2+x-1}{x}$} (Tipp: Den Bruch zuerst aufteilen!) \\
        Für $x \neq 0$: $g_2(x) = \frac{x^2}{x} + \frac{x}{x} - \frac{1}{x} = x + 1 - \frac{1}{x}$.
        \begin{align*} G_2(x) &= \int \left(x + 1 - \frac{1}{x}\right) \,dx \\ &= \int x \,dx + \int 1 \,dx - \int \frac{1}{x} \,dx \\ &= \frac{x^2}{2} + x - \ln(x) + C \\ &= \mathbf{\frac{1}{2}x^2 + x - \ln(x) + C} \quad (\text{für } x>0) \end{align*}
    \end{itemize}
\end{enumerate}

\end{loesungsumgebung}



\begin{aufgabenumgebung}{Kettenregel mit \texorpdfstring{$\ln(h(x))$}{ln(h(x))} üben}
Bilde die erste Ableitung der folgenden Funktionen. Gib auch den maximalen Definitionsbereich an.
\begin{enumerate}
    \item $f_1(x) = \ln(5x)$
    \item $f_2(x) = \ln(x^3+x)$ (für $x>0$)
    \item $f_3(x) = x \cdot \ln(2x+1)$ (Produkt- und Kettenregel!)
    \item $f_4(x) = e^{\ln(x^2)}$ (Tipp: Vereinfache zuerst mit den Logarithmus-/Exponentialgesetzen! Was ist $e^{\ln A}$?)
\end{enumerate}
\end{aufgabenumgebung}

\begin{loesungsumgebung}[loes:kettenregel-ln-h-von-x]{Kettenregel mit \texorpdfstring{$\ln(h(x))$}{ln(h(x))} üben}
Wir bilden die erste Ableitung der gegebenen Funktionen und geben den maximalen Definitionsbereich an.

\begin{enumerate}[label=(\alph*)]
    \item \textbf{Funktion $f_1(x) = \ln(5x)$}
    \begin{itemize}
        \item \textbf{Maximaler Definitionsbereich:}
        Damit $\ln(5x)$ definiert ist, muss $5x > 0$ gelten, was $x > 0$ bedeutet.
        $D_{f1} = (0, \infty)$.
        \item \textbf{Erste Ableitung:}
        Wir verwenden die Kettenregel. Äußere Funktion $a(u) = \ln(u) \Rightarrow a'(u) = \frac{1}{u}$.
        Innere Funktion $i(x) = 5x \Rightarrow i'(x) = 5$.
        $$ f_1'(x) = a'(i(x)) \cdot i'(x) = \frac{1}{5x} \cdot 5 = \frac{5}{5x} = \mathbf{\frac{1}{x}} $$
        \textit{Alternative durch Logarithmengesetze vor dem Ableiten:}
        $f_1(x) = \ln(5x) = \ln(5) + \ln(x)$.
        $f_1'(x) = \frac{d}{dx}(\ln(5)) + \frac{d}{dx}(\ln(x)) = 0 + \frac{1}{x} = \frac{1}{x}$.
    \end{itemize}

    \item \textbf{Funktion $f_2(x) = \ln(x^3+x)$ (für $x>0$)}
    \begin{itemize}
        \item \textbf{Maximaler Definitionsbereich:}
        Damit $\ln(x^3+x)$ definiert ist, muss $x^3+x > 0 \Rightarrow x(x^2+1) > 0$.
        Da $x^2+1$ immer positiv ist, muss $x>0$ gelten. Dies ist bereits in der Aufgabenstellung gegeben.
        $D_{f2} = (0, \infty)$.
        \item \textbf{Erste Ableitung:}
        Äußere Funktion $a(u) = \ln(u) \Rightarrow a'(u) = \frac{1}{u}$.
        Innere Funktion $i(x) = x^3+x \Rightarrow i'(x) = 3x^2+1$.
        $$ f_2'(x) = a'(i(x)) \cdot i'(x) = \frac{1}{x^3+x} \cdot (3x^2+1) = \mathbf{\frac{3x^2+1}{x^3+x}} $$
        Dies kann auch als $\frac{3x^2+1}{x(x^2+1)}$ geschrieben werden.
    \end{itemize}

    \item \textbf{Funktion $f_3(x) = x \cdot \ln(2x+1)$} (Produkt- und Kettenregel!)
    \begin{itemize}
        \item \textbf{Maximaler Definitionsbereich:}
        Damit $\ln(2x+1)$ definiert ist, muss $2x+1 > 0 \Rightarrow 2x > -1 \Rightarrow x > -\frac{1}{2}$.
        $D_{f3} = (-\frac{1}{2}, \infty)$.
        \item \textbf{Erste Ableitung:}
        Wir verwenden die Produktregel: $(uv)' = u'v + uv'$.
        Sei $u(x) = x \Rightarrow u'(x) = 1$.
        Sei $v(x) = \ln(2x+1)$. Für $v'(x)$ verwenden wir die Kettenregel:
        Äußere Funktion $a_v(w) = \ln(w) \Rightarrow a_v'(w) = \frac{1}{w}$.
        Innere Funktion $i_v(x) = 2x+1 \Rightarrow i_v'(x) = 2$.
        $v'(x) = \frac{1}{2x+1} \cdot 2 = \frac{2}{2x+1}$.
        Nun die Produktregel:
        \begin{align*}
        f_3'(x) &= 1 \cdot \ln(2x+1) + x \cdot \frac{2}{2x+1} \\
                &= \mathbf{\ln(2x+1) + \frac{2x}{2x+1}}
        \end{align*}
    \end{itemize}

    \item \textbf{Funktion $f_4(x) = e^{\ln(x^2)}$} (Tipp: Vereinfache zuerst!)
    \begin{itemize}
        \item \textbf{Maximaler Definitionsbereich:}
        Damit $\ln(x^2)$ definiert ist, muss $x^2 > 0$, was bedeutet $x \neq 0$.
        $D_{f4} = \mathbb{R} \setminus \{0\}$.
        \item \textbf{Vereinfachung:}
        Nach den Logarithmus-/Exponentialgesetzen gilt $e^{\ln A} = A$ (für $A>0$).
        Hier ist $A=x^2$. Da $x \neq 0$, ist $x^2 > 0$.
        Also $f_4(x) = e^{\ln(x^2)} = x^2$ (für $x \neq 0$).
        \item \textbf{Erste Ableitung:}
        $f_4(x) = x^2$ (mit $D_{f4} = \mathbb{R} \setminus \{0\}$).
        $$ f_4'(x) = \frac{d}{dx}(x^2) = \mathbf{2x} $$
        Die Ableitung $2x$ ist für alle $x \in D_{f4}$ definiert.
        \textit{Alternative ohne Vereinfachung (zur Übung der Kettenregel):}
        $f_4(x) = e^{\ln(x^2)}$.
        Äußere Funktion $a(u) = e^u \Rightarrow a'(u) = e^u$.
        Innere Funktion $i(x) = \ln(x^2)$. Für $i'(x)$ (erneut Kettenregel, oder $2\ln|x|$):
        $i'(x) = \frac{1}{x^2} \cdot (2x) = \frac{2x}{x^2} = \frac{2}{x}$ (für $x \neq 0$).
        $f_4'(x) = a'(i(x)) \cdot i'(x) = e^{\ln(x^2)} \cdot \frac{2}{x}$.
        Da $e^{\ln(x^2)} = x^2$:
        $f_4'(x) = x^2 \cdot \frac{2}{x} = 2x$ (für $x \neq 0$).
        Beide Wege führen zum selben Ergebnis. Die Vereinfachung zu Beginn ist hier deutlich vorteilhafter.
    \end{itemize}
\end{enumerate}

\end{loesungsumgebung}

%nur eine abbildung, also alle graphen in eine abbildung
\begin{aufgabenumgebung}{Kurvendiskussionen mit Logarithmusfunktionen}
Führe eine vollständige Kurvendiskussion für die folgenden Funktionen durch.
\begin{enumerate}
    \item $f(x) = \ln(x^2+2)$ (Beachte den Definitionsbereich!)
    \item $g(x) = \frac{\ln x}{x}$ (für $x>0$)
    \item \textbf{Für Experten:} $h(x) = x^2 \ln(x)$ (für $x>0$)
\end{enumerate}
\end{aufgabenumgebung}


\begin{loesungsumgebung}[loes:kurvendiskussion-ln-funktionen]{Kurvendiskussionen mit Logarithmusfunktionen}
Wir führen für jede der folgenden Funktionen eine vollständige Kurvendiskussion durch. Die Graphen aller drei Funktionen sind in Abbildung \ref{fig:kurvendiskussion_ln_kombiniert} am Ende dieser Lösung zusammengefasst.

\begin{enumerate}[label=(\alph*)]
    \item \textbf{Funktion $f(x) = \ln(x^2+2)$}

    \subsubsection*{1. Definitionsbereich ($D_f$)}
    Der Ausdruck im Logarithmus, $x^2+2$, muss positiv sein. Da $x^2 \ge 0$ für alle reellen $x$, ist $x^2+2 \ge 2$ und somit immer positiv.
    $D_f = \mathbb{R}$.

    \subsubsection*{2. Symmetrie}
    $f(-x) = \ln((-x)^2+2) = \ln(x^2+2) = f(x)$.
    Die Funktion ist \textbf{achsensymmetrisch zur y-Achse}.

    \subsubsection*{3. Verhalten im Unendlichen}
    Für $x \to \pm\infty$ geht $x^2+2 \to \infty$. Da $\ln(u) \to \infty$ für $u \to \infty$, gilt:
    $\lim_{x \to \pm\infty} f(x) = \mathbf{\infty}$. Keine horizontalen Asymptoten.

    \subsubsection*{4. Achsenschnittpunkte}
    \begin{itemize}
        \item y-Achsenabschnitt ($x=0$): $f(0) = \ln(0^2+2) = \ln(2)$. $P_y(0|\ln 2 \approx 0.693)$.
        \item Nullstellen ($f(x)=0$): $\ln(x^2+2)=0 \Rightarrow x^2+2 = e^0 = 1 \Rightarrow x^2 = -1$.
        Diese Gleichung hat keine reellen Lösungen. Es gibt \textbf{keine Nullstellen}.
    \end{itemize}

    \subsubsection*{5. Erste Ableitung $f'(x)$}
    Mit der Kettenregel ($u=x^2+2, u'=2x$):
    $f'(x) = \frac{1}{x^2+2} \cdot 2x = \mathbf{\frac{2x}{x^2+2}}$.

    \subsubsection*{6. Extrempunkte}
    Notwendige Bedingung: $f'(x)=0 \Rightarrow \frac{2x}{x^2+2}=0$. Da $x^2+2 \neq 0$, muss $2x=0 \Rightarrow x_E=0$.
    $y_E = f(0) = \ln(2)$.
    Zweite Ableitung (siehe unten): $f''(0) = 1 > 0$.
    Lokaler (und wegen Symmetrie und Globalverhalten globaler) \textbf{Tiefpunkt bei $TP(0|\ln 2)$}.

    \subsubsection*{7. Monotonieverhalten}
    Vorzeichen von $f'(x)=\frac{2x}{x^2+2}$ (Nenner ist immer positiv):
    \begin{itemize}
        \item $x < 0 \Rightarrow 2x < 0 \Rightarrow f'(x) < 0 \Rightarrow f$ ist streng monoton fallend.
        \item $x > 0 \Rightarrow 2x > 0 \Rightarrow f'(x) > 0 \Rightarrow f$ ist streng monoton steigend.
    \end{itemize}

    \subsubsection*{8. Zweite Ableitung $f''(x)$}
    Mit der Quotientenregel für $f'(x)=\frac{2x}{x^2+2}$ ($u_1=2x, u_1'=2; v_1=x^2+2, v_1'=2x$):
    $f''(x) = \frac{2(x^2+2) - 2x(2x)}{(x^2+2)^2} = \frac{2x^2+4-4x^2}{(x^2+2)^2} = \mathbf{\frac{4-2x^2}{(x^2+2)^2}}$.

    \subsubsection*{9. Wendepunkte}
    Notwendige Bedingung: $f''(x_W)=0 \Rightarrow \frac{4-2x_W^2}{(x_W^2+2)^2}=0$.
    $4-2x_W^2=0 \Rightarrow 2x_W^2=4 \Rightarrow x_W^2=2 \Rightarrow x_W=\pm\sqrt{2}$.
    Dritte Ableitung: $f'''(x) = \frac{-4x(x^2+2)^2 - (4-2x^2) \cdot 2(x^2+2)(2x)}{((x^2+2)^2)^2} = \frac{-4x(x^2+2) - 4x(4-2x^2)}{(x^2+2)^3} = \frac{-4x^3-8x-16x+8x^3}{(x^2+2)^3} = \frac{4x^3-24x}{(x^2+2)^3}$.
    $f'''(\sqrt{2}) = \frac{4(2\sqrt{2})-24\sqrt{2}}{(2+2)^3} = \frac{8\sqrt{2}-24\sqrt{2}}{64} = \frac{-16\sqrt{2}}{64} \neq 0$.
    $f'''(-\sqrt{2}) = \frac{4(-2\sqrt{2})-24(-\sqrt{2})}{(2+2)^3} = \frac{-8\sqrt{2}+24\sqrt{2}}{64} = \frac{16\sqrt{2}}{64} \neq 0$.
    Wendepunkte bei $x_W=\pm\sqrt{2}$.
    $y_W = f(\pm\sqrt{2}) = \ln((\pm\sqrt{2})^2+2) = \ln(2+2) = \ln(4)$.
    $\mathbf{WP_{1,2}(\pm\sqrt{2}|\ln 4)}$. ($\ln 4 \approx 1.386$)

    \subsubsection*{10. Krümmungsverhalten}
    Vorzeichen von $f''(x)=\frac{4-2x^2}{(x^2+2)^2}$ hängt von $4-2x^2 = -2(x^2-2)$ ab (nach unten geöffnete Parabel).
    \begin{itemize}
        \item $x < -\sqrt{2}$: $4-2x^2 < 0 \Rightarrow f''(x) < 0 \Rightarrow f$ ist rechtsgekrümmt.
        \item $-\sqrt{2} < x < \sqrt{2}$: $4-2x^2 > 0 \Rightarrow f''(x) > 0 \Rightarrow f$ ist linksgekrümmt.
        \item $x > \sqrt{2}$: $4-2x^2 < 0 \Rightarrow f''(x) < 0 \Rightarrow f$ ist rechtsgekrümmt.
    \end{itemize}

    \subsubsection*{11. Wertebereich}
    $W_f = [\ln 2, \infty)$.

    \item \textbf{Funktion $g(x) = \frac{\ln x}{x}$ (für $x>0$)}

    \subsubsection*{1. Definitionsbereich}
    Gegeben $x>0$. Für $\ln x$ muss $x>0$ sein. Für den Nenner $x \neq 0$.
    $D_g = (0, \infty)$.

    \subsubsection*{2. Symmetrie}
    Der Definitionsbereich ist nicht symmetrisch zum Ursprung, daher keine einfache Symmetrie.

    \subsubsection*{3. Verhalten im Unendlichen und an den Rändern des Definitionsbereichs}
    \begin{itemize}
        \item Für $x \to \infty$: $\lim_{x \to \infty} \frac{\ln x}{x} \stackrel{L'H}{=} \lim_{x \to \infty} \frac{1/x}{1} = 0$.
        Horizontale Asymptote $\mathbf{y=0}$ für $x \to \infty$.
        \item Für $x \to 0^+$: $\lim_{x \to 0^+} \frac{\ln x}{x}$. Da $\ln x \to -\infty$ und $x \to 0^+$, ist der Grenzwert $\frac{-\infty}{0^+} = -\infty$.
        Senkrechte Asymptote $\mathbf{x=0}$ (y-Achse), der Graph geht gegen $-\infty$.
    \end{itemize}

    \subsubsection*{4. Achsenschnittpunkte}
    \begin{itemize}
        \item y-Achsenabschnitt: $x=0$ ist nicht im Definitionsbereich.
        \item Nullstellen ($g(x)=0$): $\frac{\ln x}{x}=0 \Rightarrow \ln x = 0 \Rightarrow x = e^0 = 1$.
        Nullstelle $\mathbf{N(1|0)}$.
    \end{itemize}

    \subsubsection*{5. Erste Ableitung $g'(x)$}
    Mit der Quotientenregel ($u=\ln x, u'=1/x; v=x, v'=1$):
    $g'(x) = \frac{(1/x)x - (\ln x)(1)}{x^2} = \mathbf{\frac{1-\ln x}{x^2}}$.

    \subsubsection*{6. Extrempunkte}
    $g'(x)=0 \Rightarrow \frac{1-\ln x}{x^2}=0 \Rightarrow 1-\ln x = 0 \Rightarrow \ln x = 1 \Rightarrow x_E=e$.
    $y_E = g(e) = \frac{\ln e}{e} = \frac{1}{e}$.
    Zweite Ableitung (siehe unten): $g''(e) = \frac{2\ln e - 3}{e^3} = \frac{2-3}{e^3} = -\frac{1}{e^3} < 0$.
    Lokaler (und globaler) \textbf{Hochpunkt bei $HP(e|1/e \approx 0.368)$}.

    \subsubsection*{7. Monotonieverhalten}
    Vorzeichen von $g'(x)=\frac{1-\ln x}{x^2}$ (Nenner $x^2>0$ für $x \in D_g$).
    \begin{itemize}
        \item $0 < x < e$: $\ln x < 1 \Rightarrow 1-\ln x > 0 \Rightarrow g'(x) > 0 \Rightarrow g$ ist streng monoton steigend.
        \item $x > e$: $\ln x > 1 \Rightarrow 1-\ln x < 0 \Rightarrow g'(x) < 0 \Rightarrow g$ ist streng monoton fallend.
    \end{itemize}

    \subsubsection*{8. Zweite Ableitung $g''(x)$}
    Mit der Quotientenregel für $g'(x)=\frac{1-\ln x}{x^2}$ ($u_1=1-\ln x, u_1'=-1/x; v_1=x^2, v_1'=2x$):
    $g''(x) = \frac{(-1/x)x^2 - (1-\ln x)(2x)}{(x^2)^2} = \frac{-x - 2x + 2x\ln x}{x^4} = \frac{-3x + 2x\ln x}{x^4} = \mathbf{\frac{2\ln x - 3}{x^3}}$.

    \subsubsection*{9. Wendepunkte}
    $g''(x_W)=0 \Rightarrow \frac{2\ln x_W - 3}{x_W^3}=0 \Rightarrow 2\ln x_W - 3 = 0 \Rightarrow \ln x_W = 3/2 \Rightarrow x_W = e^{3/2}$.
    $x_W = e^{1.5} \approx 4.482$.
    $g'''(x) = \frac{(2/x)x^3 - (2\ln x - 3)(3x^2)}{(x^3)^2} = \frac{2x^2 - 6x^2\ln x + 9x^2}{x^6} = \frac{11-6\ln x}{x^4}$.
    $g'''(e^{3/2}) = \frac{11-6(3/2)}{(e^{3/2})^4} = \frac{11-9}{e^6} = \frac{2}{e^6} \neq 0$. Wendepunkt.
    $y_W = g(e^{3/2}) = \frac{\ln(e^{3/2})}{e^{3/2}} = \frac{3/2}{e^{3/2}} = \frac{3}{2e^{3/2}} \approx 0.335$.
    $\mathbf{WP(e^{3/2}|\frac{3}{2e^{3/2}})}$.

    \subsubsection*{10. Krümmungsverhalten}
    Vorzeichen von $g''(x)=\frac{2\ln x - 3}{x^3}$ (Nenner $x^3>0$ für $x \in D_g$).
    \begin{itemize}
        \item $0 < x < e^{3/2}$: $\ln x < 3/2 \Rightarrow 2\ln x - 3 < 0 \Rightarrow g''(x) < 0 \Rightarrow g$ ist rechtsgekrümmt.
        \item $x > e^{3/2}$: $\ln x > 3/2 \Rightarrow 2\ln x - 3 > 0 \Rightarrow g''(x) > 0 \Rightarrow g$ ist linksgekrümmt.
    \end{itemize}

    \subsubsection*{11. Wertebereich}
    $W_g = (-\infty, 1/e]$.

    \item \textbf{Für Experten: Funktion $h(x) = x^2 \ln(x)$ (für $x>0$)}

    \subsubsection*{1. Definitionsbereich}
    Gegeben $x>0$. Für $\ln x$ muss $x>0$.
    $D_h = (0, \infty)$.

    \subsubsection*{2. Symmetrie}
    Der Definitionsbereich ist nicht symmetrisch zum Ursprung. Keine einfache Symmetrie.

    \subsubsection*{3. Verhalten im Unendlichen und an den Rändern des Definitionsbereichs}
    \begin{itemize}
        \item Für $x \to \infty$: $\lim_{x \to \infty} x^2 \ln x = \infty \cdot \infty = \infty$.
        \item Für $x \to 0^+$: $\lim_{x \to 0^+} x^2 \ln x$. Dies ist vom Typ $0 \cdot (-\infty)$.
        Umschreiben: $\lim_{x \to 0^+} \frac{\ln x}{x^{-2}} \stackrel{L'H}{=} \lim_{x \to 0^+} \frac{1/x}{-2x^{-3}} = \lim_{x \to 0^+} \frac{1/x}{-2/x^3} = \lim_{x \to 0^+} \frac{1}{x} \cdot \frac{x^3}{-2} = \lim_{x \to 0^+} \frac{x^2}{-2} = 0$.
        Der Graph startet im Ursprung $(0|0)$ (bzw. nähert sich diesem von rechts).
    \end{itemize}

    \subsubsection*{4. Achsenschnittpunkte}
    \begin{itemize}
        \item y-Achsenabschnitt: $x=0$ ist nicht im Definitionsbereich. $\lim_{x \to 0^+} h(x) = 0$.
        \item Nullstellen ($h(x)=0$): $x^2 \ln x = 0$. Da $x>0$, ist $x^2 \neq 0$.
        Also $\ln x = 0 \Rightarrow x = e^0 = 1$.
        Nullstelle $\mathbf{N(1|0)}$.
    \end{itemize}

    \subsubsection*{5. Erste Ableitung $h'(x)$}
    Mit der Produktregel ($u=x^2, u'=2x; v=\ln x, v'=1/x$):
    $h'(x) = 2x \ln x + x^2 \cdot \frac{1}{x} = 2x \ln x + x = \mathbf{x(2\ln x + 1)}$.

    \subsubsection*{6. Extrempunkte}
    $h'(x_E)=0 \Rightarrow x_E(2\ln x_E + 1) = 0$.
    Da $x_E>0$, muss $2\ln x_E + 1 = 0 \Rightarrow 2\ln x_E = -1 \Rightarrow \ln x_E = -1/2 \Rightarrow x_E = e^{-1/2} = 1/\sqrt{e}$.
    $x_E \approx 0.607$.
    $y_E = h(e^{-1/2}) = (e^{-1/2})^2 \ln(e^{-1/2}) = e^{-1} \cdot (-1/2) = -\frac{1}{2e}$.
    Zweite Ableitung (siehe unten): $h''(e^{-1/2}) = 2 > 0$.
    Lokaler (und globaler, da $\lim_{x\to 0^+} h(x)=0$ und $\lim_{x\to\infty} h(x)=\infty$) \textbf{Tiefpunkt bei $TP(e^{-1/2} | -1/(2e) \approx -0.184)$}.

    \subsubsection*{7. Monotonieverhalten}
    Vorzeichen von $h'(x)=x(2\ln x+1)$ ($x>0$). Hängt von $2\ln x+1$ ab.
    \begin{itemize}
        \item $0 < x < e^{-1/2}$: $\ln x < -1/2 \Rightarrow 2\ln x+1 < 0 \Rightarrow h'(x) < 0 \Rightarrow h$ ist streng monoton fallend.
        \item $x > e^{-1/2}$: $\ln x > -1/2 \Rightarrow 2\ln x+1 > 0 \Rightarrow h'(x) > 0 \Rightarrow h$ ist streng monoton steigend.
    \end{itemize}

    \subsubsection*{8. Zweite Ableitung $h''(x)$}
    $h'(x) = 2x\ln x + x$.
    $h''(x) = (2\ln x + 2x \cdot \frac{1}{x}) + 1 = 2\ln x + 2 + 1 = \mathbf{2\ln x + 3}$.

    \subsubsection*{9. Wendepunkte}
    $h''(x_W)=0 \Rightarrow 2\ln x_W + 3 = 0 \Rightarrow 2\ln x_W = -3 \Rightarrow \ln x_W = -3/2 \Rightarrow x_W = e^{-3/2}$.
    $x_W \approx 0.223$.
    Dritte Ableitung: $h'''(x) = 2/x$. $h'''(e^{-3/2}) = 2/e^{-3/2} = 2e^{3/2} \neq 0 \implies$ Wendepunkt.
    $y_W = h(e^{-3/2}) = (e^{-3/2})^2 \ln(e^{-3/2}) = e^{-3} \cdot (-3/2) = -\frac{3}{2e^3} \approx -0.075$.
    $\mathbf{WP(e^{-3/2}|-3/(2e^3))}$.

    \subsubsection*{10. Krümmungsverhalten}
    Vorzeichen von $h''(x)=2\ln x+3$.
    \begin{itemize}
        \item $0 < x < e^{-3/2}$: $\ln x < -3/2 \Rightarrow 2\ln x+3 < 0 \Rightarrow h''(x) < 0 \Rightarrow h$ ist rechtsgekrümmt.
        \item $x > e^{-3/2}$: $\ln x > -3/2 \Rightarrow 2\ln x+3 > 0 \Rightarrow h''(x) > 0 \Rightarrow h$ ist linksgekrümmt.
    \end{itemize}

    \subsubsection*{11. Wertebereich}
    $W_h = [-1/(2e), \infty)$.
\end{enumerate}

\subsubsection*{Gemeinsame Skizze der Graphen}
Eine aussagekräftige gemeinsame Skizze aller drei Funktionen $f(x)=\ln(x^2+2)$, $g(x)=\frac{\ln x}{x}$ und $h(x)=x^2\ln x$ in einem einzigen Koordinatensystem ist aufgrund der sehr unterschiedlichen Wertebereiche und Definitionsbereiche (insbesondere für $g(x)$ und $h(x)$ nur $x>0$) anspruchsvoll. Es wäre sinnvoller, jede Funktion einzeln oder $g(x)$ und $h(x)$ gemeinsam zu skizzieren und $f(x)$ separat.
\begin{itemize}
    \item $f(x)=\ln(x^2+2)$: Symmetrisch zur y-Achse, Minimum bei $(0|\ln 2)$, steigt beidseitig ins Unendliche, keine Nullstellen.
    \item $g(x)=\frac{\ln x}{x}$: Definiert für $x>0$. Startet bei $-\infty$ für $x \to 0^+$, Nullstelle bei $(1|0)$, Maximum bei $(e|1/e)$, nähert sich $y=0$ für $x \to \infty$.
    \item $h(x)=x^2\ln x$: Definiert für $x>0$. Startet bei $(0|0)$ (Grenzwert), Nullstelle bei $(1|0)$, Minimum bei $(e^{-1/2}|-1/(2e))$, steigt dann ins Unendliche.
\end{itemize}
Für eine kombinierte Skizze müsste man Kompromisse bei der Skalierung eingehen oder mehrere Detailausschnitte verwenden.

\begin{center}
\includegraphics[width=1\textwidth]{grafiken/kurvendiskussion_ln_kombiniert.png}
% --- Beschreibung der Skizze ---
% Die Skizze sollte versuchen, die drei Graphen darzustellen, ggf. mit unterschiedlichen y-Achsen oder in separaten Panels innerhalb einer Abbildung.
% f(x) = ln(x^2+2): Symmetrische Kurve, Min(0, ln2), keine Nst.
% g(x) = ln(x)/x: Für x>0, Asymptote x=0 (nach -unendlich), Nst(1,0), Max(e, 1/e), Asymptote y=0 (nach +unendlich).
% h(x) = x^2 ln(x): Für x>0, startet bei (0,0), Nst(1,0), Min(1/sqrt(e), -1/(2e)), geht nach +unendlich.
% Die Darstellung aller drei in einem sinnvollen gemeinsamen Maßstab ist herausfordernd.
\captionof{figure}{Skizze der Graphen von $f(x)=\ln(x^2+2)$, $g(x)=\frac{\ln x}{x}$ und $h(x)=x^2\ln x$ (konzeptionell).}
\label{fig:kurvendiskussion_ln_kombiniert}
\end{center}

\end{loesungsumgebung}



\begin{aufgabenumgebung}{Integration mit Logarithmusfunktionen üben}
\begin{enumerate}
    \item Berechne die folgenden unbestimmten Integrale:
        \begin{itemize}
            \item $\int (3\ln(x) - 2x) \,dx$
            \item $\int \frac{5x^4}{x^5+1} \,dx$
            \item $\int \frac{e^x}{e^x+3} \,dx$
            \item $\int \frac{1}{2x+7} \,dx$ (Tipp: Erweitere so, dass der Zähler die Ableitung des Nenners ist.)
        \end{itemize}
    \item \textbf{Flächenberechnung:}
        Die Funktion $f(x) = \frac{1}{x}$ ist im ersten Quadranten gegeben.
        \begin{itemize}
            \item Berechne den Inhalt der Fläche, die der Graph von $f(x)$ mit der x-Achse im Intervall $[1, e]$ einschließt.
            \item Berechne den Inhalt der Fläche, die der Graph von $f(x)$ mit der x-Achse im Intervall $[1, b]$ für ein beliebiges $b>1$ einschließt. Was passiert mit dieser Fläche, wenn $b \to \infty$? (Uneigentliches Integral)
        \end{itemize}
    \item \textbf{Anwendung partielle Integration (anspruchsvoller):}
        Berechne $\int x^2 \ln(x) \,dx$. (Tipp: Wähle $v(x)=\ln(x)$ und $u'(x)=x^2$).
\end{enumerate}
\end{aufgabenumgebung}



\begin{loesungsumgebung}[loes:integration-ln-funktionen-ueben]{Integration mit Logarithmusfunktionen üben}

\begin{enumerate}[label=(\alph*)]
    \item \textbf{Berechne die folgenden unbestimmten Integrale:}
    \begin{itemize}
        \item $\mathbf{\int (3\ln(x) - 2x) \,dx}$ \\
        Für diesen Integranden muss $x>0$ sein, damit $\ln(x)$ definiert ist.
        Wir verwenden $\int \ln(x) \,dx = x\ln(x) - x + C_0$ (herleitbar durch partielle Integration).
        \begin{align*} \int (3\ln(x) - 2x) \,dx &= 3\int \ln(x) \,dx - 2\int x \,dx \\ &= 3(x\ln(x) - x) - 2\frac{x^2}{2} + C \\ &= \mathbf{3x\ln(x) - 3x - x^2 + C} \end{align*}

        \item $\mathbf{\int \frac{5x^4}{x^5+1} \,dx}$ \\
        Dieser Integrand ist von der Form $k \cdot \frac{u'(x)}{u(x)}$.
        Sei $u = x^5+1$. Dann ist $\frac{du}{dx} = 5x^4$, also $du = 5x^4 \,dx$.
        \begin{align*} \int \frac{5x^4}{x^5+1} \,dx &= \int \frac{1}{x^5+1} \cdot (5x^4 \,dx) \\ &= \int \frac{1}{u} \,du \\ &= \ln|u| + C \end{align*}
        Rücksubstitution $u=x^5+1$:
        $$ \mathbf{\ln|x^5+1| + C} $$

        \item $\mathbf{\int \frac{e^x}{e^x+3} \,dx}$ \\
        Sei $u = e^x+3$. Dann ist $\frac{du}{dx} = e^x$, also $du = e^x \,dx$.
        \begin{align*} \int \frac{e^x}{e^x+3} \,dx &= \int \frac{1}{e^x+3} \cdot (e^x \,dx) \\ &= \int \frac{1}{u} \,du \\ &= \ln|u| + C \end{align*}
        Rücksubstitution $u=e^x+3$. Da $e^x > 0$ für alle $x$, ist $e^x+3 > 3$ und somit immer positiv. Der Betrag kann entfallen.
        $$ \mathbf{\ln(e^x+3) + C} $$

        \item $\mathbf{\int \frac{1}{2x+7} \,dx}$ \\
        Sei $u = 2x+7$. Dann ist $\frac{du}{dx} = 2$, also $du = 2 \,dx \Rightarrow dx = \frac{1}{2} \,du$.
        \begin{align*} \int \frac{1}{2x+7} \,dx &= \int \frac{1}{u} \cdot \frac{1}{2} \,du \\ &= \frac{1}{2} \int \frac{1}{u} \,du \\ &= \frac{1}{2}\ln|u| + C \end{align*}
        Rücksubstitution $u=2x+7$:
        $$ \mathbf{\frac{1}{2}\ln|2x+7| + C} $$
    \end{itemize}

    \item \textbf{Flächenberechnung mit $f(x) = \frac{1}{x}$ im ersten Quadranten:}
    Im ersten Quadranten ist $x>0$, somit ist $f(x)=\frac{1}{x} > 0$. Die Stammfunktion ist $\ln(x)$.
    \begin{itemize}
        \item \textbf{Flächeninhalt im Intervall $[1, e]$:}
        $$ A_1 = \int_1^e \frac{1}{x} \,dx = [\ln(x)]_1^e = \ln(e) - \ln(1) = 1 - 0 = \mathbf{1} $$
        Der Flächeninhalt beträgt $1$ Flächeneinheit.

        \item \textbf{Flächeninhalt im Intervall $[1, b]$ für $b>1$ und Grenzwert für $b \to \infty$:}
        $$ A_b = \int_1^b \frac{1}{x} \,dx = [\ln(x)]_1^b = \ln(b) - \ln(1) = \ln(b) - 0 = \mathbf{\ln(b)} $$
        Für $b \to \infty$:
        $$ \lim_{b \to \infty} A_b = \lim_{b \to \infty} \ln(b) = \mathbf{\infty} $$
        Die Fläche unter dem Graphen von $f(x)=\frac{1}{x}$ ab $x=1$ bis ins Unendliche ist unendlich groß (das uneigentliche Integral divergiert).
    \end{itemize}

    \item \textbf{Anwendung partielle Integration (anspruchsvoller): $\int x^2 \ln(x) \,dx$} \\
    Für $\ln(x)$ muss $x>0$ sein. Wir wählen (gemäß Tipp, $v(x)$ wird abgeleitet, $u'(x)$ wird integriert, was der Standardwahl $f(x)=\ln(x)$ und $g'(x)=x^2$ entspricht):
    \begin{itemize}
        \item $f(x) = \ln(x) \Rightarrow f'(x) = \frac{1}{x}$
        \item $g'(x) = x^2 \Rightarrow g(x) = \frac{x^3}{3}$
    \end{itemize}
    Anwendung der Formel $\int f(x)g'(x) \,dx = f(x)g(x) - \int f'(x)g(x) \,dx$:
    \begin{align*} \int x^2 \ln(x) \,dx &= \ln(x) \cdot \frac{x^3}{3} - \int \frac{1}{x} \cdot \frac{x^3}{3} \,dx \\ &= \frac{x^3}{3}\ln(x) - \int \frac{x^2}{3} \,dx \\ &= \frac{x^3}{3}\ln(x) - \frac{1}{3} \int x^2 \,dx \\ &= \frac{x^3}{3}\ln(x) - \frac{1}{3} \cdot \frac{x^3}{3} + C \\ &= \mathbf{\frac{x^3}{3}\ln(x) - \frac{x^3}{9} + C} \quad \text{oder} \quad \mathbf{\frac{x^3}{9}(3\ln(x) - 1) + C} \end{align*}
\end{enumerate}

\end{loesungsumgebung}








\begin{aufgabenumgebung}{Checkliste: Das Wesen des Logarithmus und der $\ln$-Funktion verstehen}
Der Logarithmus, insbesondere der natürliche Logarithmus $\ln(x)$, ist ein Schlüsselkonzept mit vielen Facetten. Diese Fragen helfen dir, die Grundlagen zu festigen:

\begin{enumerate}[label=(\alph*)]
    \item \textbf{Definition und Umkehrbeziehung:}
    \begin{itemize}
        \item Erkläre mit eigenen Worten, was $\log_b(y)=x$ bedeutet. Was ist die spezielle Basis beim natürlichen Logarithmus $\ln(x)$?
        \item Warum muss gelten: $e^{\ln(x)} = x$ (für $x>0$) und $\ln(e^x) = x$ (für alle $x \in \mathbb{R}$)? Erläutere dies am Konzept der Umkehrfunktion.
        \item Begründe, warum $\ln(1)=0$ und $\ln(e)=1$ sein muss.
    \end{itemize}
    \item \textbf{Definitionsbereich und Graph:}
    \begin{itemize}
        \item Warum ist der Definitionsbereich der Funktion $f(x)=\ln(x)$ auf positive reelle Zahlen ($x>0$) beschränkt? Wie hängt das mit dem Wertebereich der $e$-Funktion zusammen?
        \item Beschreibe die wesentlichen Unterschiede im Graphenverlauf zwischen $g(x)=e^x$ und $f(x)=\ln(x)$ bezüglich Monotonie, Krümmung und Asymptoten. Wie kommt die geometrische Beziehung (Spiegelung an $y=x$) zustande?
    \end{itemize}
    \item \textbf{Logarithmengesetze anwenden und verstehen:}
    \begin{itemize}
        \item Die Regel $\ln(u^r) = r \cdot \ln(u)$ ist besonders nützlich beim Lösen von Exponentialgleichungen. Erkläre, warum diese Regel das 'Herunterholen' des Exponenten ermöglicht.
        \item Ist die Aussage $\ln(a+b) = \ln(a) + \ln(b)$ korrekt? Überprüfe mit einem Zahlenbeispiel und vergleiche mit der Produktregel $\ln(a \cdot b) = \ln(a) + \ln(b)$.
    \end{itemize}
\end{enumerate}
\end{aufgabenumgebung}

\begin{loesungsumgebung}[loes:checkliste-logarithmus-ln-verstehen]{Checkliste: Das Wesen des Logarithmus und der $\ln$-Funktion verstehen}

\begin{enumerate}[label=(\alph*)]
    \item \textbf{Definition und Umkehrbeziehung:}
    \begin{itemize}
        \item \textbf{Bedeutung von $\log_b(y)=x$ und die Basis des natürlichen Logarithmus:} \\
        Die Gleichung $\log_b(y)=x$ ist die logarithmische Schreibweise für die Exponentialgleichung $b^x = y$. Sie bedeutet: 'Der Logarithmus von $y$ zur Basis $b$ ist der Exponent $x$, mit dem die Basis $b$ potenziert werden muss, um $y$ zu erhalten.' Dabei muss die Basis $b$ positiv und ungleich 1 sein ($b>0, b \neq 1$), und das Argument des Logarithmus $y$ muss positiv sein ($y>0$).
        Der \textbf{natürliche Logarithmus}, geschrieben als $\ln(x)$, hat als spezielle Basis die \textbf{Eulersche Zahl $e \approx 2.71828$}. Es gilt also: $\ln(x) = \log_e(x)$.

        \item \textbf{Warum $e^{\ln(x)} = x$ (für $x>0$) und $\ln(e^x) = x$ (für alle $x \in \mathbb{R}$)?} \\
        Dies ergibt sich direkt aus der Definition des Logarithmus als Umkehrfunktion zur Exponentialfunktion.
        \begin{itemize}
            \item $f(x)=e^x$ und $g(x)=\ln(x)$ sind Umkehrfunktionen zueinander.
            \item Für Umkehrfunktionen gilt allgemein: $f(g(x))=x$ für $x$ im Definitionsbereich von $g$, und $g(f(x))=x$ für $x$ im Definitionsbereich von $f$.
            \item $e^{\ln(x)} = x$: Hier wird $\ln(x)$ (der Exponent, mit dem $e$ potenziert $x$ ergibt) wieder als Exponent von $e$ verwendet. Das Ergebnis muss $x$ sein. Dies gilt für $x>0$, da $\ln(x)$ nur für positive $x$ definiert ist.
            \item $\ln(e^x) = x$: Hier wird der Logarithmus von $e^x$ gebildet. Die Frage ist: 'Mit welcher Zahl muss $e$ potenziert werden, um $e^x$ zu erhalten?' Die Antwort ist $x$. Dies gilt für alle reellen Zahlen $x$, da $e^x$ immer positiv und somit im Definitionsbereich des Logarithmus ist.
        \end{itemize}

        \item \textbf{Begründung für $\ln(1)=0$ und $\ln(e)=1$:}
        \begin{itemize}
            \item $\ln(1)=0$: Wir suchen die Zahl $x$, für die $e^x=1$ gilt. Da jede Zahl (außer 0) hoch 0 gleich 1 ist, also $e^0=1$, muss $\ln(1)=0$ sein.
            \item $\ln(e)=1$: Wir suchen die Zahl $x$, für die $e^x=e$ gilt. Dies ist offensichtlich $x=1$, da $e^1=e$. Also muss $\ln(e)=1$ sein.
        \end{itemize}
    \end{itemize}

    \item \textbf{Definitionsbereich und Graph:}
    \begin{itemize}
        \item \textbf{Warum ist der Definitionsbereich von $f(x)=\ln(x)$ auf $x>0$ beschränkt?} \\
        Der Logarithmus $\ln(x)=y$ ist definiert als die Umkehrung von $e^y=x$. Die Exponentialfunktion $e^y$ nimmt für alle reellen Exponenten $y$ ausschließlich positive Werte an (der Wertebereich von $e^y$ ist $(0, \infty)$). Da $x$ das Ergebnis von $e^y$ ist, muss $x$ zwingend positiv sein. Es gibt keine reelle Zahl $y$, für die $e^y$ Null oder negativ wäre.
        Daher ist der Definitionsbereich der natürlichen Logarithmusfunktion die Menge der positiven reellen Zahlen, $D_f = (0, \infty)$.

        \item \textbf{Unterschiede im Graphenverlauf zwischen $g(x)=e^x$ und $f(x)=\ln(x)$:}
        \begin{itemize}
            \item \textbf{Monotonie:} Beide Funktionen sind streng monoton steigend in ihrem gesamten Definitionsbereich.
            \item \textbf{Krümmung:}
            $g(x)=e^x$: $g''(x)=e^x > 0$ für alle $x$. Der Graph ist immer linksgekrümmt (konvex).
            $f(x)=\ln(x)$: $f''(x)=-1/x^2 < 0$ für $x>0$. Der Graph ist immer rechtsgekrümmt (konkav).
            \item \textbf{Asymptoten:}
            $g(x)=e^x$: Hat eine horizontale Asymptote $y=0$ für $x \to -\infty$. Keine senkrechten Asymptoten.
            $f(x)=\ln(x)$: Hat eine senkrechte Asymptote $x=0$ (die y-Achse) für $x \to 0^+$, wobei $f(x) \to -\infty$. Keine horizontalen Asymptoten ($f(x) \to \infty$ für $x \to \infty$).
            \item \textbf{Achsenschnittpunkte:}
            $g(x)=e^x$: y-Achsenabschnitt $(0|1)$, keine Nullstellen.
            $f(x)=\ln(x)$: x-Achsenabschnitt (Nullstelle) $(1|0)$, keinen y-Achsenabschnitt.
            \item \textbf{Definitions-/Wertebereich:}
            $g(x)=e^x$: $D_g=\mathbb{R}$, $W_g=(0, \infty)$.
            $f(x)=\ln(x)$: $D_f=(0, \infty)$, $W_f=\mathbb{R}$. (Definitions- und Wertebereich sind vertauscht).
        \end{itemize}
        \textbf{Geometrische Beziehung:} Da $f(x)=\ln(x)$ die Umkehrfunktion von $g(x)=e^x$ ist, gehen ihre Graphen durch \textbf{Spiegelung an der Geraden $y=x$} (der Winkelhalbierenden des I. und III. Quadranten) auseinander hervor.
    \end{itemize}

    \item \textbf{Logarithmengesetze anwenden und verstehen:}
    \begin{itemize}
        \item \textbf{Regel $\ln(u^r) = r \cdot \ln(u)$ und das 'Herunterholen' des Exponenten:}
        Diese Regel besagt, dass der Logarithmus einer Potenz gleich dem Exponenten multipliziert mit dem Logarithmus der Basis dieser Potenz ist.
        Beim Lösen von Exponentialgleichungen der Form $b^x=y$ ist diese Regel zentral:
        1. Man logarithmiert beide Seiten der Gleichung: $\ln(b^x) = \ln(y)$.
        2. Mit der Regel $\ln(u^r) = r \cdot \ln(u)$ kann man den Exponenten $x$ als Faktor vor den Logarithmus ziehen: $x \cdot \ln(b) = \ln(y)$.
        3. Nun kann man nach $x$ auflösen: $x = \frac{\ln(y)}{\ln(b)}$.
        Die Regel ermöglicht es also, den Exponenten $x$ aus der Potenz 'herauszuholen' und ihn so einer direkten Berechnung zugänglich zu machen.

        \item \textbf{Ist die Aussage $\ln(a+b) = \ln(a) + \ln(b)$ korrekt?}
        Die Aussage $\ln(a+b) = \ln(a) + \ln(b)$ ist \textbf{falsch}.
        \textit{Zahlenbeispiel:} Seien $a=1$ und $b=e-1$ (beide positiv).
        $\ln(a+b) = \ln(1 + e-1) = \ln(e) = 1$.
        $\ln(a) + \ln(b) = \ln(1) + \ln(e-1) = 0 + \ln(e-1)$.
        Da $e-1 \approx 1.718$, ist $\ln(e-1) \approx \ln(1.718) \approx 0.541$.
        Somit ist $1 \neq 0.541$.
        \textit{Vergleich mit der Produktregel:} Die korrekte Regel für die Summe von Logarithmen ist das Logarithmus eines Produkts: $\mathbf{\ln(a \cdot b) = \ln(a) + \ln(b)}$. Für den Logarithmus einer Summe ($\ln(a+b)$) gibt es keine vergleichbare einfache Zerlegungsregel.
    \end{itemize}
\end{enumerate}

\end{loesungsumgebung}





\begin{aufgabenumgebung}{Checkliste: Analysis mit der $\ln$-Funktion – Ableiten, Integrieren, Anwenden}
Die $\ln$-Funktion und ihre Ableitung $1/x$ tauchen in vielen analytischen Zusammenhängen auf.

\begin{enumerate}[label=(\alph*)]
    \item \textbf{Ableiten mit $\ln(x)$:}
    \begin{itemize}
        \item Die Ableitung von $\ln(x)$ ist $1/x$. Was sagt dir das über die Steigung des Graphen von $\ln(x)$ für $x$-Werte nahe Null (aber $x>0$) im Vergleich zu sehr großen $x$-Werten?
        \item Bei der Ableitung von $\ln(h(x))$ lautet die Regel $\frac{h'(x)}{h(x)}$. Erkläre, wie die Kettenregel hier zur Anwendung kommt ($g(u)=\ln u \implies g'(u)=1/u$).
    \end{itemize}
    \item \textbf{Integrieren mit $\ln(x)$:}
    \begin{itemize}
        \item Warum schreiben wir $\int \frac{1}{x} \,dx = \ln|x|+C$ mit Betragsstrichen, obwohl der Definitionsbereich von $\ln(x)$ selbst $x>0$ ist?
        \item Für die Berechnung von $\int \ln(x) \,dx$ wurde die partielle Integration mit $u'(x)=1$ und $v(x)=\ln(x)$ genutzt. Warum wäre die Wahl $u'(x)=\ln(x)$ und $v(x)=1$ nicht zielführend, wenn du $\int \ln(x) \,dx$ noch nicht kennst?
        \item Erkläre, warum das Integral $\int \frac{g'(x)}{g(x)} \,dx$ auf $\ln|g(x)|+C$ führt. Welche Substitution steckt dahinter?
    \end{itemize}
    \item \textbf{Kurvendiskussion und Grenzwerte:}
    \begin{itemize}
        \item Wenn du eine Funktion wie $f(x)=x^2 \ln(x)$ untersuchst: Welcher der beiden Faktoren ($x^2$ oder $\ln(x)$) 'dominiert' das Verhalten für $x \to \infty$? Und für $x \to 0^+$? (Denke an die bekannten Grenzwerte $\lim_{x \to \infty} \frac{\ln x}{x^n}=0$ und $\lim_{x \to 0^+} x^n \ln x = 0$).
        \item Welche Schritte sind bei einer Kurvendiskussion einer Funktion mit $\ln(x)$ besonders wichtig im Hinblick auf den Definitionsbereich?
    \end{itemize}
\end{enumerate}
\end{aufgabenumgebung}

\begin{loesungsumgebung}[loes:checkliste-ln-funktion-analysis]{Checkliste: Analysis mit der $\ln$-Funktion – Ableiten, Integrieren, Anwenden}

\begin{enumerate}[label=(\alph*)]
    \item \textbf{Ableiten mit $\ln(x)$:}
    \begin{itemize}
        \item \textbf{Ableitung von $\ln(x)$ und Steigung des Graphen:} \\
        Die Ableitung von $f(x)=\ln(x)$ ist $f'(x)=\frac{1}{x}$.
        \begin{itemize}
            \item Für $x$-Werte nahe Null (aber $x>0$, z.B. $x=0.01$): $f'(x) = \frac{1}{x}$ wird sehr groß (z.B. $1/0.01 = 100$). Das bedeutet, der Graph von $\ln(x)$ ist für $x$-Werte nahe Null (von rechts kommend) \textbf{sehr steil ansteigend} (obwohl die Funktionswerte $f(x)$ gegen $-\infty$ gehen).
            \item Für sehr große $x$-Werte (z.B. $x=100$): $f'(x) = \frac{1}{x}$ wird sehr klein (z.B. $1/100 = 0.01$). Das bedeutet, der Graph von $\ln(x)$ wird für große $x$-Werte \textbf{immer flacher}, obwohl er weiterhin streng monoton steigt.
        \end{itemize}

        \item \textbf{Ableitung von $\ln(h(x))$ mit der Kettenregel:} \\
        Sei $f(x) = \ln(h(x))$. Wir wenden die Kettenregel $(a(i(x)))' = a'(i(x)) \cdot i'(x)$ an.
        Die äußere Funktion ist $a(u) = \ln(u)$, deren Ableitung $a'(u) = \frac{1}{u}$ ist.
        Die innere Funktion ist $i(x) = h(x)$, deren Ableitung $i'(x) = h'(x)$ ist.
        Somit ist die Ableitung von $f(x) = \ln(h(x))$:
        $$ f'(x) = a'(h(x)) \cdot h'(x) = \frac{1}{h(x)} \cdot h'(x) = \mathbf{\frac{h'(x)}{h(x)}} $$
    \end{itemize}

    \item \textbf{Integrieren mit $\ln(x)$:}
    \begin{itemize}
        \item \textbf{Warum $\int \frac{1}{x} \,dx = \ln|x|+C$ mit Betragsstrichen?} \\
        Die Funktion $f(x)=\frac{1}{x}$ ist für alle $x \neq 0$ definiert. Ihre Stammfunktion muss also ebenfalls für alle $x \neq 0$ definiert sein (bzw. auf getrennten Intervallen $(-\infty,0)$ und $(0,\infty)$).
        \begin{itemize}
            \item Für $x>0$: $(\ln(x))' = \frac{1}{x}$. Also ist $\ln(x)$ eine Stammfunktion für $x>0$.
            \item Für $x<0$: Sei $y=-x$. Dann ist $y>0$. Die Ableitung von $\ln(-x)$ ist nach der Kettenregel: $(\ln(-x))' = \frac{1}{-x} \cdot (-1) = \frac{1}{x}$. Also ist $\ln(-x)$ eine Stammfunktion für $x<0$.
        \end{itemize}
        Beide Fälle können zusammengefasst werden als $(\ln|x|)' = \frac{1}{x}$ für $x \neq 0$. Die Betragsstriche stellen sicher, dass das Argument des Logarithmus positiv ist und der Definitionsbereich der Stammfunktion (bis auf $x=0$) dem von $\frac{1}{x}$ entspricht. Daher ist die allgemeinste Stammfunktion $\ln|x|+C$.

        \item \textbf{Partielle Integration von $\int \ln(x) \,dx$:} \\
        Für $\int \ln(x) \,dx = \int 1 \cdot \ln(x) \,dx$ wählt man typischerweise:
        $u'(x)=1 \Rightarrow u(x)=x$
        $v(x)=\ln(x) \Rightarrow v'(x)=\frac{1}{x}$
        Dies führt zu: $\int \ln(x) \,dx = x\ln(x) - \int x \cdot \frac{1}{x} \,dx = x\ln(x) - \int 1 \,dx = x\ln(x) - x + C$.
        Die Wahl $u'(x)=\ln(x)$ und $v(x)=1$ wäre nicht zielführend, wenn man $\int \ln(x) \,dx$ noch nicht kennt, weil man dann $u(x) = \int \ln(x) \,dx$ bestimmen müsste, was ja gerade das gesuchte Integral ist. Man würde also die Kenntnis der Lösung voraussetzen:
        $\int \ln(x) \cdot 1 \,dx = (\int \ln(x) \,dx) \cdot 1 - \int (\int \ln(x) \,dx) \cdot 0 \,dx = \int \ln(x) \,dx$. Dies ist eine triviale Gleichung und hilft nicht weiter.

        \item \textbf{Warum $\int \frac{g'(x)}{g(x)} \,dx = \ln|g(x)|+C$?} \\
        Dies ergibt sich direkt aus der Substitutionsregel (Umkehrung der Kettenregel).
        Wir wählen die Substitution $u = g(x)$.
        Dann ist das Differential $\frac{du}{dx} = g'(x)$, also $du = g'(x) \,dx$.
        Das Integral transformiert sich damit zu:
        $$ \int \frac{g'(x)}{g(x)} \,dx = \int \frac{1}{g(x)} \cdot (g'(x) \,dx) = \int \frac{1}{u} \,du $$
        Die Stammfunktion von $\frac{1}{u}$ ist $\ln|u|$. Setzt man die Konstante $C$ hinzu und substituiert $u=g(x)$ zurück, erhält man:
        $$ \ln|g(x)| + C $$
    \end{itemize}

    \item \textbf{Kurvendiskussion und Grenzwerte:}
    \begin{itemize}
        \item \textbf{Dominanz bei $f(x)=x^2 \ln(x)$ für $x \to \infty$ und $x \to 0^+$:}
        \begin{itemize}
            \item Für $\mathbf{x \to \infty}$: Beide Faktoren $x^2$ und $\ln(x)$ gehen gegen $\infty$. Ihr Produkt $x^2\ln(x)$ geht daher ebenfalls gegen $\mathbf{\infty}$. Obwohl $x^2$ schneller wächst als $\ln(x)$, 'dominieren' hier beide Faktoren das Wachstum gemeinsam ins Unendliche.
            \item Für $\mathbf{x \to 0^+}$: $x^2 \to 0$ und $\ln(x) \to -\infty$. Wir haben einen Ausdruck vom Typ $0 \cdot (-\infty)$, der unbestimmt ist. Der Grenzwert $\lim_{x \to 0^+} x^n \ln x = 0$ (für $n>0$) ist bekannt. Hier ist $n=2$.
            Also $\lim_{x \to 0^+} x^2 \ln(x) = \mathbf{0}$.
            In diesem Fall 'dominiert' der Potenzterm $x^2$ (der gegen Null geht) über den logarithmischen Term $\ln(x)$ (der gegen $-\infty$ geht) und zieht das Produkt gegen Null.
        \end{itemize}

        \item \textbf{Wichtige Schritte bei Kurvendiskussion einer Funktion mit $\ln(x)$ bezüglich Definitionsbereich:}
        \begin{enumerate}
            \item \textbf{Argument des Logarithmus bestimmen:} Identifiziere alle Terme der Form $\ln(h(x))$.
            \item \textbf{Bedingung für das Argument aufstellen:} Für jeden Logarithmusterm muss gelten, dass sein Argument strikt positiv ist: $h(x) > 0$.
            \item \textbf{Ungleichung(en) lösen:} Löse diese Ungleichung(en), um die erlaubten $x$-Werte zu finden. Dies ergibt den Definitionsbereich der Funktion.
            \item \textbf{Verhalten an den Rändern des Definitionsbereichs untersuchen:} Bestimme die Grenzwerte der Funktion, wenn sich $x$ den Rändern des Definitionsbereichs nähert (insbesondere an Stellen, wo das Argument des Logarithmus gegen Null geht, da dies oft zu senkrechten Asymptoten führt, weil $\ln(u) \to -\infty$ für $u \to 0^+$).
            \item \textbf{Definitionsbereich bei Ableitungen beachten:} Die Ableitungen können zusätzliche Einschränkungen im Definitionsbereich haben (z.B. wenn $h(x)$ im Nenner von $f'(x)$ auftaucht), aber der Definitionsbereich der ursprünglichen Funktion $f(x)$ ist maßgeblich für die Analyse von $f(x)$ selbst.
        \end{enumerate}
    \end{itemize}
\end{enumerate}

\end{loesungsumgebung}
% Hier beginnt der Code für das Kapitel Trigonometrische Funktionen.
% Annahme: Alle Pakete und Definitionen aus dem Hauptdokument (math_learning_material_v2) 
% sind hier gültig und werden im Hauptdokument geladen.

\section{Trigonometrische Funktionen – Die Mathematik der Schwingungen und Wellen}
\label{sec:trigonometrische_funktionen}

Nachdem wir uns mit Polynomen, Exponential- und Logarithmusfunktionen beschäftigt haben, wenden wir uns nun einer weiteren wichtigen Funktionsklasse zu: den \textbf{trigonometrischen Funktionen}. Die bekanntesten Vertreter sind \textbf{Sinus ($\sin x$)}, \textbf{Kosinus ($\cos x$)} und \textbf{Tangens ($\tan x$)}.

\begin{tcolorbox}[colback=blue!5!white, colframe=blue!75!black, title=Was du in diesem Kapitel lernen wirst:]
Nachdem du dieses Kapitel durchgearbeitet hast, wirst du in der Lage sein:
\begin{itemize}[noitemsep, topsep=0pt, leftmargin=*, itemsep=2pt]
    \item die Bedeutung des \textbf{Bogenmaßes} für Winkel zu verstehen und es sicher im Zusammenhang mit trigonometrischen Funktionen anzuwenden sowie zwischen Grad- und Bogenmaß umzurechnen.
    \item die Funktionen \textbf{Sinus ($\sin x$)} und \textbf{Kosinus ($\cos x$)} am Einheitskreis zu definieren und ihre grundlegenden Eigenschaften (Definitions- und Wertebereich, Periode $2\pi$, Nullstellen, Extremstellen, Symmetrie) sowie den trigonometrischen Pythagoras ($\sin^2x + \cos^2x = 1$) zu kennen und zu nutzen.
    \item die \textbf{Tangensfunktion ($\tan x$)} als Quotient aus Sinus und Kosinus zu definieren und ihre wesentlichen Eigenschaften (Definitionsbereich mit Polstellen, Wertebereich, Periode $\pi$, Nullstellen, Symmetrie) zu beschreiben.
    \item die \textbf{Ableitungen} der Grundfunktionen $(\sin x)' = \cos x$, $(\cos x)' = -\sin x$ und $(\tan x)' = \frac{1}{\cos^2 x} = 1+\tan^2x$ zu kennen und anzuwenden.
    \item die grundlegenden \textbf{Stammfunktionen} $\int \cos x \,dx = \sin x + C$ und $\int \sin x \,dx = -\cos x + C$ zu bilden und für einfache bestimmte Integrale zu nutzen.
    \item \textbf{transformierte Sinus- und Kosinusfunktionen} der Form $f(x) = A \sin(B(x-C)) + D$ zu analysieren, indem du die Bedeutung der Parameter für Amplitude, Periode, Phasen- und y-Verschiebung interpretierst, und solche Funktionen zu skizzieren oder aus Graphen abzuleiten.
    \item die allgemeinen \textbf{Ableitungsregeln} (Produkt-, Quotienten- und Kettenregel) sicher auf Funktionen anzuwenden, die trigonometrische Terme enthalten (oft in Kombination mit Polynomen oder Exponentialfunktionen).
    \item eine vollständige \textbf{Kurvendiskussion} für Funktionen mit trigonometrischen Anteilen durchzuführen, um deren Graphen und charakteristische Merkmale zu analysieren.
    \item die fundamentale Rolle trigonometrischer Funktionen bei der Modellierung \textbf{periodischer Vorgänge} und Schwingungen zu verstehen.
\end{itemize}
Du wirst somit die Mathematik der Wellen und Zyklen meistern und dein Repertoire an analysierbaren Funktionen entscheidend erweitern!
\end{tcolorbox}
\bigskip

Diese Funktionen sind unerlässlich, um \textbf{periodische Vorgänge} zu beschreiben – also Phänomene, die sich in regelmäßigen Abständen wiederholen. Denk zum Beispiel an:
\begin{itemize}
    \item Schwingungen eines Pendels oder einer Gitarrensaite.
    \item Wellenbewegungen wie Wasserwellen oder Schallwellen.
    \item Kreisbewegungen, z.B. die Position eines Punktes auf einem sich drehenden Rad.
    \item Jahreszeitliche Schwankungen von Temperaturen oder Tageslängen.
    \item Wechselstrom in der Elektrotechnik.
\end{itemize}
Die trigonometrischen Funktionen sind die mathematische Sprache, um diese periodischen Muster präzise zu erfassen und zu analysieren.

\begin{infoboxumgebung}{Winkel im Bogenmaß – Die 'natürliche' Einheit für Winkel}
In der höheren Mathematik und besonders bei der Analysis von trigonometrischen Funktionen werden Winkel fast ausschließlich im \textbf{Bogenmaß} (Radiant, abgekürzt rad) angegeben, nicht im Gradmaß (z.B. $30^\circ, 90^\circ, 360^\circ$).
\textbf{Was ist das Bogenmaß?}
Stell dir einen Kreis mit Radius $r=1$ vor (den Einheitskreis). Das Bogenmaß eines Winkels $\alpha$ ist die Länge des Kreisbogens, den dieser Winkel auf dem Einheitskreis 'ausschneidet'.
\begin{itemize}
    \item Ein Vollkreis hat $360^\circ$. Der Umfang des Einheitskreises ist $U = 2\pi r = 2\pi \cdot 1 = 2\pi$.
    Also entspricht $360^\circ$ einem Bogenmaß von $2\pi$.
    \item Daraus folgt: $180^\circ \widehat{=} \pi \text{ rad}$
    \item $90^\circ \widehat{=} \frac{\pi}{2} \text{ rad}$
    \item $60^\circ \widehat{=} \frac{\pi}{3} \text{ rad}$
    \item $45^\circ \widehat{=} \frac{\pi}{4} \text{ rad}$
    \item $30^\circ \widehat{=} \frac{\pi}{6} \text{ rad}$
\end{itemize}
\textbf{Umrechnung:}
\begin{itemize}
    \item Gradmaß in Bogenmaß: $\text{Bogenmaß} = \text{Gradmaß} \cdot \frac{\pi}{180^\circ}$
    \item Bogenmaß in Gradmaß: $\text{Gradmaß} = \text{Bogenmaß} \cdot \frac{180^\circ}{\pi}$
\end{itemize}
Wenn bei trigonometrischen Funktionen keine Einheit angegeben ist (z.B. $\sin(2)$), ist immer das Bogenmaß gemeint! Die Ableitungsregeln, die wir kennenlernen werden, gelten nur, wenn der Winkel im Bogenmaß gegeben ist.
\end{infoboxumgebung}

\subsection{Sinus ($\sin x$) und Kosinus ($\cos x$) – Die Grundpfeiler}
\label{subsec:sinus_kosinus_grundlagen}

Die Funktionen $f(x)=\sin(x)$ und $g(x)=\cos(x)$ sind die fundamentalsten trigonometrischen Funktionen.

\subsubsection{Geometrische Bedeutung am Einheitskreis}
Eine anschauliche Definition von Sinus und Kosinus erhält man am \textbf{Einheitskreis} (Kreis mit Radius $r=1$ um den Ursprung eines Koordinatensystems).
Betrachtet man einen Winkel $\alpha$ (im Bogenmaß), der von der positiven x-Achse gegen den Uhrzeigersinn gemessen wird, so schneidet der freie Schenkel des Winkels den Einheitskreis in einem Punkt $P(x_P|y_P)$. Dann gilt:
\begin{itemize}
    \item $\cos(\alpha) = x_P$ (die x-Koordinate des Punktes $P$)
    \item $\sin(\alpha) = y_P$ (die y-Koordinate des Punktes $P$)
\end{itemize}
\begin{center}
    \includegraphics[width=0.7\textwidth]{grafiken/Trig_Einheitskreis.png}
    \captionof{figure}{Definition von Sinus und Kosinus am Einheitskreis}
    \label{fig:einheitskreis}
\end{center}
Aus dieser Definition am Einheitskreis ergeben sich viele wichtige Eigenschaften.

\subsubsection{Graphen und Eigenschaften von $\sin(x)$ und $\cos(x)$}

\begin{merksatzumgebung}{Eigenschaften von $f(x)=\sin(x)$ und $g(x)=\cos(x)$}
\begin{itemize}
    \item \textbf{Definitionsbereich:} $D_f = D_g = \mathbb{R}$ (man kann jeden reellen Winkel einsetzen).
    \item \textbf{Wertebereich:} $W_f = W_g = [-1, 1]$ (die Funktionswerte liegen immer zwischen -1 und 1, inklusive).
    \item \textbf{Periodizität:} Beide Funktionen sind periodisch mit der \textbf{Periode $2\pi$}. Das bedeutet, ihre Werte wiederholen sich alle $2\pi$:
        \[ \sin(x + 2\pi) = \sin(x) \quad \text{und} \quad \cos(x + 2\pi) = \cos(x) \]
    \item \textbf{Nullstellen:}
        \begin{itemize}
            \item $\sin(x) = 0$ für $x = k \cdot \pi$, wobei $k \in \mathbb{Z}$ (alle ganzen Zahlen). Also bei $x=0, \pm\pi, \pm 2\pi, \dots$
            \item $\cos(x) = 0$ für $x = \frac{\pi}{2} + k \cdot \pi$, wobei $k \in \mathbb{Z}$. Also bei $x=\pm\frac{\pi}{2}, \pm\frac{3\pi}{2}, \dots$
        \end{itemize}
    \item \textbf{Extremstellen:}
        \begin{itemize}
            \item $\sin(x)$ hat Maxima ($+1$) bei $x = \frac{\pi}{2} + 2k\pi$ und Minima ($-1$) bei $x = \frac{3\pi}{2} + 2k\pi$.
            \item $\cos(x)$ hat Maxima ($+1$) bei $x = 2k\pi$ und Minima ($-1$) bei $x = \pi + 2k\pi$.
        \end{itemize}
    \item \textbf{Symmetrie:}
        \begin{itemize}
            \item $\sin(x)$ ist \textbf{punktsymmetrisch zum Ursprung}: $\sin(-x) = -\sin(x)$.
            \item $\cos(x)$ ist \textbf{achsensymmetrisch zur y-Achse}: $\cos(-x) = \cos(x)$.
        \end{itemize}
    \item \textbf{Zusammenhang:} Die Graphen von Sinus und Kosinus sind zueinander phasenverschoben:
        $\cos(x) = \sin(x + \frac{\pi}{2})$ (Kosinus ist Sinus um $\pi/2$ nach links verschoben).
        $\sin(x) = \cos(x - \frac{\pi}{2})$ (Sinus ist Kosinus um $\pi/2$ nach rechts verschoben).
    \item \textbf{Wichtiger Grundzusammenhang (Trigonometrischer Pythagoras):} Für jeden Winkel $x$ gilt:
        \[ \sin^2(x) + \cos^2(x) = 1 \]
        (Das folgt direkt aus dem Satz des Pythagoras am Einheitskreis, da $x_P^2+y_P^2=r^2=1^2$).
\end{itemize}
\end{merksatzumgebung}

\begin{center}
    \includegraphics[width=0.9\textwidth]{grafiken/Trig_SinCos_Graphen.png}
    \captionof{figure}{Graphen der Funktionen $f(x)=\sin(x)$ und $g(x)=\cos(x)$}
    \label{fig:sincos_graphen}
\end{center}




\begin{tippumgebung}{Pythagoras im Test – Eine kleine Selbstüberprüfung}
Du hast gerade gelernt, dass für jeden Winkel $x$ gilt: $\sin^2(x) + \cos^2(x) = 1$.
Überlege einmal: Könnte es einen Winkel $x$ geben, für den gleichzeitig $\sin(x) = 0,8$ und $\cos(x) = 0,7$ gilt? Überprüfe dies, indem du die Werte in die Gleichung einsetzt. Was stellst du fest?
(Zur Erinnerung: $\sin^2(x)$ bedeutet $(\sin x)^2$.)
\end{tippumgebung}

% ... (Weiter im Text, z.B. mit der Abbildung der Sinus/Kosinus-Graphen oder dem nächsten Unterabschnitt)
\subsubsection{Ein tieferer Einblick: Sinus und Kosinus als unendliche Reihen (Taylorreihen)}
Eine faszinierende Eigenschaft vieler Funktionen in der Mathematik ist, dass sie als unendliche Summen von Potenzfunktionen dargestellt werden können – sogenannte \textbf{Taylorreihen} (oder Maclaurinreihen, wenn die Entwicklung um den Punkt $x=0$ erfolgt). Für Sinus und Kosinus sehen diese Reihen besonders elegant aus:

\begin{infoboxumgebung}{Taylorreihen für Sinus und Kosinus}
Für alle reellen Zahlen $x$ (im Bogenmaß!) gilt:
\[ \sin(x) = x - \frac{x^3}{3!} + \frac{x^5}{5!} - \frac{x^7}{7!} + \frac{x^9}{9!} - \dots = \sum_{n=0}^{\infty} (-1)^n \frac{x^{2n+1}}{(2n+1)!} \]
\[ \cos(x) = 1 - \frac{x^2}{2!} + \frac{x^4}{4!} - \frac{x^6}{6!} + \frac{x^8}{8!} - \dots = \sum_{n=0}^{\infty} (-1)^n \frac{x^{2n}}{(2n)!} \]
Dabei bedeutet $k!$ (gelesen 'k Fakultät') das Produkt aller ganzen Zahlen von 1 bis $k$: $k! = 1 \cdot 2 \cdot 3 \cdot \dots \cdot k$. (Und $0!=1$).
Zum Beispiel: $3! = 1 \cdot 2 \cdot 3 = 6$, $5! = 1 \cdot 2 \cdot 3 \cdot 4 \cdot 5 = 120$.

\textbf{Was bedeuten diese Reihen?}
\begin{itemize}
    \item Sie zeigen, dass Sinus und Kosinus im Grunde 'unendlich lange Polynome' sind.
    \item Man kann mit diesen Reihen die Werte von $\sin(x)$ und $\cos(x)$ für jedes $x$ beliebig genau annähern, indem man genügend Glieder der Summe berücksichtigt. So berechnen auch Taschenrechner diese Werte!
    \item Aus diesen Reihen lassen sich viele Eigenschaften der Funktionen ableiten, z.B. ihre Ableitungen.
\end{itemize}
Die Herleitung dieser Reihen ist fortgeschrittene Mathematik, aber ihre Existenz zu kennen, öffnet ein Fenster zu tieferen Zusammenhängen.
\end{infoboxumgebung}

\textit{Selbst-Check:} Setze $x=0$ in die Reihen für $\sin(x)$ und $\cos(x)$ ein. Was erhältst du? Vergleiche mit $\sin(0)$ und $\cos(0)$ vom Einheitskreis. (Antwort: $\sin(0)=0$, $\cos(0)=1$. Das passt!)
\begin{funfactbox}{$\pi$ aus dem Nichts? Eine unendliche Summe enthüllt die Kreiszahl!}
Die Kreiszahl $\pi \approx 3,14159\dots$ ist dir sicher bekannt. Sie beschreibt das Verhältnis des Umfangs eines Kreises zu seinem Durchmesser. Aber wie kommt man eigentlich auf diese unendlich vielen Nachkommastellen? Es gibt viele faszinierende Wege, $\pi$ zu berechnen, und einige davon haben Verbindungen zur Differential- und Integralrechnung!

Eine berühmte Methode verwendet eine unendliche Summe, die mit dem Mathematiker Gottfried Wilhelm Leibniz (einem der 'Erfinder' der Differentialrechnung) in Verbindung gebracht wird, aber auch schon früher in Indien bekannt war:
\[ \frac{\pi}{4} = 1 - \frac{1}{3} + \frac{1}{5} - \frac{1}{7} + \frac{1}{9} - \frac{1}{11} + \dots \]
Das ist die sogenannte \textbf{Leibniz-Reihe}. Sie besagt, dass man sich dem Wert von $\pi/4$ immer genauer annähert, je mehr Terme dieser abwechselnden (alternierenden) Reihe man addiert und subtrahiert. Multipliziert man das Ergebnis mit 4, erhält man eine Annäherung für $\pi$.

\textbf{Wo steckt da die Verbindung zur Analysis?}
Diese Reihe ist eigentlich ein Spezialfall der Taylorreihe für die Funktion $f(x) = \arctan(x)$ (Arkustangens, die Umkehrfunktion des Tangens), ausgewertet an der Stelle $x=1$, denn $\arctan(1) = \frac{\pi}{4}$.
Die Taylorreihe einer Funktion ist eine Darstellung dieser Funktion als unendliche Summe von Potenztermen, wobei die Koeffizienten dieser Terme durch die \textbf{Ableitungen} der Funktion an einem bestimmten Punkt bestimmt werden.

Man kann zeigen:
\[ \arctan(x) = x - \frac{x^3}{3} + \frac{x^5}{5} - \frac{x^7}{7} + \dots \]
Setzt man hier $x=1$ ein, erhält man genau die Leibniz-Reihe für $\frac{\pi}{4}$.

Die Herleitung solcher unendlicher Reihen für Funktionen ist ein wichtiges Anwendungsfeld der Differential- (und auch der Integral-) rechnung. Sie zeigt, wie man komplexe Zahlen wie $\pi$ oder Funktionswerte durch einfachere Bausteine (Potenzen) annähern und berechnen kann. Auch wenn diese spezielle Reihe für die praktische Berechnung von $\pi$ nicht sehr schnell konvergiert (man braucht viele Terme für eine gute Genauigkeit), ist sie ein wunderschönes Beispiel für die tiefen Verbindungen in der Mathematik!

\begin{center}
    \includegraphics[width=0.8\textwidth]{grafiken/Pi_Leibniz_Reihe.png}
    % Beschreibung für die Grafik 'Pi_Leibniz_Reihe.png':
    % Die Grafik könnte visualisieren, wie sich die Summe der ersten Terme 
    % der Leibniz-Reihe (multipliziert mit 4) langsam dem Wert von Pi annähert.
    % Z.B. ein Zahlenstrahl, auf dem die Näherungswerte für S_1, S_2, S_3 etc. 
    % aufgetragen sind und wie sie um Pi 'oszillieren' und sich annähern.
    % Alternativ: Eine stilisierte Darstellung der Zahl Pi mit der Leibniz-Reihe daneben.
    \captionof{figure}{Annäherung an $\pi$ durch die Leibniz-Reihe (konzeptionelle Darstellung)}
    \label{fig:pi_leibniz_funfact}
\end{center}
\end{funfactbox}

\subsubsection{Ableitungen von Sinus und Kosinus}
Eine der schönsten Symmetrien in der Differentialrechnung findet sich bei den Ableitungen von Sinus und Kosinus.

\begin{infoboxumgebung}{Beobachtung: Steigung von $\sin x$ und Werte von $\cos x$}
Bevor wir die Ableitungsregeln für Sinus und Kosinus formal kennenlernen, lass uns eine kleine Beobachtung am Graphen von $f(x)=\sin(x)$ und $g(x)=\cos(x)$ machen (siehe Abbildung \ref{fig:sincos_graphen}):
\begin{itemize}
    \item Wo hat der Graph von $\sin(x)$ seine Hoch- und Tiefpunkte? Welche Steigung hat der Graph an diesen Stellen offensichtlich? Welchen Wert hat $\cos(x)$ an genau diesen x-Stellen?
    \item Wo schneidet der Graph von $\sin(x)$ die x-Achse mit positiver Steigung (z.B. bei $x=0, 2\pi, \dots$)? Welchen Wert hat $\cos(x)$ dort? Wo ist die Steigung von $\sin(x)$ am größten?
    \item Wo schneidet der Graph von $\sin(x)$ die x-Achse mit negativer Steigung (z.B. bei $x=\pi, 3\pi, \dots$)? Welchen Wert hat $\cos(x)$ dort? Wo ist die Steigung von $\sin(x)$ am stärksten negativ (also betragsmäßig am größten, aber negativ)?
\end{itemize}
Vielleicht erkennst du schon ein Muster im Zusammenhang zwischen der Steigung von $\sin(x)$ und den Funktionswerten von $\cos(x)$. Genau diesen Zusammenhang werden wir mit der Ableitung präzisieren!
\end{infoboxumgebung}

\begin{merksatzumgebung}{Ableitungen von $\sin(x)$ und $\cos(x)$}
Für Winkel $x$ im Bogenmaß gilt:
\begin{itemize}
    \item Die Ableitung von $f(x) = \sin(x)$ ist $f'(x) = \cos(x)$.
    \[ (\sin x)' = \cos x \]
    \item Die Ableitung von $g(x) = \cos(x)$ ist $g'(x) = -\sin(x)$.
    \[ (\cos x)' = -\sin x \]
\end{itemize}
\end{merksatzumgebung}

\begin{tippumgebung}{Ableitungen aus den Taylorreihen (für Interessierte)}
Wenn man die Taylorreihen von $\sin(x)$ und $\cos(x)$ Glied für Glied mit der Potenzregel ableitet (was bei Potenzreihen unter bestimmten Bedingungen erlaubt ist), erhält man genau diese Ableitungsregeln!
Beispiel für Sinus:
$\sin(x) = x - \frac{x^3}{3!} + \frac{x^5}{5!} - \frac{x^7}{7!} + \dots$
$(\sin(x))' = (x)' - (\frac{x^3}{6})' + (\frac{x^5}{120})' - \dots$
$= 1 - \frac{3x^2}{6} + \frac{5x^4}{120} - \dots$
$= 1 - \frac{x^2}{2} + \frac{x^4}{24} - \dots$
$= 1 - \frac{x^2}{2!} + \frac{x^4}{4!} - \dots = \cos(x)$!
\end{tippumgebung}

Mit diesen Ableitungen können wir nun auch die höheren Ableitungen bestimmen:
\begin{itemize}
    \item $f(x) = \sin(x)$
    \item $f'(x) = \cos(x)$
    \item $f''(x) = (\cos x)' = -\sin(x)$
    \item $f'''(x) = (-\sin x)' = -(\cos x) = -\cos(x)$
    \item $f^{(4)}(x) = (-\cos x)' = -(-\sin x) = \sin(x)$
\end{itemize}
Nach der vierten Ableitung wiederholt sich der Zyklus! Das Gleiche gilt für $\cos(x)$.

\subsubsection{Stammfunktionen von Sinus und Kosinus}
Aus den Ableitungsregeln ergeben sich direkt die Stammfunktionen:

\begin{merksatzumgebung}{Stammfunktionen von $\sin(x)$ und $\cos(x)$}
\begin{itemize}
    \item Eine Stammfunktion von $f(x) = \cos(x)$ ist $F(x) = \sin(x)$, denn $(\sin x)' = \cos x$.
    Also: \[\int \cos(x) \,dx = \sin(x) + C \]
    \item Eine Stammfunktion von $g(x) = \sin(x)$ ist $G(x) = -\cos(x)$, denn $(-\cos x)' = -(-\sin x) = \sin x$.
    Also: \[\int \sin(x) \,dx = -\cos(x) + C \]
\end{itemize}
Achte auf das Minuszeichen bei der Stammfunktion von $\sin(x)$!
\end{merksatzumgebung}

\begin{beispielumgebung}{Ableiten und Integrieren mit $\sin x$ und $\cos x$}
\begin{enumerate}
    \item Leite $f(x) = 3\sin(x) - 2\cos(x) + x^2$ ab.
        $f'(x) = (3\sin x)' - (2\cos x)' + (x^2)'$
        $f'(x) = 3\cos(x) - 2(-\sin x) + 2x = 3\cos(x) + 2\sin(x) + 2x$.
    \item Bestimme $\int (4\cos(x) + \sin(x) - 2) \,dx$.
        $\int (4\cos(x) + \sin(x) - 2) \,dx = 4\int\cos(x)dx + \int\sin(x)dx - \int 2dx$
        $= 4\sin(x) + (-\cos x) - 2x + C = 4\sin(x) - \cos(x) - 2x + C$.
\end{enumerate}
\end{beispielumgebung}

\begin{aufgabenumgebung}{Erste Übungen mit $\sin x$ und $\cos x$}
\begin{enumerate}
    \item Bilde die erste und zweite Ableitung der folgenden Funktionen:
        \begin{itemize}
            \item $f_1(x) = -5\cos(x) + 2\sin(x) - e^x$
            \item $f_2(x) = \sin(x) + \ln(x)$ (Beachte den Definitionsbereich!)
        \end{itemize}
    \item Bestimme die Menge aller Stammfunktionen:
        \begin{itemize}
            \item $g_1(x) = \frac{1}{2}\sin(x) - 3\cos(x) + x^3$
            \item $g_2(x) = \cos(x) - \frac{1}{x}$
        \end{itemize}
    \item Berechne das bestimmte Integral $\int_0^{\pi} \sin(x) \,dx$. Was stellt dieser Wert geometrisch dar? Skizziere den Graphen von $\sin(x)$ im Intervall $[0, 2\pi]$ und markiere die berechnete Fläche.
    \item Was ist $\int_0^{2\pi} \sin(x) \,dx$? Erkläre das Ergebnis anhand der Symmetrie und der orientierten Fläche.
\end{enumerate}
\end{aufgabenumgebung}

% Vorheriger Inhalt des Kapitels bis zur aufgabenumgebung 'Erste Übungen mit sin x und cos x'
% ... (siehe vorherige Canvas-Version) ...

% Hier geht es dann weiter mit Transformationen von sin/cos, Tangens, 
% Anwendung der Produkt-/Quotienten-/Kettenregel und Kurvendiskussionen.
% Dieser Kommentar wird durch den folgenden Inhalt ersetzt:

\subsection{Transformationen von Sinus- und Kosinusfunktionen}
\label{subsec:trafo_sincos}

Die Grundfunktionen $f(x)=\sin(x)$ und $g(x)=\cos(x)$ können durch verschiedene Parameter verändert (transformiert) werden, um eine größere Vielfalt an periodischen Vorgängen zu modellieren. Die allgemeine Form einer transformierten Sinusfunktion (analog für Kosinus) ist:
\[ f(x) = A \cdot \sin(B(x-C)) + D \]
oder auch oft geschrieben als:
\[ f(x) = A \cdot \sin(Bx - \varphi) + D \quad \text{mit } \varphi = B \cdot C \]

\begin{merksatzumgebung}{Parameter der allgemeinen Sinusfunktion $f(x) = A \sin(B(x-C)) + D$}
\begin{itemize}
    \item \textbf{Amplitude $|A|$:}
        \begin{itemize}
            \item Der Faktor $A$ streckt oder staucht den Graphen in y-Richtung.
            \item $|A|$ ist die \textbf{Amplitude}, d.h. die maximale Auslenkung von der Mittellage. Der Wertebereich ist $[D-|A|, D+|A|]$.
            \item Wenn $A < 0$, wird der Graph zusätzlich an der Mittellage (Gerade $y=D$) gespiegelt.
        \end{itemize}
    \item \textbf{Periode $P$ (beeinflusst durch $B$):}
        \begin{itemize}
            \item Der Faktor $B$ beeinflusst die Periode der Schwingung. $B$ wird auch Kreisfrequenz genannt (oft als $\omega$ bezeichnet).
            \item Die \textbf{Periode} $P$ (die Länge einer vollständigen Schwingung) berechnet sich als:
            \[ P = \frac{2\pi}{|B|} \]
            \item Wenn $|B|>1$, wird die Periode kürzer (die Schwingung ist 'schneller', gestaucht in x-Richtung).
            \item Wenn $0<|B|<1$, wird die Periode länger (die Schwingung ist 'langsamer', gestreckt in x-Richtung).
            \item \textit{Zum Weiterdenken:} Was passiert anschaulich mit der Periode $P$, wenn der Betrag von $B$ sehr groß wird (z.B. $|B|=100$)? Die Schwingung wird dann sehr \dots (schnell/kurz oder langsam/lang)? Und was passiert, wenn $|B|$ sehr klein (aber positiv) wird (z.B. $|B|=0.1$)? Die Schwingung wird dann sehr \dots? Passt das zu deiner Vorstellung von 'gestaucht' bzw. 'gestreckt' in x-Richtung?
        \end{itemize}
    \item \textbf{Verschiebung in x-Richtung (Phasenverschiebung $C$):}
        \begin{itemize}
            \item Der Parameter $C$ bewirkt eine Verschiebung des Graphen entlang der x-Achse.
            \item Wenn $C>0$ (also im Term $(x-C)$), wird der Graph um $C$ Einheiten nach \textbf{rechts} verschoben.
            \item Wenn $C<0$ (also im Term $(x+ |C|)$), wird der Graph um $|C|$ Einheiten nach \textbf{links} verschoben.
            \item Diese Verschiebung wird auch \textbf{Phasenverschiebung} genannt.
        \end{itemize}
    \item \textbf{Verschiebung in y-Richtung (Mittellage $D$):}
        \begin{itemize}
            \item Der Parameter $D$ verschiebt den gesamten Graphen um $D$ Einheiten in y-Richtung.
            \item Die Gerade $y=D$ ist die neue \textbf{Mittellage} oder Ruhelage der Schwingung.
        \end{itemize}
\end{itemize}
Die gleichen Interpretationen gelten für die Kosinusfunktion $f(x) = A \cos(B(x-C)) + D$.
\end{merksatzumgebung}

\begin{beispielumgebung}{Analyse einer transformierten Sinusfunktion}
Betrachte die Funktion $f(x) = 2 \sin\left(0.5(x - \frac{\pi}{2})\right) + 1$.
\begin{itemize}
    \item \textbf{Amplitude $A$:} $A=2$. Die maximale Auslenkung von der Mittellage ist 2.
    \item \textbf{Parameter $B$ und Periode $P$:} $B=0.5$. Die Periode ist $P = \frac{2\pi}{|0.5|} = \frac{2\pi}{1/2} = 4\pi$. Die Schwingung ist also langsamer als die Grundschwingung.
    \item \textbf{Verschiebung in x-Richtung $C$:} $C=\frac{\pi}{2}$. Der Graph der Sinusfunktion $2\sin(0.5x)$ wird um $\frac{\pi}{2}$ nach rechts verschoben.
    \item \textbf{Verschiebung in y-Richtung $D$:} $D=1$. Die Mittellage ist die Gerade $y=1$.
    \item \textbf{Wertebereich:} Die Mittellage ist $y=1$, die Amplitude ist $2$. Also schwingt die Funktion zwischen $1-2=-1$ und $1+2=3$. Der Wertebereich ist $[-1, 3]$.
\end{itemize}
\begin{center}
    % Platzhalter für die Grafik 'Transformierte Sinusfunktion'
    \includegraphics[width=0.9\textwidth]{grafiken/Trig_Trafo_Sinus.png}
    \captionof{figure}{Graph der transformierten Sinusfunktion.}
    \label{fig:trafo_sin}
\end{center}
\end{beispielumgebung}

\begin{fehlerboxumgebung}{Trigonometrische Funktionen – Typische Fallstricke}
Beim Rechnen mit Sinus, Kosinus und Co. gibt es einige typische Fehlerquellen:
\begin{itemize}
    \item \textbf{Gradmaß vs. Bogenmaß:} Die Ableitungs- und Integrationsregeln $(\sin x)'=\cos x$ etc. gelten \textbf{nur}, wenn $x$ im Bogenmaß angegeben ist! Viele Taschenrechner sind standardmäßig auf Gradmaß (DEG) eingestellt. Für die Analysis muss er auf Radiant (RAD) umgestellt sein, sonst sind numerische Ergebnisse falsch.
    \item \textbf{Vorzeichen bei Ableitungen/Stammfunktionen:} Merke dir gut: $(\cos x)' = \mathbf{-}\sin x$ und $\int \sin x \,dx = \mathbf{-}\cos x + C$. Diese Minuszeichen werden oft vergessen!
    \item \textbf{Parameter bei Transformationen falsch interpretiert:}
        \begin{itemize}
            \item Bei $A \sin(B(x-C))+D$: Die Periode ist $P=\frac{2\pi}{|B|}$, nicht $2\pi \cdot B$ oder Ähnliches.
            \item Die Phasenverschiebung bei $B(x-C)$ ist $C$ nach rechts. Steht z.B. $\sin(2x-\pi)$, musst du erst $2$ ausklammern zu $\sin(2(x-\frac{\pi}{2}))$, um die korrekte Phasenverschiebung $C=\frac{\pi}{2}$ nach rechts zu erkennen.
        \end{itemize}
    \item \textbf{Definitionsbereich von $\tan x$ ignoriert:} $\tan x$ ist nicht für $x = \frac{\pi}{2} + k\pi$ definiert. An diesen Stellen sind Polstellen (senkrechte Asymptoten).
    \item \textbf{Verwechslung von $\sin^2 x$ mit $\sin(x^2)$:} $\sin^2 x = (\sin x)^2$, während $\sin(x^2)$ bedeutet, dass zuerst $x$ quadriert und dann der Sinus davon genommen wird. Beim Ableiten erfordern sie unterschiedliche Anwendungen der Kettenregel.
\end{itemize}
Achte auf diese Punkte, um typische Fehler zu vermeiden!
\end{fehlerboxumgebung}

\begin{aufgabenumgebung}{Transformierte Sinus- und Kosinusfunktionen}
\begin{enumerate}
    \item Bestimme für die folgenden Funktionen Amplitude, Periode, Phasenverschiebung (Richtung und Betrag) und Verschiebung in y-Richtung. Gib den Wertebereich an.
        \begin{itemize}
            \item $f_1(x) = 3 \cos(2x - \pi) - 1$ (Tipp: Klammere zuerst den Faktor vor dem $x$ in der Klammer aus, um die Form $B(x-C)$ zu erhalten: $2x-\pi = 2(x-\frac{\pi}{2})$)
            \item $f_2(x) = -0.5 \sin(\pi x + \frac{\pi}{4}) + 2$
            \item $f_3(t) = 100 \cos(2\pi \cdot 50 t)$ (Modell für Wechselspannung)
        \end{itemize}
    \item Skizziere den Graphen von $g(x) = \sin(2(x+\frac{\pi}{4}))$ für eine Periode. Beginne mit dem Graphen von $\sin(x)$ und führe die Transformationen schrittweise durch.
    \item Gegeben ist ein Graph einer Sinusfunktion. Bestimme aus dem Graphen die Parameter $A, B, C, D$ und stelle eine mögliche Funktionsgleichung auf.
        \begin{center}
            \includegraphics[width=0.7\textwidth]{grafiken/Trig_Graph_Ablesen.png}
            \captionof{figure}{Graph zum Ablesen der Parameter}
            \label{fig:graph_ablesen}
        \end{center}
\end{enumerate}
\end{aufgabenumgebung}

\subsection{Die Tangensfunktion ($\tan x$) und Kotangensfunktion ($\cot x$)}
\label{subsec:tangens_kotangens}

Neben Sinus und Kosinus gibt es weitere wichtige trigonometrische Funktionen. Die bekannteste davon ist die Tangensfunktion.

\begin{merksatzumgebung}{Definition der Tangensfunktion ($\tan x$)}
Die \textbf{Tangensfunktion} ist definiert als das Verhältnis von Sinus zu Kosinus:
\[ \tan(x) = \frac{\sin(x)}{\cos(x)} \]
\textbf{Geometrische Deutung am Einheitskreis:}
Die Tangensfunktion entspricht der y-Koordinate des Punktes, an dem der verlängerte freie Schenkel des Winkels $x$ die Tangente an den Einheitskreis im Punkt $(1|0)$ schneidet. Sie kann auch als Steigung des freien Schenkels des Winkels $x$ interpretiert werden.

\textbf{Eigenschaften von $f(x)=\tan(x)$:}
\begin{itemize}
    \item \textbf{Definitionsbereich:} $D_f = \mathbb{R} \setminus \{ \frac{\pi}{2} + k\pi \,|\, k \in \mathbb{Z} \}$.
    Der Tangens ist nicht definiert, wenn $\cos(x)=0$ ist (also an den Nullstellen des Kosinus), da man nicht durch Null teilen darf. An diesen Stellen hat der Graph \textbf{senkrechte Asymptoten}.
    \item \textbf{Wertebereich:} $W_f = \mathbb{R}$ (der Tangens kann jeden reellen Wert annehmen).
    \item \textbf{Periodizität:} Die Tangensfunktion ist periodisch mit der \textbf{Periode $\pi$}.
        \[ \tan(x + \pi) = \tan(x) \]
        (Beachte: kürzere Periode als Sinus und Kosinus!)
    \item \textbf{Nullstellen:} $\tan(x) = 0 \Leftrightarrow \sin(x)=0$. Also für $x = k \cdot \pi$, wobei $k \in \mathbb{Z}$.
    \item \textbf{Symmetrie:} $\tan(x)$ ist \textbf{punktsymmetrisch zum Ursprung}: $\tan(-x) = \frac{\sin(-x)}{\cos(-x)} = \frac{-\sin x}{\cos x} = -\tan(x)$.
    \item \textbf{Monotonie:} Die Tangensfunktion ist in jedem ihrer Definitionsintervalle streng monoton steigend.
\end{itemize}
\end{merksatzumgebung}

\begin{center}
    \includegraphics[width=0.9\textwidth]{grafiken/Trig_Tangens_Graph.png}
    \captionof{figure}{Graph der Funktion $f(x)=\tan(x)$}
    \label{fig:tangens_graph}
\end{center}

\begin{infoboxumgebung}{Die Kotangensfunktion ($\cot x$)}
Die \textbf{Kotangensfunktion} ist definiert als $\cot(x) = \frac{\cos(x)}{\sin(x)} = \frac{1}{\tan(x)}$.
Sie ist nicht definiert, wenn $\sin(x)=0$ ist (also bei $x=k\pi$). Ihre Periode ist ebenfalls $\pi$.
Der Kotangens spielt in der Schulmathematik oft eine geringere Rolle als der Tangens, ist aber in manchen Anwendungen nützlich.
\end{infoboxumgebung}

\subsubsection{Ableitungen von Tangens (und Kotangens)}
Die Ableitung der Tangensfunktion können wir mit der Quotientenregel herleiten.

\begin{merksatzumgebung}{Ableitung von $\tan(x)$ und $\cot(x)$}
\begin{itemize}
    \item Die Ableitung von $f(x) = \tan(x)$ ist:
    \[ (\tan x)' = \frac{1}{\cos^2(x)} = 1 + \tan^2(x) \]
    \item Die Ableitung von $g(x) = \cot(x)$ ist:
    \[ (\cot x)' = -\frac{1}{\sin^2(x)} = -(1 + \cot^2(x)) \]
\end{itemize}
\end{merksatzumgebung}

\textbf{Herleitung der Ableitung von $\tan(x)$:}
Wir verwenden $\tan(x) = \frac{\sin(x)}{\cos(x)}$ und die Quotientenregel $(\frac{u}{v})' = \frac{u'v - uv'}{v^2}$.
Sei $u(x) = \sin(x) \implies u'(x) = \cos(x)$.
Sei $v(x) = \cos(x) \implies v'(x) = -\sin(x)$.
$(\tan x)' = \frac{(\cos x)(\cos x) - (\sin x)(-\sin x)}{(\cos x)^2} = \frac{\cos^2(x) + \sin^2(x)}{\cos^2(x)}$.
Mit dem trigonometrischen Pythagoras $\sin^2(x) + \cos^2(x) = 1$ folgt:
$(\tan x)' = \frac{1}{\cos^2(x)}$.
Alternative Form: $\frac{1}{\cos^2(x)} = \frac{\cos^2(x) + \sin^2(x)}{\cos^2(x)} = \frac{\cos^2(x)}{\cos^2(x)} + \frac{\sin^2(x)}{\cos^2(x)} = 1 + \left(\frac{\sin x}{\cos x}\right)^2 = 1 + \tan^2(x)$.

\begin{aufgabenumgebung}{Ableiten mit Tangens}
Bilde die erste Ableitung der folgenden Funktionen:
\begin{enumerate}
    \item $f(x) = 3\tan(x) - x$
    \item $g(x) = x \cdot \tan(x)$ (Produktregel!)
    \item $h(x) = \tan(2x+1)$ (Kettenregel! Äußere Funktion $\tan(u)$, innere $u=2x+1$)
\end{enumerate}
\end{aufgabenumgebung}

\subsection{Anwendung der Ableitungsregeln auf trigonometrische Funktionen}
\label{subsec:ableitungsregeln_trig}

Jetzt kombinieren wir Sinus, Kosinus (und Tangens) mit Polynomen oder anderen Funktionen und wenden unsere bekannten Ableitungsregeln (Produkt-, Quotienten-, Kettenregel) an.

\begin{beispielumgebung}{Kombinierte Ableitungsregeln mit trigonometrischen Funktionen}
\begin{enumerate}
    \item \textbf{Produktregel:} $f(x) = x^2 \sin(x)$.
        $u(x)=x^2 \implies u'(x)=2x$.
        $v(x)=\sin(x) \implies v'(x)=\cos(x)$.
        $f'(x) = u'v + uv' = 2x \sin(x) + x^2 \cos(x)$.

    \item \textbf{Quotientenregel:} $g(x) = \frac{\cos(x)}{x}$. (Für $x \neq 0$)
        $u(x)=\cos(x) \implies u'(x)=-\sin(x)$.
        $v(x)=x \implies v'(x)=1$.
        $g'(x) = \frac{u'v - uv'}{v^2} = \frac{-\sin(x) \cdot x - \cos(x) \cdot 1}{x^2} = \frac{-x\sin(x) - \cos(x)}{x^2}$.

    \item \textbf{Kettenregel:} $h(x) = \sin(3x^2+2)$.
        Äußere Funktion: $a(u)=\sin(u) \implies a'(u)=\cos(u)$.
        Innere Funktion: $b(x)=3x^2+2 \implies b'(x)=6x$.
        $h'(x) = a'(b(x)) \cdot b'(x) = \cos(3x^2+2) \cdot 6x = 6x \cos(3x^2+2)$.

    \item \textbf{Kombination:} $k(x) = e^x \cos(2x)$. (Produkt- und Kettenregel)
        $u(x)=e^x \implies u'(x)=e^x$.
        $v(x)=\cos(2x)$. Für $v'(x)$ brauchen wir die Kettenregel:
            Äußere: $\cos(u) \implies -\sin(u)$. Innere: $2x \implies 2$. Also $v'(x) = -\sin(2x) \cdot 2 = -2\sin(2x)$.
        $k'(x) = u'v + uv' = e^x \cos(2x) + e^x (-2\sin(2x)) = e^x(\cos(2x) - 2\sin(2x))$.
\end{enumerate}
\end{beispielumgebung}

\begin{aufgabenumgebung}{Ableiten trigonometrischer Funktionskombinationen}
Bilde die erste Ableitung der folgenden Funktionen und vereinfache, wenn möglich.
\begin{enumerate}
    \item $f_1(x) = (x^3+1)\cos(x)$
    \item $f_2(x) = \frac{\sin(x)}{e^x}$
    \item $f_3(x) = \cos(x^2+1)$
    \item $f_4(x) = \sin^2(x)$ (Tipp: $\sin^2(x) = (\sin x)^2$. Kettenregel!)
    \item $f_5(x) = \ln(\cos x)$ (Für welche $x$ ist dies definiert?)
    \item $f_6(x) = e^{\sin(x)}$
\end{enumerate}
\end{aufgabenumgebung}

\subsection{Kurvendiskussion von trigonometrischen Funktionen (Beispiele)}
\label{subsec:kurvendiskussion_trig}

Die Kurvendiskussion von reinen Sinus- oder Kosinusfunktionen ist oft einfach, da ihre Eigenschaften (Periode, Amplitude, Nullstellen, Extrema) bekannt sind. Interessanter wird es, wenn sie mit anderen Funktionen kombiniert werden oder transformiert sind.

\begin{beispielumgebung}{Kurvendiskussion von $f(x) = \sin(x) + \cos(x)$ im Intervall $[0, 2\pi]$}
\begin{enumerate}
    \item \textbf{Definitionsbereich:} $D_f = \mathbb{R}$, hier betrachten wir $[0, 2\pi]$.
    \item \textbf{Symmetrie:} $f(-x) = \sin(-x)+\cos(-x) = -\sin(x)+\cos(x)$. Weder achsen- noch punktsymmetrisch zum Ursprung.
    \item \textbf{Grenzwerte:} Nicht relevant für ein abgeschlossenes Intervall. Periodisch mit $P=2\pi$.
    \item \textbf{y-Achsenabschnitt:} $f(0) = \sin(0)+\cos(0) = 0+1=1$. $P_y(0|1)$.
    \item \textbf{Nullstellen:} $f(x)=0 \implies \sin(x)+\cos(x)=0 \implies \sin(x)=-\cos(x)$.
        Wenn $\cos(x) \neq 0$, können wir teilen: $\frac{\sin x}{\cos x} = -1 \implies \tan(x)=-1$.
        Im Intervall $[0, 2\pi]$ sind die Lösungen $x_1 = \frac{3\pi}{4}$ und $x_2 = \frac{7\pi}{4}$.
        $N_1(\frac{3\pi}{4}|0)$, $N_2(\frac{7\pi}{4}|0)$.
    \item \textbf{Erste Ableitung:} $f'(x) = \cos(x) - \sin(x)$.
    \item \textbf{Extremstellen:} $f'(x)=0 \implies \cos(x) - \sin(x)=0 \implies \cos(x)=\sin(x)$.
        Wenn $\cos(x) \neq 0$: $1 = \tan(x)$.
        Im Intervall $[0, 2\pi]$ sind die Lösungen $x_{E1} = \frac{\pi}{4}$ und $x_{E2} = \frac{5\pi}{4}$.
    \item \textbf{Zweite Ableitung:} $f''(x) = -\sin(x) - \cos(x) = -(\sin x + \cos x) = -f(x)$.
    \item \textbf{Art der Extremstellen:}
        $f''(\frac{\pi}{4}) = -\sin(\frac{\pi}{4}) - \cos(\frac{\pi}{4}) = -\frac{\sqrt{2}}{2} - \frac{\sqrt{2}}{2} = -\sqrt{2} < 0 \implies$ Hochpunkt.
        $y_H = f(\frac{\pi}{4}) = \sin(\frac{\pi}{4})+\cos(\frac{\pi}{4}) = \frac{\sqrt{2}}{2} + \frac{\sqrt{2}}{2} = \sqrt{2} \approx 1.414$. $H(\frac{\pi}{4}|\sqrt{2})$.
        $f''(\frac{5\pi}{4}) = -\sin(\frac{5\pi}{4}) - \cos(\frac{5\pi}{4}) = -(-\frac{\sqrt{2}}{2}) - (-\frac{\sqrt{2}}{2}) = \sqrt{2} > 0 \implies$ Tiefpunkt.
        $y_T = f(\frac{5\pi}{4}) = \sin(\frac{5\pi}{4})+\cos(\frac{5\pi}{4}) = -\frac{\sqrt{2}}{2} - \frac{\sqrt{2}}{2} = -\sqrt{2}$. $T(\frac{5\pi}{4}|-\sqrt{2})$.
    \item \textbf{Wendepunkte:} $f''(x_W)=0 \implies -(\sin(x_W)+\cos(x_W))=0 \implies \sin(x_W)+\cos(x_W)=0$.
        Das sind dieselben Stellen wie die Nullstellen der Funktion: $x_{W1}=\frac{3\pi}{4}, x_{W2}=\frac{7\pi}{4}$.
        Dritte Ableitung: $f'''(x) = -f'(x) = -(\cos x - \sin x) = \sin x - \cos x$.
        $f'''(\frac{3\pi}{4}) = \sin(\frac{3\pi}{4}) - \cos(\frac{3\pi}{4}) = \frac{\sqrt{2}}{2} - (-\frac{\sqrt{2}}{2}) = \sqrt{2} \neq 0$.
        $f'''(\frac{7\pi}{4}) = \sin(\frac{7\pi}{4}) - \cos(\frac{7\pi}{4}) = -\frac{\sqrt{2}}{2} - \frac{\sqrt{2}}{2} = -\sqrt{2} \neq 0$.
        Wendepunkte bei $W_1(\frac{3\pi}{4}|0)$ und $W_2(\frac{7\pi}{4}|0)$ (also die Nullstellen).
    \item \textbf{Skizze:}
        \begin{center}
            \includegraphics[width=0.9\textwidth]{grafiken/Trig_Kurvendiskussion_SinPlusCos.png}
            \captionof{figure}{Graph von $f(x)=\sin(x)+\cos(x)$}
            \label{fig:kurvendisk_sinpluscos}
        \end{center}
\end{enumerate}
\end{beispielumgebung}

\begin{aufgabenumgebung}{Kurvendiskussionen mit trigonometrischen Funktionen}
Führe eine möglichst vollständige Kurvendiskussion für die folgenden Funktionen im angegebenen Intervall durch und skizziere den Graphen. Bestimme alle Eigenschaften (Definitionsbereich, Symmetrie, Verhalten an den Rändern, y-Achsenabschnitt, Ableitungen, Monotonie, Krümmung) analytisch, soweit dies mit den dir bekannten Methoden möglich ist. 

Für Nullstellen oder die x-Koordinaten von Extrem- und Wendepunkten, deren exakte Berechnung auf \textbf{transzendente Gleichungen} führt (Gleichungen, die algebraisch nicht einfach nach der Variablen aufgelöst werden können, z.B. wenn $x$ sowohl innerhalb als auch außerhalb einer trigonometrischen Funktion steht), ist eine exakte algebraische Lösung oft nicht das Ziel.
\begin{itemize}
    \item Versuche in solchen Fällen, die \textit{Existenz} von Lösungen durch Überlegungen (z.B. Zwischenwertsatz, falls bekannt, oder Monotonie) zu begründen oder zumindest plausibel zu machen.
    \item Du kannst dann \textbf{digitale Werkzeuge} (wie Wolfram Alpha, GeoGebra oder einen grafikfähigen Taschenrechner) verwenden, um Näherungswerte für diese speziellen x-Werte zu ermitteln. Notiere in deiner Lösung, wenn du solche Näherungswerte verwendest, um deine Skizze zu vervollständigen und die Analyse abzurunden.
\end{itemize}
Der Schwerpunkt liegt auf dem Verständnis der analytischen Schritte und der Interpretation der Ergebnisse.

\begin{enumerate}
    \item $f(x) = 2\sin(x) - x$ im Intervall $[0, 2\pi]$.
        \begin{tippumgebung}{Umgang mit Nullstellen von $f(x)$}
        Du wirst feststellen, dass $x=0$ eine Nullstelle ist. Die Gleichung $2\sin(x) = x$ für weitere Nullstellen ist transzendent. Du kannst grafisch argumentieren, dass es eine weitere Nullstelle im Intervall gibt (z.B. indem du $y=2\sin x$ und $y=x$ vergleichst) oder deren ungefähre Lage mit einem digitalen Werkzeug bestimmen. Konzentriere dich ansonsten auf die exakte Berechnung der Ableitungen, Extrem- und Wendestellen, soweit möglich.
        \end{tippumgebung}
    \item $g(x) = x \cdot \cos(x)$ im Intervall $[-\pi, \pi]$.
        \begin{tippumgebung}{Umgang mit Extremstellen von $g(x)$}
        Die Nullstellen von $g(x)$ sind exakt bestimmbar. Die notwendige Bedingung für Extremstellen ($g'(x)=0$) führt hier jedoch auf die transzendente Gleichung $\cos(x) = x\sin(x)$. Untersuche die Ableitung $g'(x)$ an markanten Punkten (z.B. Nullstellen von $g(x)$ oder Ränder des Intervalls), um Bereiche mit unterschiedlichem Monotonieverhalten zu identifizieren. Für die genaue Lage der Extremstellen kannst du Näherungswerte aus digitalen Werkzeugen verwenden und dies vermerken.
        \end{tippumgebung}
\end{enumerate}
\end{aufgabenumgebung}

% \begin{kurzknappumgebung}{Trigonometrische Funktionen – Analysis}
% \begin{itemize}
%     \item \textbf{Bogenmaß} ist Standard für die Analysis.
%     \item \textbf{Ableitungen:} $(\sin x)' = \cos x$, $(\cos x)' = -\sin x$, $(\tan x)' = \frac{1}{\cos^2 x}$.
%     \item \textbf{Stammfunktionen:} $\int \sin x \,dx = -\cos x + C$, $\int \cos x \,dx = \sin x + C$.
%     \item \textbf{Transformationen $A \sin(B(x-C))+D$:} Amplitude $|A|$, Periode $P=\frac{2\pi}{|B|}$, Verschiebung $C$ (x-Richtung), $D$ (y-Richtung).
%     \item \textbf{Kettenregel} ist entscheidend für verkettete trigonometrische Funktionen (z.B. $\sin(kx)$ oder $\cos(x^2)$).
%     \item \textbf{Produkt- und Quotientenregel} für Kombinationen mit anderen Funktionen (z.B. $x \sin x$ oder $\frac{\sin x}{x}$).
%     \item \textbf{Kurvendiskussion} folgt den bekannten Schritten, Periodizität und Symmetrie sind oft hilfreich.
% \end{itemize}
% \end{kurzknappumgebung}



% Dieser Block sollte am ENDE des Kapitels zu den trigonometrischen Funktionen eingefügt werden,
% bevor das allgemeine Abschlusskapitel des gesamten Dokuments beginnt.

% Annahme: Der Hauptteil des Kapitels zu trigonometrischen Funktionen wurde bereits behandelt
% (Definitionen, Eigenschaften, Ableitungen, Stammfunktionen, Kurvendiskussionen von sin, cos, tan etc.)

\begin{kurzknappumgebung}{Trigonometrische Funktionen – Das Wichtigste im Überblick}
\begin{itemize}
    \item \textbf{Grundfunktionen:} $\sin(x)$, $\cos(x)$, $\tan(x) = \frac{\sin(x)}{\cos(x)}$.
    \item \textbf{Bogenmaß:} Standard für Winkelangaben in der Analysis ($2\pi \widehat{=} 360^\circ$).
    \item \textbf{Eigenschaften $\sin(x), \cos(x)$:}
        \begin{itemize}
            \item Definitionsbereich: $\mathbb{R}$. Wertebereich: $[-1, 1]$.
            \item Periode: $2\pi$.
            \item Nullstellen: $\sin(x)=0$ für $x=k\pi$; $\cos(x)=0$ für $x=\frac{\pi}{2}+k\pi$.
            \item Symmetrie: $\sin(x)$ ist punktsymmetrisch zum Ursprung, $\cos(x)$ ist achsensymmetrisch zur y-Achse.
            \item Trigonometrischer Pythagoras: $\sin^2(x) + \cos^2(x) = 1$.
        \end{itemize}
    \item \textbf{Eigenschaften $\tan(x)$:}
        \begin{itemize}
            \item Definitionsbereich: $\mathbb{R} \setminus \{ \frac{\pi}{2} + k\pi \}$. Polstellen bei $x = \frac{\pi}{2} + k\pi$.
            \item Wertebereich: $\mathbb{R}$. Periode: $\pi$.
            \item Nullstellen: $x=k\pi$. Punktsymmetrisch zum Ursprung.
        \end{itemize}
    \item \textbf{Wichtige Ableitungen:}
        \begin{itemize}
            \item $(\sin x)' = \cos x$
            \item $(\cos x)' = -\sin x$
            \item $(\tan x)' = \frac{1}{\cos^2 x} = 1 + \tan^2 x$
        \end{itemize}
    \item \textbf{Wichtige Stammfunktionen:}
        \begin{itemize}
            \item $\int \sin x \,dx = -\cos x + C$
            \item $\int \cos x \,dx = \sin x + C$
            \item $\int \frac{1}{\cos^2 x} \,dx = \tan x + C$ (seltener benötigt)
        \end{itemize}
    \item \textbf{Transformationen:} Funktionen der Form $A \cdot \sin(B(x-C)) + D$ beschreiben allgemeine Sinusschwingungen mit Amplitude $|A|$, Periode $P=\frac{2\pi}{|B|}$, Phasenverschiebung $C$ und Mittellage $D$.
    \item \textbf{Anwendungen:} Modellierung von periodischen Vorgängen, Schwingungen, Wellen.
\end{itemize}
\end{kurzknappumgebung}

\begin{aufgabenumgebung}{Integrationsregeln im Mix}
Nachdem du nun vielfältige Funktionen und Integrationstechniken kennengelernt hast, soll diese Aufgabe dein Verständnis und deine Fertigkeiten bei der Kombination verschiedener Regeln prüfen.

Berechne die folgenden unbestimmten Integrale. Gib an, welche Methode(n) (z.B. partielle Integration, Substitution, Grundintegrale) du verwendest.
\begin{enumerate}[label=(\alph*)]
    \item $\int x^2 \cos(x) \,dx$
        \begin{tippumgebung}{Mehrfache partielle Integration}
        Hier ist die partielle Integration zweimal anzuwenden, ähnlich wie bei $\int x^2 e^x \,dx$. Wähle den Polynomteil zum Ableiten.
        \end{tippumgebung}
    \item $\int \sin(x) \cdot e^{\cos(x)} \,dx$
        \begin{tippumgebung}{Substitution mit $e$-Funktion}
        Erkennst du eine innere Funktion $g(x)$, deren Ableitung $g'(x)$ (oder ein Vielfaches davon) ebenfalls im Integranden vorkommt? Denke an die Ableitung des Exponenten.
        \end{tippumgebung}
    \item $\int e^{2x} \cos(x) \,dx$
        \begin{tippumgebung}{Der 'Trick-Integral' mit Varianten}
        Diese Aufgabe ähnelt dem Integral $\int e^x \sin(x) \,dx$. Du wirst zweimal partiell integrieren müssen, bevor das ursprüngliche Integral (oder ein Vielfaches davon) wieder auf der rechten Seite erscheint und du die Gleichung danach auflösen kannst.
        \end{tippumgebung}
    \item \textbf{Für Experten:} $\int \sin(\ln x) \,dx$
        \begin{tippumgebung}{Knifflige Kombination}
        Versuche zuerst eine Substitution $u = \ln x$, woraus $x = e^u$ und $dx = e^u du$ folgt. Das führt zu einem Integral der Form $\int \sin(u) e^u du$, das du bereits aus einer früheren Herausforderung (Aufgabe 7.12, Teil f) kennst oder mit der dort beschriebenen Methode lösen kannst. Vergiss am Ende die Rücksubstitution nicht!
        \end{tippumgebung}
    \item \textbf{Flächenberechnung (Herausforderung):} Bestimme den Inhalt der Fläche, die vom Graphen der Funktion $f(x) = e^x \sin(x)$ und der x-Achse im Intervall $[0, \pi]$ eingeschlossen wird. Fertige eine Skizze an. Du benötigst die Stammfunktion aus Aufgabe (c) (oder der früheren Herausforderung). Beachte, dass $\sin(x)$ im Intervall $[0,\pi]$ nicht negativ ist.
\end{enumerate}
Diese Aufgaben erfordern sorgfältiges Anwenden der Regeln und oft auch mehrere Schritte. Viel Erfolg!
\end{aufgabenumgebung}





\begin{infoboxumgebung}{Die Welt der Wellen und Kreise – Ein Nachklang zu den trigonometrischen Funktionen}
Mit den trigonometrischen Funktionen hast du die mathematischen Werkzeuge kennengelernt, um die vielfältigen periodischen Rhythmen und zyklischen Muster zu beschreiben, die uns in der Natur und Technik überall begegnen – von den harmonischen Schwingungen einer gestimmten Saite über die Wellen des Lichts bis hin zu den komplexen Überlagerungen von Signalen in der Kommunikationstechnik.

Die Eleganz, mit der Sinus und Kosinus Kreisbewegungen und periodische Phänomene erfassen, ist ein weiteres wunderbares Beispiel für die tiefe Verbindung zwischen Geometrie und Analysis. Die Untersuchung ihrer Ableitungen und Integrale hat dir gezeigt, wie die Werkzeuge der Analysis auch auf diese speziellen Funktionen angewendet werden können, um ihr Verhalten genau zu verstehen und Vorhersagen zu treffen.

Vielleicht hast du bei der Beschäftigung mit Sinus und Kosinus auch schon eine Ahnung von noch tieferen Zusammenhängen bekommen, wie beispielsweise der Eulerschen Formel ($e^{ix} = \cos x + i \sin x$). Diese verblüffende Gleichung schlägt eine Brücke zwischen der Welt der Exponentialfunktionen (mit der Basis $e$) und der Welt der Trigonometrie und öffnet gleichzeitig die Tür zum Reich der komplexen Zahlen – ein weiteres faszinierendes Feld der Mathematik.

Die Reise in die Welt der Schwingungen, Wellen und periodischen Muster ist voller Entdeckungen und zeigt eindrücklich, wie universell und mächtig die Sprache der Mathematik ist, um die Phänomene um uns herum zu beschreiben und zu verstehen.
\end{infoboxumgebung}




\begin{aufgabenumgebung}{Checkliste: Bogenmaß, Einheitskreis und Grundeigenschaften verstehen}
Die Basis der trigonometrischen Funktionen liegt im Einheitskreis und der Verwendung des Bogenmaßes. Überprüfe dein Verständnis:

\begin{enumerate}[label=(\alph*)]
    \item \textbf{Bogenmaß vs. Gradmaß:}
    \begin{itemize}
        \item Erkläre mit eigenen Worten, was das Bogenmaß eines Winkels darstellt. Warum ist $180^\circ = \pi \text{ rad}$?
        \item Warum ist es in der Analysis (insbesondere beim Ableiten) wichtig, Winkel im Bogenmaß anzugeben? (Tipp: Denke an die Einfachheit der Ableitungsformeln.)
    \end{itemize}
    \item \textbf{Sinus und Kosinus am Einheitskreis:}
    \begin{itemize}
        \item Wie hängen die Koordinaten eines Punktes $P(x_P|y_P)$ auf dem Einheitskreis mit $\sin(\alpha)$ und $\cos(\alpha)$ zusammen, wenn $\alpha$ der Winkel zwischen der positiven x-Achse und dem Strahl zum Punkt $P$ ist?
        \item Erkläre mithilfe des Einheitskreises, warum der Wertebereich von $\sin(x)$ und $\cos(x)$ das Intervall $[-1, 1]$ ist.
        \item Leite die Identität $\sin^2(x) + \cos^2(x) = 1$ (Trigonometrischer Pythagoras) aus der Definition am Einheitskreis her.
    \end{itemize}
    \item \textbf{Periodizität und Symmetrie:}
    \begin{itemize}
        \item Was bedeutet es, dass $\sin(x)$ und $\cos(x)$ periodisch mit der Periode $2\pi$ sind? Wie zeigt sich das am Einheitskreis und im Graphen?
        \item Erkläre die Punktsymmetrie von $\sin(x)$ zum Ursprung ($\sin(-x)=-\sin x$) und die Achsensymmetrie von $\cos(x)$ zur y-Achse ($\cos(-x)=\cos x$) anhand des Einheitskreises.
    \end{itemize}
\end{enumerate}
\end{aufgabenumgebung}

\begin{aufgabenumgebung}{Checkliste: Transformationen und Analysis trigonometrischer Funktionen}
Die Grundfunktionen $\sin(x)$ und $\cos(x)$ können transformiert und mit den Werkzeugen der Differential- und Integralrechnung analysiert werden.

\begin{enumerate}[label=(\alph*)]
    \item \textbf{Transformierte Sinusfunktion $f(x) = A \sin(B(x-C)) + D$:}
    \begin{itemize}
        \item Wenn du eine Schwingung modellieren möchtest, die doppelt so hoch ausschlägt wie $\sin(x)$ und nur halb so schnell schwingt (also die doppelte Periode hat), wie müsstest du $A$ und $B$ wählen (angenommen $A>0, B>0$)?
        \item Wie verschiebt sich der 'Standard-Startpunkt' (der bei $\sin(x)$ im Ursprung mit positiver Steigung liegt) durch die Parameter $C$ und $D$? Was ist die Gleichung der Mittellage?
    \end{itemize}
    \item \textbf{Ableitungen und ihre Bedeutung:}
    \begin{itemize}
        \item Es gilt $(\sin x)' = \cos x$. Betrachte die Graphen von $\sin x$ und $\cos x$: An welchen Stellen hat $\sin x$ seine Extrempunkte (Hoch-/Tiefpunkte)? Was ist an diesen Stellen der Wert von $\cos x$ (also der Ableitung)? Passt das zur Regel, dass die Ableitung an Extremstellen Null ist?
        \item Umgekehrt: Wo hat $\cos x$ seine Nullstellen? Was für Punkte (bezüglich Steigung) hat $\sin x$ an diesen Stellen?
        \item Warum ist das Vorzeichen bei $(\cos x)' = -\sin x$ negativ? Versuche, dies anhand der Steigung des Kosinusgraphen zu erklären, wenn $\sin x$ positiv ist.
    \end{itemize}
    \item \textbf{Integration und Fläche:}
    \begin{itemize}
        \item Das bestimmte Integral $\int_0^{2\pi} \cos(x) \,dx = 0$. Erkläre dieses Ergebnis geometrisch anhand des Graphen von $\cos(x)$ und dem Konzept der orientierten Fläche.
        \item Wie würdest du vorgehen, um den \textit{gesamten geometrischen Flächeninhalt} zu berechnen, den der Graph von $f(x)=\sin(x)$ mit der x-Achse im Intervall $[0, 2\pi]$ einschließt? (Tipp: Nullstellen und Beträge).
    \end{itemize}
    \item \textbf{Tangensfunktion:}
    \begin{itemize}
        \item Warum ist die Tangensfunktion $\tan(x) = \frac{\sin x}{\cos x}$ nicht für alle $x \in \mathbb{R}$ definiert? Was befindet sich an den Definitionslücken im Graphen?
        \item Die Ableitung von $\tan x$ ist $(\tan x)' = \frac{1}{\cos^2 x}$. Ist diese Ableitung jemals negativ? Was bedeutet das für die Monotonie der Tangensfunktion in ihren Definitionsintervallen?
    \end{itemize}
\end{enumerate}
\end{aufgabenumgebung}



\section*{Ein letztes Wort zu den Lösungen – Dein Erfolg!}
\label{sec:abschluss_loesungsbuch}

Liebe Leserin, lieber Leser,

wenn du bis hierher gekommen bist und dich nicht nur durch das Hauptskript, sondern auch intensiv mit den Aufgaben und diesen Lösungswegen auseinandergesetzt hast, dann hast du allen Grund, stolz auf dich zu sein! \smiley{} Du hast Ausdauer bewiesen, dich Herausforderungen gestellt und einen weiten Weg in der Welt der Analysis zurückgelegt – vom einfachen Dreisatz bis hin zu den komplexen Ideen der Differential- und Integralrechnung sowie der Untersuchung spezieller Funktionen.

\begin{center}
    \Large\bfseries Herzlichen Glückwunsch zu dieser beachtlichen Leistung!
\end{center}
\vspace{1em}

Dieses Lösungsbuch war dafür gedacht, dich auf deiner Reise zu begleiten – nicht als Abkürzung, sondern als eine helfende Hand, wenn du sie brauchtest, als eine Möglichkeit zur Überprüfung deiner eigenen Gedanken und als eine Quelle für neue Perspektiven auf Lösungsansätze. Wir hoffen, es hat dir geholfen:
\begin{itemize}
    \item deine eigenen Lösungswege zu überprüfen und zu festigen.
    \item Denkfehler zu erkennen und aus ihnen zu lernen.
    \item die Struktur und Logik hinter den mathematischen Methoden noch tiefer zu verstehen.
    \item Sicherheit im Umgang mit den verschiedenen Konzepten zu gewinnen.
\end{itemize}

Wenn du die Ratschläge aus dem Vorwort dieses Lösungsbuchs beherzigt und stets versucht hast, die Aufgaben erst eigenständig zu lösen, dann hast du den größten Nutzen aus diesem Material gezogen. Denke daran: Der Prozess des Lernens, das eigene Ringen um Lösungen und das Verstehen von Zusammenhängen sind das, was dich mathematisch wirklich voranbringt.

\begin{infoboxumgebung}{Der Wert des Durcharbeitens}
Sich durch ein umfangreiches mathematisches Thema von den Grundlagen bis zu anspruchsvollen Anwendungen durchzuarbeiten, ist eine Leistung, die weit über das reine Auswendiglernen von Formeln hinausgeht. Du hast gelernt, analytisch zu denken, Probleme strukturiert anzugehen und komplexe Ideen zu durchdringen. Diese Fähigkeiten sind nicht nur in der Mathematik, sondern in unzähligen Lebensbereichen von unschätzbarem Wert.

Die Tatsache, dass du dich auch mit den Lösungswegen aktiv beschäftigt hast – sei es durch Vergleichen, Nachrechnen oder sogar durch das kritische Hinterfragen – zeigt dein Engagement und deinen Willen, die Dinge wirklich zu verstehen.
\end{infoboxumgebung}

Behalte dir diese Neugier und diesen Forschergeist! Die Mathematik bietet noch unzählige weitere spannende Gebiete und Herausforderungen. Mit dem Fundament, das du dir erarbeitet hast, und der Fähigkeit, dich auch in komplexe Lösungswege einzuarbeiten, bist du bestens gerüstet für alles, was noch kommt – sei es im weiteren Schulverlauf, im Studium oder auf deinem ganz persönlichen Wissensweg.

Wir hoffen, dieses Lösungsbuch war dir ein guter und verlässlicher Partner.

\vspace{1em}
\begin{center}
    \Large\itshape Weiterhin viel Freude und Erfolg auf all deinen Wegen!
\end{center}
\vspace{2em}

% Ende des Abschlusses für das Lösungsbuch
% Fügen Sie hier weitere Lösungen für andere Aufgaben ein.
% Sie können die Umgebungen aufgabenumgebung und beispielumgebung verwenden,
% um die Aufgabenstellung und die entsprechende Lösung zu strukturieren.
% Nutzen Sie auch merksatzumgebung, tippumgebung, fehlerboxumgebung etc.,
% um zusätzliche Erklärungen und Hinweise zu geben.

\end{document}
